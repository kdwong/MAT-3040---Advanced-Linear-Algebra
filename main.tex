\documentclass[11pt,openany]{book}

% Encoding and fonts
\usepackage[utf8]{inputenc}
\usepackage[T1]{fontenc}
\usepackage{lmodern}

% Math packages
\usepackage{amsmath, amssymb, amsthm, mathtools, bm}

% Layout and spacing
\usepackage{geometry}
\geometry{margin=1in}
\usepackage{setspace}
\onehalfspacing

% Theorem environments with cleveref support
\usepackage{aliascnt}


\theoremstyle{plain}
\newtheorem{theorem}{Theorem}[chapter]
\newaliascnt{lemma}{theorem}
\newtheorem{lemma}[lemma]{Lemma}
\aliascntresetthe{lemma}
\providecommand*{\lemmaautorefname}{Lemma}

\newaliascnt{proposition}{theorem}
\newtheorem{proposition}[proposition]{Proposition}
\aliascntresetthe{proposition}
\providecommand*{\propositionautorefname}{Proposition}

\newaliascnt{corollary}{theorem}
\newtheorem{corollary}[corollary]{Corollary}
\aliascntresetthe{corollary}
\providecommand*{\corollaryautorefname}{Corollary}

\theoremstyle{definition}
\newaliascnt{definition}{theorem}
\newtheorem{definition}[definition]{Definition}
\aliascntresetthe{definition}
\providecommand*{\definitionautorefname}{Definition}

\newaliascnt{example}{theorem}
\newtheorem{example}[example]{Example}
\aliascntresetthe{example}
\providecommand*{\exampleautorefname}{Example}

\theoremstyle{remark}
\newaliascnt{remark}{theorem}
\newtheorem{remark}[remark]{Remark}
\aliascntresetthe{remark}
\providecommand*{\remarkautorefname}{Remark}

% Enumeration
\usepackage{enumitem}
\setlist{topsep=4pt,itemsep=2pt,parsep=2pt}

% Graphics and diagrams
\usepackage{graphicx}
\usepackage{tikz-cd}

% Header and footer
\usepackage{fancyhdr}
\pagestyle{fancy}
\fancyhead[L]{\leftmark}
\fancyhead[R]{\thepage}
\fancyfoot[C]{}

% Colors and hyperlinks
\usepackage{xcolor}
\definecolor{darkblue}{rgb}{0,0,0.5}
\usepackage{hyperref}
\hypersetup{
  colorlinks=true,
  linkcolor=darkblue,
  citecolor=darkblue,
  urlcolor=darkblue,
  pdfborder={0 0 0}
}

% Title
\title{MAT 3040 - Advanced Linear Algebra}
\date{}
\begin{document}
\maketitle
\chapter*{Foreword}
\addcontentsline{toc}{chapter}{Foreword}


This book is taken notes from the MAT 3040 in spring semester, 2019. These lecture notes were taken and compiled in \LaTeX\ by Jie Wang, an undergraduate student in spring 2019. The tex writter would like to thank Prof. Daniel Wong and some students for their detailed and valuable comments and suggestions, which significantly improved the quality of this notebook. Students taking this course may use the notes as part of their reading and reference materials. This version of the lecture notes were revised and extended for many times, but may still contain many mistakes and typos, including English grammatical and spelling errors. It would be greatly appreciated if those students, who will use the notes as their reading or reference material, tell any mistakes and typos to Jie Wang for improving this notebook.
\tableofcontents

% Include chapters

\chapter*{Notations and Conventions}
\addcontentsline{toc}{chapter}{Notations and Conventions}


\begin{tabular}{rl}
\({\mathbb{F}}^n\) & \(n\)-dimensional \(\mathbb{F}\)-valued space\\

 \({M}_{m \times  n}\left( \mathbb{F}\right)\) &set of all \((m \times  n)\) \(\mathbb{F}\)-valued matrices\\

 \(\oplus  \) & Direct sum \\

 \(\ker \left( T\right) \) & The null space of \(T\)\\

 \(V \cong  W\) & vector spaces \(V\) and \(W\) are isomorphic\\

 \({\left( T\right) }_{\mathcal{B},\mathcal{A}}\) & Matrix representation of \(T\) with respect to \(\mathcal{A}\) and \(\mathcal{B}\)\\

 \(\mathbf{v} + W\) & coset of \(\mathbf{v}\) in $W \leq V$, i.e., \(\{ \mathbf{v} + \mathbf{w} \mid  \mathbf{w} \in  W\}\)\\

 \({A}_{i}^{\mathrm{T}}\)& \(i^{th}\) row of matrix \(A\)\\

  \(V/W\) & Quotient space of \(V\) by the subspace \(W \leq V\)\\

  \(V^*\)& Dual space of \(V\), i.e. the set of linear transformations from \(V\) to \(\mathbb{F}\)\\

  \(\mathrm{Ann}\left( S\right) \) & The annihilator of \(S \subseteq  V\), i.e. \(\left\{  {f \in  {V}^{ * } \mid  f\left( s\right)  = 0,\forall s \in  S}\right\}\)\\

 \({T}^{ * } : {W}^{ * } \rightarrow  {V}^{ * }\) & Adjoint map  for the mapping \(T : V \rightarrow  W\)\\

\(A^t\) & transpose of \(A\), i.e, \(B = A^t\) means \({b}_{ji} = {{a}}_{ij}\) for all \(i,j\)\\

 \(A^{\mathrm{H}}\) & Hermitian transpose of \(A\), i.e, \(B = A^{\mathrm{H}}\) means \({b}_{ji} = {\bar{a}}_{ij}\) for all \(i,j\)\\

 \({\mathcal{X}}_{T}\left( x\right) \) & characteristic polynomial of \(T\)\\

 \({m}_{T}\left( x\right) \) & Minimal polynomial of the linear operator \(T\)\\

 \({T}^{\prime } : V \rightarrow  V\) & Hermitian Adjoint for \(T : V \rightarrow  V\)\\

 \(\langle \mathbf{v},\mathbf{w}\rangle \) & Inner product between vectors \(\mathbf{v}\) and \(\mathbf{w}\)\\

 \(V \otimes  W\) & Tensor product between vector spaces \(V\) and \(W\)\\

 \(V \land  V\) & Wedge product for vector space \(V\)
\end{tabular}
\chapter{Review of Vector Space and Basis}

\section{Introduction}

Advanced Linear Algebra (MAT 3040) is one of the most important courses in MATH major, with prerequisite MAT 2040 and MAT 1001/1011. This course will offer {\bf true} linear algebra knowledge. 

\medskip

{\bf What will be covered?}
\begin{itemize}
\item In MAT 2040 we have studied the space \({\mathbb{R}}^n\) ; while in MAT 3040 we will study the general vector space \(V\).

\item In MAT 2040 we have studied the linear transformation between Euclidean spaces, i.e., \(T : {\mathbb{R}}^n \rightarrow  {\mathbb{R}}^{m}\) ; while in MAT 3040 we will study the linear transformation from vector spaces to vector spaces: \(T : V \rightarrow  W\)

\item In MAT 2040 we have studied the eigenvalues of \(n \times  n\) matrix \(A\) ; while in MAT 3040 we will study the eigenvalues of a linear operator \(T : V \rightarrow  V\).

\item In MAT 2040 we have studied the dot product \(\mathbf{x} \cdot  \mathbf{y} = \mathop{\sum }\limits_{{i = 1}}^n{x}_{i}{y}_{i}\) ; while in MAT 3040 we will study the inner product \(\left\langle  {{\bf v}_1,{\bf v}_2}\right\rangle\).
\end{itemize}

{\bf Why generalize?} In mathematics (and beyond), we come across many other spaces than $\mathbb{R}^n$, for example:
\begin{itemize}
\item \(\mathcal{C}\left( \mathbb{R}\right)\), the space of all functions on \(\mathbb{R}\),
\item \({\mathcal{C}}^{\infty }\left( \mathbb{R}\right)\), the space of all infinitely differentiable functions on \(\mathbb{R}\),
\item \(\mathbb{R}[x]\), the space of polynomials of one real variable.
\end{itemize}

Here are some applications in partial differential equations and physics:
\begin{example}
1. Consider the Laplace equation \({\Delta f} = 0\) with Laplace operator:

\[
\Delta  : {C}^{\infty }\left( {\mathbb{R}}^{3}\right)  \rightarrow  {C}^{\infty }\left( {\mathbb{R}}^{3}\right), \quad \quad f \mapsto  \left( {\frac{{\partial }^2}{\partial {x}^2} + \frac{{\partial }^2}{\partial {y}^2} + \frac{{\partial }^2}{\partial {z}^2}}\right) f
\]
The solution to the partial differential equation \({\Delta f} = 0\) is the $0$-eigenspace of \(\Delta\).

\medskip
2. Consider the Schrödinger equation \(\widehat{H}f = {Ef}\) with the Schrödinger operator
\[
\widehat{H} : {C}^{\infty }\left( {\mathbb{R}}^{3}\right)  \rightarrow  {C}^{\infty }\left( {\mathbb{R}}^{3}\right),\quad \quad f \mapsto  \left(  {\frac{-{\hslash }^2}{2\mu }{\nabla }^2 + V\left( {x,y,z}\right) }\right)  f
\]

Solving the equation \(\widehat{H}f = \lambda f\) is equivalent to finding the $\lambda$-eigenspace of \(\widehat{H}\). In fact, a major observation in quantum mechanics is that the eigenvalues of \(\widehat{H}\) are discrete.    
\end{example}


\section{Vector Spaces}

\begin{definition}[Vector Space] A {\bf vector space} over a field \(\mathbb{F}\) (in particular, \(\mathbb{F} = \mathbb{R}\) or \(\mathbb{C}\)) is a set of objects \(V\) equipped with vector addiction $+$ and scalar multiplication $\cdot$ such that the following rules are satisfied:
\begin{itemize}
    \item Commutativity: \({\bf v}_1 + {\bf v}_2 = {\bf v}_2 + {\bf v}_1\).

    \item Associativity: \({\bf v}_1 + \left( {{\bf v}_2 + {\bf v}_{3}}\right)  = \left( {{\bf v}_1 + {\bf v}_2}\right)  + {\bf v}_{3}\).

    \item Zero vector: there exists \(\mathbf{0} \in  V\) such that \(\mathbf{0} + \mathbf{v} = \mathbf{v}\) for all \(\mathbf{v} \in  V\).

    \item Additive inverse: for all ${\bf v} \in V$, there exists \(-{\bf v} \in V\) such that ${\bf v} + (-{\bf v}) = {\bf 0}$.

    \item Distributivity of vectors: \(\alpha \left( {{\bf v}_1 + {\bf v}_2}\right)  = \alpha {\bf v}_1 + \alpha {\bf v}_2\).

    \item Distributivity of scalars: \(\left( {\alpha_1 + \alpha_2}\right) \mathbf{v} = \alpha_1\mathbf{v} + \alpha_2\mathbf{v}\).

    \item Compatibility of scalar multiplication: \(\alpha_1\left( {\alpha_2\mathbf{v}}\right)  = \left( {\alpha_1\alpha_2}\right) \mathbf{v}\).

    \item Multiplicative identity: \(1\mathbf{v} = \mathbf{v}\).
\end{itemize}
\end{definition}

Here we study several examples of vector spaces:

\begin{example} For \(V = {\mathbb{F}}^n\), we can define scalar multiplication:
\[
\alpha \left( \begin{matrix} {x}_1 \\  \vdots \\  {x}_n \end{matrix}\right)  = \left( \begin{matrix} \alpha {x}_1 \\  \vdots \\  \alpha {x}_n \end{matrix}\right)
\]
and vector addition:
\[
\left( \begin{matrix} {x}_1 \\  \vdots \\  {x}_n \end{matrix}\right)  + \left( \begin{matrix} {y}_1 \\  \vdots \\  {y}_n \end{matrix}\right)  = \left( \begin{matrix} {x}_1 + {y}_1 \\  \vdots \\  {x}_n + {y}_n \end{matrix}\right).
\]
In such a case,
\(
\mathbf{0} = \left( \begin{matrix} 0 \\  \vdots \\  0 \end{matrix}\right)
\) is the zero vector.
\end{example}

\begin{example}
It is clear that the set \(V = {M}_{m \times  n}\left( \mathbb{F}\right)\) (the set of all \(m \times  n\) real matrices) is a vector space as well. In such a case, the zero `vector' is the zero matrix ${\bf 0}_{m \times n}$.
\end{example}

\begin{example}
The set \(V = \mathcal{C}\left( \mathbb{R}\right)\) is a vector space with scalar multiplication:
\[
\left( {\alpha f}\right) \left( x\right)  = {\alpha f}\left( x\right) ,\forall \alpha  \in  \mathbb{R},f \in  V
\]    
and vector addition:
\[
\left( {f + g}\right) \left( x\right)  = f\left( x\right)  + g\left( x\right) ,\forall f,g \in  V
\]
Here the zero vector is the zero function, i.e., \(\mathbf{0}\left( x\right)  = 0\) for all \(x \in  \mathbb{R}\).
\end{example}


\begin{definition} A sub-collection \(W \subseteq  V\) of a vector space \(V\) is called a {\bf vector subspace} of \(V\) if \(W\) itself forms a vector space, denoted by \(W \leq  V\).
\end{definition}

\begin{example} 
For \(V = {\mathbb{R}}^{3}\), \(W := \{ \left( {x,y,0}\right)  \mid  x,y \in  \mathbb{R}\}  \leq  V\), but \(W' := \{ \left( {x,y,1}\right)  \mid  x,y \in  \mathbb{R}\}\) is not a vector subspace of \(V\).
\end{example}

\begin{proposition}\label{prop:vecsubspceiff} A subset \(W \subseteq  V\) is a vector subspace of \(V\) iff for all \({\bf w}_1,{\bf w}_2 \in  W\), we have 
\[\alpha \mathbf{\bf w}_1 + \beta {\bf w}_2 \in  W\] 
for all \(\alpha ,\beta  \in  \mathbb{F}\).
\end{proposition}
The proof is identical to that of MAT 2040.


\begin{example} 
We look at a few more examples and non-examples of vector subspaces:
\begin{itemize}
    \item Let \(V = {M}_{n \times  n}\left( \mathbb{F}\right)\), then the subset of all symmetric matrices \(W = \left\{  A \in  V \mid  {A^t = A}\right\}   \leq  V\) is a subspace of $V$.

    \item Let \(V = {C}^{\infty }\left( \mathbb{R}\right)\). Define \(W = \left\{  {f \in  V \mid  \frac{{\mathrm{d}}^2}{\mathrm{\;d}{x}^2}f + f = 0}\right\}   \leq  V\). For \(f,g \in  W\), we have
\[
{\left( \alpha f + \beta g\right) }^{\prime \prime } = \alpha {f}^{\prime \prime } + \beta {g}^{\prime \prime } = \alpha \left( {-f}\right)  + \beta \left( {-g}\right)  =  - \left( {{\alpha f} + {\beta g}}\right) ,
\]

which implies \({\left( \alpha f + \beta g\right) }^{\prime \prime } + \left( {{\alpha f} + {\beta g}}\right)  = 0\). Therefore, $W \leq V$.

\item Let \(V = {\mathbb{R}}^2\). Then the set \( {\mathbb{R}}_{ + }^2 := \{(x,y) \in \mathbb{R}^2\ | x, y > 0\}\) is not a vector subspace, since \(W\) is not closed under scalar multiplication.

\item Moreover, the set \({\mathbb{R}}_{ + }^2 \cup {\mathbb{R}}_{ - }^2 := \{(x,y) \in \mathbb{R}^2\ | xy > 0\} \) is not a vector subspace of $V = \mathbb{R}^2$ since it is not closed under addition.

\item For \(V = {\mathbb{M}}_{3 \times  3}\left( \mathbb{R}\right)\), the set of invertible \(3 \times  3\) matrices is not a vector subspace, since the zero `vector' ${\bf0}_{3 \times 3}$ is not invertible (Exercise: How about the set of all singular matrices?).
\end{itemize}
\end{example}

\subsection{Spanning Set and Linear Independence}

\begin{definition} Let \(V\) be a vector space over \(\mathbb{F}\) :

1. A {\bf linear combination} of a subset \(S\) in \(V\) is of the form
\[
\mathop{\sum }\limits_{{i = 1}}^n\alpha_{i}{\mathbf{s}}_{i},\quad \alpha_{i} \in  \mathbb{F},{\mathbf{s}}_{i} \in  S
\]
for $n \in \mathbb{N}$. Note that the summation should be finite.

2. The {\bf span} of a subset \(S \subseteq  V\) is
\[
\operatorname{span}\left( S\right)  = \left\{  {\mathop{\sum }\limits_{{i = 1}}^n\alpha_{i}{\mathbf{s}}_{i} \Big|  \alpha_{i} \in  \mathbb{F},{\mathbf{s}}_{i} \in  S}\right\}
\]

3. \(S\) is a {\bf spanning set} of \(V\), or say \(S\) {\bf spans} \(V\), if
\[
\operatorname{span}\left( S\right)  = V\text{ . }
\]
\end{definition}

\begin{example} 
\leavevmode
\begin{enumerate}
    \item For \(V = \mathbb{R}\left\lbrack  x\right\rbrack\), define the set \(S := \left\{  {1,{x}^2,{x}^{4},\ldots ,{x}^{6}}\right\} \subseteq V\). Then \(2 + {x}^{4} + \pi {x}^{106} \in  \operatorname{span}\left( S\right)\), while the series \(1 + {x}^2 + {x}^{4} + \cdots  \notin  \operatorname{span}\left( S\right)\). 

It is clear that \(\operatorname{span}\left( S\right)  \neq  V\), but \(S\) is the spanning set of \(W = \{ p \in  V \mid  p\left( x\right)  = p\left( {-x}\right) \}\).

\item For \(V = {M}_{3 \times  3}\left( \mathbb{R}\right)\), let \({\bf w}_1 = \left\{  {A \in  V \mid  {A}^{\mathrm{T}} = A}\right\}\) and \({\bf w}_2 = \left\{  {B \in  V \mid  {B}^{\mathrm{T}} =  - B}\right\}\) (the set of skew-symmetric matrices) be two vector subspaces. {\bf Exercise:} show that the set \( S := {\bf w}_1\cup {\bf w}_2\) spans \(V\).
\end{enumerate}
\end{example}


\begin{proposition} \label{prop:exchangelem}
Let \(S\) be a subset in a vector space \(V\).

\begin{enumerate}
\item \(S \subseteq  \operatorname{span}\left( S\right)\)

\item \(\operatorname{span}\left( S\right)  = \operatorname{span}\left( {\operatorname{span}\left( S\right) }\right)\)

\item (Exchange Lemma) If \(\mathbf{w} \in  \operatorname{span}\left\{  {{\bf v}_1,\ldots ,{\bf v}_n}\right\}   \smallsetminus  \operatorname{span}\left\{  {{\bf v}_2,\ldots ,{\bf v}_n}\right\}\), then

\[
{\bf v}_1 \in  \operatorname{span}\left\{  {\mathbf{w},{\bf v}_2,\ldots ,{\bf v}_n}\right\}   \smallsetminus  \operatorname{span}\left\{  {{\bf v}_2,\ldots ,{\bf v}_n}\right\}
\]
\end{enumerate}
\end{proposition}

\begin{proof} 
\begin{enumerate}
    \item For each \(s \in  S\), we have

\[
\mathbf{s} = 1 \cdot  \mathbf{s} \in  \operatorname{span}\left( S\right)
\]

    \item From (1), it’s clear that \(\operatorname{span}\left( S\right)  \subseteq  \operatorname{span}\left( {\operatorname{span}\left( S\right) }\right)\), and therefore suffices to show

\(\operatorname{span}\left( {\operatorname{span}\left( S\right) }\right)  \subseteq  \operatorname{span}\left( S\right)\) :

Pick \(\mathbf{v} = \mathop{\sum }\limits_{{i = 1}}^n\alpha_{i}{\bf v}_{i} \in  \operatorname{span}\left( {\operatorname{span}\left( S\right) }\right)\), where \({\bf v}_{i} \in  \operatorname{span}\left( S\right)\). Rewrite

\[
{\bf v}_{i} = \mathop{\sum }\limits_{{j = 1}}^{n_{i}}\beta_{ij}{\mathbf{s}}_{j},\;{\mathbf{s}}_{j} \in  S,
\]

which implies

\[
\mathbf{v} = \mathop{\sum }\limits_{{i = 1}}^n\alpha_{i}\mathop{\sum }\limits_{{j = 1}}^{n_{i}}\beta_{ij}{\mathbf{s}}_{j} = \mathop{\sum }\limits_{{i = 1}}^n\mathop{\sum }\limits_{{j = 1}}^{n_{i}}\left( {\alpha_{i}\beta_{ij}}\right) {\mathbf{s}}_{j},
\]

i.e., \(\bf v\) is a finite linear combination of elements in \(S\), which implies \({\bf v} \in  \operatorname{span}\left( S\right)\).

    \item  By hypothesis, \(\mathbf{w} = \alpha_1{\bf v}_1 + \cdots  + \alpha_n{\bf v}_n\) with \(\alpha_1 \neq  0\), which implies
    \[
    {\bf v}_1 =  - \frac{\alpha_2}{\alpha_1}{\bf v}_2 + \cdots  + \left( {-\frac1{\alpha_1}\mathbf{w}}\right)
    \]
    which implies \({\bf v}_1 \in  \operatorname{span}\left\{  {\mathbf{w},{\bf v}_2,\ldots ,{\bf v}_n}\right\}\). It suffices to show \({\bf v}_1 \notin  \operatorname{span}\left\{  {{\bf v}_2,\ldots ,{\bf v}_n}\right\}\). Suppose on the contrary that \({\bf v}_1 \in  \operatorname{span}\left\{  {{\bf v}_2,\ldots ,{\bf v}_n}\right\}\). It’s clear that \(\operatorname{span}\left\{  {{\bf v}_1,\ldots ,{\bf v}_n}\right\}   =\)  \(\operatorname{span}\left\{  {{\bf v}_2,\ldots ,{\bf v}_n}\right\}\). (left as exercise). Therefore,
    \[
    \varnothing  = \operatorname{span}\left\{  {{\bf v}_1,\ldots ,{\bf v}_n}\right\}   \smallsetminus  \operatorname{span}\left\{  {{\bf v}_2,\ldots ,{\bf v}_n}\right\}  ,
    \]
    
    which is a contradiction.
\end{enumerate}

\end{proof}

\begin{definition}[Linear Independence] Let \(S\) be a (not necessarily finite) subset of \(V\). Then \(S\) is {\bf linearly independent} (l.i.) on \(V\) if for any finite subset \(\left\{  {{\mathbf{s}}_1,\ldots ,{\mathbf{s}}_{k}}\right\}\) in \(S\),

\[
\mathop{\sum }\limits_{{i = 1}}^{k}\alpha_{i}{\mathbf{s}}_{i} = 0 \Leftrightarrow  \alpha_{i} = 0\ \forall i.
\]
Otherwise, we say $S$ is {\bf linearly dependent}.
\end{definition}

\begin{example}
\leavevmode
\begin{itemize}
\item For \(V = C\left( \mathbb{R}\right)\), \({S}_1 = \{ \sin x,\cos x\}\) is linearly independent, since
\[
\alpha \sin x + \beta \cos x = \mathbf{0}\text{ (means zero function) }
\]
Taking \(x = 0\) both sides leads to \(\beta  = 0\) ; taking \(x = \frac{\pi }2\) both sides leads to \(\alpha  = 0\).

\item For \(V = C\left( \mathbb{R}\right)\), \({S}_2 = \left\{  {{\sin }^2x,{\cos }^2x,1}\right\}\) is linearly dependent, since
\[
1 \cdot  {\sin }^2x + 1 \cdot  {\cos }^2x + \left( {-1}\right)  \cdot  1 = 0,\forall x
\]

\item For \(V = \mathbb{R}\left\lbrack  x\right\rbrack\), \(S = \left\{  {1,x,{x}^2,{x}^{3},\ldots ,}\right\}\) is linearly independent: Pick \({x}^{{k}_1},\ldots ,{x}^{{k}_n} \in  S\) with \({k}_1 < \cdots  < {k}_n\). Consider that the equation

\[
\alpha_1{x}^{{k}_1} + \cdots  + \alpha_n{x}^{{k}_n} = \mathbf{0}
\]

holds for all \(x\), and try to solve for \(\alpha_1,\ldots ,\alpha_n\) (one way is differentation.)
\end{itemize}
\end{example}


\begin{definition}[Basis] A subset \(S\) is a {\bf basis} of \(V\) if

(a) \(S\) spans \(V\); and

(b) \(S\) is linearly independent.
\end{definition}

\begin{example} 
\leavevmode
1. For \(V = {\mathbb{R}}^n,S = \left\{  {{\mathbf{e}}_1,\ldots ,{\mathbf{e}}_n}\right\}\) is a basis of \(V\)

2. For \(V = \mathbb{R}\left\lbrack  x\right\rbrack  ,S = \left\{  {1,x,{x}^2,\ldots }\right\}\) is a basis of \(V\)

3. For \(V = {M}_{2 \times  2}\left( \mathbb{R}\right)\),
\[
S = \left\{  {\left( \begin{array}{ll} 1 & 0 \\  0 & 0 \end{array}\right) ,\left( \begin{array}{ll} 0 & 1 \\  0 & 0 \end{array}\right) ,\left( \begin{array}{ll} 0 & 0 \\  1 & 0 \end{array}\right) ,\left( \begin{array}{ll} 0 & 0 \\  0 & 1 \end{array}\right) }\right\}
\]
is a basis of \(V\).
\end{example}

\begin{proposition} Let \(V = \operatorname{span}\left\{  {{\bf v}_1,\ldots ,{\bf v}_{m}}\right\}\). Then there exists a subset of \(\left\{  {{\bf v}_1,\ldots ,{\bf v}_{m}}\right\}\) which is a basis of \(V\).
\end{proposition}

\begin{proof} If \(\left\{  {{\bf v}_1,\ldots ,{\bf v}_{m}}\right\}\) is linearly independent, the proof is complete. Otherwise, \(\alpha_1{\bf v}_1 + \cdots  + \alpha_{m}{\bf v}_{m} = \mathbf{0}\) has a non-trivial solution. w.l.o.g., \(\alpha_1 \neq  0\), which implies

\[
{\bf v}_1 =  \left(- \frac{\alpha_2}{\alpha_1}\right){\bf v}_2 + \cdots  + \left(- \frac{\alpha_{m}}{\alpha_1}\right) {\bf v}_{m} \Rightarrow  {\bf v}_1 \in  \operatorname{span}\left\{  {{\bf v}_2,\ldots ,{\bf v}_{m}}\right\}
\]

By the proof in \autoref{prop:exchangelem} (3),
\[
V =\operatorname{span}\left\{  {{\bf v}_1,\ldots ,{\bf v}_{m}}\right\}   = \operatorname{span}\left\{  {{\bf v}_2,\ldots ,{\bf v}_{m}}\right\}  ,
\]
which implies \(V = \operatorname{span}\left\{  {{\bf v}_2,\ldots ,{\bf v}_{m}}\right\}\).
So one can continue this argument finitely many times until one has \(\left\{  {{\bf v}_{i},{\bf v}_{i + 1},\ldots ,{\bf v}_{m}}\right\}\) is linearly independent, and spans \(V\). The proof is complete.
\end{proof}


\begin{corollary} \label{cor:fghasbasis}
Suppose \(V\) is {\bf finitely-generated}, i.e. there exists a finite set $\{{\bf v}_1,\ldots ,{\bf v}_{m}\}$ such that $V = \operatorname{span}\left\{{\bf v}_1,\ldots ,{\bf v}_{m}\right\}$.
Then \(V\) has a basis. 
\end{corollary}
The same conclusion holds for non-finitely generated \(V\) if one assumes \href{https://en.wikipedia.org/wiki/Zorn%27s_lemma}{Zorn's lemma}.


\begin{proposition} If \(\left\{  {{\bf v}_1,\ldots ,{\bf v}_n}\right\}\) is a basis of \(V\), then every \(v \in  V\) can be expressed uniquely as

\[
\mathbf{v} = \alpha_1{\bf v}_1 + \cdots  + \alpha_n{\bf v}_n
\]
\end{proposition}

\begin{proof} Since \(\left\{  {{\bf v}_1,\ldots ,{\bf v}_n}\right\}\) spans \(V\), so \(\mathbf{v} \in  V\) can be written as in the form given in the Proposition.
Suppose further that
\[
\mathbf{v} = \beta_1{\bf v}_1 + \cdots  + \beta_n{\bf v}_n,
\]
for some $\beta_i \in \mathbb{F}$, it suffices to show that \(\alpha_{i} = \beta_{i}\) for all \(i\): Indeed, by subtracting the above equations, one has
\[
\left( {\alpha_1 - \beta_1}\right) {\bf v}_1 + \cdots  + \left( {\alpha_n - \beta_n}\right) {\bf v}_n = 0.
\]
By the hypothesis of linear independence, we have \(\alpha_{i} - \beta_{i} = 0\) for all \(i\), i.e., \(\alpha_{i} = \beta_{i}\).
\end{proof}

\subsection{Basis and Dimension}

We begin by showing that the bases of any finitely generated vector space have the same number of vectors:

\begin{theorem} \label{thm:basisdim}
Let $V$ be a finitely generated vector space (so that $V$ has a basis by \autoref{cor:fghasbasis}. Suppose $\{ \mathbf{v}_1,\ldots, \mathbf{v}_{m} \}$ and $\{ \mathbf{w}_1,\ldots, \mathbf{w}_n \}$ are two bases of $V$. Then $m = n$.
\end{theorem}

\begin{proof}
Suppose on the contrary that $m \neq n$. Without loss of generality, assume that $m < n$. Let
\[
\mathbf{v}_1 = \alpha_1 \mathbf{w}_1 + \cdots + \alpha_n \mathbf{w}_n
\]
with some $\alpha_i \neq 0$. By rearranging the terms if necessary, assume $\alpha_1 \neq 0$. Then
\begin{equation}\label{eq:replacement}
\mathbf{v}_1 \in \operatorname{span}\{ \mathbf{w}_1, \mathbf{w}_2, \ldots, \mathbf{w}_n \} \setminus \operatorname{span}\{ \mathbf{w}_2, \ldots, \mathbf{w}_n \}.
\end{equation}

This implies that
\[
\mathbf{w}_1 \in \operatorname{span}\{ \mathbf{v}_1, \mathbf{w}_2, \ldots, \mathbf{w}_n \} \setminus \operatorname{span}\{ \mathbf{w}_2, \ldots, \mathbf{w}_n \}.
\]
We now claim that $\{ \mathbf{v}_1, \mathbf{w}_2, \ldots, \mathbf{w}_n \}$ is a basis of $V$.

\begin{enumerate}
  \item \textbf{Spanning Set:} Firstly, note that
  $\mathbf{w}_1 \in \operatorname{span}\{ \mathbf{v}_1, \mathbf{w}_2, \ldots, \mathbf{w}_n \}$, so 
  \[\operatorname{span}\{ \mathbf{w}_1, \mathbf{w}_2, \ldots, \mathbf{w}_n \} \subseteq \operatorname{span}\{ \mathbf{v}_1, \mathbf{w}_2, \ldots, \mathbf{w}_n \}.
  \]
  Since $\operatorname{span}\{ \mathbf{w}_1, \ldots, \mathbf{w}_n \} = V$, we have $\operatorname{span}\{ \mathbf{v}_1, \mathbf{w}_2, \ldots, \mathbf{w}_n \} = V$.

  \item \textbf{Linear independence:} Suppose
  \[
  \beta_1 \mathbf{v}_1 + \beta_2 \mathbf{w}_2 + \cdots + \beta_n \mathbf{w}_n = \mathbf{0}.
  \]
  \begin{itemize}
    \item If $\beta_1 \neq 0$, then
    \[
    \mathbf{v}_1 = -\frac{\beta_2}{\beta_1} \mathbf{w}_2 - \cdots - \frac{\beta_n}{\beta_1} \mathbf{w}_n \in \operatorname{span}\{ \mathbf{w}_2, \ldots, \mathbf{w}_n \},
    \]
    contradicting \textcolor{darkblue}{($\ref{eq:replacement}$)}.

    \item If $\beta_1 = 0$, then independence of $\{ \mathbf{w}_2, \ldots, \mathbf{w}_n \}$ implies $\beta_2 = \cdots = \beta_n = 0$.
  \end{itemize}
\end{enumerate}

Thus $\{ \mathbf{v}_1, \mathbf{w}_2, \ldots, \mathbf{w}_n \}$ is a basis. By repeating this replacement process, we ultimately construct
\[\mathcal{B} :=
\{ \mathbf{v}_1, \ldots, \mathbf{v}_m, \mathbf{w}_{m+1}, \ldots, \mathbf{w}_n \}
\]
as a basis of $V$. But then $\mathbf{w}_{m+1}$ would be a linear combination of $\mathbf{v}_1, \ldots, \mathbf{v}_m$, contradicting the linear independence the basis $\mathcal{B}$. Hence, $m = n$.
\end{proof}

\begin{definition}[Dimension] \label{def:dimension}
    Let $V$ be a vector space over $\mathbb{F}$. 
    \begin{itemize}
        \item If $V$ is finitely-generated, then the {\bf dimension} of $V$ is defined as $\dim(V) = n$, where $n$ the size of {\it any} basis of $V$ (c.f. \autoref{thm:basisdim}).
        \item If $V$ is not finitely-generated, then the dimension of $V$ is infinite, i.e. $\dim(V) = \infty$.
    \end{itemize}
\end{definition}


\begin{example}
A vector space may have more than one basis. For instance, suppose \(V = \mathbb{F}^n\). Then clearly \(\dim(V) = n\), and
\[
\left\{ \mathbf{e}_1, \ldots, \mathbf{e}_n \right\}
\]
is a basis of \(V\), where \(\mathbf{e}_i\) denotes the standard unit vector. Also, another basis of \(V\) can be taken as:
\[
\left\{
\begin{pmatrix} 1 \\ 0 \\ \vdots \\ 0 \end{pmatrix},
\begin{pmatrix} 1 \\ 1 \\ \vdots \\ 0 \end{pmatrix},
\ldots,
\begin{pmatrix} 1 \\ 1 \\ \vdots \\ 1 \end{pmatrix}
\right\}.
\]

In fact, the columns of any invertible \(n \times n\) matrix form a basis of \(V\).
\end{example}

\begin{example}
Let \(V = M_{m \times n}(\mathbb{R})\). Then \(\dim(V) = mn\). More explicitly, a basis of $V$ is given by the set
\[
\left\{ E_{ij} \;\middle|\; 1 \leq i \leq m,\; 1 \leq j \leq n \right\},
\]
where \(E_{ij}\) is the \(m \times n\) matrix with a 1 in the \((i,j)\)-th entry and 0s elsewhere.
\end{example}

\begin{example}
Let \(V\) be the space of all real polynomials of degree \(\leq n\). Then
\[
\dim(V) = n + 1.
\]
A basis of $V$ is given by \(\{1, x, x^2, \ldots, x^n\}\).
\end{example}

\begin{example}
Let \(V = \left\{ A \in M_{n \times n}(\mathbb{R}) \mid A^\mathrm{T} = A \right\}\), the space of real symmetric matrices. Then
\[
\dim(V) = \frac{n(n + 1)}{2}.
\]
\end{example}

\begin{example}
Let \(W = \left\{ B \in M_{n \times n}(\mathbb{R}) \mid B^\mathrm{T} = -B \right\}\), the space of real skew-symmetric matrices. Then
\[
\dim(W) = \frac{n(n - 1)}{2}.
\]
\end{example}

\begin{example}
Sometimes we need to specify the field \(\mathbb{F}\) for the scalar multiplication when we define a vector space. Consider the example below:
\begin{enumerate}
    \item Let \(V = \mathbb{C}\), then \(\dim _{\mathbb{C}}\left( \mathbb{C}\right)  = 1\) for as a vector space over the field \(\mathbb{F} = \mathbb{C}\).
    \item Let \(V = \operatorname{span}_{\mathbb{R}}\{ 1,i\}  = \mathbb{C}\), then \(\dim_{\mathbb{R}}\left( \mathbb{C}\right)  = 2\) as a vector space over \(\mathbb{F} = \mathbb{R}\), since all \(z \in  V\) can be uniquely written as \(z = a + {bi}\), for some \(a,b \in  \mathbb{R}\).
    In cases where there may be confusion, it is safer to write
    \[
    {\dim }_{\mathbb{C}}\left( \mathbb{C}\right)  = 1,\;{\dim }_{\mathbb{R}}\left( \mathbb{C}\right)  = 2.
    \]
\end{enumerate}    
\end{example}

Note that a basis for a vector space is characterized as the maximal linearly independent set.

\begin{theorem}[Basis Extension]\label{thm: basis-extension}
Let $V$ be a finite-dimensional vector space, and let $\{ \mathbf{v}_1, \ldots, \mathbf{v}_k \}$ be a linearly independent subset of $V$. Then we can extend it to a basis $\{ \mathbf{v}_1, \ldots, \mathbf{v}_k, \mathbf{v}_{k+1}, \ldots, \mathbf{v}_n \}$ of $V$.
\end{theorem}

\begin{proof}
Suppose $\dim(V) = n > k$, and let $\{ \mathbf{w}_1, \ldots, \mathbf{w}_n \}$ be a basis of $V$. Consider the set
\[
\{ \mathbf{w}_1, \ldots, \mathbf{w}_n \} \cup \{ \mathbf{v}_1, \ldots, \mathbf{v}_k \},
\]
which spans $V$ but is linearly dependent, i.e., there exists non-trivial solutions of the linear equation
\[
\alpha_1 \mathbf{w}_1 + \cdots + \alpha_n \mathbf{w}_n + \beta_1 \mathbf{v}_1 + \cdots + \beta_k \mathbf{v}_k = \mathbf{0}.
\]
Note that there must be some $\alpha_i \neq 0$, otherwise the above equation will only involve the linearly independent set ${\bf v}_1$, $\dots$, ${\bf v}_k$, which forces $\beta_1 = \dots = \beta_k = 0$. 


Without loss of generality, assume $\alpha_1 \neq 0$, so that ${\bf w}_1 \in  \operatorname{span}(\{{\bf w}_2, \dots, {\bf w}_n\} \cup \{ {\bf v}_1 \dots, {\bf v}_k\})$. Consider the set
\[
\{ \mathbf{w}_2, \ldots, \mathbf{w}_n \} \cup \{ \mathbf{v}_1, \ldots, \mathbf{v}_k \}.
\]
with ${\bf w}_1$ removed. As in the proof of spanning set in \autoref{thm:basisdim}, one has
$$\operatorname{span}(\{ \mathbf{w}_2, \ldots, \mathbf{w}_n \} \cup \{ \mathbf{v}_1, \ldots, \mathbf{v}_k \}) = \operatorname{span}(\{ \mathbf{w}_1, \ldots, \mathbf{w}_n \} \cup \{ \mathbf{v}_1, \ldots, \mathbf{v}_k \}) = V.$$
In other words, it is still a spanning set. If it is linearly dependent, we repeat the argument by removing ${\bf w}_2$ and get a smaller spanning set
$$\{ \mathbf{w}_3, \ldots, \mathbf{w}_n \} \cup \{ \mathbf{v}_1, \ldots, \mathbf{v}_k \}$$
Continue this process by removing more vectors $\mathbf{w}_3$, $\dots$, so that the remaining set still spans $V$, until we hit the first occasion when the set
\[
 \{ \mathbf{w}_{\ell}, \ldots, \mathbf{w}_m \} \cup \{ \mathbf{v}_1, \ldots, \mathbf{v}_k \}
\]
is linearly independent and spans $V$. Then this becomes a basis of $V$. Moreover, one has $\ell = k+1$ by \autoref{thm:basisdim}.
\end{proof}

\subsection{Intersection and Sum of Vector Subspaces}

\begin{definition} \label{def:sumintersect} Let \(W_1\), \(W_2\) be two vector subspaces of \(V\). Then:

\begin{enumerate}
  \item \(W_1 \cap W_2 := \left\{ \mathbf{w} \in V \mid \mathbf{w} \in W_1 \text{ and } \mathbf{w} \in W_2 \right\}\)

  \item \(W_1 + W_2 := \left\{ \mathbf{w}_1 + \mathbf{w}_2 \mid \mathbf{w}_i \in W_i\ \text{for}\ i = 1,2 \right\}\)

  \item Furthermore, if \(W_1 \cap W_2 = \{ \mathbf{0} \}\), then \(W_1 + W_2 = W_1 \oplus W_2\) is called the \emph{direct sum} of $W_1$ and $W_2$.
\end{enumerate}
\end{definition}

\begin{proposition}
\(W_1 \cap W_2\) and \(W_1 + W_2\) are vector subspaces of \(V\).
\end{proposition}
\begin{proof}
    Exercise (use \autoref{prop:vecsubspceiff}).
\end{proof}

\begin{proposition}
Let \( W_1, W_2 \) be subspaces of a vector space \( V \). The following are equivalent (TFAE):
\begin{enumerate}
\item $W_1 + W_2 = W_1 \oplus W_2$ is a direct sum
\item Every \( \mathbf{w} \in W_1 + W_2 \) can be uniquely written as $\mathbf{w} = \mathbf{w}_1 + \mathbf{w}_2$ for \( \mathbf{w}_1 \in W_1 \) and \( \mathbf{w}_2 \in W_2 \).
\item For \( \mathbf{w}_1 \in W_1 \) and \( \mathbf{w}_2 \in W_2 \), 
$$\mathbf{w}_1 + \mathbf{w}_2 = \mathbf{0} \quad \Leftrightarrow \quad \mathbf{w}_1 = \mathbf{w}_2 = \mathbf{0}.$$
\end{enumerate}
\end{proposition}

\begin{proof}
    XXX
\end{proof}

Obviously, one can generalize \autoref{def:sumintersect} to any number of (even for uncountably many) vector subspaces:
\begin{definition}
 Let \(\{W_i\ |\ i \in I\}\)\ be a collection of vector subspaces of \(V\). Then:

\begin{enumerate}
  \item \(\bigcap_{i \in I} W_i  := \left\{ \mathbf{w} \in V \mid \mathbf{w} \in W_i \text{ for all } i \in I \right\}\)

  \item \(\sum_{i \in I} W_i  := \left\{ \mathbf{w}_{i_1} + \dots + \mathbf{w}_{i_n} \mid i_k \in I \text{ for all } k,\ n \in \mathbb{N} \right\}\) (Note that the sum in the definition above must be finite).
\end{enumerate}   
\end{definition}
 

Here is a generalization of direct sum of more than two subspaces:
\begin{definition}
 Let \( \{ W_i\ |\ i \in I \} \) be a collection of subspaces of a vector space \( V \). We write
\[
\bigoplus_{i \in I} W_i = \sum_{i \in I} W_i 
\]
as the \textbf{direct sum} of all \( W_i\)'s if for {\bf any} finite subset $\{i_1, \dots, i_n\} \subseteq I$ and ${\bf w}_{i_k} \in W_{i_k}$,
\[
\mathbf{w}_{i_1} + \cdots + \mathbf{w}_{i_n} = \mathbf{0} \quad \Leftrightarrow \quad \mathbf{w}_{i_1} = \dots  = \mathbf{w}_{i_n} = \mathbf{0},
\]

\end{definition}
As in the two vector subspaces case, every element \( {\bf w} \in \bigoplus_{i \in I} W_i\) in the direct sum has a \textbf{unique} expression as a finite sum \( {\bf w} = \mathbf{w}_{i_1} + \cdots + \mathbf{w}_{i_n} \) of vectors \( \mathbf{w}_{i_k} \in W_{i_k} \).


\begin{proposition}[Complementation]\label{prop:complementation}
Let \( W \leq V \) be a subspace of a finite-dimensional vector space \( V \). Then there exists a subspace \( W' \leq V \) such that
\[
V = W \oplus W',
\]
i.e., every vector in \( V \) can be uniquely written as a sum of vectors from \( W \) and \( W' \).
\end{proposition}

\begin{proof}
Let \( \dim(W) = k \leq n = \dim(V) \), and let \( \{ \mathbf{v}_1, \ldots, \mathbf{v}_k \} \) be a basis of \( W \).
By the basis extension theorem, we can extend this to a basis of \( V \):
\[
\{ \mathbf{v}_1, \ldots, \mathbf{v}_k, \mathbf{v}_{k+1}, \ldots, \mathbf{v}_n \}.
\]
Define
\[
W' := \operatorname{span} \{ \mathbf{v}_{k+1}, \ldots, \mathbf{v}_n \}.
\]

We now verify the two conditions for a direct sum:

\begin{enumerate}
\item \textbf{Sum:} For every \( \mathbf{v} \in V \), we can write
\[
\mathbf{v} = \sum_{i=1}^{k} \alpha_i \mathbf{v}_i + \sum_{j=k+1}^{n} \alpha_j \mathbf{v}_j,
\]
where the first sum lies in \( W \) and the second in \( W' \), so \( V = W + W' \).

\item \textbf{Intersection:} Suppose \( \mathbf{v} \in W \cap W' \). Then
\[
\mathbf{v} = \sum_{i=1}^{k} \beta_i \mathbf{v}_i = \sum_{j=k+1}^{n} \gamma_j \mathbf{v}_j.
\]
But since the \( \mathbf{v}_i \) form a basis of \( V \), their representation is unique. Hence all coefficients must vanish:
\[
\beta_1 = \cdots = \beta_k = \gamma_{k+1} = \cdots = \gamma_n = 0,
\]
so \( \mathbf{v} = \mathbf{0} \), and \( W \cap W' = \{ \mathbf{0} \} \).
\end{enumerate}

Thus, \( V = W \oplus W' \).
\end{proof}

\chapter{Linear Transformations}
In linear algebra, we will study a special kind of functions between vector spaces:
\begin{definition}[Linear Transformation]
Let \( V \) and \( W \) be vector spaces over a field \( \mathbb{F} \). A map \( T : V \to W \) is called a \emph{linear transformation} if for all \( \alpha, \beta \in \mathbb{F} \) and all \( \mathbf{v}_1, \mathbf{v}_2 \in V \), we have
\[
T(\alpha \mathbf{v}_1 + \beta \mathbf{v}_2) = \alpha T(\mathbf{v}_1) + \beta T(\mathbf{v}_2).
\]
\end{definition}

\begin{proposition}
Let \( V, W, U \) be vector spaces and let \( T, S \) be linear transformations. Then:
\begin{enumerate}
    \item If \( S : V \to W \) and \( T : W \to U \), then the composition \( T \circ S : V \to U \) is also a linear transformation.
    \item For any linear transformation \( T : V \to W \), we have
    \[
    T(\mathbf{0}_V) = \mathbf{0}_W.
    \]
\end{enumerate}
\end{proposition}
\begin{proof} \begin{enumerate}
    \item Let \( \mathbf{v}_1, \mathbf{v}_2 \in V \) and \( \alpha, \beta \in \mathbb{F} \). Then,
    \[
    (T \circ S)(\alpha \mathbf{v}_1 + \beta \mathbf{v}_2) = T(S(\alpha \mathbf{v}_1 + \beta \mathbf{v}_2)) = T(\alpha S(\mathbf{v}_1) + \beta S(\mathbf{v}_2)) = \alpha T(S(\mathbf{v}_1)) + \beta T(S(\mathbf{v}_2)).
    \]
    Hence \( T \circ S \) is linear.

    \item Since \( T \) is linear,
    \[
    T(\mathbf{0}_V) = T(0 \cdot \mathbf{v}) = 0 \cdot T(\mathbf{v}) = \mathbf{0}_W
    \]
    for any \( \mathbf{v} \in V \).
\end{enumerate}
\end{proof}

\begin{example}
\hfill
\begin{enumerate}
    \item Let \( A \in \mathbb{R}^{m \times n} \). The map \( T : \mathbb{R}^n \to \mathbb{R}^m \) defined by
    \[
    T(\mathbf{x}) = A \mathbf{x}
    \]
    is a linear transformation.

    \item Let \( T : \mathbb{R}[x] \to \mathbb{R}[x] \) be defined by either of the following:
    \[
    T(p(x)) = \frac{d}{dx}p(x), \quad \text{or} \quad T(p(x)) = \int_0^x p(t) \, dt.
    \]
    Both are linear transformations.

    \item The map \( T : M_{n \times n}(\mathbb{R}) \to \mathbb{R} \) defined by
    \[
    T(A) = \operatorname{trace}(A) := \sum_{i=1}^n a_{ii}
    \]
    is a linear transformation.

    \item However, the map \( A \mapsto \det(A) \) is \emph{not} a linear transformation.
\end{enumerate}
\end{example}

As we saw above, many of the functions we studied in other branches of mathematics are linear transformations. One of the many advantages of linear transformation over a general function is the following:
\begin{remark} \label{rmk:linear_trans_basis}
Let $T:V \to W$ be a linear transformation, and $\mathcal{B} = \{{\bf v}_i\ |\ i \in I\}$ be a basis of $V$, then 
\begin{center}
\fbox{\bf the image of $T$ is uniquely determined by the values of $\{T({\bf v}_i)\ |\ {\bf v}_i \in \mathcal{B}\}$}.
\end{center}
Namely, since every element ${\bf v} \in V$ can be expressed as 
${\bf v} = \sum_{\ell = 1}^k \alpha_{\ell} {\bf v}_{i_{\ell}},$ then 
\begin{equation} \label{eq:linear_trans_basis}
T({\bf v}) = T\left(\sum_{\ell = 1}^k \alpha_{\ell} {\bf v}_{i_{\ell}}\right) = \sum_{\ell = 1}^k \alpha_{\ell} T\left({\bf v}_{i_{\ell}}\right).
\end{equation}
This is certainly not true for a general function - for instance, if I tell you $f: \mathbb{R} \to \mathbb{R}$ satisfies $f(1) = 1$ and $f(2) = 4$, one cannot determine $f(x)$ for all $x \in \mathbb{R}$. 

Conversely, one can
\begin{center}
\fbox{\bf Define linear transformation $T: V \to W$ by {\it just} specifying $\{T({\bf v}_i) = {\bf w}_i \ |\ {\bf v}_i \in \mathcal{B}\}$.}
\end{center}
 In such a case, the image $T({\bf v})$ for all ${\bf v} \in V$ can be obtained by extending $T$ linearly using \autoref{eq:linear_trans_basis}.
\end{remark}


\section{Kernel and Image}
\begin{definition}[Kernel and Image]
Let \( T : V \rightarrow W \) be a linear transformation.
\begin{enumerate}
    \item The \textbf{kernel} of \( T \) is
    \[
    \ker(T) = T^{-1}(\mathbf{0}) = \{ \mathbf{v} \in V \mid T(\mathbf{v}) = \mathbf{0} \}.
    \]

    \item The \textbf{image} (or range) of \( T \) is
    \[
    \operatorname{Im}(T) = T(V) =\{ T(\mathbf{v}) \in W \mid \mathbf{v} \in V \}.
    \]
\end{enumerate}
\end{definition}

\begin{example}
\hfill
\begin{enumerate}
    \item Let \( T : \mathbb{R}^n \rightarrow \mathbb{R}^n \) be a linear transformation defined by \( T(\mathbf{x}) = A\mathbf{x} \), where \( A \in \mathbb{R}^{n \times n} \). Then
    \[
    \ker(T) = \{ \mathbf{x} \in \mathbb{R}^n \mid A\mathbf{x} = \mathbf{0} \} = \operatorname{Null}(A)
    \]
    and
    \[
    \operatorname{Im}(T) = \{ A\mathbf{x} \mid \mathbf{x} \in \mathbb{R}^n \} = \operatorname{Col}(A) = \operatorname{span}\{\text{columns of } A\} \quad \text{(Column Space)}.
    \]

    \item For \( T(p(x)) = \frac{d}{dx}p(x) \), we have
    \[
    \ker(T) = \{ \text{constant polynomials} \}, \quad \operatorname{Im}(T) = \mathbb{R}[x].
    \]
\end{enumerate}
\end{example}

\begin{proposition}
The kernel and image of a linear transformation \( T : V \rightarrow W \) are vector subspaces:
\[
\ker(T) \leq V, \quad \operatorname{Im}(T) \leq W.
\]
\end{proposition}

\begin{proof}
Let \( \mathbf{v}_1, \mathbf{v}_2 \in \ker(T) \). Then for any \( \alpha, \beta \in \mathbb{F} \), we have
\[
T(\alpha \mathbf{v}_1 + \beta \mathbf{v}_2) = \alpha T(\mathbf{v}_1) + \beta T(\mathbf{v}_2) = \alpha \cdot \mathbf{0} + \beta \cdot \mathbf{0} = \mathbf{0},
\]
so \( \alpha \mathbf{v}_1 + \beta \mathbf{v}_2 \in \ker(T) \). The case for \( \operatorname{Im}(T) \) is similar.
\end{proof}

\begin{definition}
[Rank/Nullity] Let \(V, W\) be finite dimensional vector spaces over a field \(\mathbb{F}\) and \(T : V \rightarrow W\) a linear transformation. Then we define
\[
\operatorname{rank}\left( T\right) = \dim \left( \operatorname{im}\left( T\right) \right),
\]
\[
\operatorname{nullity}\left( T\right) = \dim \left( \ker \left( T\right) \right).
\]
\end{definition}

\begin{proposition}
There are alternative characterizations for the injectivity and surjectivity of a linear transformation \(T\):
\begin{enumerate}
  \item The linear transformation \(T\) is injective if and only if
  \[ \ker(T) = \{\bf 0\} \quad \Leftrightarrow \quad \operatorname{nullity}(T) = 0. \]

  \item The linear transformation \(T\) is surjective if and only if
  \[ \operatorname{im}(T) = W \quad \Leftrightarrow \quad \operatorname{rank}(T) = \dim(W). \]

  \item If \(T\) is bijective, then \(T^{-1}\) is a linear transformation.
\end{enumerate}
\end{proposition}

\begin{proof}
\begin{enumerate}
  \item[(1)]
  \begin{enumerate}
    \item For the forward direction:
    \[
    \mathbf{x} \in \ker(T) \Rightarrow T(\mathbf{x}) = 0 = T(\mathbf{0}) \Rightarrow \mathbf{x} = \mathbf{0}.
    \]
    \item For the reverse direction:
    \[
    T(\mathbf{x}) = T(\mathbf{y}) \Rightarrow T(\mathbf{x} - \mathbf{y}) = 0 \Rightarrow \mathbf{x} - \mathbf{y} \in \ker(T) = \{0\} \Rightarrow \mathbf{x} = \mathbf{y}.
    \]
  \end{enumerate}

  \item[(2)] The proof follows a similar idea as in (1).

  \item[(3)] Let \(T^{-1} : W \rightarrow V\). For all \(\mathbf{w}_1, \mathbf{w}_2 \in W\), there exist \(\mathbf{v}_1, \mathbf{v}_2 \in V\) such that \(T(\mathbf{v}_i) = \mathbf{w}_i\), i.e.,
  \[ T^{-1}(\mathbf{w}_i) = \mathbf{v}_i, \quad i = 1, 2. \]
  Consider the mapping:
  \[
  T(\alpha \mathbf{v}_1 + \beta \mathbf{v}_2) = \alpha T(\mathbf{v}_1) + \beta T(\mathbf{v}_2) = \alpha \mathbf{w}_1 + \beta \mathbf{w}_2,
  \]
  which implies
  \[
  \alpha \mathbf{v}_1 + \beta \mathbf{v}_2 = T^{-1}(\alpha \mathbf{w}_1 + \beta \mathbf{w}_2),
  \]
  i.e.,
  \[
  \alpha T^{-1}(\mathbf{w}_1) + \beta T^{-1}(\mathbf{w}_2) = T^{-1}(\alpha \mathbf{w}_1 + \beta \mathbf{w}_2).
  \]
\end{enumerate}
\end{proof}


\section{Space of Linear Transformations}
\begin{definition}[Hom space]
Let
\[ {\operatorname{Hom}}_{\mathbb{F}}\left( V, W\right) = \{ \text{all linear transformations } T : V \rightarrow W \}, \]
and we can define the addition and scalar multiplication to make it a vector space:

\begin{enumerate}
  \item For \(T, S \in \operatorname{Hom}_{\mathbb{F}}\left( V, W\right)\), define
  \[
  (T + S)(\mathbf{v}) := T(\mathbf{v}) + S(\mathbf{v}),
  \]
  one can check easily that $(T + S)(\alpha{\bf v} + \beta{\bf w}) = \alpha(T + S)({\bf v}) + \beta(T + S)({\bf w})$. Therefore, \(T + S \in \operatorname{Hom}_{\mathbb{F}}\left( V, W\right)\).

  \item Also, define
  \[
  (\gamma T)(\mathbf{v}) := \gamma T(\mathbf{v}), \quad \text{for all } \gamma \in \mathbb{F},
  \]
  as before, one can easily check that \(\gamma T \in \operatorname{Hom}_{\mathbb{F}}\left( V, W\right)\).
\end{enumerate}
Therefore, $T+S$, $\gamma T$ are the addition and scalar multiplication operations on the vector space $\operatorname{Hom}_{\mathbb{F}}\left( V, W\right)$.

\medskip   
In particular, if \(V = \mathbb{F}^n, W = \mathbb{F}^{m}\), then by MAT2040, all linear transformations between $V$ and $W$ are just matrix multiplications, i.e.:
\[ \operatorname{Hom}_{\mathbb{F}}\left( V, W\right) = \{T:V \to W\ |\ T({\bf v}) = A{\bf v} \text{ for some } A\in M_{m \times n}(\mathbb{F})\}. \]
So 
$$\operatorname{Hom}_{\mathbb{F}}\left( V, W\right)\xleftrightarrow{1:1} M_{m \times n}(\mathbb{F}).$$ 
In fact, the relationship between them is more than just a bijection - as we will see later, it is an {\bf isomorphism} (\autoref{def:isomorphism}) between vector spaces!
\end{definition}

\begin{proposition}
If \(\dim(V) = n, \dim(W) = m\), then \(\dim\left( \operatorname{Hom}_{\mathbb{F}}(V, W) \right) = mn\).
\end{proposition}



\section{Isomorphism of Vector Spaces}
\begin{definition} \label{def:isomorphism}
[Isomorphism] We say that the vector subspaces 
\[V \cong W\]
are {\bf isomorphic} if there exists a bijective linear transformation \(T : V \rightarrow W\). This mapping \(T\) is called an \emph{isomorphism} from \(V\) to \(W\).
\end{definition}

\begin{example} \label{eg:isodim}
Let \(\dim(V) = n\) and  \(\dim(W) =m \) with \( n,m < \infty\). Then $n = m$ implies \(V \cong W\). More precisely, suppose \(\{ \mathbf{v}_1, \ldots, \mathbf{v}_n \}, \{ \mathbf{w}_1, \ldots, \mathbf{w}_n \}\) are bases of \(V\) and \(W\) respectively, define \(T : V \rightarrow W\) by:
\[
T(\alpha_1 \mathbf{v}_1 + \cdots + \alpha_n \mathbf{v}_n) = \alpha_1 \mathbf{w}_1 + \cdots + \alpha_n \mathbf{w}_n, \quad \forall \alpha_i \in \mathbb{F}.
\]
In particular, \(T(\mathbf{v}_i) = \mathbf{w}_i\) for all \(i\), and it is clear that our constructed \(T\) is a bijective linear transformation.

Note that the converse also holds, i.e. if $V \cong W$ then $n = m$. This is given by the proposition below.
\end{example}


\begin{proposition}\label{prop: isomorphism-properties}
If \(T : V \rightarrow W\) is an isomorphism, then:
\begin{enumerate}
  \item The set \(\{ \mathbf{v}_1, \ldots, \mathbf{v}_k \}\) is linearly independent in \(V\) if and only if \(\{ T\mathbf{v}_1, \ldots, T\mathbf{v}_k \}\) is linearly independent in \(W\).
  Note the same holds if we replace ``linearly independent'' with ``spans.''
  \item If \(\dim(V) = n\), then \(\{ \mathbf{v}_1, \ldots, \mathbf{v}_n \}\) forms a basis of \(V\) if and only if \(\{ T\mathbf{v}_1, \ldots, T\mathbf{v}_n \}\) forms a basis of \(W\). In particular, \(\dim(V) = \dim(W)\).
  \item Two vector spaces with finite dimensions are isomorphic if and only if they have the same dimension.
\end{enumerate}
\end{proposition}

\begin{proof}
Exercise. %It suffices to show the reverse direction. Let \(\{ \mathbf{v}_1, \ldots, \mathbf{v}_n \}\) and \(\{ \mathbf{w}_1, \ldots, \mathbf{w}_n \}\) be bases of \(V\) and \(W\), respectively. Define the linear transformation \(T : V \rightarrow W\) by
%\[
%T(a_1 \mathbf{v}_1 + \cdots + a_n \mathbf{v}_n) = a_1 \mathbf{w}_1 + \cdots + a_n \mathbf{w}_n.
%\]
%Then \(T\) is surjective since \(\{ \mathbf{w}_1, \ldots, \mathbf{w}_n \}\) spans \(W\), and injective since \(\{ \mathbf{w}_1, \ldots, \mathbf{w}_n \}\) is linearly independent.
\end{proof}


 

\begin{remark}
Note that \(V \cong W\) does not imply that any linear transformation \(T : V \rightarrow W\) is an isomorphism. For example, \(T(\mathbf{v}) = \mathbf{0}\) is not an isomorphism if \(W \neq \{ \mathbf{0} \}\).
\end{remark}


\begin{theorem}[Rank–Nullity Theorem]
Let \(T : V \rightarrow W\) be a linear transformation with \(\dim(V) < \infty\). Then
\[
\operatorname{rank}(T) + \operatorname{nullity}(T) = \dim(V).
\]
\end{theorem}

\begin{proof}
Since \(\ker(T) \leq V\), by \autoref{prop:complementation}, there exists a subspace \(U \leq V\) such that
\[
V = \ker(T) \oplus U.
\]
\begin{enumerate}
  \item Consider the restricted map \(T|_U : U \rightarrow T(U)\). This is an isomorphism because:
  \begin{itemize}
    \item It is surjective by definition of the codomain \(T(U)\).
    \item If \({\bf v} \in \ker(T|_U)\), then \(T({\bf v}) = 0\), so \({\bf v} \in \ker(T)\). But \({\bf v} \in U\) as well, and since \(U \cap \ker(T) = \{{\bf 0}\}\), we conclude \({\bf v} = {\bf 0}\).
  \end{itemize}
  Therefore, \(\dim(U) = \dim(T(U))\) by the end of \autoref{eg:isodim}.
  
  \item Note that \(\operatorname{im}(T) = T(V) = T(U)\), because for any \(v \in V\), we can write \({\bf v} = {\bf v}_0 + {\bf u}\) with \({\bf v}_0 \in \ker(T), {\bf u} \in U\), and then
  \[
  T({\bf v}) = T({\bf v}_0 + {\bf u}) = T({\bf v}_0) + T({\bf u}) = {\bf 0} + T({\bf u}) = T({\bf u}).
  \]
  Hence \(T(V) \subseteq T(U)\). But obviously one has $U \subseteq V \Rightarrow T(U) \subseteq T(V)$ as well. So $T(V) = T(U)$.
  
  \item Since \(V = \ker(T) \oplus U\), we have:
  \[
  \dim(V) = \dim(\ker(T)) + \dim(U).
  \]
  by the proof of \autoref{prop:complementation}.
\end{enumerate}

Therefore,
\[
\dim(V) = \operatorname{nullity}(T) + \dim(T(U)) = \operatorname{nullity}(T) + \dim(T(V)) = \operatorname{nullity}(T) + \operatorname{rank}(T). \qedhere
\]
\end{proof}
\chapter{Change of Basis}
\section{Change of Basis and Matrix Representation}
\subsection{Coordinate Vector}

\begin{definition}[Coordinate Vector]\label{def:coordinate_vector}
 Let \(V\) be a finite-dimensional vector space and \(\mathcal{B} = \{ \mathbf{v}_1, \ldots, \mathbf{v}_n \}\) an ordered basis of \(V\). Any vector \(\mathbf{v} \in V\) can be uniquely written as
\[
\mathbf{v} = \alpha_1 \mathbf{v}_1 + \cdots + \alpha_n \mathbf{v}_n.
\]
We define the map \([ \cdot ]_{\mathcal{B}} : V \rightarrow \mathbb{F}^n\), which maps any vector \(\mathbf{v}\) to its coordinate vector:
\[
[ \mathbf{v} ]_{\mathcal{B}} = \begin{pmatrix} \alpha_1 \\ \vdots \\ \alpha_n \end{pmatrix}.
\]
Note that \(\{ \mathbf{v}_1, \ldots, \mathbf{v}_n \}\) and \(\{ \mathbf{v}_2, \mathbf{v}_1, \ldots, \mathbf{v}_n \}\) are distinct ordered bases.
\end{definition}

\begin{example}\label{ex:coordinate_matrix_example}
Given \(V = M_{2 \times 2}(\mathbb{F})\) and the ordered basis
\[
\mathcal{B} = \left\{ \begin{pmatrix} 1 & 0 \\ 0 & 0 \end{pmatrix}, \begin{pmatrix} 0 & 1 \\ 0 & 0 \end{pmatrix}, \begin{pmatrix} 0 & 0 \\ 1 & 0 \end{pmatrix}, \begin{pmatrix} 0 & 0 \\ 0 & 1 \end{pmatrix} \right\},
\]
any matrix has a coordinate vector with respect to \(\mathcal{B}\), e.g.,
\[
\left[ \begin{pmatrix} 1 & 4 \\ 2 & 3 \end{pmatrix} \right]_{\mathcal{B}} = \begin{pmatrix} 1 \\ 4 \\ 2 \\ 3 \end{pmatrix}.
\]
However, if given another ordered basis
\[
\mathcal{B}_1 = \left\{ \begin{pmatrix} 0 & 1 \\ 0 & 0 \end{pmatrix}, \begin{pmatrix} 1 & 0 \\ 0 & 0 \end{pmatrix}, \begin{pmatrix} 0 & 0 \\ 1 & 0 \end{pmatrix}, \begin{pmatrix} 0 & 0 \\ 0 & 1 \end{pmatrix} \right\},
\]
the matrix may have a different coordinate vector:
\[
\left[ \begin{pmatrix} 1 & 4 \\ 2 & 3 \end{pmatrix} \right]_{\mathcal{B}_1} = \begin{pmatrix} 4 \\ 1 \\ 2 \\ 3 \end{pmatrix}.
\]
\end{example}

\begin{theorem}\label{thm: coordinate-isomorphism}
The mapping \([ \cdot ]_{\mathcal{B}} : V \to \mathbb{F}^n\) is an isomorphism.
\end{theorem}

\begin{proof}
We prove this in several steps:

\begin{enumerate}
  \item \textbf{Well-definedness.} Let
   \(
  \mathbf{v}  = \alpha_1 \mathbf{v}_1 + \cdots + \alpha_n \mathbf{v}_n.
  \) and \(
  {\bf v'} = \alpha_1' \mathbf{v}_1 + \cdots + \alpha_n' \mathbf{v}_n.
  \)
  Suppose ${\bf v} = {\bf v'}$,  then by uniqueness of coordinates \(\alpha_i = \alpha'_i\) for all \(i = 1, \ldots, n\) and hence
  \[
  [\mathbf{v}]_{\mathcal{B}} = \begin{pmatrix} \alpha_1 \\ \vdots \\ \alpha_n \end{pmatrix}
  = \begin{pmatrix} \alpha'_1 \\ \vdots \\ \alpha'_n \end{pmatrix} = [\mathbf{v'}]_{\mathcal{B}}.
  \]
 


  \item \textbf{Linearity.} It is clear that the operator is a linear transformation:
  \[
  [p\mathbf{v} + q\mathbf{w}]_{\mathcal{B}} = p[\mathbf{v}]_{\mathcal{B}} + q[\mathbf{w}]_{\mathcal{B}} \quad \forall\, p, q \in \mathbb{F}.
  \]

  \item \textbf{Injectivity.} If
  \[
  [\mathbf{v}]_{\mathcal{B}} = \begin{pmatrix} 0 \\ \vdots \\ 0 \end{pmatrix},
  \]
  then \(\mathbf{v} = 0\mathbf{v}_1 + \cdots + 0\mathbf{v}_n = \mathbf{0}\).

  \item \textbf{Surjectivity.} Given any \(\mathbf{x} = \begin{pmatrix} x_1 \\ \vdots \\ x_n \end{pmatrix} \in \mathbb{F}^n\), let
  \[
  \mathbf{v} := x_1 \mathbf{v}_1 + \cdots + x_n \mathbf{v}_n.
  \]
  Then \([ \mathbf{v} ]_{\mathcal{B}} = \mathbf{x}\), so every element in \(\mathbb{F}^n\) is hit.

\end{enumerate}

Therefore, \([ \cdot ]_{\mathcal{B}}\) is an isomorphism.
\end{proof}

\begin{example}\label{ex:coordinate_change}
Given a vector space \(V = P_3[x]\) and its basis \(B = \{1, x, x^2, x^3\}\).

To check if the set \(\{1 + x^2, 3 - x^3, x - x^3\}\) is linearly independent, by part (1) in \autoref{prop: isomorphism-properties} and \autoref{thm: coordinate-isomorphism}, it suffices to check whether the corresponding coordinate vectors
\[
\left\{
\begin{bmatrix}
1 \\ 0 \\ 1 \\ 0
\end{bmatrix},
\begin{bmatrix}
3 \\ 0 \\ 0 \\ -1
\end{bmatrix},
\begin{bmatrix}
0 \\ 1 \\ 0 \\ -1
\end{bmatrix}
\right\}
\]
are linearly independent, i.e., do Gaussian Elimination and check the number of pivots.
\end{example}

\begin{remark}
Here gives rise to the question: if \(\mathcal{B}_1, \mathcal{B}_2\) form two bases of \(V\), then how are \([\mathbf{v}]_{\mathcal{B}_1}, [\mathbf{v}]_{\mathcal{B}_2}\) related to each other?

Here we consider an easy example first:
\end{remark}

\begin{example}\label{ex: change-of-basis}
Consider \(V = \mathbb{R}^n\) and its basis \(\mathcal{B}_1 = \{ \mathbf{e}_1, \ldots, \mathbf{e}_n \}\). For any \(\mathbf{v} \in V\),
\[
\mathbf{v} = \begin{bmatrix} \alpha_1 \\ \vdots \\ \alpha_n \end{bmatrix}
= \alpha_1 \mathbf{e}_1 + \cdots + \alpha_n \mathbf{e}_n
\Rightarrow [\mathbf{v}]_{\mathcal{B}_1} = \begin{bmatrix} \alpha_1 \\ \vdots \\ \alpha_n \end{bmatrix}
\]

Also, we can construct a different basis \(\mathcal{B}_2\) of \(V\) as:
\[
\mathcal{B}_2 = \left\{
\begin{bmatrix} 1 \\ 0 \\ \vdots \\ 0 \end{bmatrix},
\begin{bmatrix} 1 \\ 1 \\ \vdots \\ 0 \end{bmatrix},
\ldots,
\begin{bmatrix} 1 \\ 1 \\ \vdots \\ 1 \end{bmatrix}
\right\}
\]

which gives a different coordinate vector of \(\mathbf{v}\):
\[
[\mathbf{v}]_{B_2} =
\begin{bmatrix}
\alpha_1 - \alpha_2 \\
\alpha_2 - \alpha_3 \\
\vdots \\
\alpha_{n-1} - \alpha_n \\
\alpha_n
\end{bmatrix}
\]
\end{example}

\begin{proposition}[Change of Basis]\label{prop: change-of-basis}
Let \( \mathcal{A} = \{ \mathbf{v}_1, \ldots, \mathbf{v}_n \} \) and \( \mathcal{A}' = \{ \mathbf{w}_1, \ldots, \mathbf{w}_n \} \) be two ordered bases of a vector space \( V \). Define the change of basis matrix from \( \mathcal{A} \) to \( \mathcal{A}' \), say \( C_{\mathcal{A}',\mathcal{A}} := [\alpha_{ij}] \), where
\[
\mathbf{v}_j = \sum_{i=1}^n \alpha_{ij} \mathbf{w}_i.
\]
Then for any vector \( \mathbf{v} \in V \), the change of basis amounts to left-multiplying the change of basis matrix:
\begin{equation}\label{eq:change-basis}
C_{\mathcal{A}',\mathcal{A}}[\mathbf{v}]_{\mathcal{A}} = [\mathbf{v}]_{\mathcal{A}'}.
\end{equation}
Define matrix \( C_{\mathcal{A},\mathcal{A}'} := [\beta_{ij}] \), where
\[
\mathbf{w}_j = \sum_{i=1}^n \beta_{ij} \mathbf{v}_i.
\]
Then \( \left( C_{\mathcal{A},\mathcal{A}'} \right)^{-1} = C_{\mathcal{A}',\mathcal{A}} \).
\end{proposition}
\begin{proof}
\begin{enumerate}
    \item \textbf{Validity check.} Show \eqref{eq:change-basis} holds for \( \mathbf{v} = \mathbf{v}_j \), \( j = 1, \ldots, n \):
    \begin{align*}
        \text{LHS of \eqref{eq:change-basis}} &= C_{\mathcal{A}'\mathcal{A}} [{\bf v}_j]_{\mathcal{A}} =  [\alpha_{ij}] {\bf e}_j = \begin{pmatrix} \alpha_{1j} \\ \vdots \\ \alpha_{nj} \end{pmatrix}, \\
        \text{RHS of \eqref{eq:change-basis}} &= [\mathbf{v}_j]_{\mathcal{A}'} 
        = \left[ \sum_{i=1}^n \alpha_{ij} \mathbf{w}_i \right]_{\mathcal{A}'} 
        = \begin{pmatrix} \alpha_{1j} \\ \vdots \\ \alpha_{nj} \end{pmatrix}.
    \end{align*}
    So \eqref{eq:change-basis} holds for all basis vectors \( \mathbf{v}_j \).

    \item \textbf{Linearity extension.} Let \( \mathbf{v} = \sum_{j=1}^n r_j \mathbf{v}_j \). Then:

    \begin{align*}
        C_{\mathcal{A}',\mathcal{A}}[\mathbf{v}]_{\mathcal{A}} 
        &= C_{\mathcal{A}',\mathcal{A}} \left[ \sum_{j=1}^n r_j \mathbf{v}_j \right]_{\mathcal{A}}\\
        &= C_{\mathcal{A}',\mathcal{A}} \sum_{j=1}^n r_j [\mathbf{v}_j]_{\mathcal{A}} \\
        &= \sum_{j=1}^n r_j C_{\mathcal{A}',\mathcal{A}} [\mathbf{v}_j]_{\mathcal{A}}  \\
        &= \sum_{j=1}^n r_j [\mathbf{v}_j]_{\mathcal{A}'} \\
        &= \left[ \sum_{j=1}^n r_j \mathbf{v}_j \right]_{\mathcal{A}'} = [\mathbf{v}]_{\mathcal{A}'} \notag
    \end{align*}
     where the first and last line follow from the linearity of the coordinate map \( [\,\cdot\,]_{\mathcal{A}} \) and \( [\,\cdot\,]_{\mathcal{A}'} \), and the fourth equality follows from (1) above. Therefore, \eqref{eq:change-basis} is shown for all \( \mathbf{v} \in V \). 
    \item \textbf{Matrix inverse.} We verify \( C_{\mathcal{A}',\mathcal{A}} C_{\mathcal{A},\mathcal{A}'} = I_n \):

    \begin{align*}
    \mathbf{v}_j 
    &= \sum_{i=1}^n \alpha_{ij} \mathbf{w}_i 
    = \sum_{i=1}^n \alpha_{ij} \left( \sum_{k=1}^n \beta_{ki} \mathbf{v}_k \right) 
    = \sum_{i=1}^n \sum_{k=1}^n \alpha_{ij} \beta_{ki} \mathbf{v}_k \\
    &= \sum_{k=1}^n \left( \sum_{i=1}^n \beta_{ki} \alpha_{ij} \right) \mathbf{v}_k.
    \end{align*}
    
    By the uniqueness of coordinates, we imply
    \[
    \left( \sum_{i=1}^n \beta_{ki} \alpha_{ij} \right) = \delta_{jk} := 
    \begin{cases}
    1, & j = k \\
    0, & j \ne k
    \end{cases}
    \]
    
    By matrix multiplication, the \( (k,j) \)-th entry of \( C_{\mathcal{A}',\mathcal{A}} C_{\mathcal{A},\mathcal{A}'} \) is
    \[
    [C_{\mathcal{A}',\mathcal{A}} C_{\mathcal{A},\mathcal{A}'}]_{kj} = \left( \sum_{i=1}^n \beta_{ki} \alpha_{ij} \right) = \delta_{jk}
    \Rightarrow C_{\mathcal{A}',\mathcal{A}} C_{\mathcal{A},\mathcal{A}'} = I_n.
    \]

  \item \textbf{Symmetry.} Similarly,
  \begin{align*}
    \mathbf{w}_j &= \sum_{i=1}^n \beta_{ij} \mathbf{v}_i 
    = \sum_{i=1}^n \beta_{ij} \left( \sum_{k=1}^n \alpha_{ki} \mathbf{w}_k \right) 
    = \sum_{k=1}^n \left( \sum_{i=1}^n \alpha_{ki} \beta_{ij} \right) \mathbf{w}_k,
  \end{align*}
  giving \( \sum_{i=1}^n \alpha_{ki} \beta_{ij} = \delta_{kj} \), so
  \[
  C_{\mathcal{A},\mathcal{A}'} C_{\mathcal{A}',\mathcal{A}} = I_n.
  \]
\end{enumerate}
\end{proof}

\begin{example}
Back to \autoref{ex: change-of-basis}, write \( \mathcal{B}_1, \mathcal{B}_2 \) as
\[
\mathcal{B}_1 = \{ \mathbf{e}_1, \ldots, \mathbf{e}_n \}, 
\quad 
\mathcal{B}_2 = \{ \mathbf{w}_1, \ldots, \mathbf{w}_n \}
\]
and therefore \( \mathbf{w}_i = \mathbf{e}_1 + \cdots + \mathbf{e}_i \). The change of basis matrix is given by
\[
C_{\mathcal{B}_1, \mathcal{B}_2} = 
\begin{pmatrix}
1 & 1 & \cdots & 1 \\
0 & 1 & \cdots & 1 \\
\vdots & \vdots & \ddots & \vdots \\
0 & 0 & \cdots & 1
\end{pmatrix}.
\]
For instance, take \( \mathbf{v} \) in the example, then one has:
\[
C_{\mathcal{B}_1, \mathcal{B}_2} [\mathbf{v}]_{\mathcal{B}_2} = 
\begin{pmatrix}
1 & 1 & \cdots & 1 \\
0 & 1 & \cdots & 1 \\
\vdots & \vdots & \ddots & \vdots \\
0 & 0 & \cdots & 1
\end{pmatrix}
\begin{pmatrix}
\alpha_1 - \alpha_2 \\
\vdots \\
\alpha_{n-1} - \alpha_n \\
\alpha_n
\end{pmatrix}
=
\begin{pmatrix}
\alpha_1 \\
\vdots \\
\alpha_n
\end{pmatrix}
= [\mathbf{v}]_{\mathcal{B}_1}
\]
\end{example}

\subsection{Matrix Representations}
\begin{definition}
Let \( T : V \to W \) be a linear transformation, and
\[
\mathcal{A} = \{ \mathbf{v}_1, \ldots, \mathbf{v}_n \}, 
\quad 
\mathcal{B} = \{ \mathbf{w}_1, \ldots, \mathbf{w}_m \}
\]
be bases of \( V \) and \( W \), respectively. The matrix representation of \( T \) with respect to (w.r.t.) \( \mathcal{A} \) and \( \mathcal{B} \) is defined as \( (T)_{\mathcal{B}\mathcal{A}} := (\alpha_{ij}) \in M_{m \times n}(\mathbb{F}) \), where
\[
T(\mathbf{v}_j) = \sum_{i=1}^m \alpha_{ij} \mathbf{w}_i
\]
for $j = 1, \dots, n$.
\end{definition}


\begin{example} \label{eg:matrix_transformation}
In MAT2040, we studied \emph{matrix transformations} $T:\mathbb{F}^n \to \mathbb{F}^m$ given by ${\bf x} \mapsto A{\bf x}$ for some $A \in M_{m \times n}(\mathbb{F})$. In this section, we want to put it in the perspective of linear transformations:

    Let $A = \begin{pmatrix}
        1 & 2 \\ 3 & 4
    \end{pmatrix}$, and $T: \mathbb{R}^2 \to \mathbb{R}^2$ given by $T{\bf x} = A{\bf x}$, and $\mathcal{B} = \left\{\begin{pmatrix}
        1 \\ -1
    \end{pmatrix}, \begin{pmatrix}
        1 \\ 1
    \end{pmatrix}\right\}$, then the matrix representation of $T: \mathbb{R}^2_{\mathcal{B}} \to \mathbb{R}^2_{\mathcal{B}}$ can be calculated by:
    \begin{align*}T\begin{pmatrix}
        1 \\ -1
    \end{pmatrix} &= \begin{pmatrix}
        -1 \\ -1
    \end{pmatrix} = 0\begin{pmatrix}
        1 \\ -1
    \end{pmatrix}+ -1\begin{pmatrix}
        1 \\ 1
    \end{pmatrix}\\
    T\begin{pmatrix}
        1 \\ 1
    \end{pmatrix} &= \begin{pmatrix}
        3 \\ 7
    \end{pmatrix} = -2\begin{pmatrix}
        1 \\ -1
    \end{pmatrix}+ 5\begin{pmatrix}
        1 \\ 1
    \end{pmatrix}
    \end{align*}
Therefore,
$$T_{\mathcal{B},\mathcal{B}} = \begin{pmatrix}
        0 & -2 \\ -1 & 5
    \end{pmatrix}$$
So we end up having a different matrix than $A$!

However, if we use the usual basis $\mathcal{E} = \{{\bf e}_1, {\bf e}_2\}$, then the matrix representation of $T: \mathbb{R}^2_{\mathcal{E}} \to \mathbb{R}^2_{\mathcal{E}}$ becomes:
    \begin{align*}T({\bf e}_1) &= \begin{pmatrix}
        1 \\ 3
    \end{pmatrix} = 1\{{\bf e}_1+ 3{\bf e}_2\\
    T({\bf e}_2) &= \begin{pmatrix}
        2 \\ 4
    \end{pmatrix} = 2{\bf e}_1+ 4{\bf e}_2,
    \end{align*}
so that 
$$T_{\mathcal{E},\mathcal{E}} = \begin{pmatrix}
        1 & 2 \\ 3 & 4
    \end{pmatrix} = A$$
and one can get back our original matrix $A$. 

This generalized perspective is extremely useful in our latter studies - it allows us to understand matrix transformations using different bases (coordinate systems). In the special case when one chooses the usual basis, one can retrieve the original matrix (see \autoref{sec:similar_basis} below.
\end{example}


Here is an example of matrix representation of a non-matrix transformation:
\begin{example}
Let \( V = \mathbb{P}_3[x] \) and \( \mathcal{A} = \{ 1, x, x^2, x^3 \} \).

Let \( T : V \to V \) be defined as \( p(x) \mapsto p'(x) \) :
\[
\left\{
\begin{aligned}
T(1) &= 0 \cdot 1 + 0 \cdot x + 0 \cdot x^2 + 0 \cdot x^3 \\
T(x) &= 1 \cdot 1 + 0 \cdot x + 0 \cdot x^2 + 0 \cdot x^3 \\
T(x^2) &= 0 \cdot 1 + 2 \cdot x + 0 \cdot x^2 + 0 \cdot x^3 \\
T(x^3) &= 0 \cdot 1 + 0 \cdot x + 3 \cdot x^2 + 0 \cdot x^3
\end{aligned}
\right.
\]

We can define the change of basis matrix for the linear transformation \( T: V_{\mathcal{A}} \to V_{\mathcal{A}} \) with respect to the bases \( \mathcal{A} \) and \( \mathcal{A} \) as:
\[
T_{\mathcal{A},\mathcal{A}} =
\begin{pmatrix}
0 & 1 & 0 & 0 \\
0 & 0 & 2 & 0 \\
0 & 0 & 0 & 3 \\
0 & 0 & 0 & 0
\end{pmatrix}
\]

Suppose a different basis \( \mathcal{A}' = \{ x^3, x^2, x, 1 \} \) for the output space of \( T \), i.e., \( T : V_{\mathcal{A}} \to V_{\mathcal{A}'} \). Then,
\[
T_{\mathcal{A}', \mathcal{A}} =
\begin{pmatrix}
0 & 0 & 0 & 0 \\
0 & 0 & 0 & 3 \\
0 & 0 & 2 & 0 \\
0 & 1 & 0 & 0
\end{pmatrix}
\]
for instance, since $T(x^2) = 0\cdot x^3 + 0\cdot x^2 + 2\cdot x + 0\cdot 1$ (written in the order of $\mathcal{A}'$, the third column of $T_{\mathcal{A}', \mathcal{A}}$ reads $\begin{pmatrix}
    0 \\ 0 \\ 2 \\0 
\end{pmatrix}$.
\end{example}
Observe that the coordinate vectors before and after applying \( T \) has the same matrix multiplication. For instance, if we express $T(2x^2 + 4x^3) = (4x + 12x^2)$ in terms of matrices and vectors:
\[
[2x^2 + 4x^3]_{\mathcal{A}} =
\begin{pmatrix}
0 \\ 0 \\ 2 \\ 4
\end{pmatrix},
\quad
[4x + 12x^2]_{\mathcal{A}} =
\begin{pmatrix}
0 \\ 4 \\ 12 \\ 0
\end{pmatrix},\]
Then one can easily check that the matrix representation $T_{\mathcal{A}, \mathcal{A}}$ satisfies 
$$T_{\mathcal{A}, \mathcal{A}} \cdot 
[2x^2 + 4x^3]_{\mathcal{A}}
=
[4x + 12x^2]_{\mathcal{A}} = [T(2x^2 + 4x^3)]_{\mathcal{A}}.$$
In other words, one can compute $T:V \to V$ by using $T_{\mathcal{A}, \mathcal{A}}$. More generally:
\begin{theorem}[Matrix Representation]\label{thm: matrix-Representation}
Let \( T : V \to W \) be a linear transformation of finite-dimensional vector spaces.  
Let \( \mathcal{A}, \mathcal{B} \) be the ordered bases of \( V \) and \( W \), respectively.  
Then the following diagram holds:
\begin{figure}[h!]
\centering
\[
\begin{tikzcd}[row sep=large, column sep=huge]
V \arrow[r, "T"] \arrow[d, "{[\cdot]_{\mathcal{A}}}"'] 
  & W \arrow[d, "{[\cdot]_{\mathcal{B}}}"] \\
\mathbb{F}^n \arrow[r, "T_{\mathcal{B}\mathcal{A}}"'] 
  & \mathbb{F}^m
\end{tikzcd}
\]
\end{figure}

Namely, for any \( \mathbf{v} \in V \), 
$$ T_{\mathcal{B},\mathcal{A}} [\mathbf{v}]_{\mathcal{A}} = [T (\mathbf{v})]_{\mathcal{B}}.$$
(Informally, the linear transformation $T$ `corresponds' to multiplication of coordinate matrix $T_{\mathcal{B},\mathcal{A}}$).
\end{theorem}
\begin{proof}
Suppose \( \mathcal{A} = \{ \mathbf{v}_1, \dots, \mathbf{v}_n \} \) and \( \mathcal{B} = \{ \mathbf{w}_1, \dots, \mathbf{w}_m \} \). The proof of this theorem follows the same procedure as in \autoref{prop: change-of-basis}.

\begin{enumerate}
    \item \textbf{Basis vector} We compare both sides:
\begin{align*}
\text{LHS} = T_{\mathcal{B}\mathcal{A}} [{\bf v}_j]_{\mathcal{A}} = [\alpha_{ij}] \cdot \mathbf{e}_j = \begin{pmatrix} \alpha_{1j} \\ \vdots \\ \alpha_{nj} \end{pmatrix} = \left( \sum_{i=1}^{m} \alpha_{ij} \mathbf{w}_i \right)_{\mathcal{B}} = [T (\mathbf{v}_j)]_{\mathcal{B}} = \text{RHS} 
\end{align*}
So the identity \( T_{\mathcal{B}\mathcal{A}} [\mathbf{v}_j]_{\mathcal{A}} = [T\mathbf{v}_j]_{\mathcal{B}} \) holds for all basis vectors \( \mathbf{v}_j \).

\item \textbf{Extension to arbitrary vector}
Let \( \mathbf{v} = \sum_{j=1}^n r_j \mathbf{v}_j \) with coefficients \( r_j \in \mathbb{F} \). Then:
\begin{align*}
[T]_{\mathcal{B}\mathcal{A}}([\mathbf{v}]_{\mathcal{A}}) = (T)_{\mathcal{B}\mathcal{A}}\left( \sum_{j=1}^n r_j [\mathbf{v}_j]_{\mathcal{A}} \right) 
&= \sum_{j=1}^n r_j [T]_{\mathcal{B}\mathcal{A}}[\mathbf{v}_j]_{\mathcal{A}}  \\
&= \sum_{j=1}^n r_j [T (\mathbf{v}_j)]_{\mathcal{B}}  \\
&= \left[ \sum_{j=1}^n r_j T(\mathbf{v}_j)\right]_{\mathcal{B}}\\
&= [T(\sum_{j=1}^n r_j \mathbf{v}_j)]_{\mathcal{B}} = [T( \mathbf{v})]_{\mathcal{B}}.
\end{align*}
\end{enumerate}

Thus, for all \( \mathbf{v} \in V \), we have
\[
[T]_{\mathcal{B}\mathcal{A}} [\mathbf{v}]_{\mathcal{A}} = [T(\mathbf{v})]_{\mathcal{B}},
\]
completing the proof.
\end{proof}

\begin{remark} \label{rmk:changeofbasis_matrixtrans}
Suppose 
\[ \operatorname{id}: V_{\mathcal{A}} \to V_{\mathcal{A}'}\]
is the identity transformation $\operatorname{id}({\bf v}) := {\bf v}$ for all ${\bf v} \in V$, and \( \mathcal{A}, \mathcal{A}' \) are two ordered bases of \( V \). We would like to study
the matrix representation 
$$\operatorname{id}_{\mathcal{A}',\mathcal{A}}$$
(from now on, we write $T: V_{\mathcal{A}} \to W_{\mathcal{B}}$ instead of $T:V \to W$ whenever we study the matrix representation $T_{\mathcal{B},\mathcal{A}}$). 

Firstly, \autoref{thm: matrix-Representation} implies that
\[
\operatorname{id}_{\mathcal{A}', \mathcal{A}} ([\mathbf{v}]_{\mathcal{A}}) = [\operatorname{id}(\mathbf{v})]_{\mathcal{A}'} = [\mathbf{v}]_{\mathcal{A}'}.
\]
On the other hand, \autoref{prop: change-of-basis} implies that the change of basis matrix $C_{\mathcal{A}', \mathcal{A}}$ satisfies:
\[
C_{\mathcal{A}', \mathcal{A}} ([\mathbf{v}]_{\mathcal{A}}) = [\mathbf{v}]_{\mathcal{A}'}.
\]
Therefore, 
\[ \operatorname{id}_{\mathcal{A}', \mathcal{A}} = C_{\mathcal{A}', \mathcal{A}} \] 
This shows that \autoref{thm: matrix-Representation} generalizes the change-of-basis theorem \autoref{prop: change-of-basis}.
\end{remark}

Another advantage of using matrix representation to understand linear transformation is that:
\begin{center}
\fbox{
    Composing linear transformations $S \circ T$ \quad  $\longleftrightarrow$ \quad Multiplying matrix representations $S_{\mathcal{C},\mathcal{B}} \cdot T_{\mathcal{B},\mathcal{A}}$}
\end{center}
\begin{proposition}[Functoriality] \label{prop:functoriality}
Let \( T : V \to W \), $S: W \to U$ be linear transformations, and $\mathcal{A}$, $\mathcal{B}$, $\mathcal{C}$ be ordered bases of $V$, $W$ and $U$ respectively. Then the corresponding matrix representations satisfy
\begin{equation}\label{eq:change-of-basis-matrix}
(S \circ T)_{\mathcal{C}, \mathcal{A}} = S_{\mathcal{C}, \mathcal{B}} \cdot T_{\mathcal{B}, \mathcal{A}}.
\end{equation}
\end{proposition}

\begin{proof}
Let \( \mathcal{A} = \{ \mathbf{v}_1, \ldots, \mathbf{v}_n \} \), \( \mathcal{B} = \{ \mathbf{w}_1, \ldots, \mathbf{w}_m \} \), \( \mathcal{C} = \{ \mathbf{u}_1, \ldots, \mathbf{u}_p \} \). As before, it suffices to check \eqref{eq:change-of-basis-matrix} holds on each column:
$$(S \circ T)_{\mathcal{C}, \mathcal{A}}{\bf e}_j = (S_{\mathcal{C}, \mathcal{B}} \cdot T_{\mathcal{B}, \mathcal{A}}){\bf e}_j.$$

Recall that 
\[
T\left( {\mathbf{v}}_{j}\right)  = \mathop{\sum }\limits_{i}{\left( {T}_{\mathcal{B},\mathcal{A}}\right) }_{ij}{\mathbf{w}}_{i}, \quad \quad S\left( {\mathbf{w}}_{i}\right)  = \mathop{\sum }\limits_{k}{\left( {S}_{C,\mathcal{B}}\right) }_{ki}{\mathbf{u}}_{k}
\]
Therefore,
\begin{align*}
{\left( S \circ  T\right) }_{\mathcal{C},\mathcal{A}}\mathbf{e}_{j}={\left( S \circ  T\right) }_{\mathcal{C},\mathcal{A}}{\left( {\mathbf{v}}_{j}\right) }_{\mathcal{A}} &= {\left( S \circ  T\left( {\mathbf{v}}_{j}\right) \right) }_{\mathcal{C}}\\
&= {\left\lbrack  S \circ  \left( \mathop{\sum }\limits_{i}{\left( {T}_{\mathcal{B},\mathcal{A}}\right) }_{ij}{\bf w}_{i}\right) \right\rbrack  }_{\mathcal{C}}\\
&= \mathop{\sum }\limits_{i}{\left( {T}_{\mathcal{B},\mathcal{A}}\right) }_{ij}{\left( S\left( {\bf w}_{i}\right) \right) }_{\mathcal{C}} \\
&= \mathop{\sum }\limits_{i}{\left( {T}_{\mathcal{B},\mathcal{A}}\right) }_{ij}{\left( \mathop{\sum }\limits_{k}{\left( {S}_{C,\mathcal{B}}\right) }_{ki}{\bf u}_{k}\right) }_{\mathcal{C}} \\
&= \mathop{\sum }\limits_{k}\mathop{\sum }\limits_{i}{\left( {S}_{C,\mathcal{B}}\right) }_{ki}{\left( {T}_{\mathcal{B},\mathcal{A}}\right) }_{ij}{\left( {\mathbf{u}}_{k}\right) }_{\mathcal{C}} \\
&= \mathop{\sum }\limits_{k}{\left( {S}_{\mathcal{C},\mathcal{B}}\cdot{T}_{\mathcal{B},\mathcal{A}}\right) }_{kj}{\left( {\mathbf{u}}_{k}\right) }_{C} \\
&= \mathop{\sum }\limits_{k}{\left( {S}_{C,\mathcal{B}}\cdot{T}_{\mathcal{B},\mathcal{A}}\right) }_{kj}{\mathbf{e}}_{k} \\ 
&=({S}_{C\mathcal{B}}\cdot{T}_{\mathcal{B},\mathcal{A}})\cdot {\bf e}_j
\end{align*}
Consequently, the result follows.
\end{proof}


\begin{example}
Let $\operatorname{id}:V \to V$ be the identity map, and $\mathcal{A}, \mathcal{A}'$ are two ordered bases of $V$. By \autoref{rmk:changeofbasis_matrixtrans}, $\operatorname{id}_{\mathcal{A}',\mathcal{A}} = C_{\mathcal{A}',\mathcal{A}}$ and $\operatorname{id}_{\mathcal{A},\mathcal{A}'} = C_{\mathcal{A},\mathcal{A}'}$. Now \autoref{prop:functoriality} implies that
$$(\operatorname{id} \circ \operatorname{id})_{\mathcal{A},\mathcal{A}} = \operatorname{id}_{\mathcal{A},\mathcal{A}'} \cdot \operatorname{id}_{\mathcal{A}',\mathcal{A}} = C_{\mathcal{A},\mathcal{A}'} \cdot C_{\mathcal{A}',\mathcal{A}}$$
But obviously $(\operatorname{id} \circ \operatorname{id})_{\mathcal{A},\mathcal{A}} = \operatorname{id}_{\mathcal{A},\mathcal{A}} = I_{\dim(V) \times \dim(V)}$
is the identity matrix. Therefore, we have reproved that
$$I_{\dim(V) \times \dim(V)} = C_{\mathcal{A},\mathcal{A}'} \cdot C_{\mathcal{A}',\mathcal{A}}.$$
\end{example}




\section{Relation to Similar Matrices} \label{sec:similar_basis}
{\bf We end this section with a very important observations in linear algebra:} 
Let \( T : V \to V \) be a linear operator with \( \mathcal{A}, \mathcal{A}' \) being two ordered bases of \( V \).  We wish to relate
\begin{center}
    $T_{\mathcal{A}',\mathcal{A}'}$ and $T_{\mathcal{A},\mathcal{A}}$.
\end{center}
More explicitly, rewrite $T$ by $T = \operatorname{id} \circ T \circ \operatorname{id}$. By specifying which bases we are working on, $T: V_{\mathcal{A}'} \to V_{\mathcal{A}'}$ can be understood as:
$$V_{\mathcal{A}'} \xrightarrow{\operatorname{id}} V_{\mathcal{A}} \xrightarrow{T} V_{\mathcal{A}}\xrightarrow{\operatorname{id}} V_{\mathcal{A}'}$$
By \autoref{prop:functoriality},
\[
T_{\mathcal{A}', \mathcal{A}'} = \operatorname{id}_{\mathcal{A}', \mathcal{A}} \cdot 
T_{\mathcal{A}, \mathcal{A}} \cdot \operatorname{id}_{\mathcal{A}, \mathcal{A}'}
= C_{\mathcal{A}', \mathcal{A}} \cdot T_{\mathcal{A}, \mathcal{A}} \cdot C_{\mathcal{A}, \mathcal{A}'} = (C_{\mathcal{A}, \mathcal{A}'})^{-1} T_{\mathcal{A}, \mathcal{A}} C_{\mathcal{A}, \mathcal{A}'}.
\]
Therefore, the two matrix representations \( T_{\mathcal{A}', \mathcal{A}'} \) and \( T_{\mathcal{A}, \mathcal{A}} \) are {\bf similar matrices}. In particular, they share the same characteristic polynomials, eigenvalues, determinant, etc.

\medskip
Putting this into the perspective of matrix representations, recall that we do the following in \autoref{eg:matrix_transformation}:
\begin{itemize}
    \item Let $T: \mathbb{F}^n \to \mathbb{F}^n$ be defined by $T({\bf x}) := A{\bf x}$. 
    \item Let $\mathcal{B} = \{{\bf m}_1, \dots, {\bf m}_n\}$ be an ordered basis of $\mathbb{F}^n$, then
    $$T_{\mathcal{E},\mathcal{E}} = B$$
    \item Let $\mathcal{E} = \{{\bf e}_1, \dots {\bf e}_n\}$ be the usual ordered basis of $\mathbb{F}^n$. Then one gets back the original matrix 
    $$T_{\mathcal{E},\mathcal{E}} = A$$
\end{itemize}
Let $M$ be the matrix whose columns are given by $\mathcal{B}$, then \autoref{prop: change-of-basis} implies that $C_{\mathcal{E},\mathcal{B}} = M,$
and our discussions above gives
    \[B = M^{-1}AM.\]
In particular, $A$ and $B$ are similar.

To showcase the above arguments in the setting of  \autoref{eg:matrix_transformation}, note that
$$T_{\mathcal{E},\mathcal{E}} = \begin{pmatrix}
        1 & 2 \\ 3 & 4
    \end{pmatrix} \quad \quad T_{\mathcal{B},\mathcal{B}} = \begin{pmatrix}
        0 & -2 \\ -1 & 5
    \end{pmatrix}$$
and the change of basis matrix is
$C_{\mathcal{E},\mathcal{B}} = \begin{pmatrix}
        1 & 1 \\ -1 & 1
    \end{pmatrix}$. Then one can check that
$$\begin{pmatrix}
        0 & -2 \\ -1 & 5
    \end{pmatrix} = \begin{pmatrix}
        1 & 1 \\ -1 & 1
    \end{pmatrix}^{-1}\begin{pmatrix}
        1 & 2 \\ 3 & 4
    \end{pmatrix}\begin{pmatrix}
        1 & 1 \\ -1 & 1
    \end{pmatrix}$$
Now one can understand diagonalizable matrices better: Recall in MAT2040, the following are equivalent:
\begin{itemize}
    \item $A$ is diagonalizable;
    \item There exists a basis $\mathcal{B}$ of eigenvectors;
    \item There exists an invertible $M$ such that
    $$\begin{pmatrix}
        \lambda_1 & & \\
        & \ddots & \\
        & & \lambda_n
    \end{pmatrix} = M^{-1}AM,$$
    i.e. $A$ is similar to a diagonal matrix.
\end{itemize}
Under our new perspective, we have a natural account (rather than direct computation) on why the columns of $M$ are given by the eigen-basis $\mathcal{B}$, and diagonalization is simply a change-of-basis of the {\bf same} linear transformation $T({\bf x}) = A{\bf x}$ from the usual basis $\mathcal{E}$ to the (eigen)-basis $\mathcal{B}$.

To conclude:
\begin{center}
\fbox{$A$, $B$ are {\bf similar}\quad $\longleftrightarrow$\ \quad $\begin{matrix} A, B\ \text{are matrix representations of the {\bf same} linear } \\
\text{transformation under {\bf different} coordinate systems.}  \end{matrix}$}
\end{center}



\chapter{Quotient Spaces \& Its Universal Properties}

\section{Cosets and Quotient Space}
One important aspect in mathematics is to construct {\it new} objects from {\it known} ones. In this chapter, we will study the quotient space of a (known) vector space $V$ by a (known) vector subspace $W$. Informally speaking, we will `divide' the big vector space $V$ into many `slices' of $W$.


\begin{definition}[Coset]
Let \( V \) be a vector space and \( W \leq V \). For any element \( \mathbf{v} \in V \), the (right) coset determined by \( \mathbf{v} \) is the set
\[
\mathbf{v} + W := \{ \mathbf{v} + \mathbf{w} \mid \mathbf{w} \in W \}.
\]
\end{definition}

\begin{example}
Consider \( V = \mathbb{R}^3 \) and \( W = \mathrm{span}\{(1,2,0)\} \). Then the coset determined by \( \mathbf{v} = (5,6,-3) \) can be written as
\[
\mathbf{v} + W = \{ (5 + t, 6 + 2t, -3) \mid t \in \mathbb{R} \}.
\]

It’s interesting that the coset determined by \( \mathbf{v}' = (4,4,-3) \) is exactly the same as the coset shown above:
\[
\mathbf{v}' + W = \{ (4 + t, 4 + 2t, -3) \mid t \in \mathbb{R} \} = \mathbf{v} + W.
\]

Therefore, writing the exact expression of \( \mathbf{v} + W \) may sometimes become tedious and hard to check the equivalence. We say \( \mathbf{v} \) is a \emph{representative} of a coset \( \mathbf{v} + W \). 
\end{example}

\begin{example}
One motivation to understand cosets is to describe the solutions of the linear system \( A \mathbf{x} = \mathbf{b} \) with \( A \in \mathbb{R}^{m \times n} \). 
Recall in MAT2040 that the general step for solving this linear system is as follows:
\begin{enumerate}
    \item Find the solution set for \( A \mathbf{x} = 0 \), i.e., the set \( \ker(A) \leq \mathbb{R}^n\).
    \item Find a particular solution \( \mathbf{x}_0 \) such that \( A \mathbf{x}_0 = \mathbf{b} \).
\end{enumerate}
\noindent Then the general solution set to this linear system is the coset 
$\mathbf{x}_0 + \ker(A)$
of $\mathbb{R}^n$.

Recall that the particular solution ${\bf x}_0$ is {\bf not} unique. Indeed, if ${\bf y}_0$ is another particular solution of $A{\bf x} = {\bf b}$, then 
$${\bf x}_0 + \ker(A) = {\bf y}_0 + \ker(A)$$
describes the same set of solutions of the system of equations.
\end{example}

As directed by the previous examples, we have:
\begin{proposition}\label{prop: coset-equality}
Two cosets are the same if and only if the subtraction for the corresponding representatives is in \( W \), i.e.,
\[
\mathbf{v}_1 + W = \mathbf{v}_2 + W \iff \mathbf{v}_1 - \mathbf{v}_2 \in W.
\]
\end{proposition}

\begin{proof}
\textbf{Necessity.} Suppose that \( \mathbf{v}_1 + W = \mathbf{v}_2 + W \),  
then \( \mathbf{v}_1 + \mathbf{w}_1 = \mathbf{v}_2 + \mathbf{w}_2 \) for some \( \mathbf{w}_1, \mathbf{w}_2 \in W \),  
which implies
\[
\mathbf{v}_1 - \mathbf{v}_2 = \mathbf{w}_2 - \mathbf{w}_1 \in W.
\]

\textbf{Sufficiency.} Suppose that \( \mathbf{v}_1 - \mathbf{v}_2 = \mathbf{w} \in W \).  
It suffices to show \( \mathbf{v}_1 + W \subseteq \mathbf{v}_2 + W \).  
For any \( \mathbf{v}_1 + \mathbf{w}' \in \mathbf{v}_1 + W \), this element can be expressed as
\[
\mathbf{v}_1 + \mathbf{w}' = (\mathbf{v}_2 + \mathbf{w}) + \mathbf{w}' = \mathbf{v}_2 + \underbrace{(\mathbf{w} + \mathbf{w}')}_{\in W} \in \mathbf{v}_2 + W.
\]
Therefore, \( \mathbf{v}_1 + W \subseteq \mathbf{v}_2 + W \). Similarly, we can show that \( \mathbf{v}_2 + W \subseteq \mathbf{v}_1 + W \),  
and thus \( \mathbf{v}_1 + W = \mathbf{v}_2 + W \).
\end{proof}

\noindent (Exercise: If two cosets \(\mathbf{v}_1 +W \neq \mathbf{v}_2+W \) are not equal, then they have no intersection, i.e. \( (\mathbf{v}_1 +W) \cap   (\mathbf{v}_2 + W) = \phi \)).

\begin{remark} \label{rmk:different_expression}
    To many of us, this may be the first occasion to come across some mathematical objects that has {\bf different expressions of the same element}, i.e. one may have 
    \begin{center}
    ${\bf v}_1 \neq {\bf v}_2$ but ${\bf v}_1 + W = {\bf v}_2 + W$. 
    \end{center}
    Therefore, extra care is needed whenever we study such objects (see {\bf WARNING} above \autoref{prop:coset_well_defined}).
\end{remark}


\begin{definition}[Quotient Space]\label{def:quotient-space}
The quotient space $V/W$ of \( V \) by the subspace \( W \) is the collection of all cosets \( \mathbf{v} + W \), i.e. 
\[ V/W := \{{\bf v} + W\ | {\bf v} \in V\}.\]
\end{definition}

To make the quotient space a vector space structure, we define the addition and scalar multiplication on \( V/W \) by:
\[
(\mathbf{v}_1 + W) + (\mathbf{v}_2 + W) := (\mathbf{v}_1 + \mathbf{v}_2) + W
\]
\[
\alpha \cdot (\mathbf{v} + W) := (\alpha \cdot \mathbf{v}) + W
\]

For example, consider \( V = \mathbb{R}^2 \) and \( W = \operatorname{span}\{\begin{pmatrix}
    1 \\ 1
\end{pmatrix}\} \). Then note that:
\[
\left( \left( \begin{array}{c} 1 \\ 0 \end{array} \right) + W \right)
+ \left( \left( \begin{array}{c} 2  \\ 0 \end{array} \right) + W \right)
=  \left( \begin{array}{c} 3  \\ 0 \end{array} \right) + W 
\]

\[
\pi \cdot \left( \left( \begin{array}{c} 1 \\ 0 \end{array} \right) + W \right)
= \left( \left( \begin{array}{c} \pi \\ 0 \end{array} \right) + W \right)
\]

\noindent {\bf WARNING:} As mentioned in \autoref{rmk:different_expression}, one has different expression of the same element in $V/W$. For instance,
$$\left( \begin{array}{c} 1 \\ 0 \end{array} \right) + W =  \left( \begin{array}{c} 11 \\ 10 \end{array} \right) + W , \quad \quad \left( \begin{array}{c} 2 \\ 0 \end{array} \right) + W =  \left( \begin{array}{c} -98 \\ -100 \end{array} \right) + W $$
by \autoref{prop: coset-equality}. It is unclear from our definition of addition that
$$\left( \left( \begin{array}{c} 1 \\ 0 \end{array} \right) + W \right)
+ \left( \left( \begin{array}{c} 2  \\ 0 \end{array} \right) + W \right) = \left( \left( \begin{array}{c} 11 \\ 10 \end{array} \right) + W \right)
+ \left( \left( \begin{array}{c} -98  \\ -100 \end{array} \right) + W \right)$$
are both equal to $\left( \begin{array}{c} 3  \\ 0 \end{array} \right) + W$ (similar problem also occurs in scalar multiplication). In other words, we do not know if the functions $+:\ V/W \times V/W \to V/W$ given by
$$({\bf v}_1+W, {\bf v}_2+W) \mapsto ({\bf v}_1+{\bf v}_2) +W$$
and $\cdot :\ \mathbb{F} \times V/W \to V/W$ given by
$$(\alpha, {\bf v}+W) \mapsto (\alpha\cdot {\bf v}) +W$$
are {\bf well-defined} \footnote{A function $f:A \to B$ is well-defined if for $a_1 = a_2 \in A$, $f(a_1) = f(a_2) \in B$.}.
\begin{proposition} \label{prop:coset_well_defined}
Addition and scalar multiplication on \( V/W \) is well-defined.
\end{proposition}

\begin{proof}
\textbf{Addition.} Suppose that
\begin{equation}\label{eq: coset-addition}
\left\{
\begin{aligned}
\mathbf{v}_1 + W &= \mathbf{v}_1' + W \\
\mathbf{v}_2 + W &= \mathbf{v}_2' + W
\end{aligned}
\right.
\end{equation}

We need to show that
\[
(\mathbf{v}_1 + \mathbf{v}_2) + W = (\mathbf{v}_1' + \mathbf{v}_2') + W.
\]
From \eqref{eq: coset-addition} and \autoref{prop: coset-equality}, we have:
\[
\mathbf{v}_1 - \mathbf{v}_1' \in W, \quad \mathbf{v}_2 - \mathbf{v}_2' \in W,
\]
which implies
\[
(\mathbf{v}_1 - \mathbf{v}_1') + (\mathbf{v}_2 - \mathbf{v}_2') = (\mathbf{v}_1 + \mathbf{v}_2) - (\mathbf{v}_1' + \mathbf{v}_2') \in W.
\]
By \autoref{prop: coset-equality} again, we conclude:
\[
(\mathbf{v}_1 + \mathbf{v}_2) + W = (\mathbf{v}_1' + \mathbf{v}_2') + W.
\]

\textbf{Multiplication.} For scalar multiplication, similarly, we can show that \( \mathbf{v}_1 + W = \mathbf{v}_1' + W \) implies
\[
\alpha \mathbf{v}_1 + W = \alpha \mathbf{v}_1' + W \quad \text{for all } \alpha \in \mathbb{F}.
\]
\end{proof}

Consequently
$$(V/W, +,\ \cdot)$$
is a vector space, and is called the {\bf quotient space}. In particular, its zero vector is
$${\bf 0}_{V/W} = {\bf 0}_V + W.$$


\begin{proposition}\label{prop:canonical-projection}
The {\bf canonical projection} mapping defined by
\[
\pi_W : V \to V/W, \quad \mathbf{v} \mapsto \mathbf{v} + W
\]
is a surjective linear transformation, with \( \ker(\pi_W) = W \).
\end{proposition}

\begin{proof} We follow a few steps:
\begin{enumerate}
    \item To show the mapping \( \pi_W \) is a linear transformation, note that
    \begin{align*}
    \pi_W(\alpha \mathbf{v}_1 + \beta \mathbf{v}_2) &= (\alpha \mathbf{v}_1 + \beta \mathbf{v}_2) + W \\
    &= (\alpha \mathbf{v}_1 + W) + (\beta \mathbf{v}_2 + W) \\
    &= \alpha (\mathbf{v}_1 + W) + \beta (\mathbf{v}_2 + W) \\
    &= \alpha \pi_W(\mathbf{v}_1) + \beta \pi_W(\mathbf{v}_2).
    \end{align*}
    \item Then we show that \( \ker(\pi_W) = W \):
    \[
    \pi_W(\mathbf{v}) = {\bf 0}_{V/W}\  \Leftrightarrow \ \mathbf{v} + W = 0 + W\ \Leftrightarrow\ \ \mathbf{v} = (\mathbf{v} - 0) \in W.
    \]
    \item Finally, for any \( \mathbf{v}_0 + W \in V/W \), we can construct \( \mathbf{v}_0 \in V \) such that \( \pi_W(\mathbf{v}_0) = \mathbf{v}_0 + W \). Therefore the mapping \( \pi_W \) is surjective.
\end{enumerate}
So the proposition is proved.
\end{proof}

\section{First Isomorphism Theorem}
\begin{proposition}[Universal Property of the Quotient Map I]\label{prop: universal-property-quotient}
Suppose that \( T : V \to U \) is a linear transformation, and that \( W \leq \ker(T) \). Then the mapping
\[
\widetilde{T} : V/W \to U, \quad \mathbf{v} + W \mapsto T(\mathbf{v})
\]
is a well-defined linear transformation. As a result, the following diagram commutes:
\[
\begin{tikzcd}[row sep=large, column sep=large]
V \arrow[r, "\pi_W", blue] \arrow[rd, "T"'] & V/W \arrow[d, "\widetilde{T}"', dashed, red] \\
& U
\end{tikzcd}
\]
In other words, we have \( T = \widetilde{T} \circ \pi_W \).
\end{proposition}

\begin{proof}
First we show the well-definedness. Suppose that \( \mathbf{v}_1 + W = \mathbf{v}_2 + W \), and we suffice to show 
\[
\widetilde{T}(\mathbf{v}_1 + W) = \widetilde{T}(\mathbf{v}_2 + W),
\]
i.e., 
\[
T(\mathbf{v}_1) = T(\mathbf{v}_2).
\]
By \autoref{prop: coset-equality}, we imply
\[
\mathbf{v}_1 - \mathbf{v}_2 \in W \leq \ker(T) \Rightarrow T(\mathbf{v}_1 - \mathbf{v}_2) = 0 \Rightarrow T(\mathbf{v}_1) - T(\mathbf{v}_2) = \mathbf{0}.
\]

Then we show \( \widetilde{T} \) is a linear transformation:
\begin{align*}
\widetilde{T}(\alpha(\mathbf{v}_1 + W) + \beta(\mathbf{v}_2 + W)) 
&= \widetilde{T}((\alpha \mathbf{v}_1 + \beta \mathbf{v}_2) + W) \\
&= T(\alpha \mathbf{v}_1 + \beta \mathbf{v}_2) \\
&= \alpha T(\mathbf{v}_1) + \beta T(\mathbf{v}_2) \\
&= \alpha \widetilde{T}(\mathbf{v}_1 + W) + \beta \widetilde{T}(\mathbf{v}_2 + W)
\end{align*}
\end{proof}

We now study a special case of \autoref{prop: universal-property-quotient}: under the same setting, we let
\( W = \ker(T) \) and $U = T(V) = \mathrm{im}(T)$.
Then one has a linear transformation mapping 
\[
\widetilde{T}: V/\ker{(T)} \to \mathrm{im}(T).
\]
In fact, more is true:
\begin{theorem}[First Isomorphism Theorem]\label{thm: first-isomorphism}
Let \( T: V \to U \) be a linear transformation. Then the mapping
\[
\widetilde{T}: V/\ker(T) \to \mathrm{im}(T), \quad \mathbf{v} + \ker(T) \mapsto T(\mathbf{v})
\]
is an isomorphism.
\end{theorem}

\begin{proof}
\textbf{Injectivity.} Suppose that 
\[
\widetilde{T}(\mathbf{v}_1 + \ker(T)) = \widetilde{T}(\mathbf{v}_2 + \ker(T)),
\]
then we imply
\[
T(\mathbf{v}_1) = T(\mathbf{v}_2) \Rightarrow T(\mathbf{v}_1 - \mathbf{v}_2) = 0_U \Rightarrow \mathbf{v}_1 - \mathbf{v}_2 \in \ker(T),
\]
i.e., 
\[
\mathbf{v}_1 + \ker(T) = \mathbf{v}_2 + \ker(T).
\]

\textbf{Surjectivity.} For \( \mathbf{u} \in U \), due to the surjectivity of \( T \), we can find a \( \mathbf{v}_0 \in V \) such that \( T(\mathbf{v}_0) = \mathbf{u} \). Therefore, we can construct a set \( \mathbf{v}_0 + \ker(T) \) such that
\[
\widetilde{T}(\mathbf{v}_0 + \ker(T)) = \mathbf{u}.
\]
\end{proof}

\begin{example}
Suppose that \( U, W \leq V \) with \( U \cap W = \{ \mathbf{0} \} \), the mapping
\[
\phi : U \oplus W \to U, \quad \phi(\mathbf{u} + \mathbf{w}) = \mathbf{u}.
\]
is a well-defined surjective linear transformation with \( \ker(\phi) = W \). 
\noindent (Exercise: If  \( U \cap W \neq \{ \mathbf{0} \} \), then 
\(\phi : U + W \to U\) given by \(\phi(\mathbf{u} + \mathbf{w}) := \mathbf{u}\) is \textbf{not} well-defined.)

%Suppose that \( \mathbf{0} \neq \mathbf{v} \in U \cap W \), and for any \( \mathbf{u} \in U, \mathbf{w} \in W \), we construct
%\[\mathbf{u}' = \mathbf{u} - \mathbf{v} \in U, \quad \mathbf{w}' = \mathbf{w} + \mathbf{v} \in V \Rightarrow \phi(\mathbf{u}' + \mathbf{w}') = \mathbf{u} - \mathbf{v}.\]
%Therefore we get \( \mathbf{u} + \mathbf{w} = \mathbf{u}' + \mathbf{w}' \), but \( \phi(\mathbf{u} + \mathbf{w}) \neq \phi(\mathbf{u}' + \mathbf{w}') \).

%\medskip

%Back to the situation \( U \cap W = \{ \mathbf{0} \} \), then it’s clear that
By \autoref{thm: first-isomorphism}, the new linear transformation
\[
\widetilde{\phi} : (U \oplus W)/W \to U, \quad \widetilde{\phi}((\mathbf{u} + \mathbf{w}) + W) = {\bf u}.
\]
is an isomorphism of vector spaces.
\end{example}

\section{Universal Property of Quotient Spaces}
\begin{definition}[Universal Property of Quotient Spaces]\label{def:universal-property-quotient}
Let \( V \) be a vector space and let \( V' \leq V \) be a subspace. Consider the collection of linear transformations:
\[
\mathrm{Obj} = \left\{ T : V \to W \;\middle|\;
T \text{ is linear},\ V' \leq \ker(T)
\right\}.
\]
({\it Important example:} the canonical projection \( \pi_{V'} : V \to V/V' \) in \autoref{prop:canonical-projection} belongs to \( \mathrm{Obj} \).)

\medskip
An element \( (\phi : V \to U) \in\mathrm{Obj} \) is said to satisfy the \textbf{universal property} if the following holds: Given any \( T : V \to W \) in \( \mathrm{Obj} \), there exists a unique linear transformation \( \widetilde{T} : U \to W \) such that the following diagram commutes:

\begin{figure}[h!]
\centering
\begin{tikzcd}[row sep=large, column sep=large]
V \arrow[r, "\phi", blue] \arrow[rd, swap, "T"] & U \arrow[d, dashed, red, "\widetilde{T}"] \\
& W
\end{tikzcd}
\label{fig:universal-quotient}
\end{figure}
Equivalently, for any \( T : V \to W \) in \( \mathrm{Obj} \), there exists a unique map \( \widetilde{T} : U \to W \) such that
\[
T = \widetilde{T} \circ \phi.
\]
\end{definition}

\begin{theorem}[Universal Property of Quotient Space II]\label{thm: universal-property-quotient}
Let \( V \) be a vector space and \( V' \leq V \). Then:

\begin{enumerate}
    \item The canonical projection \( \pi_{V'} : V \to V/V' \) is a universal object in the category \( \mathrm{Obj} \), i.e., it satisfies the universal property of \autoref{def:universal-property-quotient}.
    
    \item If \( \phi : V \to U \) is a universal object in \( \mathrm{Obj} \), then \( U \cong V/V' \). In other words, there is intrinsically “one” element (up to isomorphism) in the set of universal objects.
\end{enumerate}
\end{theorem}

\begin{proof} \begin{enumerate}
  \item Consider any linear transformation \( T : V \to W \) such that \( V' \leq \ker(T) \). By \autoref{prop: universal-property-quotient}, there is a linear transformation \( \widetilde{T} : V/V' \to W \) such that
  \[
  T = \widetilde{T} \circ \pi_{V'},
  \]
  i.e., \( \pi_{V'} \) satisfies the commuting diagram in the theorem.

  To show uniqueness of \( \widetilde{T} \), suppose there exists another map \( \widetilde{S} : V/V' \to W \) such that
  \[\widetilde{T} \circ \pi_{V'} = T = \widetilde{S} \circ \pi_{V'}\]
%  \begin{figure}[h!]
%  \centering
%  \begin{tikzcd}[row sep=large, column sep=large]
%  V \arrow[r, "\pi_{V'}", blue] \arrow[rd, "T"'] & V/V' \arrow[d, dashed, "\widetilde{S}", red] \\
%  & W
%  \end{tikzcd}
%  \label{fig:universal-quotient-alt}
%  \end{figure}
  Then for any \( \mathbf{v} + V' \in V/V' \), we compute:
  \[
  \widetilde{S}(\mathbf{v} + V') := \widetilde{S}(\pi_{V'}(\mathbf{v})) = T(\mathbf{v}),
  \]
  and by definition of \( \widetilde{T} \), we also have \( T(\mathbf{v}) = \widetilde{T}(\mathbf{v} + V') \). Hence,
  \[
  \widetilde{S}(\mathbf{v} + V') = \widetilde{T}(\mathbf{v} + V')
  \]
  for all \( \mathbf{v} + V' \in V/V' \), proving $\widetilde{S} = \widetilde{T}$ is unique.

  \item Suppose there exists another \((\phi : V \to U) \in \mathrm{Obj}\) also satisfies the universal property, i.e. for all $(T:V \to W) \in \mathrm{Obj}$, one has:
  
  \begin{figure}[h!]
  \centering
  \begin{tikzcd}[row sep=large, column sep=large]
  V \arrow[r, "\phi", blue] \arrow[rd, "T"] & U \arrow[d, dashed, red, "\widetilde{T}"] \\
  & W
  \end{tikzcd}
  \end{figure}
  
    So one can take $T = \pi_{V'}: V \to V/V'$, i.e.
  \begin{figure}[h!]
  \centering
  \begin{tikzcd}[row sep=large, column sep=large]
  V \arrow[r, "\phi", blue] \arrow[rd, "\pi_{V'}"] & U \arrow[d, dashed, red, "\alpha"] \\
  & V/V'
  \end{tikzcd}
  \end{figure}
  
  for $\alpha = \widetilde{\pi}_{V'}$. Similarly, by the fact that $(\pi_{V'}: V \to V/V') \in \mathrm{Obj}$ is universal, one can take $T = \phi: V \to U$ and get

\begin{figure}[h!]
  \centering
  \begin{tikzcd}[row sep=large, column sep=large]
  V \arrow[r, "\pi_{V'}", blue] \arrow[rd, "\phi"] & V/V' \arrow[d, dashed, red, "\beta"] \\
  & U
  \end{tikzcd}
  \end{figure}

  for $\beta = \widetilde{\phi}$. Combining the two, we obtain the following diagram:

  \begin{figure}[h!]
  \centering
  \begin{tikzcd}[row sep=large, column sep=large]
  V \arrow[r, "\pi_{V'}", blue] \arrow[rd, "\phi", blue] 
  \arrow["\pi_{V'}", rdd] & V/V' \arrow[d, dashed, red, "\beta"] \\
  & U \arrow[d, "\alpha", red, dashed] \\
  & V/V'
  \end{tikzcd}
  \end{figure}

  Therefore, 
  $${\color{red} \alpha \circ \beta} \circ \pi_{V'} = \pi_{V'} = \mathrm{id}_{V/V'} \circ \pi_{V'}.$$ 
\end{enumerate}
By (1), this implies ${\color{red} \alpha \circ \beta} = \mathrm{id}_{V/V'}$. Similarly, one can show that ${\color{red} \beta \circ \alpha} = \mathrm{id}_U$ and hence
$$\alpha: U \to V/V' \quad \quad \beta:V/V' \to U$$
are inverses of each other, i.e.
$$V/V' \cong U.$$
\end{proof}



\chapter{Dual Spaces}
\section{Dual Spaces}
\begin{definition}[Dual Space]\label{def:dual-space}
Let $V$ be a vector space over a field $\mathbb{F}$. The \emph{dual vector space} $V^*$ is defined as
\[
V^* = \mathrm{Hom}_{\mathbb{F}}(V, \mathbb{F}) = \{ f : V \to \mathbb{F} \mid f \text{ is a linear transformation} \}.
\]
\end{definition}

\begin{example}
\leavevmode
\begin{enumerate}
  \item Let $V = \mathbb{R}^n$, and define $\phi_i : V \to \mathbb{R}$ to be the $i$-th coordinate function:
  \[
  \phi_i \begin{pmatrix} x_1 \\ \vdots \\ x_n \end{pmatrix} = x_i.
  \]
  Then clearly $\phi_i \in V^*$. On the contrary, the function
  \[
  \phi_i^2 \begin{pmatrix} x_1 \\ \vdots \\ x_n \end{pmatrix} = x_i^2
  \]
  is not linear, hence $\phi_i^2 \notin V^*$.

  \item Let $V = \mathbb{F}[x]$, and define $\phi : V \to \mathbb{F}$ by $\phi(p(x)) = p(1)$. Then $\phi \in V^*$ because:
  \[
  \phi(ap(x) + bq(x)) = ap(1) + bq(1) = a \phi(p(x)) + b \phi(q(x)).
  \]

  \item Let $\psi : V \to \mathbb{F}$ be defined by $\psi(p(x)) = \int_0^1 p(x)\,dx$. Then $\psi \in V^*$.

  \item Let $V = M_{n \times n}(\mathbb{F})$, and define $\mathrm{tr} : V \to \mathbb{F}$ by $\mathrm{tr}(M) = \sum_{i=1}^n M_{ii}$. Then $\mathrm{tr} \in V^*$. However, the determinant function $\det : V \to \mathbb{F}$ is not linear, hence $\det \notin V^*$.
\end{enumerate}
\end{example}

\subsection{Dual Basis}
\begin{definition}[Dual Basis]\label{def: dual-basis}
Let $V$ be a vector space with basis $\mathcal{B} = \{{\bf v}_i \mid i \in I\}$ (where $I$ may be finite, countable, or uncountable). Define
\[
\mathcal{B}^* = \{ f_i : V \to \mathbb{F} \mid i \in I \},
\]
where the functionals $f_i \in V^*$ satisfies:
\[
f_i({\bf v}_j) = \delta_{ij} = \begin{cases} 1, & \text{if } i = j, \\ 0, & \text{if } i \ne j. \end{cases}
\]
for all ${\bf v}_j \in \mathcal{B}$; and for all ${\bf v} = \sum_{\ell=1}^k \alpha_{\ell} v_{i_{\ell}} \in V$:
\[
f_i\left(\sum_{\ell=1}^k \alpha_{\ell} {\bf v}_{i_{\ell}}\right) = \sum_{\ell=1}^k \alpha_{\ell} f_i({\bf v}_{i_{\ell}}) = \alpha_i.
\]
(c.f. \autoref{rmk:linear_trans_basis}). The natural question is: \textbf{Does the set $\mathcal{B}^*$ form a basis for $V^*$?}
\end{definition}

\begin{proposition}\label{prop:dual-basis}
Let \( B = \{ \mathbf{v}_i \mid i \in I \} \) be a basis of a vector space \( V \), and define the dual set
\[
\mathcal{B}^* = \{ f_i : V \to \mathbb{F} \mid i \in I \}
\]
as in \autoref{def: dual-basis}. Then:
\begin{enumerate}
  \item \( \mathcal{B}^* \) is always linearly independent. That is, any finite subset of \( \mathcal{B}^* \) is linearly independent.
  \item If \( V \) is finite-dimensional, then \( \mathcal{B}^* \) forms a basis of \( V^* \).
\end{enumerate}
\end{proposition}

\begin{proof}
\begin{enumerate}
  \item Let \( f_1, \ldots, f_k \in \mathcal{B}^* \), and suppose there exists a linear dependence:
  \[
  \alpha_1 f_1 + \alpha_2 f_2 + \cdots + \alpha_k f_k = 0_{V^*}.
  \]
  We will show all \( \alpha_i = 0 \). Since each \( f_i \in \mathcal{B}^* \) satisfies \( f_i(\mathbf{v}_j) = \delta_{ij} \), evaluate the left-hand side on \( \mathbf{v}_1 \in B \):
  \[
  \alpha_1 f_1(\mathbf{v}_1) + \alpha_2 f_2(\mathbf{v}_1) + \cdots + \alpha_k f_k(\mathbf{v}_1) = \alpha_1 = 0.
  \]
  Repeating this for \( \mathbf{v}_2, \ldots, \mathbf{v}_k \) gives \( \alpha_2 = \cdots = \alpha_k = 0 \). Thus, \( \{ f_1, \ldots, f_k \} \) is linearly independent.

  \item Suppose \( V \) is finite-dimensional with basis \( B = \{ \mathbf{v}_1, \ldots, \mathbf{v}_n \} \), and let \( \mathcal{B}^* = \{ f_1, \ldots, f_n \} \). For any \( f \in V^* \), define
  \[
  g := \sum_{i=1}^n f(\mathbf{v}_i) f_i \in \mathrm{span}(\mathcal{B}^*).
  \]
  For any \( j = 1, \ldots, n \), we compute:
  \[
  g(\mathbf{v}_j) = \sum_{i=1}^n f(\mathbf{v}_i) f_i(\mathbf{v}_j) = \sum_{i=1}^n f(\mathbf{v}_i) \delta_{ij} = f(\mathbf{v}_j).
  \]
  Hence \( g(\mathbf{v}) = f(\mathbf{v}) \) for all \( \mathbf{v} \in V \), by linearity and since the \( \mathbf{v}_j \)'s form a basis. Therefore, \( f = g \in \mathrm{span}(\mathcal{B}^*) \), and we conclude:
  \[
  V^* = \mathrm{span}(\mathcal{B}^*).
  \]
  Since \( \mathcal{B}^* \) is linearly independent and spans \( V^* \), it forms a basis.
\end{enumerate}
\end{proof}

\begin{corollary}\label{cor: dual-dim-equality}
If \( \dim(V) = n \), then \( \dim(V^*) = n \).
\end{corollary}

\begin{proof}
Define the map
\[
\Phi: V \longrightarrow V^*, \qquad \Phi(\mathbf{v}_i) := f_i,
\]
where \( f_i \in \mathcal{B}^* \) is the dual basis corresponding to the basis \( \mathcal{B} = \{ \mathbf{v}_1, \dots, \mathbf{v}_n \} \) of \( V \) (once again, we use \autoref{rmk:linear_trans_basis} to define $\Phi$). This is a linear isomorphism.

Alternatively, recall that $V^* = \mathrm{Hom}(V,\mathbb{F})$ and hence
$$\dim(V^*) = \dim(V)\dim(\mathbb{F}) = \dim(V) \cdot 1 = \dim(V).$$
\end{proof}

\begin{remark}\label{rem: non-canonical-isomorphism}
The above definition of the isomorphism $\Phi$ requires a specified choice of basis in \( V \), so it is not a \emph{natural isomorphism}.
\end{remark}
\begin{remark}
Part (2) of \autoref{prop:dual-basis} does not hold when \( V \) is infinite-dimensional. That is, the set \( \mathcal{B}^* \) may not span all of \( V^* \), since we only allow finite linear combinations in the definition of span. The following example illustrates a counterexample.
\end{remark}

\begin{example}\label{ex:dual-nonfinite}
Let \( V = \mathbb{F}[x] \), the space of polynomials, and define a basis
\[
B = \{ 1, x, x^2, x^3, \dots \}, \quad \text{so} \quad \mathcal{B}^* = \{ \phi_0, \phi_1, \phi_2, \dots \},
\]
where \( \phi_i : V \to \mathbb{F} \) is defined by evaluation on monomials:
\[
\phi_i(x^j) = \begin{cases}
1, & \text{if } i = j, \\
0, & \text{otherwise}.
\end{cases}
\]

Consider a functional \( \phi \in V^* \) given by:
\[
\phi(p(x)) = p(1),
\]
i.e., \( \phi \) evaluates the polynomial at \( x = 1 \). Then we have:
\[
\phi(1) = 1, \quad \phi(x) = 1, \quad \phi(x^2) = 1, \quad \cdots, \quad \phi(x^n) = 1 \quad \text{for all } n \in \mathbb{N}.
\]

If \( \phi \) were in the span of \( \{ \phi_0, \phi_1, \phi_2, \dots \} \), then it could be written as a finite linear combination:
\[
g := \sum_{n=0}^{\infty} \phi(x^n)\phi_n = \sum_{n=0}^{\infty} \phi_n.
\]
This is a contradiction, since \( \operatorname{span}(\mathcal{B}^*) \) only consists of finite linear combinations of the \( \phi_i \)'s.

\textbf{Conclusion:} If \( V \) is infinite-dimensional, then \( V^* \) strictly contains more elements than \( \operatorname{span}(\mathcal{B}^*) \). In fact, \( \dim(V^*) \) is uncountable, while \( \mathcal{B}^* \) is countable.

Any subspace of a given vector space has some gap with the whole space when considered from the perspective of the dual space. We now turn to describing this more precisely.
\end{example}

\subsection{Annihilators}\label{sec:annihilators}

\begin{definition}[Annihilator]\label{def:annihilator}
Let \( V \) be a vector space and \( S \subseteq V \) a subset. The \emph{annihilator} of \( S \), denoted \( \mathrm{Ann}(S) \), is defined as:
\[
\mathrm{Ann}(S) = \{ f \in V^* \mid f(s) = 0, \, \forall s \in S \}.
\]
\end{definition}

\begin{example}
Let \( V = \mathbb{R}^4 \), with basis \( B = \{ \mathbf{e}_1, \dots, \mathbf{e}_4 \} \), and let the dual basis be \( \mathcal{B}^* = \{ f_1, \dots, f_4 \} \). Consider the subset \( S = \{ \mathbf{e}_3, \mathbf{e}_4 \} \).

Then \( f_1 \in \mathrm{Ann}(S) \), since
\[
f_1(\mathbf{e}_3) = 0, \quad f_1(\mathbf{e}_4) = 0.
\]

Indeed, any linear combination \( a f_1 + b f_2 \in V^* \) such that the coefficients only involve functionals vanishing on \( S \), will be in \( \mathrm{Ann}(S) \).
\end{example}

\begin{proposition}[Properties of Annihilators]\label{prop:annihilator-properties}
Let \( V \) be a vector space. Then:
\begin{enumerate}
  \item The set \( \mathrm{Ann}(S) \subseteq V^* \) is a vector subspace.
  \item The annihilator is \textbf{inclusion-reversing}, i.e., if \( W_1 \subseteq W_2 \subseteq V \), then
  \[
  \mathrm{Ann}(W_1) \supseteq \mathrm{Ann}(W_2).
  \]
  \item The annihilator is \textbf{idempotent}, i.e.,
  \[
  \mathrm{Ann}(S) = \mathrm{Ann}(\operatorname{span}(S)).
  \]
  \item If \( V \) is finite-dimensional and \( W \leq V \), then the annihilator of \( W \) “fills in the gap”:
  \[
  \dim(W) + \dim(\mathrm{Ann}(W)) = \dim(V).
  \]
\end{enumerate}
\end{proposition}

\begin{proof}
\begin{enumerate}
  \item Suppose that \(f,g \in  \operatorname{Ann}\left( S\right)\), i.e., \(f\left( s\right)  = g\left( s\right)  = 0,\forall s \in  S\). It’s clear that \(({af} + {bg}) \in  \operatorname{Ann}\left( S\right)\).

  \item Suppose that \(f \in  \operatorname{Ann}\left( W_2\right)\), we imply \(f\left( \mathbf{w}\right)  = 0\) for any \(\mathbf{w} \in  W_2\). Therefore, \(f\left( {\bf w}_1\right)  = 0\) for any \({\bf w}_1 \in W_1 \subseteq  W_2\), i.e., \(f \in  \operatorname{Ann}\left( W_1\right)\).


  \item Note that \(S \subseteq  \operatorname{span}\left( S\right)\). Therefore we imply \(\operatorname{Ann}\left( S\right)  \supseteq  \operatorname{Ann}\left( {\operatorname{span}\left( S\right) }\right)\). It suffices to show \(\operatorname{Ann}\left( S\right)  \subseteq  \operatorname{Ann}\left( {\operatorname{span}\left( S\right) }\right)\):

  For any \(f \in  \operatorname{Ann}\left( S\right)\) and any \(\sum_{i = 1}^n{k}_{i}{\mathbf{s}}_{i} \in  \operatorname{span}\left( S\right)\), we imply
  \[
  f\left( \sum_{i = 1}^n{k}_{i}{\mathbf{s}}_{i}\right)  = \sum_{i = 1}^n{k}_{i}f\left( {\mathbf{s}}_{i}\right) = \sum_{i = 1}^n{k}_{i} \cdot  0 = 0,
  \]
  i.e., \(f \in  \operatorname{Ann}\left( {\operatorname{span}\left( S\right) }\right)\).

  \item Let \(\left\{  \mathbf{v}_1,\ldots ,\mathbf{v}_{k}\right\}\) be a basis of \(W\). By basis extension, we construct a basis of \(V\):
  \[
  \mathcal{B} = \left\{  \mathbf{v}_1,\ldots ,\mathbf{v}_{k},\mathbf{v}_{k + 1},\ldots ,\mathbf{v}_n\right\}.
  \]
  Let \(\mathcal{B}^* = \left\{  f_1,\ldots ,f_{k},f_{k + 1},\ldots ,f_n\right\}\) be a basis of \(V^{ * }\). We claim that \(\left\{  f_{k + 1},\ldots ,f_n\right\}\) is a basis of \(\operatorname{Ann}\left( W\right)\):
  \begin{enumerate}
    \item \(f_{j}\) are elements in \(\operatorname{Ann}\left( W\right)\) for \(j = k + 1,\ldots ,n\), since for any \(\mathbf{w} = \sum_{i = 1}^{k}\alpha_{i}\mathbf{v}_{i} \in  W\), we have
    \[
    f_{j}\left( \mathbf{w}\right)  = \sum_{i = 1}^{k}\alpha_{i}f_{j}\left( \mathbf{v}_{i}\right) = 0.
    \]
    \item The set \(\left\{  f_{k + 1},\ldots ,f_n\right\}\) is linearly independent.
    \item \(\left\{  f_{k + 1},\ldots ,f_n\right\}\) spans \(\operatorname{Ann}\left( W\right)\): for any \(g \in  \operatorname{Ann}\left( W\right) \leq V^*\), it can be expressed as a linear combination of $\mathcal{B}^*$, i.e. \(g = \sum_{i = 1}^n\beta_{i}f_{i}\). Since $g \in \mathrm{Ann}(W)$
    \[
    0 = g\left( \mathbf{v}_j\right)  = \sum_{i = 1}^n\beta_{i}f_{i}\left( \mathbf{v}_j\right) = \beta_j,
    \]
    for all $1 \leq j \leq k$. We conclude
    \[
    g = \beta_{k + 1}f_{k + 1} + \cdots  + \beta_n f_n \in  \operatorname{span}\left\{ f_{k + 1},\ldots ,f_n \right\}.
    \]
  

  Therefore, \(\left\{  f_{k + 1},\ldots ,f_n\right\}\) forms a basis for \(\operatorname{Ann}\left( W\right)\), i.e., \(\dim \operatorname{Ann}\left( W\right)  = n - k\).

  Let \(W \leq  V\), with \(\dim V < \infty\), then
  \[
  \dim (\operatorname{Ann}(W)) = \dim (V) - \dim (W), \quad \dim((V/W)^*) = \dim(V/W) = \dim (V) - \dim (W).
  \]
\end{enumerate}
\end{enumerate}
So the result follows.
\end{proof}

\section{Adjoint Map}
In the previous section, if $\dim(V) < \infty$, then both
$V/W$ and $\operatorname{Ann}(W)$ has dimensional equal to $\dim(V) - \dim(W)$. One wishes to define a {\it natural} isomorphism between them without specifying any bases.

In fact, the more natural isomorphism is instead:
  \[
  (V/W)^* \cong \operatorname{Ann}(W),
  \]
{\bf and this isomorphism holds even for infinite dimensional vector spaces!}

\medskip
We now construct the natural isomorphisms $\Phi: \operatorname{Ann}(W) \to (V/W)^*$ and 
$\Psi: (V/W)^* \to \operatorname{Ann}(W)$
explicitly, so that 
$$\Phi \circ \Psi = \mathrm{id}_{(V/W)^*} \quad \quad \quad \Psi \circ \Phi = \mathrm{id}_{\operatorname{Ann}(W)}.$$
Firstly, for $\Phi: \operatorname{Ann}(W) \to (V/W)^*$, we need to define $\Phi(f)$
for $f \in \operatorname{Ann}(W) \leq V^*$. In other words, 
\begin{center}
$f: V \to \mathbb{F}$ with $W \leq \ker(f)$. 
\end{center}
By \autoref{prop: universal-property-quotient}, one can define a linear transformation \(\widetilde{f} : V/W \to \mathbb{F}\) such that the diagram commutes:

\begin{figure}[h!]
\centering
\begin{tikzcd}[row sep=large, column sep=large]
V \arrow[r, "\pi", blue] \arrow[rd, swap, "f"] & V/W \arrow[d, dashed, red, "\widetilde{f}"] \\
& \mathbb{F}
\end{tikzcd}
\end{figure}

i.e., \(\widetilde{f}(\mathbf{v} + W) = f(\mathbf{v})\). Then \(\Phi:\operatorname{Ann}(W) \to (V/W)^*\) is defined by
   \[\Phi(f) := \widetilde{f}.\]

\begin{proposition}
The map \( \Phi \) is a linear transformation, i.e.,
\[
\Phi(af + bg) = a \cdot \Phi(f) + b \cdot \Phi(g).
\]
\end{proposition}

\begin{proof}
It suffices to show that
\[
\widetilde{(af + bg)} = a\widetilde{f} + b\widetilde{g}.
\]
\end{proof}

Now we have constructed the natural linear transformation $\Phi$ without specifying any bases. To see whether \( \Phi \) is a bijection, we will construct its inverse \( \Psi \). In order to do so, we need the following:
\begin{definition}[Adjoint Map]\label{def:adjoint-map}
Let \( T : V \to W \) be a linear transformation. Define the \emph{adjoint} of \( T \) by
\[
T^* : W^* \to V^*
\]
such that for any \( f \in W^* \),
\[
[T^*(f)](v) := f(T(v)), \quad \forall v \in V.
\]
\end{definition}

\begin{enumerate}
  \item In other words, \( T^*(f) = f \circ T \), i.e., a linear transformation from \( V \) to \( \mathbb{F} \), so it belongs to \( V^* \).

  \item Moreover, the mapping \( T^* \) itself is a linear transformation. For \( f, g \in W^* \), and for all \( v \in V \),
  \begin{align*}
  [T^*(af + bg)](v) &= (af + bg)[T(v)] \\
  &= a f(T(v)) + b g(T(v)) \\
  &= a [T^*(f)](v) + b [T^*(g)](v) \\
  &= [a T^*(f) + b T^*(g)](v).
  \end{align*}
\end{enumerate}

\begin{proposition}\label{prop: adjoint-injectivity-surjectivity}
Let \( T : V \to W \) be a linear transformation.
\begin{enumerate}
  \item If \( T \) is injective, then \( T^* \) is surjective.
  \item If \( T \) is surjective, then \( T^* \) is injective.
\end{enumerate}
This statement is intuitive, since \( T^* \) reverses the dual of the output into the dual of the input:
\[
T : V \to W, \quad T^* : W^* \to V^*.
\]
\end{proposition}

\begin{proof}
We prove only part (2): If \( T \) is surjective, then \( T^* \) is injective.

Suppose \( g \in W^* \) and \( T^*(g) = 0_{V^*} \). Then, by definition of the adjoint map:
\begin{equation} \label{eq:adjoint-kernel}
[T^*(g)](v) = g(T(v)) = 0, \quad \forall v \in V.
\end{equation}

We want to show \( g = 0_{W^*} \), i.e., \( g(w) = 0 \) for all \( w \in W \).

Since \( T \) is surjective, for every \( w \in W \), there exists \( v' \in V \) such that
\[
w = T(v').
\]
Substituting into \eqref{eq:adjoint-kernel}, we get
\[
g(w) = g(T(v')) = 0.
\]
Hence, \( g = 0_{W^*} \in Ker(T*)\) and \( T^* \) is thus injective.
\end{proof}

The following proposition shows the effect of adjoint map on matrix representations:
\begin{proposition}
Let \( T : V \to W \) be a linear transformation, and let \( \mathcal{A} = \{ \mathbf{v}_1, \dots, \mathbf{v}_n \} \), \( \mathcal{B} = \{ \mathbf{w}_1, \dots, \mathbf{w}_m \} \) be bases of \( V \) and \( W \), respectively. Let \( \mathcal{A}^* = \{ f_1, \dots, f_n \} \), \( \mathcal{B}^* = \{ g_1, \dots, g_m \} \) be the dual bases of \( V^* \) and \( W^* \), respectively. Then the adjoint \( T^* : W^* \to V^* \) has a matrix representation
\[
[T^*]_{\mathcal{A}^*, \mathcal{B}^*} = \left([T]_{\mathcal{B}, \mathcal{A}}\right)^{\mathsf{T}},
\]
where \([T^*]_{\mathcal{A}^*, \mathcal{B}^*} \in \mathbb{R}^{n \times m}\) and \([T]_{\mathcal{B}, \mathcal{A}} \in \mathbb{R}^{m \times n}\).
\end{proposition}

\begin{proof}
Let \([T]_{\mathcal{B}, \mathcal{A}} = (\alpha_{ij})\) and \([T^*]_{\mathcal{A}^*, \mathcal{B}^*} = (\beta_{ij})\). By definition of the matrix representation:

\[
T(\mathbf{v}_j) = \sum_{i=1}^{m} \alpha_{ij} \mathbf{w}_i, \quad T^*(g_j) = \sum_{k=1}^{n} \beta_{kj} f_k \in V^*.
\]

Then for any \( i, j \), we compute:
\[
[T^*(g_j)](\mathbf{v}_j) = g_j(T(\mathbf{v}_j)) = g_j\left( \sum_{\ell=1}^{m} \alpha_{\ell j} \mathbf{w}_\ell \right) = \sum_{\ell=1}^{m} \alpha_{\ell j} g_j(\mathbf{w}_\ell) = \alpha_{ij}.
\]

On the other hand:
\[
[T^*(g_j)](\mathbf{v}_j) = \left( \sum_{k=1}^{n} \beta_{kj} f_k \right)(\mathbf{v}_j) = \sum_{k=1}^{n} \beta_{kj} f_k(\mathbf{v}_j) = \beta_{ji}.
\]

Therefore, we conclude that \( \beta_{ij} = \alpha_{ij} \), so \( [T^*] = [T]^{\mathsf{T}} \).
\end{proof}

\subsection{Relationship between Annihilator and Dual of Quotient Spaces}

\begin{example}
Consider the canonical projection mapping \( \pi_W : V \to V/W \) with its adjoint mapping:
\[
(\pi_W)^* : (V/W)^* \to V^*.
\]
To understand \( (\pi_W)^* \):
\begin{enumerate}
  \item Take \( h \in (V/W)^* \) and study \( (\pi_W)^*(h) \in V^* \).
  \item Take \( {\bf v} \in V \) and compute:
  \[
  \left[(\pi_W)^*(h)\right]({\bf v}) = h(\pi_W({\bf v})) = h({\bf v} + W).
  \]
\end{enumerate}
In particular, for all \( {\bf w} \in W \leq V \), we have:
  \[
  \left[(\pi_W)^*(h)\right]({\bf w}) = h({\bf w} + W) = h({\bf 0}_{V/W}) = 0_{\mathbb{F}}.
  \]
  Therefore, 
  \[
  (\pi_W)^*(h) \in \operatorname{Ann}(W).
  \]
  i.e., 
  \[(\pi_W)^*: (V/W)^* \to \operatorname{Ann}(W).\]
\end{example}

So $\Psi = (\pi_W)^*$ is our candidate of the inverse of $\Phi$. 


\begin{theorem}
\[
\Phi \circ \Psi = \operatorname{id}_{(V/W)^*}, \quad \Psi \circ \Phi = \operatorname{id}_{\operatorname{Ann}(W)}.
\]
Consequently, one has a natural isomorphism $(V/W)^* \cong \operatorname{Ann}(W)$ regardless of whether $\dim(V) < \infty$ or not.
\end{theorem}
\begin{proof}
    Recall that for $f \in \mathrm{Ann}(W)$, $\Phi(f) \in (V/W)^*$ satisfies:
    $$\Phi(f)({\bf v}+W) = f({\bf v}),$$ 
    while for $h \in (V/W)^*$, $\Psi(h) \in \mathrm{Ann}(W)$ satisfies
    $$\Psi(h)({\bf v}) = h({\bf v}+W).$$

    Therefore, for all ${\bf v} +W \in V/W$,
    $$(\Phi \circ \Psi(h))({\bf v} + W) = \Psi(h)({\bf v}) = h({\bf v}+W)$$
    and hence $\Phi \circ \Psi(h) = h$ for all $h \in (V/W)^*$, i.e. $\Phi \circ \Psi = \mathrm{id}_{(V/W)^*}$.

    On the other hand, for all ${\bf v} \in V$,
    $$(\Psi \circ \Phi(f))({\bf v}) = \Phi(f)({\bf v} + W) = f({\bf v}),$$
    which implies $\Psi \circ \Phi(f) = f$ for all $f \in \mathrm{Ann}(W)$. Consequently, $\Psi \circ \Phi = \mathrm{id}_{\mathrm{Ann}(W)}$ and the result follows.
\end{proof}

%my idea: \( \Phi : \operatorname{Ann}(W) \rightarrow (V/W)^*\)
%\begin{enumerate}    \item Since \((\pi_W)*:(V/W)^* \rightarrow \operatorname{Ann}(W)\) is an injection, we can define \(\Phi\) as its inverse, which is a surjection and allow the left identity to hold.
%    \item Similar to \[  \left[(\pi_W)^*(h)\right](w) = h(w + W) \], we see for \(f \in \operatorname{Ann}(W)\) and \(v + W \in V/W\), we define:\[\left[\Phi(f)\right](v + W) = f(v).\]
%   \item We can now show the left identity, \(h \in (V/W)^*\) and \(v\in V\)  \[\Phi((\pi_W)^*)(h)(v) = \Phi(h)(v + W) = h(v)\]
% \item And the right: \[((\pi_W)^* \circ \Phi) (f)(v) = (\pi_W)^*(f)(v) = f(v+W) = f(v) \] for \(h \in \operatorname{Ann}(W) \text{ and } v \in W\)
%\end{enumerate}

%I think there is problem in my idea since I inputted \(v \in V\) into \(h: V/W \rightarrow \mathbb{F}\) and \(v + W  \in V/W\) into \(f: W \rightarrow \{0\}\). That is mixing coset and vectors. But, i cant figure out another way to show this.





\chapter{Polynomials, Eigenvalues and Eigenvectors}
\section{Basics of Polynomials}

We recall some useful properties of polynomials before studying eigenvalues and eigenvectors.

\begin{definition}[Polynomial]\label{def:polynomial}
\leavevmode
\begin{enumerate}
  \item A polynomial over a field \(\mathbb{F}\) has the form
  \[
  p(z) = a_m z^m + \cdots + a_1 z + a_0, \quad \text{with } a_m \neq 0.
  \]
  Here \(a_m z^m\) is called the \emph{leading term} of $p(z)$; \(\deg(p(z)) =m\) the \emph{degree} of $p(z)$; \(a_m\) the \emph{leading coefficient} of $p(z)$; and \(a_0, \dots, a_m\) the \emph{coefficients} of $p(z)$.

  \item A polynomial over \(\mathbb{F}\) is said to be \emph{monic} if its leading coefficient is \(1_\mathbb{F}\).

  \item A polynomial \(p(z) \in \mathbb{F}[z]\) is \emph{irreducible} if whenever
  \[
  p(z) = a(z) b(z)
  \]
  for \(a(z), b(z) \in \mathbb{F}[z]\), then either \(\deg(a(z)) = 0\) or \(\deg(b(z)) = 0\) (i.e. $a(z)$ or $b(z)$ is a constant polynomial). Otherwise, \(p(z)\) is \emph{reducible}.
\end{enumerate}
\end{definition}

\begin{example}\label{ex:irreducible}
The polynomial \(p(x) = x^2 + 1\) is irreducible over \(\mathbb{R}\), but reducible over \(\mathbb{C}\) since
\[
p(x) = (x - i)(x + i).
\]
\end{example}

\subsection{Division Algorithm}
\begin{theorem}[Division Theorem]\label{thm:division}
Let \(p, q \in \mathbb{F}[z]\) with \(q \neq 0\). Then there exist unique \(s, r \in \mathbb{F}[z]\) such that
\[
p(z) = s(z) \cdot q(z) + r(z), \quad \deg(r) < \deg(q).
\]
Here \(r(z)\) is called the \emph{remainder}.
\end{theorem}

\begin{example}\label{ex:division}
Let \(p(x) = x^4 + 1\) and \(q(x) = x^2 + 1\). Then, using basic algebra,
\[
x^4 + 1 = (x^2 - 1)(x^2 + 1) + 2.
\]
\end{example}

\begin{theorem}[Root Theorem]\label{thm:root}
Let \(p(x) \in \mathbb{F}[x]\), and \(\lambda \in \mathbb{F}\). Then \(x - \lambda\) divides \(p(x)\) if and only if \(p(\lambda) = 0\).
\end{theorem}

\begin{proof}
\leavevmode
\begin{enumerate}
  \item[\((\Rightarrow)\)] If \(x - \lambda\) divides \(p\), then \(p = (x - \lambda) q\) for some \(q \in \mathbb{F}[x]\). Substituting \(\lambda\) yields
  \[
  p(\lambda) = (\lambda - \lambda) q(\lambda) = 0.
  \]

  \item[\((\Leftarrow)\)] Suppose \(p(\lambda) = 0\). By the Division Theorem, there exist \(s, r \in \mathbb{F}[x]\) such that
  \[
  p(x) = (x - \lambda)s(x) + r(x), \quad \deg(r) < \deg(x - \lambda) = 1. \tag{6.1}
  \]
  Therefore, \(r(x) = r\) is constant. Plugging in \(x = \lambda\),
  \[
  0 = p(\lambda) = 0 \cdot s(\lambda) + r \quad \Rightarrow \quad r = 0.
  \]
  Hence \(p(x) = (x - \lambda)s(x)\), and \(x - \lambda\) divides \(p(x)\).
\end{enumerate}
\end{proof}


\begin{corollary}[Root Bound]
A polynomial with degree \(n\) has at most \(n\) roots, counting multiplicity.
\end{corollary}

\begin{example}
The polynomial \((x - 3)^2\) has one root \(x = 3\) with multiplicity 2. When counting multiplicity, we say it has two roots.
\end{example}

\subsection{Unique Factorization}
\begin{definition}[Algebraically Closed Field]
A field \(\mathbb{F}\) is called \emph{algebraically closed} if every non-constant polynomial \(p(x) \in \mathbb{F}[x]\) has a root \(\lambda \in \mathbb{F}\).
\end{definition}

\begin{theorem}[Fundamental Theorem of Algebra] \label{thm:fundamental_theorem_of_algebra}
The field \(\mathbb{C}\) of complex numbers is algebraically closed.
\end{theorem}

\begin{proof}[Sketch]
One way to prove this is via complex analysis; another way is using the topology of \(\mathbb{C} \setminus \{0\}\).
\end{proof}

By induction, we can show that every polynomial of degree \(n\) over an algebraically closed field \(\mathbb{F}\) has exactly \(n\) roots, counting multiplicity. Therefore, any polynomial \(p(x)\) over such a field can be factorized as
\begin{equation}\label{eq:root-factorization}
p(x) = c(x - \lambda_1)\cdots(x - \lambda_n)
\end{equation}
for some \(c, \lambda_1, \dots, \lambda_n \in \mathbb{F}\).

Polynomials over general fields \(\mathbb{F}\) may not factor as in \eqref{eq:root-factorization}, but they still satisfy a unique factorization property. To begin with:
\begin{definition}[Divisibility]
Let \(p(x), q(x), s(x) \in \mathbb{F}[x]\) with \(p(x) = q(x) s(x)\). Then:
\begin{itemize}
    \item \(p(x)\) is divisible by \(s(x)\),
    \item \(s(x)\) is a factor of \(p(x)\),
    \item \(s(x) \mid p(x)\),
    \item \(s(x)\) divides \(p(x)\),
    \item \(p(x)\) is a multiple of \(s(x)\).
\end{itemize}
\end{definition}

\begin{definition}[Irreducible Polynomials]
A polynomial $f(x) \in \mathbb{F}[x]$ is {\it irreducible} if whenever
$$s(x) | f(x),$$
then either $s(x) = c$ is a constant or $s(x) = cf(x)$ is a scalar multiple of $f(x)$.
\end{definition}

\begin{theorem}[Unique Factorization]
Every \(f(x) = a_n x^n + \cdots + a_0 \in \mathbb{F}[x]\) can be uniquely written as
\[
f(x) = a_n [p_1(x)]^{e_1} \cdots [p_k(x)]^{e_k}
\]
where each \(p_i\) is monic, irreducible, and \(p_i \neq p_j\) for \(i \neq j\). This factorization is unique up to the ordering of the factors.
\end{theorem}



\subsection{Greatest Common Divisor and B\'ezout's Theorem}
\begin{definition}[Common Factor and GCD] \label{def:gcd}
Let \(f_1, \dots, f_k \in \mathbb{F}[x]\). A polynomial \(g(x)\) is a \emph{common factor} of these if \(g \mid f_i\) for all \(i\).

The \emph{greatest common divisor {GCD}} \(g = \gcd(f_1, \dots, f_k)\) of $f_1, \dots, f_k$ satisfies:
\begin{itemize}
    \item \(g\) is monic,
    \item \(g\) is a common factor of \(f_1, \dots, f_k\),
    \item \(g\) has maximal degree among all common factors.
\end{itemize}
If \(\gcd(f_1, \dots, f_k) = 1\), we say the polynomials are \emph{relatively prime}.
\end{definition}
With the definition above, it is easy to check that \(\gcd(f_1, \dots, f_k)\) is unique, and 
\[\gcd(f_1, \dots, f_k) = \gcd(\gcd(f_1, f_2), f_3, \dots, f_k) = \gcd(\gcd(\gcd(f_1, f_2), f_3), \dots, f_k) = \cdots.\]


\begin{remark}
Polynomials \(f_1, \dots, f_k\) being relatively prime does not imply \(\gcd(f_i, f_j) = 1\) for all \(i \neq j\).
\end{remark}

\begin{example}[Counter-example]
Let \(a_1, \dots, a_n\) be distinct irreducible polynomials, and define
\[
f_i(x) := a_1(x) \cdots \widehat{a_i(x)} \cdots a_n(x),
\]
where the hat means \(a_i(x)\) is omitted. Then \(\gcd(f_1, \dots, f_n) = 1\), but
\[
\gcd(f_i, f_j) = a_1 \cdots \widehat{a_i} \cdots \widehat{a_j} \cdots a_n
\]
which may not be 1.
\end{example}

\begin{example}
Let 
\[
f_1(x) = (x^2 + x + 1)^3 (x - 3)^2 x^4, \quad
f_2(x) = (x^2 + 1)(x - 3)^4 x^2 \in \mathbb{R}[x],
\]
then
\[
\gcd(f_1, f_2) = (x - 3)^2 x^2.
\]
\end{example}


As mentioned in the paragraph after \autoref{def:gcd}, \(\gcd(f_1, f_2, f_3, \cdots,f_k)\) can be computed inductively by understanding
$$\gcd(f_1, f_2), \gcd((\gcd(f_1,f_2),f_3)), \gcd(\gcd((\gcd(f_1,f_2),f_3)),f_4), \cdots$$
So it suffices to study the greatest common divisor of two polynomials. In such a case $\gcd(f_1, f_2)$ can be computed using {\bf Euclidean Algorithm} (see Homework set).

\medskip
One important theorem relating $f_1, f_2$ and $\gcd(f_1,f_2)$ is the following:
\begin{theorem}[Bézout's Identity]\label{thm:bezout}
Let \(g = \gcd(f_1, f_2)\). Then there exist polynomials \(r_1, r_2 \in \mathbb{F}[x]\) such that
\[
g(x) = r_1(x) f_1(x) + r_2(x) f_2(x).
\]
More generally, for \(g = \gcd(f_1, \ldots, f_k)\), there exist \(r_1, \ldots, r_k \in \mathbb{F}[x]\) such that
\[
g(x) = r_1(x) f_1(x) + \cdots + r_k(x) f_k(x).
\]
\end{theorem}

\begin{example}
Consider the polynomials \(f_1(x) = x^3 + 6x + 7\) and \(f_2(x) = x^2 + 3x + 2\). Applying the Euclidean algorithm:

\begin{align*}
f_1(x) - (x - 3)f_2(x) &= x^3 + 6x + 7 - (x - 3)(x^2 + 3x + 2) \\
&= x^3 + 6x + 7 - (x^3 + 3x^2 + 2x - 3x^2 - 9x - 6) \\
&= 13x + 13.
\end{align*}

Next,
\begin{align*}
f_2(x) - \frac{x + 2}{13}(13x + 13) &= x^2 + 3x + 2 - (x + 2)(x + 1) \\
&= 0.
\end{align*}

Therefore,
\[
\gcd(f_1, f_2) = \gcd(f_2, 13x + 13) = x + 2.
\]
\end{example}

By applying \autoref{thm:bezout} repeatedly, one has the following:
\begin{corollary} \label{cor:bezout}
    Let $g = \mathrm{gcd}(f_1, \dots, f_k)$. Then there exists polynomials $r_1, \dots, r_k \in \mathbb{F}[x]$ such that
    $$g(x) = r_1(x)f_1(x) + \dots + r_k(x)f_k(x).$$
\end{corollary}


\section{Eigenvalues \& Eigenvectors}

\begin{definition}[Eigenvalue and Eigenspace]\label{def:eigenvalue}
Let \( T : V \to V \) be a linear operator.

\begin{enumerate}
  \item A nonzero vector \( {\bf v} \in V \setminus \{ \mathbf{0} \} \) is called an \emph{eigenvector} of \(T\) with eigenvalue \(\lambda\) if
  \[
  T({\bf v}) = \lambda {\bf v}.
  \]
  
  \item Equivalently, \({\bf v} \in \ker(T - \lambda I)\), where \(I = \mathrm{id} : V \to V\) is the identity map, i.e. \(I({\bf v}) = {\bf v}\) for all \({\bf v} \in V\). The subspace \(\ker(T - \lambda I)\) is called the \(\lambda\)-\emph{eigenspace} of \(T\).
\end{enumerate}
\end{definition}

\begin{definition}[Generalized Eigenvector]\label{def:generalized-eigen}
A nonzero vector \({\bf v} \in V\) is a \emph{generalized eigenvector} of \(T\) with generalized eigenvalue \(\lambda\) if
\[
{\bf v} \in \ker\left( (T - \lambda I)^k \right)
\quad \text{for some } k \in \mathbb{N}_{>0}.
\]
\end{definition}

\begin{remark}
Every eigenvector is a generalized eigenvector, but the converse does not necessarily hold.
\end{remark}

\begin{example}
Let \( A : \mathbb{R}^2 \to \mathbb{R}^2 \) be the linear transformation defined by matrix 
\(A = \begin{pmatrix}
1 & 1 \\
0 & 1
\end{pmatrix}.
\).
\begin{enumerate}
  \item The vector \( \begin{pmatrix} 1 \\ 0 \end{pmatrix} \) is an eigenvector with eigenvalue \(1\), since
  \(
  A \begin{pmatrix} 1 \\ 0 \end{pmatrix} = \begin{pmatrix} 1 \\ 0 \end{pmatrix}.
  \).
  
  \item The vector \( \begin{pmatrix} 0 \\ 1 \end{pmatrix} \) is not an eigenvector of eigenvalue $1$, since
  \(
  A \begin{pmatrix} 0 \\ 1 \end{pmatrix} = \begin{pmatrix} 1 \\ 1 \end{pmatrix}.
  \)
  However, we compute:
  \[
  (A - I)^2 = \begin{pmatrix} 0 & 1 \\ 0 & 0 \end{pmatrix}^2 = \begin{pmatrix} 0 & 0 \\ 0 & 0 \end{pmatrix},
  \]
  so
  \[
  \begin{pmatrix} 0 \\ 1 \end{pmatrix} \in \ker((A - I)^2),
  \]
  i.e., it is a generalized eigenvector with generalized eigenvalue \(1\).
\end{enumerate}
\end{example}

\begin{example}
Let \( V = C^\infty(\mathbb{R}) \) be the space of infinitely differentiable functions. Define the linear operator
\[
T(f) = f''.
\]
Then \( f \in V \) is a \((-1)\)-eigenvector of \(T\) if and only if \( f'' = -f \). From ODE theory, the general solution is
\[
f(x) = a \sin x + b \cos x,
\]
so the \((-1)\)-eigenspace is spanned by \( \{ \sin x, \cos x \} \).
\end{example}

From now on, we assume that \(V\) is finite-dimensional unless otherwise stated.

\begin{definition}[Determinant of a Linear Operator]\label{def:determinant}
Let \( T : V \to V \) be a linear operator. The determinant of \(T\) is defined as
\[
\det(T) = \det(T_{\mathcal{A}, \mathcal{A}}),
\]
where \(T_{\mathcal{A}, \mathcal{A}}\) is the matrix representation of \(T\) with respect to {\bf any} choice of basis \(\mathcal{A}\) of \(V\).
\end{definition}

\begin{remark}
The determinant is independent of the choice of basis. For any other basis \(\mathcal{B}\), we have the following from \autoref{sec:similar_basis}:
\[
\det(T_{\mathcal{B}, \mathcal{B}}) = 
\det(C_{\mathcal{B}, \mathcal{A}} T_{\mathcal{A}, \mathcal{A}} C_{\mathcal{A}, \mathcal{B}}) 
= \det(C_{\mathcal{A}, \mathcal{B}}^{-1}) \cdot \det(T_{\mathcal{A}, \mathcal{A}}) \cdot \det(C_{\mathcal{A}, \mathcal{B}}) 
= \det(T_{\mathcal{A}, \mathcal{A}}).
\]
\end{remark}

\begin{definition}[Characteristic Polynomial]\label{def:char-poly}
Let \( T : V \to V \) be a linear operator. The \emph{characteristic polynomial} \( \mathcal{X}_T(x) \in \mathbb{F}[x] \) of \(T\) is defined by
\[
\mathcal{X}_T(x) := \det\left( T_{\mathcal{A},\mathcal{A}} - x I \right),
\]
where \(T_{\mathcal{A},\mathcal{A}}\) is the matrix representation of \(T\) with respect to any basis \(\mathcal{A}\) of \(V\), and \(I\) is the identity matrix of the same size.
\end{definition}

\begin{remark}
    The above definition is independent of the choice of basis of $V$. Indeed, let $\mathcal{B}$ be another basis of $V$, then 
    $$T_{\mathcal{B},\mathcal{B}} = M^{-1}T_{\mathcal{A},\mathcal{A}}M$$
    for $M := C_{\mathcal{A},\mathcal{B}}$. As a result, one has:
    $$\det(T_{\mathcal{B},\mathcal{B}} - xI) = \det(M^{-1}T_{\mathcal{A},\mathcal{A}}M - xM^{-1}IM) = \det(M)^{-1}\det(T_{\mathcal{A},\mathcal{A}} - xI)\det(M) = \det(T_{\mathcal{A},\mathcal{A}} - xI).$$
\end{remark}

In the next couple of chapters, we will study two fundamental theorems that reveal the internal structure of linear operators:

\begin{itemize}
  \item Cayley–Hamilton Theorem
  \item Jordan Canonical Form
\end{itemize}

These results are usually stated in terms of matrices, but they are truly theorems about linear operators—independent of the choice of basis. Our goal is to understand them in the more general setting of finite-dimensional vector spaces, beyond just \( \mathbb{R}^n \) or \( \mathbb{C}^n \).

\begin{remark}
Why do we care about these theorems?

\begin{enumerate}
  \item The \textbf{Cayley–Hamilton Theorem} tells us that every linear operator satisfies its own characteristic polynomial. This bridges algebraic information (the determinant and characteristic polynomial) with operator action.

  \item The \textbf{Jordan Canonical Form} provides a refined classification of linear operators up to similarity. It allows us to break down a complicated operator into blocks that reveal both eigenvalues and the structure of generalized eigenspaces.

  \item Understanding these theorems gives insight into the “shape” of an operator—how it stretches, rotates, or shears a space—and how we can simplify its action by choosing the right basis.
\end{enumerate}
\end{remark}
\chapter{Minimal Polynomial \& Cayley-Hamilton Theorem}
\section{ Minimal Polynomial}

\begin{definition}[Linear Operator Induced From Polynomial]
Let \( f(x) := a_m x^m + \cdots + a_0 \in \mathbb{F}[x] \), and let \( T : V \to V \) be a linear operator. Then the operator
\[
f(T) := a_m T^m + \cdots + a_1 T + a_0 I : V \to V
\]
is called the linear operator induced by the polynomial \( f(x) \).
\end{definition}

The composition of linear operators is not commutative, i.e. \( S \circ T \neq T \circ S \) in general. This follows similarly to the fact that matrix multiplication is not commutative. However, we always have
  \[
  f(T) T = T f(T),
  \]
for any polynomial \( f \in \mathbb{F}[x] \) - Indeed, we can show that \( T^n T = T T^n \) for all \( n \) by induction. Now for \( f(x) = \sum_i a_i x^i \), then
  \[
  f(T) T = \sum_i a_i T^i T = \sum_i a_i T T^i = T \sum_i a_i T^i = T f(T). 
  \]
More generally, one can apply induction argument again to show that the composition of polynomial-induced operators is commutative:
  \[
  f(T) g(T) = g(T) f(T),
  \]
for all \( f(x), g(x) \in \mathbb{F}[x] \).

\begin{definition}[Minimal Polynomial]
Let \( T : V \to V \) be a linear operator. The \emph{minimal polynomial} \( m_T(x) \) is the unique monic polynomial of least degree such that
\[
m_T(T) = \mathbf{0}_{V \to V},
\]
where \( \mathbf{0}_{V \to V} \) denotes the zero operator in \( \operatorname{Hom}(V, V) \).
\end{definition}

\begin{example}[Minimal Polynomials of Matrices]
\leavevmode
\begin{enumerate}
  \item Let \( A = \begin{pmatrix} 1 & 0 \\ 0 & 1 \end{pmatrix} \), then
  \[
  \mathcal{X}_A(x) = (x - 1)^2, \quad m_A(x) = x - 1,
  \]
  since \( A - I = 0 \).

  \item Let \( B = \begin{pmatrix} 1 & 1 \\ 0 & 1 \end{pmatrix} \), then
  \[
  \mathcal{X}_B(x) = (x - 1)^2.
  \]
  Could the minimal polynomial be \( x - 1 \)? No, because
  \[
  B - I = \begin{pmatrix} 0 & 1 \\ 0 & 0 \end{pmatrix} \neq 0.
  \]
  However,
  \[
  (B - I)^2 = \begin{pmatrix} 0 & 0 \\ 0 & 0 \end{pmatrix} \Rightarrow m_B(x) = (x - 1)^2.
  \]
\end{enumerate}
\end{example}

Two natural questions arise:
\begin{enumerate}
  \item Does \( m_T(x) \) always exist? Is it unique?
  \item What is the relationship between \( m_T(x) \) and \( \mathcal{X}_T(x) \)?
\end{enumerate}

Regarding the existence of minimal polynomial, note that $m_T(x)$ may not exist if \( V \) is infinite-dimensional:
\begin{example}
Let \( V = \mathbb{R}[x] \), and define
\[
T : V \to V, \quad p(x) \mapsto \int_0^x p(t)\, dt.
\]
Then, for monomials: \( T(x^n) = \frac{1}{n+1} x^{n+1} \). Suppose \( m_T(x) = x^n + \cdots + a_0 \), then
\[
m_T(T) = T^n + \cdots + a_0 I = 0.
\]
Applying this to \( x \in V \), we would have:
\[
m_T(T)(x) = \frac{1}{n!} x^n + a_{n-1} \frac{1}{(n-1)!} x^{n-1} + \cdots + a_0 x = 0.
\]
But this is a contradiction since the LHS is a nonzero polynomial while the RHS is zero. Hence, no such minimal polynomial exists.
\end{example}

\subsection{ Properties of the Minimal Polynomial}

\begin{proposition}[Existence]\label{prop:minpoly-existence}
The minimal polynomial \( m_T(x) \) always exists if \( \dim(V) = n < \infty \).
\end{proposition}

\begin{proof}
We consider the set
\[
\{ I, T, T^2, \dots, T^{n^2} \} \subseteq \operatorname{Hom}(V, V),
\]
which lies in a vector space $\mathrm{Hom}(V,V)$ of dimension \( n^2 \). Therefore, this set is linearly dependent, and there exist scalars \( a_0, \dots, a_{n^2} \), not all zero, such that
\[
a_0 I + a_1 T + \cdots + a_{n^2} T^{n^2} = 0.
\]
That is, there exists a nonzero polynomial \( g(x) \in \mathbb{F}[x] \) with \( \deg(g) < n^2 \) such that \( g(T) = 0 \). Thus a minimal-degree such polynomial exists.
\end{proof}

\begin{proposition}[Uniqueness]\label{prop:minpoly-uniqueness}
The minimal polynomial \( m_T(x) \), if it exists, is unique.
\end{proposition}

\begin{proof}
Suppose \( f_1(x) \neq f_2(x) \) are two minimal polynomials with the same minimal degree. Then their difference
\[
g(x) = f_1(x) - f_2(x) \neq 0,
\]
satisfies \( \deg(g) < \deg(f_1) \) and
\[
g(T) = f_1(T) - f_2(T) = 0.
\]
After scaling \( g(x) \) to be monic, we obtain a polynomial of strictly lower degree annihilating \( T \), contradicting the minimality of \( f_1 \). Hence \( m_T(x) \) is unique.
\end{proof}

\begin{proposition}[Divisibility]\label{prop:minpoly-divides}
Let \( f(x) \in \mathbb{F}[x] \) satisfy \( f(T) = 0 \). Then
\[
m_T(x) \mid f(x).
\]
\end{proposition}

\begin{proof}
By the division algorithm, write
\[
f(x) = q(x) m_T(x) + r(x), \quad \deg(r) < \deg(m_T).
\]
Then
\[
f(T) = q(T) m_T(T) + r(T) = 0 + r(T) = r(T).
\]
Since \( f(T) = 0 \), we get \( r(T) = 0 \). But then \( r(x) \) is a polynomial of smaller degree than \( m_T \) that annihilates \( T \), contradicting the minimality of \( m_T \). Thus \( r(x) \equiv 0 \), and \( m_T(x) \mid f(x) \).
\end{proof}

\begin{proposition}[Similarity Invariance]\label{prop:minpoly-similarity}
Let \( A, B \in \mathbb{F}^{n \times n} \) be similar, i.e., \( B = P^{-1} A P \) for some invertible matrix \( P \). Then
\[
m_A(x) = m_B(x).
\]
\end{proposition}

\begin{proof}
Let \( m_A(x) = x^k + a_{k-1}x^{k-1} + \cdots + a_0 \). Then
\[
m_A(B) = m_A(P^{-1} A P) = P^{-1} m_A(A) P = P^{-1}(0)P = 0.
\]
Hence \( m_A(x) \) annihilates \( B \), so by \autoref{prop:minpoly-divides}, \( m_B(x) \mid m_A(x) \). A symmetric argument shows \( m_A(x) \mid m_B(x) \). Since both are monic, we conclude
\[
m_A(x) = m_B(x).
\]
\end{proof}

\begin{remark}
\autoref{prop:minpoly-similarity} states that the minimal polynomial is invariant under similarity. In fact, the characteristic polynomial is similarity-invariant as well.
\end{remark}

% From now on we assume finite-dimensionality.
% Define standard notation for minimal polynomial.

We now assume \( \dim(V) < \infty \). The minimal polynomial \( m_T(x) \) denotes the unique monic polynomial of least degree such that
\[
m_T(T) = 0 \in \operatorname{Hom}(V, V).
\]



\section{Cayley–Hamilton Theorem}
\subsection{Statement of the Theorem}
We begin with an example: consider the matrix \( A = \begin{pmatrix} 1 & 0 \\ 0 & 2 \end{pmatrix} \), and its associated linear operator on \( \mathbb{F}^2 \). The characteristic polynomial of \( A \) is
\[
\chi_A(x) = (x - 1)(x - 2).
\]
Note that the minimal polynomial \( m_A(x) \) cannot have degree one. Otherwise, \( m_A(x) = x - k \) for some scalar \( k \), and we would have
\[
m_A(A) = A - k I = \begin{pmatrix} 1 - k & 0 \\ 0 & 2 - k \end{pmatrix} \neq 0,
\]
for any \( k \in \mathbb{F} \), contradicting the definition of minimal polynomial. Therefore, \( m_A(x) \) must be
\[
m_A(x) = (x - 1)(x - 2),
\]
which coincides with the characteristic polynomial. In fact, one has:

\begin{theorem}[Cayley–Hamilton]\label{thm:cayley-hamilton}
Let \( T : V \to V \) be a linear operator on a finite-dimensional vector space over a field \( \mathbb{F} \). Then the characteristic polynomial \( \mathcal{X}_T(x) \) annihilates \( T \), i.e.,
\[
\mathcal{X}_T(T) = 0.
\]
Equivalently, the minimal polynomial \( m_T(x) \) divides the characteristic polynomial:
\[
m_T(x) \mid \mathcal{X}_T(x).
\]
\end{theorem}

In the course of the proof of \autoref{thm:cayley-hamilton}, we would assume 
$\mathcal{X}_T(x) = (x - \lambda_1)^{e_1} \cdots (x-\lambda_k)^{e_k}$
can be factorized into linear terms. However, this is not always possible over arbitrary \( \mathbb{F} \). For instance, consider
\[
A = \begin{pmatrix} 0 & -1 \\ 1 & 0 \end{pmatrix} \in M_{2 \times 2}(\mathbb{R}).
\]
Its characteristic polynomial is \( \mathcal{X}_A(x) = x^2 + 1 \), which cannot be factorized into linear terms in \( \mathbb{R}[x] \). However, we may always extend the field to its {\bf algebraic closure} 
\[ \overline{\mathbb{F}} \supseteq \mathbb{F} \] (e.g. $\mathbb{C} \supseteq \mathbb{R}$) so that \( \mathcal{X}_A(x) \in \mathbb{F}[x] \subseteq \overline{\mathbb{F}}[x]\) always admits the factorization (e.g. $\mathcal{X}_A(x) = (x - i)(x +1)$ in $\mathbb{C}[x]$).

Our strategy to prove \autoref{thm:cayley-hamilton} is:
\begin{itemize}
    \item Consider the case where \( m_T(x), \mathcal{X}_T(x)\) are in \(\overline{\mathbb{F}}[x]\) 
    \item Show that \(m_T(x) \mid \mathcal{X}_T(x)\) under \(\overline{\mathbb{F}}[x]\).
    \item Since both \(m_T(x)\) and \(\mathcal{X}_T(x)\) are in $\mathbb{F}[x]$, one must also have \(m_T(x) \mid \mathcal{X}_T(x)\) under \(\mathbb{F}[x]\).
\end{itemize}

\subsection{Invariant Subspace}
Before we give the full proof, we develop tools that allow us to decompose the vector space using invariant subspaces.
\begin{definition}[Invariant Subspace]\label{def:invariant-subspace}
Let \( T : V \to V \) be a linear operator. A subspace \( W \leq V \) is called \( T \)-invariant if
\[
T(W) \leq W.
\]
\end{definition}

\begin{remark}
If \( W \leq V \) is \( T \)-invariant, then the restriction \( T|_W : W \to W \) defines a well-defined linear operator.
\end{remark}

\begin{example}\label{ex:invariant-subspace}
The following are examples of \( T \)-invariant subspaces:
\begin{enumerate}
    \item The whole space \( V \) is trivially \( T \)-invariant.
    \item For each eigenvalue \( \lambda \), the eigenspace \( \ker(T - \lambda I) \) is \( T \)-invariant.
    \item More generally, for any polynomial \( g \in \mathbb{F}[x] \), the space
    \[
    U = \ker(g(T)) \leq V
    \]
    is \( T \)-invariant.

    \emph{Proof:} If \( {\bf v} \in \ker(g(T)) \), then \( g(T)({\bf v}) = 0 \). We want to show \( T({\bf v}) \in \ker(g(T)) \), i.e., \( g(T)(T({\bf v})) = 0 \). Observe:
    \[
    g(T)(T({\bf v})) = (a_m T^m + \cdots + a_1 T + a_0 I)(T({\bf v})) = T(g(T)({\bf v})) = T({\bf 0}) = {\bf 0}.
    \]
    \item If \( {\bf v} \in \ker(T - \lambda I) \), then \( \operatorname{span}\{{\bf v}\} \) is a \( T \)-invariant subspace.
\end{enumerate}
\end{example}

\begin{proposition}\label{prop:block-diagonal}
Let \( T : V \to V \) be a linear operator, and suppose \( W \leq V \) is a \( T \)-invariant subspace. Then we have the induced linear operators:
\begin{align}
\left. T \right|_W & : W \to W, \quad \mathbf{w} \mapsto T(\mathbf{w}) \label{eq:restriction-map} \\
\widetilde{T} & : V/W \to V/W, \quad \mathbf{v} + W \mapsto T(\mathbf{v}) + W \label{eq:quotient-map}
\end{align}
(The well-definedness of \( \widetilde{T} \) is proved in Homework 2, Exercise 4.)

With respect to a suitable basis, the matrix representation of \( T \) has the block form:
\[
[T]_{\mathcal{B},\mathcal{B}} =
\begin{pmatrix}
[\left. T \right|_W]_{\mathcal{C}, \mathcal{C}} & * \\
0 & [\widetilde{T}]_{\overline{\mathcal{B}}, \overline{\mathcal{B}}}
\end{pmatrix}
\]
where \( \mathcal{C} = \{ \mathbf{v}_1, \dots, \mathbf{v}_k \} \) is a basis of \( W \), and \( \overline{\mathcal{B}} = \{ \mathbf{v}_{k+1} + W, \dots, \mathbf{v}_n + W \} \) is a basis of \( V/W \).

It follows that the characteristic polynomial of \( T \) satisfies:
\[
\mathcal{X}_T(x) = \mathcal{X}_{\left. T \right|_W}(x) \cdot \mathcal{X}_{\widetilde{T}}(x).
\]
\end{proposition}

\subsection{Triangularizable Transformation}
As mentioned before, we will now focus on the case when 
\[
\mathcal{X}_T(x) = (x - \lambda_1) \cdots (x - \lambda_n) \in \mathbb{F}[x]
\]
can be factorized into linear terms (for instance, when $\mathbb{F} = \overline{\mathbb{F}}$ is algebraically closed.
\begin{proposition}\label{prop:upper-triangular-form}
Suppose
\[
\mathcal{X}_T(x) = (x - \lambda_1) \cdots (x - \lambda_n)
\]
where the \( \lambda_i \in \mathbb{F} \) are not necessarily distinct. Then there exists a basis \( \mathcal{A} \) of \( V \) such that
\[
T_{\mathcal{A}, \mathcal{A}} =
\begin{pmatrix}
\lambda_1 & * & \cdots & * \\
0 & \lambda_2 & \cdots & * \\
\vdots & \vdots & \ddots & \vdots \\
0 & 0 & \cdots & \lambda_n
\end{pmatrix}.
\]
In such a case, we call $T$ a {\bf triangularizable} linear transformation.
\end{proposition}

Obviously, the converse of the above proposition also holds, i.e. if $T:V \to V$ is triangularizable, then $\mathcal{X}_T(x) = (x - \lambda_1) \cdots (x - \lambda_n)$ can be factorized into linear terms. We will not make use of this fact in the proof of \autoref{thm:cayley-hamilton}, though.
\begin{proof}
We proceed by induction on \( n = \dim V \).

\textbf{Base Case:} For \( n = 1 \), the matrix of \( T \) is scalar \( (\lambda_1) \), and the statement holds.

\textbf{Induction Hypothesis:} Assume the result holds for all dimensions \( < n \). We show it holds for \( \dim V = n \).

\textbf{Step 1: Existence of eigenvector.}
Let \( \lambda_1 \) be an eigenvalue of \( T \). By the fundamental theorem of algebra (applied over \( \mathbb{C} \)), there exists an eigenvector \( \mathbf{v} \in V \setminus \{{\bf 0}\} \) such that
\[
T(\mathbf{v}) = \lambda_1 \mathbf{v}.
\]

\textbf{Step 2: Invariant subspace and quotient.}
Define \( W = \operatorname{span}\{\mathbf{v}\} \leq V \), which is \( T \)-invariant. By \autoref{prop:block-diagonal}, \( T \) induces a well-defined operator \( \widetilde{T} : V/W \to V/W \) with characteristic polynomial
\[
\mathcal{X}_{\widetilde{T}}(x) = (x - \lambda_2) \cdots (x - \lambda_n).
\]

\textbf{Step 3: Apply induction.}
By the induction hypothesis, there exists a basis \( \overline{\mathcal{C}} = \{ \mathbf{w}_2 + W, \dots, \mathbf{w}_n + W \} \) of \( V/W \) such that
\[
\widetilde{T}_{\overline{\mathcal{C}}, \overline{\mathcal{C}}} =
\begin{pmatrix}
\lambda_2 & * & \cdots & * \\
0 & \lambda_3 & \cdots & * \\
\vdots & \vdots & \ddots & \vdots \\
0 & 0 & \cdots & \lambda_n
\end{pmatrix}.
\]

\textbf{Step 4: Lift basis to \( V \).}
Define \( \mathcal{A} := \{ \mathbf{v}, \mathbf{w}_2, \dots, \mathbf{w}_n \} \). Then \( \mathcal{A} \) is a basis of \( V \) (see Homework 2, Exercise 2), and in this basis we have:
\[
T_{\mathcal{A}, \mathcal{A}} =
\begin{pmatrix}
\lambda_1 & * \\
0 & \widetilde{T}_{\overline{\mathcal{C}}, \overline{\mathcal{C}}}
\end{pmatrix}
=
\begin{pmatrix}
\lambda_1 & * & * & * \\
0 & \lambda_2 & \cdots & * \\
0 & \cdots & \ddots & * \\
0 & 0 & \cdots & \lambda_n
\end{pmatrix}
\]
as given in \autoref{prop:block-diagonal}. So the result follows.
\end{proof}
\begin{proposition}\label{prop:cayley-hamilton-diagonalizable}
Suppose that the characteristic polynomial of a linear operator \( T \) is
\[
\mathcal{X}_{T}(x) = (x - \lambda_1)\cdots(x - \lambda_n) \in \mathbb{F}[x].
\]
Then
\[
\mathcal{X}_{T}(T) = \mathbf{0}_{V \to V}.
\]
In other words, \autoref{thm:cayley-hamilton} holds for $T$.
\end{proposition}


\begin{proof}
Since ${\mathcal{X}}_{T}(x) = (x - \lambda_1) \cdots (x - \lambda_n)$, we imply $T$ is triangularizable under some basis $\mathcal{A}$, i.e.
$$T_{\mathcal{A},\mathcal{A}} = \begin{pmatrix}
\lambda_1 & \times & \cdots & \times \\
0 & \lambda_2 & \ddots & \vdots \\
\vdots & \ddots & \ddots & \times \\
0 & \cdots & 0 & \lambda_n
\end{pmatrix}.$$
Note that
\begin{itemize}
\item $\ast \mapsto {\left( \ast \right)}_{\mathcal{A},\mathcal{A}}$ is an isomorphism between $\operatorname{Hom}(V,V)$ and ${M}_{n \times n}(\mathbb{F})$ (c.f. \autoref{thm: coordinate-isomorphism} with $S = T$ and $\mathcal{A} = \mathcal{B} = \mathcal{C}$).
\item ${\left( T^m \right)}_{\mathcal{A},\mathcal{A}} = \left[ T_{\mathcal{A},\mathcal{A}} \right]^m$ for any $m$ (c.f. \autoref{prop:functoriality}).
\end{itemize}
Therefore, if
${\mathcal{X}}_{T}(x) = x^n + \alpha_{n-1}x^{n-1} + \dots + \alpha_1 x + \alpha_0,$ then
\begin{align*}
[{\mathcal{X}}_{T}(T)]_{\mathcal{A},\mathcal{A}} &= 
[T^n + \alpha_{n-1}T^{n-1} + \dots + \alpha_1 T + \alpha_0I]_{\mathcal{A},\mathcal{A}} \\
&= (T_{\mathcal{A},\mathcal{A}})^n + \alpha_{n-1}(T_{\mathcal{A},\mathcal{A}})^{n-1} + \dots + \alpha_1 T_{\mathcal{A},\mathcal{A}} + \alpha_0I_{\mathcal{A},\mathcal{A}} \\
&= {\mathcal{X}}_{T}\left( T_{\mathcal{A},\mathcal{A}}\right)
\end{align*}
Therefore,
$$\mathcal{X}_{T}(T) = \mathbf{0}_{V \to V} \quad \Leftrightarrow \quad [\mathcal{X}_{T}(T)]_{\mathcal{A},\mathcal{A}} = {\bf 0}_{n \times n} \quad \Leftrightarrow \quad {\mathcal{X}}_{T}\left( T_{\mathcal{A},\mathcal{A}}\right) = {\bf 0}_{n \times n},$$
and it suffices to show ${\mathcal{X}}_{T}\left( T_{\mathcal{A},\mathcal{A}}\right)$ is the zero matrix. To see so, note that
\[
{\mathcal{X}}_{T}\left( T_{\mathcal{A},\mathcal{A}} \right) = \left( T_{\mathcal{A},\mathcal{A}} - \lambda_1 \mathbf{I} \right) \cdots \left( T_{\mathcal{A},\mathcal{A}} - \lambda_n \mathbf{I} \right),
\]
where $T_{\mathcal{A},\mathcal{A}}$ is as given above. Write

$$M_i := \left( T_{\mathcal{A},\mathcal{A}} - \lambda_i \mathbf{I} \right) = \begin{pmatrix}
\lambda_1 - \lambda_i & \times & \cdots & \cdots &\times \\
0 & \ddots & \ddots & & \vdots \\
\vdots & \ddots & {\bf 0} & \ddots& \vdots \\
\vdots &  & \ddots &\ddots  &\times \\
0 & \cdots & \cdots & 0 &\lambda_n - \lambda_i
\end{pmatrix}$$
so that ${\mathcal{X}}_{T}\left( T_{\mathcal{A},\mathcal{A}}\right) = M_1\cdots M_i \cdots M_n$. Then
\begin{align*}
{\mathcal{X}}_{T}\left( T_{\mathcal{A},\mathcal{A}}\right)
\begin{pmatrix}
x_1 \\
\vdots \\
x_{n-1} \\
x_n
\end{pmatrix}
&=
M_1\cdots M_{n-1} M_n \begin{pmatrix}
x_1 \\
\vdots \\
x_{n-1} \\
x_n
\end{pmatrix} \\
&= M_1\cdots M_{n-1}\begin{pmatrix}
\lambda_1 - \lambda_n & \times &  \cdots &\times \\
0 & \ddots & \ddots & \vdots \\
\vdots &  \ddots &\lambda_{n-1} - \lambda_n  &\times \\
0  & \cdots & 0 & 0
\end{pmatrix}\begin{pmatrix}
x_1 \\
\vdots \\
x_{n-1} \\
x_n
\end{pmatrix} \\
&= M_1\cdots M_{n-1}\begin{pmatrix}
x_1' \\
\vdots \\
x_{n-1}' \\
0
\end{pmatrix}
\end{align*}
So we `kill off' the last coordinate of $\mathbf{v} =  \begin{pmatrix}
x_1 \\
\vdots \\
x_n 
\end{pmatrix}$. Continuing the process, one will kill of the $n-1$, $n-2$, $\dots$ until the top entry of ${\bf v}$, i.e.
\[
{\mathcal{X}}_{T}\left( T_{\mathcal{A},\mathcal{A}}\right){\bf v} = M_n \cdots M_1 \mathbf{v} = {\bf 0}.
\]
Consequently, ${\mathcal{X}}_{T}\left( T_{\mathcal{A},\mathcal{A}} \right)$ is a zero matrix.
\end{proof}

\section{Proof of Cayley–Hamilton Theorem}
Now we are ready to give a proof for the Cayley-Hamilton Theorem:
\begin{theorem}[Cayley–Hamilton]\label{thm:cayley-hamilton-weaker}
Let \( T : V \to V \) be a linear operator on a finite-dimensional vector space over a field \( \mathbb{F} \). Then the characteristic polynomial \( \mathcal{X}_T(x) \) annihilates \( T \), i.e.,
\[
\mathcal{X}_T(T) = {\bf 0}.
\]
\end{theorem}

\begin{proof}
Suppose that \( \mathcal{X}_T(x) = x^n + a_{n - 1}x^{n - 1} + \cdots + a_0 \in \mathbb{F}[x] \). By considering an algebraically closed field \( \overline{\mathbb{F}} \supseteq \mathbb{F} \), we write
\begin{align}
\mathcal{X}_T(x) &= x^n + a_{n - 1}x^{n - 1} + \cdots + a_0, \quad a_i \in \mathbb{F} \label{eq:charpoly-explicit} \\
                 &= (x - \lambda_1)\cdots(x - \lambda_n), \quad \lambda_i \in \overline{\mathbb{F}}. \label{eq:charpoly-factor}
\end{align}
By applying \autoref{prop:cayley-hamilton-diagonalizable}, we conclude \( \mathcal{X}_T(T) = {\bf 0} \), where the coefficients in the formula \( \mathcal{X}_T(T) = {\bf 0} \) with respect to \( T \) lie in \( \overline{\mathbb{F}} \).

Then we argue that these coefficients are actually in \( \mathbb{F} \). Expand the full expression for \( \mathcal{X}_T(T) \):
\begin{align}
\mathcal{X}_T(T) &= (T - \lambda_1 I)\cdots(T - \lambda_n I) \label{eq:poly-subst-product} \\
                 &= T^n - (\lambda_1 + \cdots + \lambda_n) T^{n-1} + \cdots + (-1)^n \lambda_1 \cdots \lambda_n I \label{eq:poly-subst-expanded} \\
                 &= T^n + a_{n - 1} T^{n - 1} + \cdots + a_0 I. \label{eq:poly-subst-original}
\end{align}
The equality \eqref{eq:poly-subst-original} holds because the coefficients of the polynomials in \eqref{eq:charpoly-explicit} and \eqref{eq:charpoly-factor} are equal.

Therefore, we conclude that \( \mathcal{X}_T(T) = 0 \) over the field \( \mathbb{F} \).
\end{proof}

\begin{corollary}\label{cor:cayley-hamiton-cor}
$m_T(x) \mid \mathcal{X}_T(x)$. More precisely, if
\[
\mathcal{X}_T(x) = [p_1(x)]^{e_1} \cdots [p_k(x)]^{e_k}, \quad e_i > 0 \text{ for all } i,
\]
where $p_i$ are distinct, monic, and irreducible polynomials, then
\[
m_T(x) = [p_1(x)]^{f_1} \cdots [p_k(x)]^{f_k}, \quad \text{for some } 0 < f_i \leq e_i \text{ for all } i.
\]
\end{corollary}

\begin{proof}
The statement $m_T(x) \mid \mathcal{X}_T(x)$ follows from \autoref{thm:cayley-hamilton-weaker} and \autoref{prop:minpoly-divides}. Therefore, $0 < f_i \leq e_i$ for all $i$.

Suppose on the contrary that $f_i = 0$ for some $i$. Without loss of generality, let $i = 1$.

It’s clear that $\gcd(p_1, p_j) = 1$ for all $j \neq 1$, which implies
\[
a(x)p_1(x) + b(x)p_j(x) = 1, \quad \text{for some } a(x), b(x) \in \mathbb{F}[x].
\]
Considering the field extension $\overline{\mathbb{F}} \supseteq \mathbb{F}$, we have
\[
p_1(x) = (x - \mu_1) \cdots (x - \mu_\ell).
\]
For any root $\mu_m$ of $p_1$, $m = 1, \ldots, \ell$, we have
\[
a(\mu_m)p_1(\mu_m) + b(\mu_m)p_j(\mu_m) = 1 \Rightarrow b(\mu_m)p_j(\mu_m) = 1 \Rightarrow p_j(\mu_m) \neq 0,
\]
i.e., $\mu_m$ is not a root of $p_j$ for all $j \neq 1$.

Therefore, $\mu_m$ is a root of $\mathcal{X}_T(x)$ but not a root of $m_T(x)$. Then $\mu_m$ is an eigenvalue of $T$, i.e., $T \mathbf{v} = \mu_m \mathbf{v}$ for some $\mathbf{v} \neq 0$.

Recall that $m_{T,\mathbf{v}} = x - \mu_m$, so $m_{T,\mathbf{v}} = x - \mu_m \mid m_T(x)$, which contradicts the assumption $f_1 = 0$.
\end{proof}

\begin{example}\label{ex:minpoly-examples}
We can use \autoref{cor:cayley-hamiton-cor}, a stronger version of the Cayley–Hamilton Theorem, to determine minimal polynomials:

\begin{enumerate}
  \item For the matrix \( A = \begin{pmatrix} 0 & -1 \\ 1 & 1 \end{pmatrix} \), we compute
  \[
  \mathcal{X}_A(x) = \left(x^2 + x + 1\right)^1.
  \]
  Since \( x^2 + x + 1 \) is irreducible over \( \mathbb{R} \), we conclude
  \[
  m_A(x) = x^2 + x + 1.
  \]

  \item For the matrix
  \[
  A = \begin{pmatrix}
  1 & 1 & 0 & 0 \\
  0 & 1 & 0 & 0 \\
  0 & 0 & 2 & 0 \\
  0 & 0 & 0 & 2
  \end{pmatrix},
  \]
  we compute
  \[
  \mathcal{X}_A(x) = (x - 1)^2(x - 2)^2.
  \]
  By \autoref{cor:cayley-hamiton-cor}, both \( (x - 1) \) and \( (x - 2) \) must divide \( m_A(x) \), so the minimal polynomial must be one of:
  \[
  (x - 1)^2(x - 2)^2, \quad (x - 1)(x - 2)^2, \quad (x - 1)^2(x - 2), \quad (x - 1)(x - 2).
  \]
  By trial and error, we find that
  \[
  m_A(x) = (x - 1)^2(x - 2).
  \]
\end{enumerate}
\end{example}
\chapter{Primary Decomposition and Jordan Normal Form}

We begin this chapter by re-defining the well-known notion of diagonalizability for linear operators:
\begin{definition}[Diagonalizable]\label{def:diagonalizable}
The linear operator \( T : V \to V \) is \emph{diagonalizable} over \( \mathbb{F} \) if and only if there exists a basis \( \mathcal{A} \) of \( V \) such that
\[
T_{\mathcal{A}, \mathcal{A}} = \operatorname{diag}(\lambda_1, \ldots, \lambda_n),
\]
where the \( \lambda_i \)'s are not necessarily distinct.
\end{definition}

\begin{proposition}\label{prop:minpoly-diagonal}
If the linear operator \( T : V \to V \) is diagonalizable, then its minimal polynomial is
\[
m_T(x) = (x - \mu_1)\cdots(x - \mu_k),
\]
where the \( \mu_i \)'s are distinct.
\end{proposition}

\begin{proof}
Suppose \( T \) is diagonalizable. Then there exists a basis \( \mathcal{A} \) of \( V \) such that
\[
T_{\mathcal{A}, \mathcal{A}} = \operatorname{diag}(\underbrace{\mu_1, \ldots, \mu_1}_{r_1}, \underbrace{\mu_2, \ldots, \mu_2}_{r_2}, \ldots, \underbrace{\mu_k, \ldots, \mu_k}_{r_k}),
\]
for some multiplicities \( r_i > 0 \) and distinct scalars \( \mu_i \in \mathbb{F} \). Then
\[
\left( T_{\mathcal{A}, \mathcal{A}} - \mu_1 I \right)\cdots\left( T_{\mathcal{A}, \mathcal{A}} - \mu_k I \right) = 0,
\]
i.e., \( m_T(x) \mid (x - \mu_1)\cdots(x - \mu_k) \).

To show minimality, suppose we omit any factor \( (x - \mu_i) \) from the product. Then the matrix
\[
\left( T_{\mathcal{A}, \mathcal{A}} - \mu_1 I \right)\cdots \widehat{(T_{\mathcal{A}, \mathcal{A}} - \mu_iI)} \cdots \left( T_{\mathcal{A}, \mathcal{A}} - \mu_k I \right)
\]
is nonzero, since all \( \mu_i \)'s are distinct and appear on the diagonal. Hence, no proper sub-product annihilates \( T \), and we conclude
\[
m_T(x) = (x - \mu_1)\cdots(x - \mu_k).
\]
\end{proof}

\section{Primary Decomposition Theorem}
The converse of \autoref{prop:minpoly-diagonal} is also true, which is a special case of the Primary Decomposition Theorem.

\begin{theorem}[Primary Decomposition Theorem]\label{thm:primary-decomposition}
Let \( T : V \rightarrow V \) be a linear operator with minimal polynomial
\[
m_T(x) = \left[ p_1(x) \right]^{e_1} \cdots \left[ p_k(x) \right]^{e_k},
\]
where \( p_i(x) \) are distinct, monic, and irreducible polynomials over \( \mathbb{F} \). For each \( i = 1, \ldots, k \), define the subspace
\[
V_i = \ker\left( \left[ p_i(T) \right]^{e_i} \right) \leq V.
\]
Then the following hold:
\begin{enumerate}
    \item Each \( V_i \) is \( T \)-invariant, i.e.,
    \[
    T(V_i) \leq V_i.
    \]
    
    \item The space \( V \) decomposes as a direct sum:
    \[
    V = V_1 \oplus V_2 \oplus \cdots \oplus V_k.
    \]
    
    \item The minimal polynomial of the restriction \( T|_{V_i} : V_i \to V_i \) is
    \[
    m_{T|_{V_i}}(x) = \left[ p_i(x) \right]^{e_i}.
    \]
\end{enumerate}
\end{theorem}

\begin{proof}
\begin{enumerate}
    \item This follows from \autoref{ex:invariant-subspace}(3).

    \item Define
    \[
    q_i(x) := \frac{m_T(x)}{p_i(x)^{e_i}} = \prod_{\substack{j = 1 \\ j \neq i}}^k p_j(x)^{e_j}.
    \]
    Then:
    \begin{itemize}
        \item \(\gcd(q_1, \dots, q_k) = 1\),
        \item \(\gcd(q_i, p_i^{e_i}) = 1\),
        \item \(q_i \cdot p_i^{e_i} = m_T\),
        \item \(m_T(x) \mid q_i(x) q_j(x)\) for \(i \neq j\).
    \end{itemize}

    By \hyperref[cor:bezout]{Bézout’s Identity}, there exist polynomials \(a_1(x), \dots, a_k(x) \in \mathbb{F}[x]\) such that
    \begin{equation}\label{eq:bezout-decomp}
        a_1(x) q_1(x) + \cdots + a_k(x) q_k(x) = 1.
    \end{equation}
    Evaluating at \(T\), we obtain
    \[
    a_1(T) q_1(T) + \cdots + a_k(T) q_k(T) = I.
    \]
    Thus, for any \(\mathbf{v} \in V\),
    \begin{equation}\label{eq:v-sum}
        \mathbf{v} = \sum_{i=1}^k a_i(T) q_i(T) \mathbf{v},
    \end{equation}
    and each term \(a_i(T) q_i(T) \mathbf{v} \in V_i\). Therefore,
    \begin{equation}
        V = V_1 + \cdots + V_k. \label{eq:v-sum-plain}
    \end{equation}

    To show the sum is direct, suppose
    \begin{equation}
        \mathbf{0} = \mathbf{v}_1' + \cdots + \mathbf{v}_k', \quad \text{where } \mathbf{v}_i' \in V_i. \label{eq:zero-sum}
    \end{equation}
    Fix any index \(i\). Since \(\gcd(q_i, p_i^{e_i}) = 1\), Bézout’s identity gives polynomials \(b_i(x), c_i(x)\) such that
    \[
    b_i(x) q_i(x) + c_i(x) p_i(x)^{e_i} = 1 \quad \Rightarrow \quad b_i(T) q_i(T) + c_i(T) p_i(T)^{e_i} = I.
    \]
    Applying both sides to \(\mathbf{v}_i'\), and using \(p_i(T)^{e_i} \mathbf{v}_i' = 0\), we get
    \[
    b_i(T) q_i(T) \mathbf{v}_i' = \mathbf{v}_i'.
    \]

    Apply \(b_i(T) q_i(T)\) to both sides of \eqref{eq:zero-sum}:
    \[
    \mathbf{0} = b_i(T) q_i(T) (\mathbf{v}_1' + \cdots + \mathbf{v}_k') = \sum_{j=1}^k b_i(T) q_i(T) \mathbf{v}_j'.
    \]
    But for \(j \neq i\), \(q_i(T) \mathbf{v}_j' = 0\), since \(p_j(x)^{e_j} \mid q_i(x)\). So:
    \[
    \mathbf{0} = b_i(T) q_i(T) \mathbf{v}_i' = \mathbf{v}_i' \quad \Rightarrow \quad \mathbf{v}_i' = 0.
    \]
    Since this holds for each \(i\), the sum is direct:
    \begin{equation}
        V = V_1 \oplus \cdots \oplus V_k. \label{eq:direct-sum}
    \end{equation}

    \item Since \(p_i(T)^{e_i} \mathbf{v} = 0\) for any \(\mathbf{v} \in V_i\), it follows that \(m_{T|_{V_i}}(x) \mid p_i(x)^{e_i}\). By \autoref{cor:cayley-hamiton-cor}, we have
    \[
    m_{T|_{V_i}}(x) = p_i(x)^{f_i}, \quad \text{for some } 1 \leq f_i \leq e_i.
    \]
    Suppose \(f_i < e_i\) for some \(i\). Let \(\mathbf{v} := \sum_{j=1}^k \mathbf{v}_j\), with \(\mathbf{v}_j \in V_j\), and define
    \[
    f(x) := \prod_{j=1}^k p_j(x)^{f_j}.
    \]
    Then \(f(T) \mathbf{v} = 0\), so \(f(T) = 0\) on all of \(V\), and hence \(m_T(x) \mid f(x)\), contradicting the minimality of \(m_T(x)\). Therefore,
    \[
    m_{T|_{V_i}}(x) = p_i(x)^{e_i}. \qedhere
    \]
\end{enumerate}
\end{proof}

Now we can prove the converse of \autoref{prop:minpoly-diagonal}:
\begin{corollary}\label{cor:diag-criterion}
Let the minimal polynomial of \(T\) be
\[
m_T(x) = (x - \mu_1) \cdots (x - \mu_k),
\]
with \(\mu_i \in \mathbb{F}\) distinct. Then \(T\) is diagonalizable over \(\mathbb{F}\). 
\end{corollary}

\begin{proof}
By the \hyperref[thm:primary-decomposition]{Primary Decomposition Theorem }, we have
\[
V = \bigoplus_{i = 1}^{k} V_i, \quad \text{where } V_i := \ker(T - \mu_i I).
\]
In other words, each \(V_i\) is the \(\mu_i\)-eigenspace of \(T\). Let \(\mathcal{B}_i\) be a basis of \(V_i\), and define \(\mathcal{B} := \bigcup_{i = 1}^k \mathcal{B}_i\). Then \(\mathcal{B}\) is a basis of \(V\) consisting entirely of eigenvectors of \(T\).

With respect to \(\mathcal{B}\), the matrix representation of \(T\) is block-diagonal:
\[
T_{\mathcal{A}, \mathcal{A}} = \operatorname{diag}(\underbrace{\mu_1, \dots, \mu_1}_{\dim V_1}, \dots, \underbrace{\mu_k, \dots, \mu_k}_{\dim V_k}),
\]
which is diagonal. Hence \(T\) is diagonalizable.
\end{proof}

\begin{corollary}[Spectral Decomposition]\label{cor:spectral-decomposition}
Suppose \(T : V \to V\) is diagonalizable. Then there exist linear operators \(P_i : V \to V\), for \(1 \leq i \leq k\), such that:
\begin{enumerate}
    \item \(P_i P_j = 0\) for all \(i \neq j\),
    \item \(\sum_{i=1}^k P_i = I\),
    \item \(P_i^2 = P_i\),
    \item \(P_i T = T P_i\) for all \(i\),
    \item \(T = \mu_1 P_1 + \cdots + \mu_k P_k\), for distinct scalars \(\mu_1, \ldots, \mu_k \in \mathbb{F}\).
\end{enumerate}
\end{corollary}

\begin{proof}
Since \(T : V \to V\) is diagonalizable, the minimal polynomial of \(T\) splits into distinct linear factors:
\[
m_T(x) = (x - \mu_1)(x - \mu_2) \cdots (x - \mu_k),
\]
with \(\mu_1, \ldots, \mu_k \in \mathbb{F}\) distinct. Define:
\[
q_i(x) := (x - \mu_1) \cdots \widehat{(x-\mu_i)} \cdot (c-\mu_k)
\]
As in the proof of \autoref{thm:primary-decomposition} such that $m_T(x) = (x - \mu_i) q_i(x)$, and there exist polynomials \(a_1(x), \ldots, a_k(x) \in \mathbb{F}[x]\) such that
\[
\sum_{i=1}^k a_i(x) q_i(x) = 1.
\]

Define operators \(P_i : V \to V\) by
\[
P_i := a_i(T) q_i(T).
\]

We now verify the required properties:
\begin{enumerate}
    \item[\textbf{(1)}] \textbf{Orthogonality:}  
    For \(i \ne j\), we compute:
    \[
    P_i P_j = a_i(T) q_i(T) \cdot a_j(T) q_j(T) = a_i(T) a_j(T) q_i(T) q_j(T).
    \]
    Since \(q_i(x) q_j(x)\) is divisible by \(m_T(x)\), and \(m_T(T) = 0\), it follows that $P_i P_j = 0$.

    \item[\textbf{(2)}] \textbf{Completeness:}  
    By construction:
    \[
    \sum_{i=1}^k P_i = \sum_{i=1}^k a_i(T) q_i(T) = I.
    \]
    \item[\textbf{(3)}] \textbf{Idempotency:} Since \(I = \sum_{j=1}^k P_j = \sum_{j=1}^k a_j(T) q_j(T)\), and $P_iP_j = 0$ if $i \neq j$, we compute:
    \[
    P_i^2 = P_i \cdot (P_1 + \dots +P_i + \dots +P_k) = P_i \cdot I = P_i.
    \]
    \item[\textbf{(4)}] \textbf{Commutativity:}  
    Each \(P_i = a_i(T) q_i(T)\) is a polynomial in \(T\), so it commutes with \(T\). Thus:
    \[
    P_i T = T P_i.
    \]

    \item[\textbf{(5)}] \textbf{Spectral decomposition:}  
    For each \(i\), define the subspace
    \[
    V_i := \ker(T - \mu_i I) = \ker( m_i(T)),
    \]
    where \(m_i(x) := (x - \mu_i)\). Let \(\mathbf{v} \in V\). Then using \(\sum P_i = I\), write:
    \[
    \mathbf{v} = \sum_{i=1}^k P_i \mathbf{v}.
    \]
    Apply \(T\):
    \[
    T \mathbf{v} = \sum_{i=1}^k T(P_i \mathbf{v}) = \sum_{i=1}^k P_i T(\mathbf{v}) \quad \text{(since } T P_i = P_i T\text{)}.
    \]
    But \(P_i \mathbf{v} \in V_i\), and \(T|_{V_i} = \mu_i \cdot \mathrm{id}_{V_i}\), so:
    \[
    T(P_i \mathbf{v}) = \mu_i P_i \mathbf{v}.
    \]
    Hence,
    \[
    T \mathbf{v} = \sum_{i=1}^k \mu_i P_i \mathbf{v} \quad \Rightarrow \quad T = \sum_{i=1}^k \mu_i P_i.
    \]
\end{enumerate}
\end{proof}

\section{Jordan Normal Form}

In the previous section, we have proved that a linear operator \( T \) is diagonalizable if and only if its minimal polynomial $m_T(x)$ consists of distinct linear terms. One may ask What happens if the minimal polynomial 
\[ m_T(x) = (x-\lambda_1)^{e_1}\cdots (x-\lambda_k)^{e_k}\] 
contains repeated linear factors? 

\begin{theorem}[Jordan Normal Form]\label{thm:jordan-normal-form}
Let \( \mathbb{F} \) be an algebraically closed field. Suppose that \( T : V \to V \) is a linear operator whose minimal polynomial has the form
\[
m_T(x) = \prod_{i=1}^k (x - \lambda_i)^{e_i},
\]
where the \( \lambda_i \in \mathbb{F} \) are distinct eigenvalues.

Then there exists a basis \( \mathcal{A} \) of \( V \) such that the matrix representation of \( T \) with respect to \( \mathcal{A} \) is block diagonal:
\[
T_{\mathcal{A}, \mathcal{A}} = \operatorname{diag}(J_1, \ldots, J_k),
\]
where each block \( J_i \) is a \emph{Jordan block} of the form
\[
J_i = \begin{pmatrix}
\mu & 1 & 0 & \cdots & 0 \\
0 & \mu & 1 & \cdots & 0 \\
\vdots & \ddots & \ddots & \ddots & \vdots \\
0 & \cdots & 0 & \mu & 1 \\
0 & \cdots & \cdots & 0 & \mu
\end{pmatrix},
\]
for some \( \mu \in \{ \lambda_1, \ldots, \lambda_k \} \).
\end{theorem}

\subsection{First Step of Proof:}
By the \hyperref[thm:primary-decomposition]{Primary Decomposition Theorem }, we have a direct sum decomposition:
\[
V = V_1 \oplus \cdots \oplus V_k,
\quad \text{where } V_i = \ker \left( (T - \lambda_i I)^{e_i} \right),
\]
and each subspace \( \mathbf{v}_i \) is \( T \)-invariant.
Moreover, choose a basis \( \mathcal{B}_i \) of \( \mathbf{v}_i \) for each $V_i$. Then the union \(\mathcal{B} := \bigcup_{i=1}^k \mathcal{B}_i
\) forms a basis of \( V \), and the matrix representation of \( T \) in this basis is block diagonal:
\[
T_{\mathcal{B}, \mathcal{B}} =
\begin{bmatrix}
[T|_{\mathbf{v}_1}]_{\mathcal{B}_1, \mathcal{B}_1} & 0 & \cdots & 0 \\
0 & [T|_{\mathbf{v}_2}]_{\mathcal{B}_2, \mathcal{B}_2} & \cdots & 0 \\
\vdots & \vdots & \ddots & \vdots \\
0 & 0 & \cdots & [T|_{\mathbf{v}_k}]_{\mathcal{B}_k, \mathcal{B}_k}
\end{bmatrix},
\]
with \( m_{T|_{\mathbf{v}_i}}(x) = (x - \lambda_i)^{e_i} \).

\medskip

Therefore, to prove the full Jordan normal form, it suffices to treat the case where
\[
m_T(x) = (x - \lambda)^e
\]
is a single primary factor.

\subsection{Second Step of Proof}
Firstly, we consider the case where the minimal polynomial has the form \(m_T(x) = x^m\):
\begin{proposition}[Nilpotent Case of Jordan Normal Form]\label{prop:jordan-nilpotent}
Suppose \( T : V \to V \) is a linear operator with minimal polynomial
\[
m_T(x) = x^m.
\]
Then there exists a basis \( \mathcal{A} \) of \( V \) such that
\[
[T]_{\mathcal{A},\mathcal{A}} = \operatorname{diag}(J_1, \ldots, J_\ell),
\]
where each block \( J_i \) is a Jordan block of the form
\[
J_i = \begin{bmatrix}
0 & 1 &        &        \\
  & 0 & \ddots &        \\
  &   & \ddots & 1      \\
  &   &        & 0
\end{bmatrix}.
\]
\end{proposition}

\begin{proof}
Assume \( m_T(x) = x^m \). Then we have the ascending chain of kernels:
\[
\{0\} = \ker(T^0) \subsetneq \ker(T^1) \subsetneq \cdots \subsetneq \ker(T^m) = V,
\]
we have \(\ker \left( {T}^{i - 1}\right)  \subsetneq  \ker \left( {T}^{i}\right)\) for \(i = 1,\ldots ,m\) : Note that \(\ker \left( {T}^{m - 1}\right)  \subsetneq\)  \(\ker \left( {T}^{m}\right) = V\) due to the minimality of \({m}_{T}\left( x\right)\) ; and \(\ker \left( {T}^{m - 2}\right)  \subsetneqq  \ker \left( {T}^{m - 1}\right)\) since otherwise for any \(\mathbf{x} \in  \ker \left( {T}^{m}\right)\),
\[
{T}^{m - 1}\left( {T\mathbf{x}}\right)  = \mathbf{0} \Rightarrow  T\mathbf{x} \in  \ker \left( {T}^{m - 1}\right)  = \ker \left( {T}^{m - 2}\right)  \Rightarrow  {T}^{m - 2}\left( {T\mathbf{x}}\right)  = {T}^{m - 1}\left( \mathbf{x}\right)  = \mathbf{0},
\]
i.e., \(\mathbf{x} \in  \ker \left( {T}^{m - 1}\right)\) , which contradicts to the fact that \(\ker \left( {T}^{m - 1}\right)  \subsetneqq  \ker \left( {T}^{m}\right)\) . Proceeding this trick sequentially for \(i = m,m - 1,\ldots ,1\) , we proved the desired result.


For each \( i = 1, \ldots, m \), define the quotient space:
\[
W_i = \ker(T^i)/\ker(T^{i-1}),
\]
Pictorially, 
\[
\{0\} = \underbrace{\ker(T^0) \subsetneq \ker(T^1)}_{W_1} \subsetneq \cdots  \subsetneq \underbrace{\ker(T^{m-1}) \subsetneq \ker(T^m)}_{W_m} = V,
\]
Choose a basis of $W_m$:
\[
\mathcal{B}_m' = \{ {\bf u}_1 + \ker(T^{m-1}), \ldots, {\bf u}_{\ell} + \ker(T^{m-1}) \} \subseteq W_m.
\]
And consider a {\bf set} $\mathcal{C}_m'$ in $W_{m-1}$:
\[
\mathcal{C}_m' = \{ T({\bf u}_1) + \ker(T^{m-2}), \ldots, T({\bf u}_{\ell}) + \ker(T^{m-2}) \} \subseteq W_{m-1}.
\]
(Note that each $T^{m-1}(T({\bf u}_j)) = T({\bf u}_j) = {\bf 0}$ since ${\bf u}_j \in \ker(T)$, therefore $T({\bf u}_j) \in \ker(T^{m-1})$)

{\bf Claim:} \(\mathcal{C}_{m}'\) is linearly independent in $W_{m-1}$: consider the equation
\[
\mathop{\sum }\limits_{j}k_j\left( T({\bf u}_j) + \ker(T^{m-2})\right)  = {\mathbf{0}}_{{W}_{i}} \Leftrightarrow  T\left( {\mathop{\sum }\limits_{j}k_j{\bf u}_j}\right)  + \ker \left( {T}^{m - 2}\right)  = {\mathbf{0}}_{{W}_{i}}
\]
i.e.,
\[
T\left( {\mathop{\sum }\limits_{j}k_j{\bf u}_j}\right)  \in  \ker \left( {T}^{m - 2}\right)  \Leftrightarrow  {T}^{m-2}\left( T\left( {\mathop{\sum }\limits_{j}k_j{\bf u}_j}\right)\right)  = {\mathbf{0}}_{V}.
\]
In other words, ${\mathop{\sum }\limits_{j}k_j{\bf u}_j} \in  \ker \left( {T}^{m-1}\right)$ , and hence
\[
{\mathop{\sum }\limits_{j}k_j{\bf u}_j} + \ker \left( {T}^{m-1}\right)  = {\mathbf{0}}_{W_m} \Leftrightarrow  \mathop{\sum }\limits_{j}k_j\left( {{{\bf u}_j} + \ker \left( {T}^{m-1}\right) }\right)  = {\mathbf{0}}_{W_m}.
\]
Since \(\mathcal{B}_m = \left\{ {\bf u}_j + \ker \left( {T}^{m-1}\right)\ |\ 1 \leq j \leq \ell_m\right\}\) forms a basis of \({W}_m\), we have \(k_j = 0\) for all \(j\), and $\mathcal{C}_m'$ is linearly independent in $W_{m-1}$. 

By Basis Extension Theorem, one can get a basis of $W_{m-1}$ of the form
\[
\mathcal{B}_{m-1}' = \{ T({\bf u}_1) + \ker(T^{m-2}), \ldots, T({\bf u}_{\ell}) + \ker(T^{m-2}), \quad {\bf v}_1 + \ker(T^{m-2}), \ldots, {\bf v}_p + \ker(T^{m-2}) \} \subseteq W_{m-1}.
\]
By similar arguments, one can get 
\[
\mathcal{C}_{m-1}' = \{ T^2({\bf u}_1) + \ker(T^{m-3}), \ldots, T^2({\bf u}_{\ell}) + \ker(T^{m-3}), \quad T({\bf v}_1) + \ker(T^{m-3}), \ldots, T({\bf v}_p) + \ker(T^{m-3}) \}
\]
is linearly independent in $W_{m-2}$, and one can construct a basis 
\begin{center}
$\mathcal{B}_{m-2}' = \{ T^2({\bf u}_1) + \ker(T^{m-3}), \ldots, T^2({\bf u}_{\ell}) + \ker(T^{m-3}), \quad T({\bf v}_1) + \ker(T^{m-3}), \ldots, T({\bf v}_p) + \ker(T^{m-3}), 
{\bf w}_1 + \ker(T^{m-3}), \ldots, {\bf w}_q + \ker(T^{m-3})\}$
\end{center}
of $W_{m-2}$.

Consequently, we have the following collection of bases of $W_i$:
{\small\[
\begin{array}{cccccc}
    W_1 & \cdots &
    W_{m-2} &
    W_{m-1}&
    W_{m}
\\[2em]
\left\{
    \begin{array}{c}
        T^{m-1}({\bf u}_1) + \ker(T^0) \\
        \vdots \\
        T^{m-1}({\bf u}_{\ell}) + \ker(T^0)
    \end{array}
\right\}
& \cdots &
\left\{
    \begin{array}{c}
        T^2({\bf u}_1) + \ker(T^{m-3}) \\
        \vdots \\
        T^2({\bf u}_{\ell}) + \ker(T^{m-3})
    \end{array}
\right\}
& 
\left\{
    \begin{array}{c}
        T({\bf u}_1) + \ker(T^{m-2}) \\
        \vdots \\
        T({\bf u}_{\ell}) + \ker(T^{m-2})
    \end{array}
\right\}
& 
\left\{
    \begin{array}{c}
        {\bf u}_1 + \ker(T^{m-1}) \\
        \vdots \\
        {\bf u}_{\ell} + \ker(T^{m-1})
    \end{array}
\right\}
\\[2em]
\left\{
    \begin{array}{c}
        T^{m-2}({\bf v}_1) + \ker(T^0) \\
        \vdots \\
        T^{m-2}({\bf v}_p) + \ker(T^0)
    \end{array}
\right\}
& \cdots &
\left\{
    \begin{array}{c}
        T({\bf v}_1) + \ker(T^{m-3}) \\
        \vdots \\
        T({\bf v}_p) + \ker(T^{m-3})
    \end{array}
\right\}
& 
\left\{
    \begin{array}{c}
        {\bf v}_1 + \ker(T^{m-2}) \\
        \vdots \\
        {\bf v}_p + \ker(T^{m-2})
    \end{array}
\right\}
& 
\\[2em]
\left\{
    \begin{array}{c}
        T^{m-3}({\bf w}_1) + \ker(T^0) \\
        \vdots \\
        T^{m-3}({\bf w}_q) + \ker(T^0)
    \end{array}
\right\}
& \cdots &
\left\{
    \begin{array}{c}
        {\bf w}_1 + \ker(T^{m-3}) \\
        \vdots \\
        {\bf w}_q + \ker(T^{m-3})
    \end{array}
\right\}
& & 
\\[2em]
\vdots & & & &
\end{array}
\]}

Indeed, by removing the $\ker(T^i)$'s in the above list, the vectors will give a basis of $V$ (Exercise). Now choose the ordered basis \( \mathcal{A} \) of $V$ by arranging the vectors:
\[
\mathcal{A} = \left\{ 
\begin{matrix}
T^{m-1}({\bf u}_1), & T^{m-2}({\bf u}_1), & \cdots & T^2({\bf u}_1), & T({\bf u}_1), & {\bf u}_1, \\
\vdots & \vdots &  & \vdots & \vdots & \vdots\\
T^{m-1}({\bf u}_{\ell}), & T^{m-2}({\bf u}_{\ell}), & \cdots & T^2({\bf u}_1), & T({\bf u}_{\ell}), & {\bf u}_{\ell}, \\
T^{m-2}({\bf v}_1), & T^{m-3}({\bf v}_1), & \cdots & T({\bf v}_1), & {\bf v}_1, &  \\
\vdots & \vdots &  & \vdots & \vdots \\
T^{m-2}({\bf v}_p), & T^{m-3}({\bf v}_p), & \cdots & T({\bf v}_p), & {\bf v}_p, &  \\
T^{m-3}({\bf w}_1), & T^{m-4}({\bf w}_1), & \cdots & {\bf w}_1, &   \\
\vdots & \vdots &  & \vdots \\
T^{m-3}({\bf w}_q), & T^{m-4}({\bf w}_q), & \cdots & {\bf w}_q, &  \\
\vdots & \vdots
\end{matrix} \right\}
\]
Under this basis, \( T_{\mathcal{A},\mathcal{A}} \) is block diagonal with nilpotent Jordan blocks:
\begin{itemize}
  \item All diagonal entries are zero since \( T(T^{i-1}({\bf x})) = T^i({\bf x})\), and both $\{T^{i-1}({\bf x}), T^i({\bf x})\}$ are basis vectors.
  \item All superdiagonal entries are 1, corresponding to \( T(T^{k}({\bf x})) = T^{k+1}({\bf x}) \).
\end{itemize}
In particular, the matrix representation is of the form:
$$T_{\mathcal{A},\mathcal{A}} = \begin{pmatrix}
    N_m & & & & & &\\
    & \ddots & & & & & \\
    & & N_m & & & & \\
    & & & N_{m-1} & & & \\
    & & & & \ddots & & \\
    & & & & & N_{m-1} &  \\
    & & & & & & N_{m-2} \\
    & & & & & & & \ddots \\
    & & & & & & & & N_{m-2} \\
    & & & & & & & & & \ddots \\
\end{pmatrix}$$
where $N_i = \begin{pmatrix}
0 & 1       &        &        \\
        & 0 & \ddots &        \\
        &         & \ddots & 1      \\
        &         &        & 0
\end{pmatrix}$ is an $i \times i$-matrix, and $J_m$ appears $\ell$ times, $J_{m-1}$ appears $p$ times, $J_{m-2}$ appears $q$ times in the above matrix.
\end{proof}

\subsection{Third Step of Proof}
Then we consider the case where \({m}_{T}\left( x\right)  = {\left( x - \lambda \right) }^{e}\):
\begin{corollary}
Suppose \( T : V \to V \) is such that the minimal polynomial is
\[
m_T(x) = (x - \lambda)^e,
\]
for some \( \lambda \in \mathbb{F} \). Then there exists a basis \( \mathcal{A} \) of \( V \) such that
\[
T_{\mathcal{A}, \mathcal{A}} = \operatorname{diag}(J_1, \ldots, J_\ell),
\]
where each block \( J_i \) is of the form
\[
J_i = \begin{pmatrix}
\lambda & 1       &        &        \\
        & \lambda & \ddots &        \\
        &         & \ddots & 1      \\
        &         &        & \lambda
\end{pmatrix}.
\]
\end{corollary}

\begin{proof}
Let \( U := T - \lambda I \). Then the minimal polynomial of \( U \) is \( m_U(x) = x^e \). By \autoref{prop:jordan-nilpotent}, there exists a basis \( \mathcal{A} \) such that
\[
U_{\mathcal{A}, \mathcal{A}} = \operatorname{diag}(M_1, \ldots, M_\ell),
\]
where $M_i = N_{\alpha_i}$ as given in \autoref{prop:jordan-nilpotent} for some $\alpha_i \in \mathbb{N}$.  Therefore,
\[
T_{\mathcal{A}, \mathcal{A}} = [U + \lambda I]_{\mathcal{A}, \mathcal{A}} = \operatorname{diag}(M_1 + \lambda I, \ldots, M_\ell + \lambda I),
\]
which gives the desired Jordan blocks centered at \( \lambda \).
\end{proof}

\begin{corollary}
\label{cor:jordan-c-complete}
Let \( A \in M_{n \times n}(\mathbb{C}) \). Then there exists an invertible matrix \( P \) such that
\[
P^{-1} A P = \operatorname{diag}(J_1, \ldots, J_\ell),
\]
where each \( J_i \) is a Jordan block of the form described above. That is, any complex square matrix is similar to a matrix in Jordan normal form.
\end{corollary}
\begin{proof}
    By the Fundamental Theorem of Algebra, the minimal polynomial of $T$ can always be factorized into linear terms, i.e.
    $$m_{T}(x) = (x-\lambda_1)^{e_1} \cdots (x-\lambda_k)^{e_k}.$$
    Therefore, the result follows from \autoref{thm:jordan-normal-form} and the discussions in \autoref{sec:similar_basis}.
\end{proof}


\chapter{Inner Product Space}

\section{Introduction to Inner Product Space}
\begin{definition}[Bilinear Form]\label{def:bilinear-form}
Let \( V \) be a vector space over \( \mathbb{R} \). A \emph{bilinear form} on \( V \) is a map
\[
F : V \times V \to \mathbb{R}
\]
such that for all \( \mathbf{u}, \mathbf{v}, \mathbf{w} \in V \) and all scalars \( \lambda \in \mathbb{R} \),
\begin{enumerate}
    \item \( F(\mathbf{u} + \mathbf{v}, \mathbf{w}) = F(\mathbf{u}, \mathbf{w}) + F(\mathbf{v}, \mathbf{w}) \)
    \item \( F(\mathbf{u}, \mathbf{v} + \mathbf{w}) = F(\mathbf{u}, \mathbf{v}) + F(\mathbf{u}, \mathbf{w}) \)
    \item \( F(\lambda \mathbf{u}, \mathbf{v}) = \lambda F(\mathbf{u}, \mathbf{v}) = F(\mathbf{u}, \lambda \mathbf{v}) \)
\end{enumerate}

We say:
\begin{itemize}
    \item \( F \) is \emph{symmetric} if \( F(\mathbf{u}, \mathbf{v}) = F(\mathbf{v}, \mathbf{u}) \) for all \( \mathbf{u}, \mathbf{v} \in V \).
    \item \( F \) is \emph{non-degenerate} if \( F(\mathbf{u}, \mathbf{w}) = 0 \) for all \( \mathbf{u} \in V \) implies \( \mathbf{w} = 0 \).
    \item \( F \) is \emph{positive definite} if \( F(\mathbf{v}, \mathbf{v}) > 0 \) for all \( \mathbf{v} \in V \setminus \{0\} \).
\end{itemize}
\end{definition}

\begin{remark}
If \( F \) is positive definite, then \( F \) is non-degenerate. Indeed, suppose \( F(\mathbf{v}, \mathbf{v}) > 0 \) for all \( \mathbf{v} \neq 0 \), and that \( F(\mathbf{u}, \mathbf{w}) = 0 \) for all \( \mathbf{u} \in V \). Then in particular, \( F(\mathbf{w}, \mathbf{w}) = 0 \). But by positive definiteness, this implies \( \mathbf{w} = 0 \). Hence \( F \) is non-degenerate.
\end{remark}

When we say that \( V \) is a vector space over a field \( \mathbb{F} \), we treat \( \alpha \in \mathbb{F} \) as a scalar.

\begin{definition}[Sesquilinear Form]\label{def:sesquilinear}
Let \( V \) be a vector space over \( \mathbb{C} \). A \emph{sesquilinear form} on \( V \) is a function
\[
F : V \times V \to \mathbb{C}
\]
satisfying, for all \( \mathbf{u}, \mathbf{v}, \mathbf{w} \in V \) and all \( \lambda \in \mathbb{C} \),
\begin{enumerate}
    \item \( F(\mathbf{u} + \mathbf{v}, \mathbf{w}) = F(\mathbf{u}, \mathbf{w}) + F(\mathbf{v}, \mathbf{w}) \)
    \item \( F(\mathbf{u}, \mathbf{v} + \mathbf{w}) = F(\mathbf{u}, \mathbf{v}) + F(\mathbf{u}, \mathbf{w}) \)
    \item \( F(\bar{\lambda} \mathbf{v}, \mathbf{w}) = F(\mathbf{v}, \lambda \mathbf{w}) = \lambda F(\mathbf{v}, \mathbf{w}) \)
\end{enumerate}

We say that \( F \) is \emph{conjugate symmetric} if
\[
F(\mathbf{v}, \mathbf{w}) = \overline{F(\mathbf{w}, \mathbf{v})}, \quad \forall \mathbf{v}, \mathbf{w} \in V.
\]

The definitions of \emph{non-degenerate} and \emph{positive definite} are the same as those for bilinear forms:
\begin{itemize}
    \item \( F \) is non-degenerate if \( F(\mathbf{u}, \mathbf{w}) = 0 \) for all \( \mathbf{u} \in V \) implies \( \mathbf{w} = 0 \).
    \item \( F \) is positive definite if \( F(\mathbf{v}, \mathbf{v}) > 0 \) for all \( \mathbf{v} \neq 0 \).
\end{itemize}
\end{definition}

\begin{remark}
Why is the complex conjugate \( \bar{\lambda} \) necessary in the definition?

Suppose we want \( F \) to be positive definite. If we define sesquilinearity without the conjugation, then for any \( \mathbf{v} \in V \), we would have
\[
F(i\mathbf{v}, i\mathbf{v}) = i^2 F(\mathbf{v}, \mathbf{v}) = -F(\mathbf{v}, \mathbf{v}) < 0,
\]
contradicting positive definiteness.

But with the conjugation, we instead compute:
\[
F(i\mathbf{v}, i\mathbf{v}) = \bar{i} \cdot i \cdot F(\mathbf{v}, \mathbf{v}) = |i|^2 F(\mathbf{v}, \mathbf{v}) = F(\mathbf{v}, \mathbf{v}),
\]
which preserves positivity. Hence, the use of \( \bar{\lambda} \) is essential to ensure the existence of positive definite sesquilinear forms on complex vector spaces.
\end{remark}

\begin{example}[Hermitian Inner Product]\label{ex:hermitian-inner-product}
Let \( V = \mathbb{C}^n \). A fundamental sesquilinear form on \( V \) is the Hermitian inner product, defined by
\[
F(\mathbf{v}, \mathbf{w}) = \mathbf{v}^{\mathrm{H}} \mathbf{w} = 
\begin{bmatrix}
\overline{v_1} & \cdots & \overline{v_n}
\end{bmatrix}
\begin{bmatrix}
w_1 \\ \vdots \\ w_n
\end{bmatrix}
= \sum_{i=1}^n \overline{v_i} w_i.
\]
In this case, \( F \) is not symmetric in the usual sense, i.e., \( F(\mathbf{v}, \mathbf{w}) \neq F(\mathbf{w}, \mathbf{v}) \), but it is \emph{conjugate symmetric}:
\[
F(\mathbf{v}, \mathbf{w}) = \overline{F(\mathbf{w}, \mathbf{v})}.
\]
\end{example}

\begin{definition}[Inner Product]\label{def:inner-product}
Let \( V \) be a real (respectively, complex) vector space. A bilinear (respectively, sesquilinear) form
\[
\langle \cdot, \cdot \rangle : V \times V \to \mathbb{R} \text{ (respectively, } \mathbb{C} \text{)}
\]
is called an \emph{inner product} if it satisfies:
\begin{itemize}
    \item Symmetry (or conjugate symmetry): \( \langle \mathbf{v}, \mathbf{w} \rangle = \overline{\langle \mathbf{w}, \mathbf{v} \rangle} \)
    \item Positive definiteness: \( \langle \mathbf{v}, \mathbf{v} \rangle > 0 \) for all \( \mathbf{v} \neq 0 \)
\end{itemize}
A vector space equipped with an inner product is called an \emph{inner product space}.
\end{definition}

\begin{remark}
We denote the inner product using the bracket notation \( \langle \cdot, \cdot \rangle \) instead of a general form symbol like \( F(\cdot, \cdot) \).
\end{remark}

\begin{definition}[Norm]\label{def:norm}
The \emph{norm} induced by the inner product is defined as
\[
\| \mathbf{v} \| := \sqrt{\langle \mathbf{v}, \mathbf{v} \rangle}.
\]
In particular, for any scalar \( \alpha \in \mathbb{F} \),
\[
\| \alpha \mathbf{v} \| = \sqrt{ \langle \alpha \mathbf{v}, \alpha \mathbf{v} \rangle }
= \sqrt{ \bar{\alpha} \alpha \langle \mathbf{v}, \mathbf{v} \rangle }
= |\alpha| \cdot \| \mathbf{v} \|.
\]
The norm is well-defined by the positive definiteness of the inner product.
\end{definition}

\begin{definition}[Orthogonality and Orthonormality]\label{def:orthogonal}
Let \( S = \{ \mathbf{v}_i \mid i \in I \} \subset V \). We say \( S \) is an \emph{orthogonal set} if
\[
\langle \mathbf{v}_i, \mathbf{v}_j \rangle = 0 \quad \text{for all } i \neq j.
\]
If in addition each vector has unit norm, i.e.,
\[
\langle \mathbf{v}_i, \mathbf{v}_i \rangle = 1 \quad \text{for all } i,
\]
then \( S \) is called an \emph{orthonormal set}.
\end{definition}

\section{Cauchy–Schwarz Inequality}
\begin{proposition}[Cauchy–Schwarz Inequality]\label{prop:cauchy-schwarz}
Let \( V \) be an inner product space. Then for all \( \mathbf{u}, \mathbf{v} \in V \),
\[
| \langle \mathbf{u}, \mathbf{v} \rangle | \leq \| \mathbf{u} \| \cdot \| \mathbf{v} \|.
\]
\end{proposition}

\begin{proof}
We divide the proof into two cases depending on whether the inner product is real-valued or complex-valued.

\textbf{Case 1:} \( \langle \mathbf{u}, \mathbf{v} \rangle \in \mathbb{R} \): Define the function \( f(t) := \| \mathbf{u} + t\mathbf{v} \|^2 \) for real \( t \in \mathbb{R} \). Then
\[
f(t) = \langle \mathbf{u} + t\mathbf{v}, \mathbf{u} + t\mathbf{v} \rangle = \| \mathbf{u} \|^2 + 2t \langle \mathbf{u}, \mathbf{v} \rangle + t^2 \| \mathbf{v} \|^2.
\]
Since \( f(t) \geq 0 \) for all \( t \in \mathbb{R} \), the discriminant of this quadratic must be non-positive:
\[
(2 \langle \mathbf{u}, \mathbf{v} \rangle)^2 - 4 \| \mathbf{u} \|^2 \| \mathbf{v} \|^2 \leq 0,
\]
which simplifies to
\[
|\langle \mathbf{u}, \mathbf{v} \rangle|^2 \leq \| \mathbf{u} \|^2 \| \mathbf{v} \|^2,
\]
and hence
\[
|\langle \mathbf{u}, \mathbf{v} \rangle| \leq \| \mathbf{u} \| \cdot \| \mathbf{v} \|.
\]

\textbf{Case 2:} \( \langle \mathbf{u}, \mathbf{v} \rangle \in \mathbb{C} \smallsetminus \mathbb{R} \): We apply a rescaling argument to reduce to the real case. Define
\[
\mathbf{w} := \frac{1}{\langle \mathbf{u}, \mathbf{v} \rangle} \mathbf{u}.
\]
Then
\[
\langle \mathbf{w}, \mathbf{v} \rangle 
= \left\langle \frac{1}{\langle \mathbf{u}, \mathbf{v} \rangle} \mathbf{u}, \mathbf{v} \right\rangle
= \overline{ \frac{1}{\overline{ \langle \mathbf{u}, \mathbf{v} \rangle }} } \cdot \langle \mathbf{u}, \mathbf{v} \rangle = 1 \in \mathbb{R}.
\]
Now applying the Cauchy–Schwarz inequality in the real case to vectors \( \mathbf{w} \) and \( \mathbf{v} \), we have
\[
| \langle \mathbf{w}, \mathbf{v} \rangle | \leq \| \mathbf{w} \| \cdot \| \mathbf{v} \|,
\]
i.e.,
\[
1 \leq \left\| \frac{1}{\langle \mathbf{u}, \mathbf{v} \rangle} \mathbf{u} \right\| \cdot \| \mathbf{v} \| = \frac{\| \mathbf{u} \|}{| \langle \mathbf{u}, \mathbf{v} \rangle |} \cdot \| \mathbf{v} \|.
\]
Rearranging gives
\[
| \langle \mathbf{u}, \mathbf{v} \rangle | \leq \| \mathbf{u} \| \cdot \| \mathbf{v} \|,
\]
as desired.
\end{proof}

\begin{proposition}[Triangle Inequality]\label{prop:triangle-inequality}
Let \( V \) be an inner product space. Then for all \( \mathbf{u}, \mathbf{v} \in V \),
\[
\| \mathbf{u} + \mathbf{v} \| \leq \| \mathbf{u} \| + \| \mathbf{v} \|.
\]
\end{proposition}

\begin{proof}
We expand:
\[
\| \mathbf{u} + \mathbf{v} \|^2 = \langle \mathbf{u} + \mathbf{v}, \mathbf{u} + \mathbf{v} \rangle = \| \mathbf{u} \|^2 + 2\operatorname{Re}(\langle \mathbf{u}, \mathbf{v} \rangle) + \| \mathbf{v} \|^2.
\]
Now use \( \operatorname{Re}(\langle \mathbf{u}, \mathbf{v} \rangle) \leq | \langle \mathbf{u}, \mathbf{v} \rangle | \) and apply Cauchy–Schwarz:
\[
\| \mathbf{u} + \mathbf{v} \|^2 \leq \| \mathbf{u} \|^2 + 2 \| \mathbf{u} \| \| \mathbf{v} \| + \| \mathbf{v} \|^2 = (\| \mathbf{u} \| + \| \mathbf{v} \|)^2.
\]
Taking square roots on both sides completes the proof.
\end{proof}

\begin{theorem}[Gram–Schmidt Process]\label{thm:gram-schmidt}
Let \( S = \{ \mathbf{v}_1, \ldots, \mathbf{v}_n \} \subset V \) be a finite linearly independent set in an inner product space \( V \). Then there exists an orthonormal set \( \{ \mathbf{e}_1, \ldots, \mathbf{e}_n \} \) such that \( \operatorname{span}\{ \mathbf{v}_1, \ldots, \mathbf{v}_k \} = \operatorname{span}\{ \mathbf{e}_1, \ldots, \mathbf{e}_k \} \) for all \( k \).

The construction is given inductively:
\[
\mathbf{w}_1 := \mathbf{v}_1, \quad 
\mathbf{w}_{i+1} := \mathbf{v}_{i+1} - \sum_{j=1}^{i} \frac{\langle \mathbf{v}_{i+1}, \mathbf{w}_j \rangle}{\| \mathbf{w}_j \|^2} \mathbf{w}_j, \quad \text{for } i = 1, \ldots, n-1,
\]
and the orthonormal set is obtained by normalizing:
\[
\mathbf{e}_i := \frac{\mathbf{w}_i}{\| \mathbf{w}_i \|}.
\]
\end{theorem}

\begin{corollary}
Every finite-dimensional inner product space admits an orthonormal basis.
\end{corollary}

\section{Orthogonal Complement}

\begin{definition}[Orthogonal Complement] Let \(U \leq  V\) be a subspace of an inner product space. Then the orthogonal complement of \(U\) is
\[
{U}^{\perp} = \{ \mathbf{v} \in  V \mid  \langle \mathbf{v},\mathbf{u}\rangle  = 0,\forall \mathbf{u} \in  U\}
\]
\end{definition}

The analysis for orthogonal complement for vector spaces over \(\mathbb{C}\) is quite similar as what we have studied in MAT2040.

\begin{proposition} 
Let $V$ be an inner product space, $U \leq V$ a subspace of $V$, and $U_1, U_2 \subseteq V$ are subsets of $V$.
\begin{enumerate}
    \item ${U}^{\perp} \leq V$ is a subspace of \(V\).
    \item \(U \cap  {U}^{\perp} = \{ 0\}\).
    \item \({U}_{1} \subseteq  {U}_{2}\) implies \({U}_{2}^{\perp} \leq  {U}_{1}^{\perp}\) .
\end{enumerate}
\end{proposition}

\begin{proof} 
1. Suppose that \({\bf v}_{1},{\bf v}_{2} \in  {U}^{\perp}\) , where \(a,b \in  K\left( {K = \mathbb{C}\text{ or }\mathbb{R}}\right)\) , then for all \(\mathbf{u} \in  U\) ,

\[
\left\langle  {a{\mathbf{v}}_{1} + b{\mathbf{v}}_{2},\mathbf{u}}\right\rangle   = \bar{a}\left\langle  {{\mathbf{v}}_{1},\mathbf{u}}\right\rangle   + \bar{b}\left\langle  {{\mathbf{v}}_{2},\mathbf{u}}\right\rangle
= \bar{a} \cdot  0 + \bar{b} \cdot  0 = 0
\]

Therefore, \(a{\mathbf{v}}_{1} + b{\mathbf{v}}_{2} \in  {U}^{\perp}\) .

2. Suppose that \(\mathbf{u} \in  U \cap  {U}^{\perp}\) , then we imply \(\langle \mathbf{u},\mathbf{u}\rangle  = 0\) . By the positive-definiteness of inner product, \(\mathbf{u} = \mathbf{0}\).

3. Exercise.
\end{proof}

\begin{proposition} \label{prop:orthogonal_comple}
Let $U, W \leq V$ be subspaces of an inner product space $V$.
\begin{enumerate}
    \item If \(\dim \left( V\right)  < \infty\) and \(U \leq  V\) , then \(V = U \oplus  {U}^{\perp}\).
    \item One has
\begin{enumerate}
    \item[(a)] ${\left( U + W\right) }^{\perp} = {U}^{\perp} \cap  {W}^{\perp}$;
    \item[(b)] ${\left( U \cap  W\right) }^{\perp} \supseteq  {U}^{\perp} + {W}^{\perp}$;
    \item[(c)] ${\left( {U}^{\perp}\right) }^{\perp} \supseteq  U$. 
\end{enumerate}
Moreover, if \(\dim \left( V\right)  < \infty\) , then these are equalities.
\end{enumerate}
\end{proposition}


\begin{proof} 1. Suppose that \(\left\{  {{\mathbf{v}}_{1},\ldots ,{\mathbf{v}}_{k}}\right\}\) forms a basis for \(U\) , and by basis extension, we obtain \(\left\{  {{\mathbf{v}}_{1},\ldots ,{\mathbf{v}}_{k},{\mathbf{v}}_{k + 1},\ldots ,{\mathbf{v}}_{n}}\right\}\) is a basis for \(V\) .

By Gram-Schmidt Process, any finite basis induces an orthonormal basis. Therefore, suppose that \(\left\{  {{\mathbf{e}}_{1},\ldots ,{\mathbf{e}}_{k}}\right\}\) forms an orthonormal basis for \(U\) , and \(\left\{  {{\mathbf{e}}_{k + 1},\ldots ,{\mathbf{e}}_{n}}\right\}\) forms an orthonormal basis for \({U}^{\perp}\) .

It’s easy to show \(V = U + {U}^{\perp}\) using orthonormal basis.

\noindent 2. (a) The inclusion \({\left( U + W\right) }^{\perp} \supseteq  {U}^{\perp} \cap  {W}^{\perp}\) is trivial; for the other inclusion, suppose
\({\bf v} \in  {\left( U + W\right) }^{\perp}\). Then
\[
\langle \mathbf{v},\mathbf{u} + \mathbf{w}\rangle  = 0,\forall \mathbf{u} \in  U,\mathbf{w} \in  W
\]
Taking \(\mathbf{u} \equiv  \mathbf{0}\) in the equality above gives \(\langle \mathbf{v},\mathbf{w}\rangle  = 0\) , i.e., \(\mathbf{v} \in  W^{\perp}\) . Similarly, \({\bf v} \in U^{\perp}\) .

(b) Follow the similar argument as in (2a). If \(\dim \left( V\right)  < \infty\) , then write down the orthonormal basis for \({U}^{\perp} + {W}^{\perp}\) and \({\left( U \cap  W\right) }^{\perp}\) .

(c) Follow the similar argument as in (2a). If \(\dim \left( V\right)  < \infty\) , then
\[
V = {U}^{\perp} \oplus  {\left( {U}^{\perp}\right) }^{\perp} = U \oplus  {U}^{\perp}.
\]
Therefore, \(\dim{\left( {U}^{\perp}\right) }^{\perp} = \dim(U)\) and hence we have \({\left( {U}^{\perp}\right)}^{\perp} = (U)\)
\end{proof}



\section{The Riesz Representation Theorem}

\begin{theorem}[Riesz Representation]\label{thm:riesz}
Let \( V \) be an inner product space over \( \mathbb{F} \in \{ \mathbb{R}, \mathbb{C} \} \). Define the mapping
\[
\phi : V \to V^*, \quad \mathbf{v} \mapsto \phi_{\mathbf{v}},
\]
where \( \phi_{\mathbf{v}}(\mathbf{w}) := \langle \mathbf{v}, \mathbf{w} \rangle \) for all \( \mathbf{w} \in V \).

Then:
\begin{enumerate}
    \item The mapping \( \phi \) is well-defined and \(\mathbb{R}\)-linear.
    \item If \( V \) is finite-dimensional, then \( \phi \) is an isomorphism of real vector spaces.
\end{enumerate}
\end{theorem}

\begin{remark}
Note that:
\begin{itemize}
    \item If \( V \) is over \( \mathbb{R} \), then \( \phi \) is a linear map in the usual sense.
    \item If \( V \) is over \( \mathbb{C} \), then \( \phi \) is only \(\mathbb{R}\)-linear. In particular, \( \phi(i\mathbf{v}) \neq i\phi(\mathbf{v}) \), but \( \phi(2\mathbf{v}) = 2\phi(\mathbf{v}) \).
\end{itemize}
\end{remark}

\begin{example}
Let \( V \) be a complex vector space with basis \( \{ \mathbf{v}_1, \ldots, \mathbf{v}_n \} \). Any vector \( \mathbf{v} \in V \) can be written as
\[
\mathbf{v} = \sum_{j=1}^n \alpha_j \mathbf{v}_j, \quad \alpha_j = p_j + iq_j \in \mathbb{C}.
\]
Then
\[
\mathbf{v} = \sum_j p_j \mathbf{v}_j + \sum_j q_j (i \mathbf{v}_j), \quad p_j, q_j \in \mathbb{R},
\]
so the set \( \{ \mathbf{v}_1, \ldots, \mathbf{v}_n, i\mathbf{v}_1, \ldots, i\mathbf{v}_n \} \) forms a basis for \( V \) over \( \mathbb{R} \).
\end{example}

\begin{proof}
\textbf{(1) Well-definedness.}  
We need to show \( \phi_{\mathbf{v}} \in V^* \), i.e., that it defines a linear functional. Let \( \mathbf{w}_1, \mathbf{w}_2 \in V \) and \( a, b \in \mathbb{F} \). Then:
\[
\phi_{\mathbf{v}}(a\mathbf{w}_1 + b\mathbf{w}_2) = \langle \mathbf{v}, a\mathbf{w}_1 + b\mathbf{w}_2 \rangle = a \langle \mathbf{v}, \mathbf{w}_1 \rangle + b \langle \mathbf{v}, \mathbf{w}_2 \rangle = a \phi_{\mathbf{v}}(\mathbf{w}_1) + b \phi_{\mathbf{v}}(\mathbf{w}_2),
\]
so \( \phi_{\mathbf{v}} \) is linear (over \( \mathbb{R} \) or \( \mathbb{C} \), depending on \( V \)).

\textbf{(2) \(\mathbb{R}\)-linearity of \( \phi \).}  
Let \( \mathbf{v}_1, \mathbf{v}_2 \in V \) and \( c, d \in \mathbb{R} \). For any \( \mathbf{w} \in V \),
\[
\phi_{c\mathbf{v}_1 + d\mathbf{v}_2}(\mathbf{w}) = \langle c\mathbf{v}_1 + d\mathbf{v}_2, \mathbf{w} \rangle = c \langle \mathbf{v}_1, \mathbf{w} \rangle + d \langle \mathbf{v}_2, \mathbf{w} \rangle = c \phi_{\mathbf{v}_1}(\mathbf{w}) + d \phi_{\mathbf{v}_2}(\mathbf{w}),
\]
so \( \phi(c\mathbf{v}_1 + d\mathbf{v}_2) = c\phi(\mathbf{v}_1) + d\phi(\mathbf{v}_2) \), i.e., \( \phi \) is \(\mathbb{R}\)-linear.

\textbf{(3) Isomorphism when \(\dim V < \infty\).}  
Since \( V \) and \( V^* \) have the same dimension, and \( \phi \) is injective (because \( \phi_{\mathbf{v}} = 0 \) implies \( \langle \mathbf{v}, \mathbf{w} \rangle = 0 \) for all \( \mathbf{w} \), which by positive-definiteness implies \( \mathbf{v} = 0 \)), \( \phi \) is a linear isomorphism.
\end{proof}

We now relate orthogonal complement with the annihilator space:
\begin{proposition} The mapping \(\phi  : V \rightarrow  {V}^*\) defined by
$$\phi({\bf v}) := \phi_{\bf v}$$
in \autoref{thm:riesz} maps \({U}^{\perp} \leq  V\) injectively to \(\operatorname{Ann}\left( U\right)  \leq  {V}^*\). 

If \(\dim \left( V\right)  < \infty\) , then \({U}^{\perp} \cong  \operatorname{Ann}\left( U\right)\) as \(\mathbb{R}\)-vector spaces.
\end{proposition}

\begin{proof} The injectivity of \(\phi\) is given in \autoref{thm:riesz}. For any \(\mathbf{v} \in  {U}^{\perp}\) , we imply \({\phi }_{\mathbf{v}}\left( \mathbf{u}\right)  = 0\) for all \(\mathbf{u} \in  U\) , i.e. \({\phi }_{\mathbf{v}} \in  \operatorname{Ann}\left( U\right)\) . Therefore, \(\phi \left( {U}^{\perp}\right)  \leq  \operatorname{Ann}\left( U\right)\).

\medskip
Suppose \(\dim \left( V\right)  < \infty\) , by \autoref{prop:orthogonal_comple}(1),
\[
\dim \left( U\right)  + \dim \left( {U}^{\perp}\right)  = \dim \left( V\right).
\]
Since \(\dim \left( U\right)  + \dim \left( {\operatorname{Ann}\left( U\right) }\right)  = \dim \left( V\right)\) , we imply \(\dim \left( {U}^{\perp}\right)  = \dim \left( {\operatorname{Ann}\left( U\right) }\right)\) .
Consequently, the injection
\[
\phi|_{U^{\perp}}  : {U}^{\perp} \rightarrow  \operatorname{Ann}\left( U\right)
\]
is an isomorphism between \(\mathbb{R}\)-vector spaces \({U}^{\perp}\) and \(\operatorname{Ann}\left( U\right)\) .
\end{proof}
\chapter{Adjoint Operators on Inner Product Spaces}

In this chapter, we generalize the notion of taking transpose (or Hermitian transpose) of a matrix for linear operators. 


\begin{definition} Let \(T : V \rightarrow  V\) be a linear operator between inner product spaces. The \emph{adjoint} of \(T\) is defined as \(T' : V \rightarrow  V\) satisfying
\[
\left\langle  {T'\left( \mathbf{v}\right) ,\mathbf{w}}\right\rangle   = \langle \mathbf{v},T\left( \mathbf{w}\right) \rangle ,\forall \mathbf{w} \in  V \tag{10.1}
\]
\end{definition}

\begin{remark}
Previously we have another definition of the adjoint of \(T : V \rightarrow  W\), denoted as \({T}^{ * } : {W}^{ * } \rightarrow {V}^{ * }\). 
    It is an unfortunate abuse of the same terminology but with different meaning. To distinguish the two adjoints, we will use
    $$T':V \to V$$
    for the adjoint of operators of inner product spaces, and 
    $$T: W^* \to V^*$$
    for the adjoint of linear transformations.
\end{remark}

It is not clear that whether such $T':V \to V$ exists from the definition, let alone whether it is a linear operator or not. These issues are dealt with by the following:
\begin{proposition} If \(\dim \left( V\right)  < \infty\) , then \(T'\) exists, and it is unique. Moreover, \(T'\) is a linear map.
\end{proposition}

\begin{proof} Fix any \({\bf v} \in  V\) . Consider the mapping
\[
{\alpha }_{\mathbf{v}} : \mathbf{w}\overset{T}{ \rightarrow  }T\left( \mathbf{w}\right) \overset{{\phi }_{\mathbf{v}}}{ \rightarrow  }\langle \mathbf{v},T\left( \mathbf{w}\right) \rangle
\]

This is a linear transformation from \(V\) to \(\mathbb{F}\) , i.e., \({\alpha }_{\bf v} \in  V^{ * }\). By \autoref{thm:riesz}, \(\phi\) is an isomorphism from \(V\) to \(V^{ * }\). Therefore, for any \({\alpha }_{\bf v} \in  {V}^{ * }\), there exists a vector \(T'\left( {\bf v}\right)  \in  V\) such that
\[
\phi \left( {T'\left( \mathbf{ v}\right) }\right)  = {\alpha }_{\mathbf{v}} \in  {V}^{ * }
\]

Or equivalently, \({\phi }_{T'\left( \mathbf{v}\right) }\left( \mathbf{w}\right)  = {\alpha }_{\mathbf{v}}\left( \mathbf{w}\right) ,\forall \mathbf{w} \in  V\) , i.e., 
\begin{equation}\label{eq:adjoint}\left\langle  {T'\left( \mathbf{v}\right) ,\mathbf{w}}\right\rangle   = \langle \mathbf{v},T\left( \mathbf{w}\right) \rangle.\end{equation}

Therefore, from \(\bf v\) we have constructed \(T'\left(\bf v\right)\) satisfying \autoref{eq:adjoint}. Now define 
\[T' : V \rightarrow  V \quad \quad  \mathbf{v} \mapsto  T'\left( \mathbf{v}\right).\]
Since the choice of \(T'\left( v\right)\) is unique by the injectivity of \(\phi ,T'\) is well-defined. 

Now we show \(T'\) is a linear transformation: Let \({\mathbf{v}}_{1},{\mathbf{v}}_{2} \in  V,a,b \in \mathbb{F}\). For all \(\mathbf{w} \in  V\), we have
\begin{align*}
\left\langle  {T'\left( {a{\mathbf{v}}_{1} + b{\mathbf{v}}_{2}}\right) ,\mathbf{w}}\right\rangle   = \left\langle  {a{\mathbf{v}}_{1} + b{\mathbf{v}}_{2},T\left( \mathbf{w}\right) }\right\rangle
&= \bar{a}\left\langle  {{\mathbf{v}}_{1},T\left( \mathbf{w}\right) }\right\rangle   + \bar{b}\left\langle  {{\mathbf{v}}_{2},T\left( \mathbf{w}\right) }\right\rangle
\\
&= \bar{a}\left\langle  {T'\left( {\mathbf{v}}_{1}\right) ,\mathbf{w}}\right\rangle   + \bar{b}\left\langle  {T'\left( {\mathbf{v}}_{2}\right) ,\mathbf{w}}\right\rangle
\\
&= \left\langle  {aT'\left( {\mathbf{v}}_{1}\right)  + bT'\left( {\mathbf{v}}_{2}\right) ,\mathbf{w}}\right\rangle
\end{align*}

Therefore, 
\[
\left\langle  {T'\left( {a{\mathbf{v}}_{1} + b{\mathbf{v}}_{2}}\right)  - \left\lbrack  {aT'\left( {\mathbf{v}}_{1}\right)  + bT'\left( {\mathbf{v}}_{2}\right) }\right\rbrack  ,\mathbf{w}}\right\rangle   = 0,\forall \mathbf{w} \in  V.
\]
By the non-degeneracy of inner product,
\[
T'\left( {a{\mathbf{v}}_{1} + b{\mathbf{v}}_{2}}\right)  - \left\lbrack  {aT'\left( {\mathbf{v}}_{1}\right)  + bT'\left( {\mathbf{v}}_{2}\right) }\right\rbrack   = \mathbf{0},
\]
i.e., \(T'\left( {a{\mathbf{v}}_{1} + b{\mathbf{v}}_{2}}\right)  = aT'\left( {\mathbf{v}}_{1}\right)  + bT'\left( {\mathbf{v}}_{2}\right)\).
\end{proof}

\begin{example} Let \(V = ({\mathbb{R}}^{n},\langle  \cdot  , \cdot  \rangle)\) be the usual inner product. Consider the matrix-multiplication mapping \(T : \;V \rightarrow  V\) given by:
\[
T\left( \mathbf{v}\right)  = A\mathbf{v}.
\]
Then \(\left\langle  {{T}^{\prime }\left( \mathbf{v}\right) ,\mathbf{w}}\right\rangle   = \langle \mathbf{v},T\left( \mathbf{w}\right) \rangle\) implies
\[
{\left( {T}^{\prime }\left( \mathbf{v}\right) \right) }^{\mathrm{T}}\mathbf{w} = \langle \mathbf{v},\mathbf{A}\mathbf{w}\rangle
= {\mathbf{v}}^{\mathrm{T}}\mathbf{A}\mathbf{w}
= {\left( {\mathbf{A}}^{\mathrm{T}}\mathbf{v}\right) }^{\mathrm{T}}\mathbf{w}
\]
Therefore, 
\[{T}^{\prime }\left( \mathbf{v}\right)  = {A}^{\mathrm{T}}\mathbf{v}.\]

Meanwhile, for the usual inner product \(V = ({\mathbb{C}}^{n},\langle  \cdot  , \cdot  \rangle)\), the adjoint is equal to
\[{T}^{\prime }\left( \mathbf{v}\right)  = {A}^{\mathrm{H}}\mathbf{v} = \overline{A}^{\mathrm{T}}{\bf v},\]
where the extra complex conjugate comes from the sesquilinear form of $\langle  \cdot  , \cdot  \rangle$.
\end{example}

\begin{proposition} \label{prop:adjoint_matrix} Let \(T : V \rightarrow  V\) be a linear transformation, \(V\) a inner product space. Suppose that \(\mathcal{B} = \left\{  {{\mathbf{e}}_{1},\ldots ,{\mathbf{e}}_{n}}\right\}\) is an orthonormal basis of \(V\) , then
\[
{\left( {T}^{\prime }\right) }_{\mathcal{B},\mathcal{B}} = \overline{{\left( {\left( T\right) }_{\mathcal{B},\mathcal{B}}\right) }^{\mathrm{T}}}.
\]
\end{proposition}

\begin{proof} Suppose that \({\left( T\right) }_{\mathcal{B},\mathcal{B}} = \left( {a}_{ij}\right)\) , where \(T\left( {\mathbf{e}}_{j}\right)  = \mathop{\sum }\limits_{{k = 1}}^{n}{a}_{kj}{\mathbf{e}}_{k}\) , then
\[
\left\langle  {{\mathbf{e}}_{i},T\left( {\mathbf{e}}_{j}\right) }\right\rangle   = \left\langle  {{\mathbf{e}}_{i},\mathop{\sum }\limits_{{k = 1}}^{n}{a}_{kj}{\mathbf{e}}_{k}}\right\rangle
= \mathop{\sum }\limits_{{k = 1}}^{n}{a}_{kj}\left\langle  {{\mathbf{e}}_{i},{\mathbf{e}}_{k}}\right\rangle
= {a}_{ij}
\]
Also, suppose \({\left( {T}^{\prime }\right) }_{\mathcal{B},\mathcal{B}} = \left( {b}_{ij}\right)\) , we imply \({T}^{\prime }\left( {\mathbf{e}}_{j}\right)  = \mathop{\sum }\limits_{{k = 1}}^{n}{b}_{ij}{\mathbf{e}}_{k}\) , which follows that
\(
\left\langle  {{\mathbf{e}}_{i},{T}^{\prime }\left( {\mathbf{e}}_{j}\right) }\right\rangle   = {b}_{ij}
\)
i.e., \(\overline{{a}_{ji}} = {b}_{ij}\).
\end{proof}

Note that \autoref{prop:adjoint_matrix} does not hold if \(\mathcal{B}\) is not an orthonormal basis.

\section{Self-Adjoint Operators}
\begin{definition}[Self-Adjoint]
Let \(V\) be an inner product space and \(T : V \rightarrow  V\) be a linear operator. Then \(T\) is self-adjoint if \(T' = T\).
\end{definition}

\begin{example}
Let \(V = {\mathbb{C}}^{n}\), and \(\mathcal{B} = \left\{  {{\mathbf{e}}_{1},\ldots ,{\mathbf{e}}_{n}}\right\}\) be a orthonormal basis. Let \(T : V \rightarrow  V\) be given by

\[
T\left( \mathbf{v}\right)  = \mathbf{{Av}},\;\text{ where }A \in  {M}_{n \times  n}\left( \mathbb{C}\right) .
\]

Or equivalently, there exists basis \(\mathcal{B}\) such that \({\left( T\right) }_{\mathcal{B},\mathcal{B}} = A\).

In such case, \(T\) is self-adjoint if and only if \({\left( T'\right) }_{\mathcal{B},\mathcal{B}} = {\left( T\right) }_{\mathcal{B},\mathcal{B}}\), i.e., \(\overline{{\left( T\right) }_{\mathcal{B},\mathcal{B}}^{\mathrm{T}}} = {\left( T\right) }_{\mathcal{B},\mathcal{B}}\), i.e., \({A}^{\mathrm{H}} = A.\)

Therefore, \(T\left( \mathbf{v}\right)  = \mathbf{{Av}}\) is self-adjoint if and only if \({A}^{\mathrm{H}} = A\).

Moreover, if \(\mathbb{C}\) is replaced by \(\mathbb{R}\), then \(T\) is self-adjoint if and only if \(A\) is symmetric.
\end{example}

The notion of self-adjoint for linear operator is essentially the generalized notion of Hermitian for matrix that we have stuided in MAT 2040.

We also have some nice properties for self-adjoint, and the proof for which are essentially the same for the proof in the case of Hermitian matrices.

\begin{proposition}\label{prop: self-adjoint-eigenvalue}
If \(\lambda\) is an eigenvalue of a self-adjoint operator \(T\), then \(\lambda \in \mathbb{R}\).
\end{proposition}

\begin{proof}
Suppose there is an eigen-pair \(\left( {\lambda ,\mathbf{w}}\right)\) for \(\mathbf{w} \neq  \mathbf{0}\), then
\[
\lambda \langle \mathbf{w},\mathbf{w}\rangle  = \langle \mathbf{w},\lambda \mathbf{w}\rangle
= \langle \mathbf{w},T\left( \mathbf{w}\right) \rangle  = \left\langle  {T'\left( \mathbf{w}\right) ,\mathbf{w}}\right\rangle
= \langle T\left( \mathbf{w}\right) ,\mathbf{w}\rangle  = \langle \lambda \mathbf{w},\mathbf{w}\rangle
= \bar{\lambda }\langle \mathbf{w},\mathbf{w}\rangle
\]

Since \(\langle \mathbf{w},\mathbf{w}\rangle  \neq  0\) by non-degeneracy property, we have \(\lambda  = \bar{\lambda }\), i.e., \(\lambda  \in  \mathbb{R}\).
\end{proof}

\begin{proposition}\label{prop:self-adjoint-orthogonal-space}
If \(U \leq V\) is \(T\)-invariant over the self-adjoint operator \(T\), then so is \(U^\perp\).
\end{proposition}

\begin{proof}
It suffices to show \(T\left( \mathbf{v}\right)  \in  {U}^{ \bot  },\forall \mathbf{v} \in  {U}^{ \bot  }\), i.e., for any \(\mathbf{u} \in  U\), check that

\[
\langle \mathbf{u},T\left( \mathbf{v}\right) \rangle  = \left\langle  {T'\left( \mathbf{u}\right) ,\mathbf{v}}\right\rangle   = \langle T\left( \mathbf{u}\right) ,\mathbf{v}\rangle  = 0,
\]

where the last equality is because that \(T\left( \mathbf{u}\right)  \in  U\) and \(\mathbf{v} \in  {U}^{ \bot  }\). Therefore, \(T\left( \mathbf{v}\right)  \in  {U}^{ \bot  }\).
\end{proof}

\begin{theorem}\label{thm: spectral-self-aadjoint}
If \(T : V \rightarrow V\) is self-adjoint, and \(\dim(V) < \infty\), then there exists an orthonormal basis of eigenvectors of \(T\), i.e., an orthonormal basis of V such that any element from this
basis is an eigenvector of T.
\end{theorem}

\begin{proof}
We use the induction on \(\dim \left( V\right)\) :

\begin{itemize}
\item The result is trival for \(\dim \left( V\right)  = 1\).

\item Suppose that this theorem holds for all vector spaces \(V\) with \(\dim \left( V\right)  \leq  k\), then we want to show the theorem holds when \(\dim \left( V\right)  = k + 1\) :

Suppose that \(T : V \rightarrow  V\) is self-adjoint with \(\dim \left( V\right)  = k + 1\), then consider

\[
{\mathcal{X}}_{T}\left( x\right)  = det(xI - T) = {x}^{k + 1} + \cdots  + {a}_{1}x + {a}_{0},\;{a}_{i} \in  \mathbb{K}\text{ , where }\mathbb{K}\text{ denotes }\mathbb{R}\text{ or }\mathbb{C}\text{ . }
\]


\item If \(\mathbb{K} = \mathbb{C}\), then \({\mathcal{X}}_{T}\left( x\right)\) can be decomposed as

\[
{\mathcal{X}}_{T}\left( x\right)  = \left( {x - {\lambda }_{1}}\right) \cdots \left( {x - {\lambda }_{k + 1}}\right)
\]

In paricular, we obtain the eigen-pair \(\left( {{\lambda }_{1},\mathbf{v}}\right)\)

\item If \(\mathbb{K} = \mathbb{R}\), i.e., we treat real number as scalars, then
\end{itemize}

\[
{\mathcal{X}}_{T}\left( x\right)  = \left( {x - {\lambda }_{1}}\right) \cdots \left( {x - {\lambda }_{k + 1}}\right) \text{ , where }{\lambda }_{i} \in  \mathbb{C}\text{ . }
\]

By \autoref{prop: self-adjoint-eigenvalue}, all \({\lambda }_{i}\) ’s are real. Moreover, we also obtain

the eigen-pair \(\left( {{\lambda }_{1},\mathbf{v}}\right)\).
\newline

Consider \(U = \operatorname{span}\{ \mathbf{v}\}\), then

\begin{itemize}
\item \(U\) is \(T\) -invariant by \(T(v) = \lambda v\)
\item \({U}^{ \bot  }\) is \(T\) -invariant by \autoref{prop: self-adjoint-eigenvalue}
\item \(V = U \oplus  {U}^{ \bot  }\), since \(V\) is finite dimensional inner product space.
\end{itemize}

Consider \({\left. T\right| }_{{U}^{ \bot  }}\), which is a self-adjoint operator on \({U}^{ \bot  }\), with \(\dim \left( {U}^{ \bot  }\right)  = k\). By induction, there exists an orthonormal basis \(\left\{  {{\mathbf{e}}_{2},\ldots ,{\mathbf{e}}_{k + 1}}\right\}\) of eigenvectors of \(T{ \mid  }_{{U}^{ \bot  }}\).

Consider the basis \(\mathcal{B} = \left\{  {{\mathbf{v}}^{\prime } = \frac{\mathbf{v}}{\parallel \mathbf{v}\parallel} ,{\mathbf{e}}_{2},\ldots ,{\mathbf{e}}_{k + 1}}\right\}\). As a result,

\begin{enumerate}
    \item \(\mathcal{B}\) forms a basis of \(V\).
    \item All \({\mathbf{v}}^{\prime },{\mathbf{e}}_{i}\) are of norm 1 eigenvectors of \(T\).
    \item \(\mathcal{B}\) is an orthonormal set, e.g., \(\left\langle  {{\mathbf{v}}^{\prime },{\mathbf{e}}_{i}}\right\rangle   = 0\), where \({\mathbf{v}}^{\prime } \in  U\) and \({\mathbf{e}}_{i} \in  {U}^{ \bot  }\).
\end{enumerate}
Therefore, \(\mathcal{B}\) is a basis of orthonormal eigenvectors of \(V\).
\end{proof}

\begin{corollary}
If \(\dim(V) < \infty\), and \(T : V \rightarrow V\) is self-adjoint, then there exists an orthonormal basis \(\mathcal{B}\) such that
\[(T)_{\mathcal{B},\mathcal{B}} = \begin{bmatrix} \lambda_1 &  &  \\  & \ddots &  \\  &  & \lambda_n \end{bmatrix}.\]
In particular, for all real symmetric matrix \(A \in  {\mathbb{S}}^{n}\), there exists orthogonal matrix \(P\) i.e. \(\left( {{P}^{\mathrm{T}}P = {I}_{n}}\right)\) such that
\[
{P}^{-1}{AP} = \begin{bmatrix} \lambda_1 &  &  \\  & \ddots &  \\  &  & \lambda_n \end{bmatrix}.
\]
\end{corollary}
\begin{proof}
\begin{enumerate}
    \item By applying \autoref{thm: spectral-self-aadjoint}, there exists orthonormal basis of \(V\), say \(\mathcal{B} =\)  \(\left\{  {{\mathbf{v}}_{1},\ldots ,{\mathbf{v}}_{n}}\right\}\) such that \(T\left( {\mathbf{v}}_{i}\right)  = {\lambda }_{i}{\mathbf{v}}_{i}\). Directly writing the basis representation gives

    \[
    {\left( T\right) }_{\mathcal{B},\mathcal{B}} = \operatorname{diag}\left( {{\lambda }_{1},\ldots ,{\lambda }_{n}}\right) .
    \]
    
    \item For the second part, consider \(T : {\mathbb{R}}^{n} \rightarrow  {\mathbb{R}}^{n}\) by \(T\left( \mathbf{v}\right)  = \mathbf{{Av}}\). Since \({A}^{\mathrm{T}} = A\), we imply \(T\) is self-adjoint. There exists orthonormal basis \(\mathcal{B} = \left\{  {{\mathbf{v}}_{1},\ldots ,{\mathbf{v}}_{n}}\right\}\) such that

    \[
    {\left( T\right) }_{\mathcal{B},\mathcal{B}} = \operatorname{diag}\left( {{\lambda }_{1},\ldots ,{\lambda }_{n}}\right) .
    \]
    
    In particular, if \(\mathcal{A} = \left\{  {{\mathbf{e}}_{1},\ldots ,{\mathbf{e}}_{n}}\right\}\), then \({\left( T\right) }_{\mathcal{A},\mathcal{A}} = A\). We construct \(P \mathrel{\text{ := }} {\mathcal{C}}_{\mathcal{A},\mathcal{B}}\), which is the change of basis matrix from \(\mathcal{B}\) to \(\mathcal{A}\), then
    
    \[
    P = \left( \begin{array}{lll} {\mathbf{v}}_{1} & \cdots & {\mathbf{v}}_{n} \end{array}\right)
    \]
    
    and
    
    \[
    {P}^{-1}{\left( T\right) }_{\mathcal{A},\mathcal{A}}P = {\left( T\right) }_{\mathcal{B},\mathcal{B}}
    \]
    
    Or equivalently, \({P}^{-1}{AP} = \operatorname{diag}\left( {{\lambda }_{1},\ldots ,{\lambda }_{n}}\right)\), with
    
    \[
    {P}^{\mathrm{T}}P = \left( \begin{matrix} {\mathbf{v}}_{1}^{\mathrm{T}} \\  \vdots \\  {\mathbf{v}}_{n}^{\mathrm{T}} \end{matrix}\right) \left( \begin{array}{lll} {\mathbf{v}}_{1} & \cdots & {\mathbf{v}}_{n} \end{array}\right)  = I
    \]
\end{enumerate}
\end{proof}

\section{Orthogonal and Unitary Operators}

\begin{definition}[Orthogonal/Unitary Operator]
A linear operator \(T : V \rightarrow V\) over \(\mathbb{K}\) with \(\langle T(\mathbf{w}),T(\mathbf{v}) \rangle = \langle \mathbf{w},\mathbf{v}\rangle, \forall \mathbf{v},\mathbf{w} \in V\), is called:
\begin{itemize}
  \item Orthogonal if \(\mathbb{K} = \mathbb{R}\)
  \item Unitary if \(\mathbb{K} = \mathbb{C}\)
\end{itemize}
\end{definition}

\begin{proposition}\label{prop:orthogonal-unitary}
\( T \) is orthogonal (real case) or unitary (complex case) if and only if \( T' \circ T = I \).
\end{proposition}

\begin{proof}
The reverse direction is by directly checking that
\[
\langle T(\mathbf{w}), T(\mathbf{v}) \rangle 
= \langle T' \circ T(\mathbf{w}), \mathbf{v} \rangle 
= \langle \mathbf{w}, \mathbf{v} \rangle.
\]

The forward direction is by checking that \( T' \circ T(\mathbf{w}) = \mathbf{w} \) for all \( \mathbf{w} \in V \):
\[
\langle T' \circ T(\mathbf{w}), \mathbf{v} \rangle 
= \langle T(\mathbf{w}), T(\mathbf{v}) \rangle 
= \langle \mathbf{w}, \mathbf{v} \rangle 
\Rightarrow 
\langle T' \circ T(\mathbf{w}) - \mathbf{w}, \mathbf{v} \rangle = 0, \quad \forall \mathbf{v} \in V.
\]
By non-degeneracy of the inner product, \( T' \circ T(\mathbf{w}) - \mathbf{w} = 0 \), i.e.,
\[
T' \circ T(\mathbf{w}) = \mathbf{w}, \quad \forall \mathbf{w} \in V.
\]
\end{proof}

\begin{example}
Let \(T : \mathbb{K}^n \rightarrow \mathbb{K}^n\) be given by \(T(v) = Av\). Then:
\begin{itemize}
  \item If \(\mathbb{K} = \mathbb{R}\): \(T\) is orthogonal \(\Leftrightarrow A^T A = I\)
  \item If \(\mathbb{K} = \mathbb{C}\): \(T\) is unitary \(\Leftrightarrow A^H A = I\)
\end{itemize}
\end{example}

\begin{definition}[Orthogonal/Unitary Group]
\[
\text{ Orthognoal Group : }O\left( {n,\mathbb{R}}\right)  = \left\{  {A \in  {M}_{n \times  n}\left( \mathbb{R}\right)  \mid  {A}^{\mathrm{T}}A = I}\right\}
\]
\[
\text{ Unitary Group : }U\left( {n,\mathbb{C}}\right)  = \left\{  {A \in  {M}_{n \times  n}\left( \mathbb{C}\right)  \mid  {A}^{\mathrm{H}}A = I}\right\}
\]
\end{definition}


\begin{proposition}
Let \( T : V \rightarrow V \) be a linear operator on a vector space over \( \mathbb{K} \) satisfying \( T' T = I \). Then for all eigenvalues \( \lambda \) of \( T \), we have \( |\lambda| = 1 \).
\end{proposition}

\begin{proof}
Suppose we have the eigen-pair \( (\lambda, \mathbf{v}) \), then
\[
\langle T\mathbf{v}, T\mathbf{v} \rangle = \langle \mathbf{v}, \mathbf{v} \rangle 
\Leftrightarrow \langle \lambda \mathbf{v}, \lambda \mathbf{v} \rangle = \langle \mathbf{v}, \mathbf{v} \rangle 
\Leftrightarrow \overline{\lambda} \lambda \langle \mathbf{v}, \mathbf{v} \rangle = \langle \mathbf{v}, \mathbf{v} \rangle
\]

Since \( \langle \mathbf{v}, \mathbf{v} \rangle \neq 0 \) (i.e., \( \mathbf{v} \neq 0 \)), we imply \( |\lambda|^2 = 1 \), i.e., \( |\lambda| = 1 \).
\end{proof}

\begin{proposition}
Let \( T : V \to V \) be an operator on a finite dimension \( V \) over \( \mathbb{K} \) satisfying \( T'T = I \). If \( U \leq V \) is \( T \)-invariant, then \( U \) is also \( T^{-1} \)-invariant.
\end{proposition}

\begin{proof}
Since \( T'T = I \), i.e., \( T \) is invertible, we imply \( 0 \) is not a root of \( X_T(x) \), i.e., \( 0 \) is not a root of \( m_T(x) \). Since \( m_T(0) \neq 0 \), \( m_T(x) \) has the form
\[
m_T(x) = x^m + \cdots + a_1 x + a_0, \quad a_0 \neq 0,
\]
which follows that
\[
m_T(T) = T^m + \cdots + a_0 I = 0 \Rightarrow T \left( T^{m-1} + \cdots + a_1 I \right) = -a_0 I
\]
Or equivalently,
\[
T \left( -\frac{1}{a_0} (T^{m-1} + \cdots + a_1 I) \right) = I
\]
Therefore,
\[
T^{-1} = -\frac{1}{a_0} T^{m-1} - \cdots - \frac{a_2}{a_0} T - \frac{a_1}{a_0} I,
\]
i.e., the inverse \( T^{-1} \) can be expressed as a polynomial involving \( T \) only.

Since \( U \) is \( T \)-invariant, we imply \( U \) is \( T^m \)-invariant for \( m \in \mathbb{N} \), and therefore \( U \) is \( T^{-1} \)-invariant since \( T^{-1} \) is a polynomial of \( T \).
\end{proof}

\begin{proposition}
Let \( T : V \to V \) satisfy \( T'T = I \) (\( \dim(V) < \infty \)), then if \( U \leq V \) is \( T \)-invariant, it implies \( U^\perp \) is \( T \)-invariant.
\end{proposition}

\begin{proof}
Let \( {\bf v} \in U^\perp \), it suffices to show \( T({\bf v}) \in U^\perp \). Indeed, for all \({\bf  u} \in U \), we have
\[
\langle {\bf u}, T({\bf v}) \rangle = \langle T'({\bf u}), {\bf v} \rangle = \langle T^{-1}({\bf u}), {\bf v} \rangle
\]
Since \( U \) is \( T^{-1} \)-invariant, we imply \( T^{-1}({\bf u}) \in U \), and therefore
\[
\langle {\bf u}, T({\bf v}) \rangle = \langle T^{-1}({\bf u}), {\bf v} \rangle = 0 \Rightarrow T({\bf v}) \in U^\perp.
\]
\end{proof}

\begin{theorem} Let \(T : V \rightarrow  V\) be a unitary operator on finite dimension \(V\) (over \(\mathbb{C}\)),

then there exists an orthonormal basis \(\mathcal{A}\) such that

\[
{\left( T\right) }_{\mathcal{A},\mathcal{A}} = \operatorname{diag}\left( {{\lambda }_{1},\ldots ,{\lambda }_{n}}\right) ,\left| {\lambda }_{i}\right|  = 1,\forall i.
\]
\end{theorem}

\begin{proof} Note that \({\mathcal{X}}_{T}\left( x\right)\) always admits a root in \(\mathbb{C}\), so we can always find an eigenvector \(v \in  V\) of \(T\).

Then the theorem follows by the same argument before on self-adjoint operators.

\begin{itemize}
\item Consider \(U = \operatorname{span}\{ \mathbf{v}\}\)
\end{itemize}

\begin{itemize}
\item \(V = U \oplus  {U}^{ \bot  }\) and \({U}^{ \bot  }\) is \(T\) -invariant
\end{itemize}

\begin{itemize}
\item Use induction on the unitary operator \({\left. T\right| }_{{U}^{ \bot  }} : {U}^{ \bot  } \rightarrow  {U}^{ \bot  }\)
\end{itemize}
\end{proof}

The argument above fails for orthogonal operators. For instance, let \(
T : \mathbb{R} \rightarrow  {\mathbb{R}}^{2}
\) be given by 
\[
T\left( \mathbf{v}\right)  = \left( \begin{matrix} \cos \theta &  - \sin \theta \\  \sin \theta & \cos \theta  \end{matrix}\right) \mathbf{v}
\]
The matrix \(A\) is not diagonalizable over \(\mathbb{R}\). It has no real eigenvalues. However, if we treat \(A\) as \(T : {\mathbb{C}}^{2} \rightarrow  {\mathbb{C}}^{2}\) with \(T\left( \mathbf{v}\right)  = \mathbf{{Av}}\), then \({A}^{\mathrm{H}}A = I\), and therefore \(T\) is unitary. Then \(A\) is diagonalizable over \(\mathbb{C}\) with eigenvalues \({e}^{i\theta },{e}^{-{i\theta }}\)

\begin{itemize}
\item As a corollary of the theorem, for all \(A \in  {M}_{n \times  n}\left( \mathbb{C}\right)\) satisfying \({A}^{\mathrm{H}}A = I\), there exists \(P \in  {M}_{n \times  n}\left( \mathbb{C}\right)\) such that
\end{itemize}

\[
{P}^{-1}{AP} = \operatorname{diag}\left( {{\lambda }_{1},\ldots ,{\lambda }_{n}}\right) ,\;\left| {\lambda }_{i}\right|  = 1,
\]

where \(P = \left( {{\mathbf{u}}_{1},\ldots ,{\mathbf{u}}_{n}}\right)\), with \(\left\{  {{\mathbf{u}}_{1},\ldots ,{\mathbf{u}}_{n}}\right\}\) forming orthonormal basis of \({\mathbb{C}}^{n}\). In fact,

\[
{P}^{\mathrm{H}}P = \left( \begin{matrix} {\mathbf{u}}_{1}^{\mathrm{H}} \\  \vdots \\  {\mathbf{u}}_{n}^{\mathrm{H}} \end{matrix}\right) \left( {{\mathbf{u}}_{1}\cdots {\mathbf{u}}_{n}}\right)  = \left( \begin{matrix} \left\langle  {{\mathbf{u}}_{1},{\mathbf{u}}_{1}}\right\rangle  & \cdots & \left\langle  {{\mathbf{u}}_{1},{\mathbf{u}}_{n}}\right\rangle  \\  \vdots &  \ddots  & \vdots \\  \left\langle  {{\mathbf{u}}_{n},{\mathbf{u}}_{1}}\right\rangle  & \cdots & \left\langle  {{\mathbf{u}}_{n},{\mathbf{u}}_{n}}\right\rangle   \end{matrix}\right)
\]

Conclusion: all matrices \(A \in  {M}_{n \times  n}\left( \mathbb{C}\right)\) with \({A}^{\mathrm{H}}A = I\) can be written as

\[
A = P^{-1}\operatorname{diag}\left( {{\lambda }_{1},\ldots ,{\lambda }_{n}}\right) P,
\]

with some \(P\) satisfying \(P^{\mathrm{H}}P = I\).

\textbf{Notation.} Let \(U\left( n\right)  = \left\{  {A \in  {M}_{n \times  n}\left( \mathbb{C}\right)  \mid  {A}^{\mathrm{H}}A = I}\right\}\) be the unitary group, then all \(A \in  U\left( n\right)\) can be diagonalized by

\[
A = {P}^{-1}\operatorname{diag}\left( {{\lambda }_{1},\ldots ,{\lambda }_{n}}\right) P,\;P \in  U\left( n\right) .
\]

\section{Normal Operators}

\begin{definition}[Normal Operator]
Let \( T : V \to V \) be a linear operator over a \( \mathbb{C} \)-inner product vector space \( V \). We say that \( T \) is \textbf{normal} if
\[
T^\prime T = T T^\prime,
\]
where \( T^\prime \) denotes the adjoint of \( T \).
\end{definition}

\begin{example}[Examples of Normal Operators]\leavevmode
\begin{itemize}
    \item All self-adjoint operators are normal:
    \[
    T = T^\prime \Rightarrow T T^\prime = T T = T^2.
    \]

    \item All (finite-dimensional) unitary operators are normal:
    \[
    T^\prime T = T T^\prime = I.
    \]
\end{itemize}
\end{example}

\begin{proposition}
Let \( T \) be a normal operator on \( V \). Then:
\begin{enumerate}
    \item \( \|T({\bf v})\| = \|T'({\bf v})\| \) for all \( {\bf v} \in V \).\\
    In particular, \( T({\bf v}) = 0 \) if and only if \( T'({\bf v}) = 0 \).
    
    \item \( (T - \lambda I) \) is also a normal operator for any \( \lambda \in \mathbb{C} \).

    \item \( T({\bf v}) = \lambda {\bf v} \) if and only if \( T'({\bf v}) = \bar{\lambda} {\bf v} \).
\end{enumerate}
\end{proposition}

\begin{proof}
\begin{enumerate}[label=\arabic*.]
\item
\begin{align*}
\langle {\bf v}, T{\bf v} \rangle &= \langle T'{\bf v}, {\bf v} \rangle = \langle T'T {\bf v}, {\bf v} \rangle = \langle TT'{\bf v}, {\bf v} \rangle = \langle T'{\bf v}, T'{\bf v} \rangle = \|T'({\bf v})\|^2.
\end{align*}
Therefore, \( \|T({\bf v})\|^2 = \|T'({\bf v})\|^2 \Rightarrow \|T({\bf v})\| = \|T'({\bf v})\| \).

\item
By the hypothesis that \( T \) is normal, we have \( T'T = TT' \). We check:
\[
(T - \lambda I)'(T - \lambda I) = (T - \lambda I)(T - \lambda I)'.
\]
Expanding both sides:
\begin{align*}
(T - \lambda I)'(T - \lambda I) &= (T' - \bar{\lambda} I)(T - \lambda I) = T'T - \bar{\lambda} T - \lambda T' + |\lambda|^2 I, \\
(T - \lambda I)(T - \lambda I)' &= T T' - \lambda T' - \bar{\lambda} T + |\lambda|^2 I.
\end{align*}
These expressions are equal since \( T'T = TT' \), hence \( T - \lambda I \) is normal.

\item
\textbf{Forward direction:} If \( (T - \lambda I){\bf v} = 0 \), then \( T{\bf v} = \lambda {\bf v} \). By (2), \( T - \lambda I \) is normal, so:
\[
\| (T - \lambda I)'({\bf v}) \| = 0 \Rightarrow (T - \lambda I)'({\bf v}) = 0 \Rightarrow T'{\bf v} = \bar{\lambda} {\bf v}.
\]

\textbf{Reverse direction:} Suppose \( (T' - \bar{\lambda} I){\bf v} = 0 \), i.e., \( T'{\bf v} = \bar{\lambda} {\bf v} \). Then:
\[
(T' - \bar{\lambda} I) {\bf v} = 0 \Rightarrow ((T')' - \lambda I) {\bf v} = 0,
\]
but \( (T')' = T \) by the self-adjoint property in Hilbert spaces. Thus, \( (T - \lambda I){\bf v} = 0 \).

\item
Suppose \( T{\bf v} = \lambda {\bf v} \), \( T\bm{w} = \mu \bm{w} \), and \( \lambda \neq \mu \). Then:
\[
\lambda \langle {\bf v}, \bm{w} \rangle = \langle T{\bf v}, \bm{w} \rangle = \langle {\bf v}, T'\bm{w} \rangle = \langle {\bf v}, \bar{\mu} \bm{w} \rangle = \bar{\mu} \langle {\bf v}, \bm{w} \rangle.
\]
Since \( \lambda \neq \bar{\mu} \), we must have \( \langle {\bf v}, \bm{w} \rangle = 0 \). The proof is complete.
\end{enumerate}
\end{proof}

\begin{theorem}\label{thm:spectral-normal} Let \(T\) be an operator on a finite dimensional \(\left( {\dim \left( V\right)  = n}\right) \mathbb{C}\) -inner product vector space \(V\) satisfying \(T'T = TT'\). Then there is an orthonormal basis of eigenvectors of \(V\), i.e., an orthonormal basis of \(V\) such that any element from this basis is an eigenvector of \(T\).
\end{theorem}
\begin{proof} Since \({\mathcal{X}}_{T}\left( x\right)\) must have a root in \(\mathbb{C}\), there must exist an eigen-pair \(\left( {{\bf v},\lambda }\right)\) of \(T\).

\begin{itemize}
\item Construct \(U = \operatorname{span}\{ \mathbf{v}\}\), and it follows that
\end{itemize}

\[
T\mathbf{v} = \lambda \mathbf{v} \Rightarrow  U\text{ is }T\text{ -invariant. }
\]

\[
T'\mathbf{v} = \bar{\lambda }\mathbf{v} \Rightarrow  U\text{ is }T'\text{ -invariant. }
\]

\begin{itemize}
\item Moreover, we claim that \({U}^{ \bot  }\) is \(T\) and \(T'\) invariant: let \(\mathbf{w} \in  {U}^{ \bot  }\), and for all \(\mathbf{u} \in  U\),
\end{itemize}

we have

\[
\langle \mathbf{u},T\left( \mathbf{w}\right) \rangle  = \left\langle  {T'\left( \mathbf{u}\right) ,\mathbf{w}}\right\rangle   = \langle \bar{\lambda }\mathbf{u},\mathbf{w}\rangle  = \lambda \langle \mathbf{u},\mathbf{w}\rangle  = 0,
\]

i.e., \({U}^{ \bot  }\) is \(T\) invariant.

\[
\left\langle  {\mathbf{u},T'\left( \mathbf{w}\right) }\right\rangle   = \langle T\left( \mathbf{u}\right) ,\mathbf{w}\rangle  = \langle \lambda \mathbf{u},\mathbf{w}\rangle  = \bar{\lambda }\langle \mathbf{u},\mathbf{w}\rangle  = 0,
\]

which implies \({U}^{ \bot  }\) is \(T'\) invariant.

\begin{itemize}
\item Therefore, we construct the operator \({\left. T\right| }_{{U}^{ \bot  }} : {U}^{ \bot  } \rightarrow  {U}^{ \bot  }\), and
\end{itemize}

\[
TT' = T'T \Rightarrow  \left( {\left. T\right| }_{{U}^{ \bot  }}\right) \left( {T'{\left. \right| }_{{U}^{ \bot  }}}\right)  = \left( {\left. T'\right| }_{{U}^{ \bot  }}\right) \left( {\left. T\right| }_{{U}^{ \bot  }}\right) ,
\]

i.e., \(\left( {\left. T\right| }_{{U}^{ \bot  }}\right)\) is normal on \({U}^{ \bot  }\). Moreover, \(\dim \left( {U}^{ \bot  }\right)  = n - 1\).

Applying the same trick as in \autoref{thm: spectral-self-aadjoint}, we imply there exists an orthonormal basis \(\left\{  {{\mathbf{e}}_{2},\ldots ,{\mathbf{e}}_{n}}\right\}\) of eigenvectors of \(\left( {\left. T\right| }_{{U}^{ \bot  }}\right)\). Then we can argue that
\[
\mathcal{B} = \left\{  {{\mathbf{v}}^{\prime } = \frac{\mathbf{v}}{\parallel \mathbf{v}\parallel} ,{\mathbf{e}}_{2},\ldots ,{\mathbf{e}}_{k + 1}}\right\}
\]
is a basis of orthonormal eigenvectors of \(V\).
\end{proof}

\begin{corollary}[Spectral Theorem for Normal Operators]\label{cor:spectral-theorem-normal-operators}
Let \(T : V \rightarrow  V\) be a normal operator on a \(\mathbb{C}\) -inner product space with \(\dim \left( V\right)  < \infty\). Then there exists self-adjoint operators \({P}_{1},\ldots ,{P}_{k}\) such that
\[
{P}_{i}^{2} = {P}_{i},\;{P}_{i}{P}_{j} = 0,i \neq  j,\;\mathop{\sum }\limits_{{i = 1}}^{k}{P}_{i} = I,
\]
and \(T = \mathop{\sum }\limits_{{i = 1}}^{k}{\lambda }_{i}{P}_{i}\), where \({\lambda }_{i}\) ’s are the eigenvalues of \(T\).

These \({P}_{i}\) ’s are the orthogonal projections from \(V\) to the \(\lambda_{i}\) -eigenspace \(\ker (T - \lambda_{i}I)\) of \(T\), i.e., we have

\[
{\bf v} = {P}_{i}\left({\bf  v}\right)  + \left( {{\bf v} - {P}_{i}\left({\bf  v}\right) }\right)
\]

where \({P}_{i}\left({\bf  v}\right)  \in  \ker \left( {T - {\lambda }_{i}I}\right)\), and \({\bf v} - {P}_{i}\left({\bf  v}\right)  \in  {\left( \ker \left( T - {\lambda }_{i}I\right) \right) }^{ \bot  }\).
\end{corollary}
You should know how to compute the projections \( P_i \) when \( T({\bf v}) = A{\bf v} \) in MAT2040.

\textbf{Proof.}
Since \( T \) has a basis of eigenvectors, it is diagonalizable. By Proposition (8.2),
\[
m_T(x) = (x - \lambda_1) \cdots (x - \lambda_k),
\]
with distinct eigenvalues \( \lambda_i \). By Corollary (9.2), it suffices to show that each projection \( P_i \) is self-adjoint.

\begin{itemize}
\item Recall that \( P_i = a_i(T) q_i(T) := b_m T^m + \cdots + b_1 T + b_0 I \), i.e., a polynomial in \( T \). Then,
\[
P_i' = \bar{b}_m (T')^m + \cdots + \bar{b}_1 T' + \bar{b}_0 I.
\]

We claim \( P_i \) is normal. Since \( T'T = TT' \), we have:
\[
(T')^p T^q = T^q (T')^p \quad \forall\, p, q \in \mathbb{N}.
\]

Thus:
\[
P_i P_i' = \left( b_m T^m + \cdots + b_0 I \right) \left( \bar{b}_m (T')^m + \cdots + \bar{b}_0 I \right)
= \sum_{x,y=0}^m b_x \bar{b}_y T^x (T')^y
\]
\[
= \sum_{x,y=0}^m \bar{b}_y b_x (T')^y T^x = P_i' P_i.
\]

\item In general, if an operator \( S \) is self-adjoint, then it is normal. The converse is not true unless all eigenvalues of \( S \) are real. But in this case:

By \autoref{thm:spectral-normal}, a normal operator \( S \) is orthonormally diagonalizable, so its matrix representation in some orthonormal basis \( \mathcal{B} \) is
\[
S_{\mathcal{B}, \mathcal{B}} = \operatorname{diag}(\lambda_1, \ldots, \lambda_k).
\]
This same basis diagonalizes \( S' \) as well, and by part (3) of Proposition (12.1),
\[
(S')_{\mathcal{B}, \mathcal{B}} = \operatorname{diag}(\lambda_1, \ldots, \lambda_k).
\]
Therefore, \( S = S' \), so \( S \) is self-adjoint.

In particular, for \( S = P_i \), all eigenvalues are 0 or 1, which are real. Hence, each \( P_i \) is self-adjoint.
\end{itemize}

\begin{corollary}
Let \( T : V \to V \) be a linear operator on a \( \mathbb{C} \)-inner product space with \( \dim(V) < \infty \). Then \( T \) is normal if and only if \( T' = f(T) \) for some polynomial \( f(x) \in \mathbb{C}[x] \).
\end{corollary}

\textbf{Proof.} 

For the reverse direction, if \( T' = f(T) \), then
\[
T'T = f(T)T = Tf(T) = TT'.
\]

\begin{itemize}
\item For the forward direction, suppose that \( T \) is normal. Then by Corollary (12.1), we have:
\[
T = \sum_{i=1}^k \lambda_i P_i, \quad P_i = f_i(T),
\]
where the \( P_i \) are self-adjoint.

\medskip

It follows that
\[
T' = \left( \sum_{i=1}^k \lambda_i P_i \right)' 
= \sum_{i=1}^k \bar{\lambda}_i P_i' 
= \sum_{i=1}^k \bar{\lambda}_i P_i 
= \sum_{i=1}^k \bar{\lambda}_i f_i(T).
\]
\end{itemize}

\begin{remark}
The normal operator is a generalization of Hermitian matrices, and it inherits many nice properties of Hermitian operators.
\end{remark}

\chapter{Introduction to Tensor Products}

\section{Motivation of Tensor Products}
\subsection{Constrains of Bilinear Form}
Let \( U, V, W \) be vector spaces. We want to study bilinear maps \( f : U \times W \to U \), i.e., for all \( v, v_1, v_2 \in V \), \( w, w_1, w_2 \in W \), \( a, b, c, d \in \mathbb{F} \), one has
\[
f(av_1 + bv_2, w) = a f(v_1, w) + b f(v_2, w)
\]
\[
f(v, cw_1 + dw_2) = c f(v, w_1) + d f(v, w_2)
\]

\begin{example}
Let \( f : \mathbb{R}^n \times \mathbb{R}^n \to \mathbb{R} \) be with \( (u, v) \mapsto \langle u, v \rangle \).

\begin{itemize}
    \item Let \( f : M_{n \times n}(\mathbb{F}) \times M_{n \times n}(\mathbb{F}) \to M_{n \times n}(\mathbb{F}) \) be with \( f(A, B) = AB \).
    
    \item Let \( f : \mathbb{F}[x] \times \mathbb{F}[x] \to \mathbb{F} \) be with \( f(p(x), q(x)) = p(1) q(2) \).
    
    \item Let \( f : \mathbb{F}[x] \times \mathbb{F}[x] \to \mathbb{F}[x] \) be with \( f(p(x), q(x)) = p(x) q(x) \).
    
    \item \( f : \mathbb{R}^3 \times \mathbb{R}^3 \) with \( f(u, v) = u \times v \) (the \emph{cross product} in \( \mathbb{R}^3 \)).
\end{itemize}
\end{example}

\noindent Unfortunately, bilinear maps are almost always \textbf{not a linear transformation}. For instance, in the last example, one has:
\[
f(3 \mathbf{v}, 3 \mathbf{w}) = (3 \mathbf{v}) \times (3 \mathbf{w}) = 9 \mathbf{v} \times \mathbf{w} \neq 3 f(\mathbf{v}, \mathbf{w}).
\]

Since \( f \) is not a linear transformation, one cannot apply any of the tools (e.g., matrix representations, rank-nullity theorem, etc.) we developed in this course so far to study \( f \).

Indeed, the fundamental issue is that the vector space structure of \( V \times W \) is not suited to studying bilinear maps.

\medskip

As a consequence, we begin by giving an abstract, category-theoretic definition of tensor product \( V \otimes W \).

\subsection{The Universal Property of Tensor Products}

\begin{definition}[Universal Property of Tensor Product]\label{def:univtensor}
Let \(V,W\) be vector spaces. Consider the set
\[
\text{Obj} :=  \{ \phi  : V \times  W \rightarrow  U \mid  \phi \text{ is a bilinear map }\}.
\]
We say the {\bf tensor product space} \(\mathcal{T}\), or the bilinear map \(\left( {i : V \times  W \rightarrow  \mathcal{T}}\right)  \in Obj\) satisfies the {\bf universal property of tensor product} if for any \((\phi  : V \times  W \rightarrow U) \in  \mathrm{{Obj}}\), there exists an unique linear transformation \(\color{red} T_{\phi } : \mathcal{T} \rightarrow  U\) such that the diagram below commutes:


\begin{center}
\begin{tikzcd}[row sep=large, column sep=large]
V \times W \arrow[r, "i", blue] \arrow[rd, "\phi"']  & \mathcal{T}  \arrow[d, "T_{\phi}"', dashed, red]\\
& U 
\end{tikzcd} \quad \quad \quad i.e. \(\phi  = {\color{red} T_{\phi}} \circ  i\).
\end{center} 

\end{definition}

\noindent
In other words, rather than studying the \textbf{bilinear map} \( \phi \), it is better to study the \textbf{linear transformation} \( T_\phi \). Since \( \phi = T_\phi \circ i \), \( T_\phi \) contains all the information about \( \phi \), and one can apply all the theorems we know about linear algebra to study \( T_\phi \)!

\bigskip

\noindent
\textbf{\Large The question is: Does the tensor product space \( \mathcal{T} \) exist?}

\bigskip

\noindent
In the next section, we will construct \( \mathcal{T} \) explicitly, and show that it satisfies the properties we mentioned above.


\section{Tensor Product Space}

\subsection{Construction of Tensor Product Space}
We begin with defining the tensor product $\mathcal{T} = V \otimes W$ of two vector spaces $V$ and $W$. This can be generalized into any (countable) number of vector spaces.


\begin{definition} Let \(V,W\) be vector spaces. Let \(S = \{ \left( {\mathbf{v},\mathbf{w}}\right)  \mid  \mathbf{v} \in  V,\mathbf{w} \in  W\}\), we define

\[
\mathfrak{X} = \operatorname{span}\left( S\right).
\]
\end{definition}

\begin{remark}
    Note that we assume no relations on the elements \(\left( {\mathbf{v},\mathbf{w}}\right)  \in  \mathcal{S}\). In other words, $1\cdot ({\bf v}, {\bf w})$ and $1 \cdot ({\bf v}', {\bf w}') \in \mathfrak{X}$ are linearly independent unless ${\bf v} = {\bf v}'$ and ${\bf w} = {\bf w}'$ 
    
    For example, if ${\bf w} \neq {\bf 0}$, then ${\bf w} \neq 2{\bf w}$. Therefore, $(0,{\bf w})$ and $(0, 2{\bf w})$ are linearly independent and hence:
    \begin{align*} 
    2\cdot \left( {0,\mathbf{w}}\right)  &\neq  \left( {0,2\mathbf{w}}\right)\\
3\cdot \left( {0,\mathbf{w}}\right) &\neq 1\cdot \left( {0,\mathbf{w}}\right)  + 1\cdot\left( {0,{2\bf w}}\right)  \end{align*}
Similarly, if ${\bf v}, {\bf w} \neq {\bf 0}$:
\begin{align*}
\left(\mathbf{v},\mathbf{w}\right) &\neq \left( \mathbf{v},0\right)  +  \left( 0,\mathbf{w}\right). \end{align*}
The only legitimate relationship in $\mathfrak{X}$ is
\[
2\cdot \left( \mathbf{v},{\bf w}\right)  + 3\cdot \left(\mathbf{v},\mathbf{w}\right)  = 5\left( {\mathbf{v},\mathbf{w}}\right) ,
\]
yet it is not equal to $(5{\bf v},5{\bf w})$.

\noindent In other words, \(\mathcal{S}\) is a basis of \(\mathfrak{X}\), and consequently \(\mathfrak{X}\) is of uncountable dimension.
\end{remark}

\begin{definition}[Tensor Product of $V$ and $W$] Let \(\mathfrak{Y} \leq  \mathfrak{X}\) be a vector subspace spanned by vectors of the form
\[
\left\{  {1\left( {{\bf v}_1+{\bf v}_2,\mathbf{w}}\right)  - 1\left( {{\bf v}_1,\mathbf{w}}\right)  - 1\left( {{\bf v}_2,\mathbf{w}}\right) }\right\}  ,\quad \left\{  {1\left( {\mathbf{v},{\bf w}_1 + {\bf w}_2}\right)  - 1\left( {\mathbf{v},{\bf w}_1}\right)  - 1\left( {\mathbf{v},{\bf w}_2}\right) }\right\}
\]
and
\[
\{ 1\left( {k\mathbf{v},\mathbf{w}}\right)  - k\left( {\mathbf{v},\mathbf{w}}\right)  \mid  k \in  \mathbb{F}\},
\quad
\{ 1\left( {\mathbf{v},k\mathbf{w}}\right)  - k\left( {\mathbf{v},\mathbf{w}}\right)  \mid  k \in  \mathbb{F}\}
\]
for all ${\bf v}, {\bf v}_1, {\bf v}_2 \in V$, ${\bf w}, {\bf w}_1, {\bf w}_2 \in W$. Then the {\bf tensor product} \(V \otimes  W\) is defined by

\[
V \otimes  W := \mathfrak{X}/\mathfrak{Y}
\]

\noindent Also, for ${\bf v} \in V$ and ${\bf w} \in W$, we define 
\[\mathbf{v} \otimes  \mathbf{w} := \left( {\mathbf{v},\mathbf{w}}\right)  + \mathfrak{Y} \quad \in  \mathfrak{X}/\mathfrak{Y} = V \otimes W.\]
\end{definition}

\noindent Using our definition of $V \otimes W$, the expression ${\bf v} \otimes {\bf w} \in V \otimes W$ is `bilinear', for instance:
\begin{equation} \label{eq:tensorrule1}
\begin{aligned}
\left( {{\bf v}_1 + {\bf v}_2}\right)  \otimes  \mathbf{w} &= \left( {{\bf v}_1 + {\bf v}_2,\mathbf{w}}\right)  + \mathfrak{Y} \\
&= \left( {{\bf v}_1 + {\bf v}_2,\mathbf{w}}\right)  - \left\lbrack  {\left( {{\bf v}_1 + {\bf v}_2,\mathbf{w}}\right)  - \left( {{\bf v}_1,\mathbf{w}}\right)  - \left( {{\bf v}_2,\mathbf{w}}\right) }\right\rbrack   + \mathfrak{Y}
\\
&= 0\left( {{\bf v}_1 + {\bf v}_2,\mathbf{w}}\right)  + \left( {{\bf v}_1,\mathbf{w}}\right)  + \left( {{\bf v}_2,\mathbf{w}}\right)  + \mathfrak{Y}
\\
&= \left\lbrack  {\left( {{\bf v}_1,\mathbf{w}}\right)  + \mathfrak{Y}}\right\rbrack   + \left\lbrack  {\left( {{\bf v}_2,\mathbf{w}}\right)  + \mathfrak{Y}}\right\rbrack
\\
&= {\bf v}_1 \otimes  {\bf w} + {\bf v}_2 \otimes  {\bf w}
\end{aligned}
\end{equation}
Similarly, one can check that
\begin{equation} \label{eq:tensorrule2}
\begin{aligned}
\mathbf{v} \otimes  \left( {{\bf w}_1 + {\bf w}_2}\right)  &= \left( {\mathbf{v} \otimes  {\bf w}_1}\right)  + \left( {\mathbf{v} \otimes  {\bf w}_2}\right)
\\
\left( {k\mathbf{v}}\right)  \otimes  \mathbf{w} &= k\left( {\mathbf{v} \otimes  \mathbf{w}}\right)
\\
\mathbf{v} \otimes  \left( {k\mathbf{w}}\right)  &= k\left( {\mathbf{v} \otimes  \mathbf{w}}\right)
\end{aligned}
\end{equation}
Making use of the rules above, we present an example of arithmetic on tensor product spaces:
\begin{example} 
Let \(V = W = {\mathbb{R}}^2\), with
\({\mathbf{e}}_1 = \left( \begin{array}{l} 1 \\  0 \end{array}\right) ,\;{\mathbf{e}}_2 = \left( \begin{array}{l} 0 \\  1 \end{array}\right) .
\)
Then
\begin{align*}
\left( \begin{array}{l} 3 \\  1 \end{array}\right)  \otimes  \left( \begin{matrix}  - 4 \\  2 \end{matrix}\right)  &= \left( {3{\mathbf{e}}_1 + 2{\mathbf{e}}_2}\right)  \otimes  \left( {-4{\mathbf{e}}_1 + 2{\mathbf{e}}_2}\right)
\\
&= \left( {3{\mathbf{e}}_1}\right)  \otimes  \left( {-4{\mathbf{e}}_1 + 2{\mathbf{e}}_2}\right)  + \left( {\mathbf{e}}_2\right)  \otimes  \left( {-4{\mathbf{e}}_1 + 2{\mathbf{e}}_2}\right)
\\
&= \left( {3{\mathbf{e}}_1}\right)  \otimes  \left( {-4{\mathbf{e}}_1}\right)  + \left( {3{\mathbf{e}}_1}\right)  \otimes  \left( {2{\mathbf{e}}_2}\right)  + \left( {\mathbf{e}}_2\right)  \otimes  \left( {-4{\mathbf{e}}_1}\right)  + {\mathbf{e}}_2 \otimes  \left( {2{\mathbf{e}}_2}\right)
\\
&=  - {12}\left( {{\mathbf{e}}_1 \otimes  {\mathbf{e}}_1}\right)  + 6\left( {{\mathbf{e}}_1 \otimes  {\mathbf{e}}_2}\right)  - 4\left( {{\mathbf{e}}_2 \otimes  {\mathbf{e}}_1}\right)  + 2\left( {{\mathbf{e}}_2 \otimes  {\mathbf{e}}_2}\right)
\end{align*}
Exercise: Check that \({\mathbf{e}}_1 \otimes  {\mathbf{e}}_2 + {\mathbf{e}}_2 \otimes  {\mathbf{e}}_1\) cannot be re-written as
\[
\left( {a{\mathbf{e}}_1 + b{\mathbf{e}}_2}\right)  \otimes  \left( {c{\mathbf{e}}_1 + d{\mathbf{e}}_2}\right).
\]
for any $a,b,c,d \in  \mathbb{R}$.
\end{example}

\begin{remark}
The product space \(V \times  W\) is different from the tensor product space \(V \otimes  W\) in the following sense:

(a) \(\left( {\mathbf{v},\mathbf{0}}\right)  \neq  {\mathbf{0}}_{V \times  W}\) in \(V \times  W\) ; but \(\mathbf{v} \otimes  {\bf 0} \in  {\bf 0}_{V \otimes  W}\), since

\[
{\bf v} \otimes  0 = {\bf v} \otimes  \left( {0\mathbf{w}}\right)
= 0\left( {\bf v} \otimes  {\bf w}\right) = {\bf 0}_{V \otimes  W}
\]


(b) \(\left( {{\bf v}_1,{\bf w}_1}\right)  + \left( {{\bf v}_2,{\bf w}_2}\right)  = \left( {{\bf v}_1 + {\bf v}_2,{\bf w}_1 + {\bf w}_2}\right)\) ; but \({\bf v}_1 \otimes  {\bf w}_1 + {\bf v}_2 \otimes  {\bf w}_2\) cannot be

simplified further in general, unless (for instance) \({\bf v}_1 = {\bf v}_2\), so that:

\[
\mathbf{v} \otimes  {\bf w}_1 + \mathbf{v} \otimes  {\bf w}_2 = \mathbf{v} \otimes  \left( {{\bf w}_1 + {\bf w}_2}\right)
\]

As we saw in the exercise above, a general element in $V \otimes W$ {\bf cannot} is not necessarily of the form $v \otimes w$.
How does a general element in $V \otimes W$ look like?

We begin with a general element in \(\mathfrak{X}\) :
\[
{a}_1\left( {{\bf v}_1,{\bf w}_1}\right)  + \cdots  + {a}_k\left( {{\bf v}_k,{\bf w}_k}\right) ,
\]

where \(\left( {{\bf v}_{i},{\bf w}_{i}}\right)\) are distinct. Then a general element in \(\mathfrak{X}/\mathfrak{Y} \mathrel{\text{ := }} V \otimes  W\) looks like:
\begin{align*}
{a}_1\left( {{\bf v}_1,{\bf w}_1}\right)  + \cdots  + {a}_k\left( {{\bf v}_k,{\bf w}_k}\right)  + \mathfrak{Y} &= {a}_1\left( {\left( {{\bf v}_1,{\bf w}_1}\right)  + \mathfrak{Y}}\right)  + \cdots  + {a}_k\left( {\left( {{\bf v}_k,{\bf w}_k}\right)  + \mathfrak{Y}}\right)
\\
&= {a}_1\left( {{\bf v}_1 \otimes  {\bf w}_1}\right)  + \cdots  + {a}_k\left( {{\bf v}_k \otimes  {\bf w}_k}\right)
\\
&= \left( {{a}_1{\bf v}_1}\right)  \otimes  {\bf w}_1 + \cdots  + \left( {{a}_k{\bf v}_k}\right)  \otimes  {\bf w}_k
\end{align*}
Therefore, a general element in \(V \otimes  W\) is of the form
\begin{equation} \label{eq:generaltensor}
{\bf v}^{(1)} \otimes  {\bf w}^{(1)} + \cdots  + {\bf v}^{(k)} \otimes  {\bf w}^{(k)} \quad \quad ({\bf v}^{(i)} \in  V, \quad {\bf w}^{(i)} \in  W).
\end{equation}
\end{remark}

\begin{theorem}\label{thm: univ-binlinear-map} 
The bilinear map
$$i : V \times W \rightarrow V \otimes W$$ 
defined by 
$$i({\bf v}, {\bf w}) := {\bf v} \otimes {\bf w}$$ 
is in \textnormal{Obj} in \autoref{def:univtensor} (i.e. $i$ is a bilinear map). Moreover, by taking $\mathcal{T} := V \otimes W$ in \autoref{def:univtensor}, it satisfies the universal property of tensor products.
\end{theorem}
\begin{proof}
By Equations \eqref{eq:tensorrule1} and \eqref{eq:tensorrule2}, \( i \) satisfies:
\begin{align*}
    i(\mathbf{v}_1 + \mathbf{v}_2, \mathbf{w}) &= i(\mathbf{v}_1, \mathbf{w}) + i(\mathbf{v}_2, \mathbf{w}), \\
    i(\mathbf{v}, \mathbf{w}_1 + \mathbf{w}_2) &= i(\mathbf{v}, \mathbf{w}_1) + i(\mathbf{v}, \mathbf{w}_2), \\
    i(a \mathbf{v}, \mathbf{w}) &= a \cdot i(\mathbf{v}, \mathbf{w}), \quad i(\mathbf{v}, b \mathbf{w}) = b \cdot i(\mathbf{v}, \mathbf{w}),
\end{align*}
so \( i \in \textnormal{Obj} \) as in \autoref{def:univtensor}.

Now, let \( (\phi : V \times W \to U) \in \textnormal{Obj} \) be any bilinear map into some vector space \( U \). Since \( \phi \) is bilinear, it respects the relations that generate \( \mathfrak{Y} \). Therefore, we can define a linear map
\[
T_{\phi} : V \otimes W \to U, \quad T_{\phi}(\mathbf{v} \otimes \mathbf{w}) := \phi(\mathbf{v}, \mathbf{w}),
\]
and extend linearly to all of \( V \otimes W \) (recall all elements of $V \otimes W$ are of the form ${\bf v}^{(1)} \otimes {\bf w}^{(1)} + \dots + {\bf v}^{(k)} \otimes {\bf w}^{(k)}$). This map is well-defined because if two elements \( (\mathbf{v}, \mathbf{w}) \) and \( (\mathbf{v}', \mathbf{w}') \) are equivalent in \( \mathfrak{X}/\mathfrak{Y} \), i.e. $({\bf v},{\bf w}) - ({\bf v}',{\bf w}') \in \mathfrak{Y}$, then their images under \( \phi \) are equal due to bilinearity.

It follows directly from the definition that the diagram
\[
\begin{tikzcd}[row sep=large, column sep=large]
V \times W \arrow[r, "i", blue] \arrow[rd, "\phi"'] & V \otimes W \arrow[d, "T_{\phi}"', dashed, red] \\
& U
\end{tikzcd}
\]
commutes, i.e., \( \phi = T_\phi \circ i \).

For uniqueness: suppose there exists another linear map \( T' : V \otimes W \to U \) such that \( \phi = T' \circ i \). Then both \( T_\phi \) and \( T' \) agree on all pure tensors \( \mathbf{v} \otimes \mathbf{w} \), and hence, by linearity, on all of \( V \otimes W \). So \( T' = T_\phi \).

Therefore, the pair \( (V \otimes W, i) \) satisfies the universal property of the tensor product.
\end{proof}

\chapter{Multilinear \& Exterior Products}

\section{Multilinear Tensor Product}
In this section, we will generalize the tensor product of more than 2 vector spaces:
\begin{definition}[Tensor Product among multiple vector spaces]
Let $V_1, \ldots, V_p$ be vector spaces over $\mathbb{F}$. Let 
$$S = \{(\mathbf{v}_1, \ldots, \mathbf{v}_p) \mid \mathbf{v}_i \in V_i\}$$ 
(we assume no relations among distinct elements in $S$), and define $\mathfrak{X} = \text{span}(S)$. Then the tensor product space $V_1 \otimes \cdots \otimes V_p = \mathfrak{X} / \mathfrak{Y}$, where $\mathfrak{Y}$ is the vector subspace of $\mathfrak{X}$ spanned by vectors of the form
$$(\mathbf{v}_1, \ldots, \mathbf{v}_i + \mathbf{v}_i', \ldots, \mathbf{v}_p) - (\mathbf{v}_1, \ldots, \mathbf{v}_i, \ldots, \mathbf{v}_p) - (\mathbf{v}_1, \ldots, \mathbf{v}_i', \ldots, \mathbf{v}_p),$$ 
 and
$$(\mathbf{v}_1, \ldots, \alpha \mathbf{v}_i, \ldots, \mathbf{v}_p) - \alpha (\mathbf{v}_1, \ldots, \mathbf{v}_i, \ldots, \mathbf{v}_p)$$
for $i = 1, 2, \ldots, p$.

Moreover, we define
\[\mathbf{v}_1 \otimes \cdots \otimes \mathbf{v}_p := [(\mathbf{v}_1, \ldots, \mathbf{v}_p) + \mathcal{Y}] \in V_1 \otimes \cdots \otimes V_p.\]
\end{definition}

\begin{remark} \label{rmk:mult_tensor}
Similar to the tensor product of two vector spaces, we have:
  \begin{align*}
\mathbf{v}_1 \otimes \cdots \otimes (\alpha \mathbf{v}_i + \beta \mathbf{v}_i') \otimes \cdots \otimes \mathbf{v}_p = \alpha (\mathbf{v}_1 \otimes \cdots \otimes \mathbf{v}_i \otimes \cdots \otimes \mathbf{v}_p) + \beta (\mathbf{v}_1 \otimes \cdots \otimes \mathbf{v}_i' \otimes \cdots \otimes \mathbf{v}_p).
  \end{align*}
Also, a general vector in $V_1 \otimes \cdots \otimes V_p$ is:
  \[
    \sum_{i=1}^n \left( \mathbf{w}_1^{(i)} \otimes \cdots \otimes \mathbf{w}_p^{(i)} \right), \quad \text{where } \mathbf{w}_j^{(i)} \in V_j, j = 1, \ldots, p.
  \]

Finally, let $\mathcal{B}_i = \{\mathbf{v}_i^{(1)}, \ldots, \mathbf{v}_i^{(d_i)}\}$ be a basis of $V_i$, $i = 1, \ldots, p$, then
  \[
    \mathcal{B} = \left\{ \mathbf{v}_1^{(\alpha_1)} \otimes \cdots \otimes \mathbf{v}_p^{(\alpha_p)} \mid 1 \leq \alpha_i \leq d_i \right\}
  \]
  is a basis of $V_1 \otimes \cdots \otimes V_p$. As a result,
  \[
    \dim(V_1 \otimes \cdots \otimes V_p) = d_1 \times \cdots \times d_p = \dim(V_1) \times \cdots \times \dim(V_p).
  \]
\end{remark}

We now introduce the universal property of multilinear tensor product:
\begin{theorem}[Universal Property of Multilinear Tensor]\label{thm:multi_tensor_universal}
Let 
$$\mathrm{Obj} = \{ \phi : V_1 \times \cdots \times V_p \to W \mid \phi \text{ is a $p$-linear map} \},$$ 
i.e. for any $\phi \in \mathrm{Obj}$:
\[
\phi(\mathbf{v}_1, \ldots, \alpha \mathbf{v}_i + \beta \mathbf{v}_i', \ldots, \mathbf{v}_p) 
= \alpha \phi(\mathbf{v}_1, \ldots, \mathbf{v}_i, \ldots, \mathbf{v}_p)
+ \beta \phi(\mathbf{v}_1, \ldots, \mathbf{v}_i', \ldots, \mathbf{v}_p),
\]
for all $\mathbf{v}_i, \mathbf{v}_i' \in \mathbf{v}_i$, $i = 1, \ldots, p$, $\forall \alpha, \beta \in \mathbb{F}$. Then
\[i : V_1 \times \cdots \times V_p \to V_1 \otimes \cdots \otimes V_p\]
defined by
\[({\bf v}_1, \ldots, {\bf v}_p) \mapsto {\bf v}_1 \otimes \cdots \otimes {\bf v}_p,
\]
is in $\mathrm{Obj}$. Moreover, $(i : V_1 \times \cdots \times V_p \to V_1 \otimes \cdots \otimes V_p) \in \mathrm{Obj}$ satisfies the \emph{universal property} of multilinear tensor product, that is, for all $(\phi: V_1 \times \dots \times V_p \to W) \in \mathrm{Obj}$, there exists a unique linear transformation
\[
\bar{\phi} : V_1 \otimes \cdots \otimes V_p \to W
\]
such that the diagram commutes:
\[
\begin{tikzcd}[column sep=large, row sep=large]
V_1 \times \cdots \times V_p 
\arrow[r, "i", blue] \arrow[dr, "\phi"] &
V_1 \otimes \cdots \otimes V_p 
\arrow[d, "\bar{\phi}", red] \\
& W
\end{tikzcd}
\]
\noindent In other words, \(\phi = \bar{\phi} \circ i\). 
\end{theorem}

As an application of the universal property of multilinear tensor product, we can construct linear transformation of multilinear tensor product spaces:
\begin{corollary} \label{cor:multi_linear_trans}
Let \( T_i : V_i \to V_i' \) be a linear transformation for \( 1 \leq i \leq p \). Then there is a unique linear transformation
\[
(T_1 \otimes \cdots \otimes T_p) : V_1 \otimes \cdots \otimes V_p \to V_1' \otimes \cdots \otimes V_p'
\]
such that
\[
(T_1 \otimes \cdots \otimes T_p)(\mathbf{v}_1 \otimes \cdots \otimes \mathbf{v}_p) = T_1(\mathbf{v}_1) \otimes \cdots \otimes T_p(\mathbf{v}_p).
\]
\end{corollary}

\begin{proof}
Construct the mapping
\[
\phi : V_1 \times \cdots \times V_p \to V_1' \otimes \cdots \otimes V_p'
\]
defined by
\[
(\mathbf{v}_1, \ldots, \mathbf{v}_p) \mapsto T_1(\mathbf{v}_1) \otimes \cdots \otimes T_p(\mathbf{v}_p),
\]
which is indeed \( p \)-linear.  
By the universal property, we obtain the unique linear transformation
\[
\bar{\phi} : V_1 \otimes \cdots \otimes V_p \to V_1' \otimes \cdots \otimes V_p'
\]
satisfying our desired properties.
\end{proof}

\section{Exterior Products}
From now on, we only consider \( V_1 = \cdots = V_p = V \).  
Then for any linear transformation \( T : V \to W \), we define
\[
T^{\otimes p} : V \otimes \cdots \otimes V \to W \otimes \cdots \otimes W.
\]
We use the shorthand notation \( V^{\otimes p} \) to denote
\[
\underbrace{V \otimes \cdots \otimes V}_{\text{$p$ terms in total}}.
\]
\begin{definition}[Alternating Map]
A $p$-linear map $\phi : V \times \cdots \times V \to W$ is called \textbf{alternating} if
\[
\phi(\mathbf{v}_1, \ldots, \mathbf{v}_i, \ldots, \mathbf{v}_j, \ldots, \mathbf{v}_p) = 0_W, \quad \text{whenever } \mathbf{v}_i = \mathbf{v}_j \text{ for } i \neq j.
\]
We also say $\phi$ is \emph{$p$-alternating}.
\end{definition}

\begin{example}
The cross product map is an alternating bilinear map:
\[
\phi : \mathbb{R}^3 \times \mathbb{R}^3 \to \mathbb{R}^3, \quad (\mathbf{v}, \mathbf{w}) \mapsto \mathbf{v} \times \mathbf{w}.
\]
It satisfies:
\begin{itemize}
  \item $\phi$ is bilinear,
  \item $\phi(\mathbf{v}, \mathbf{v}) = \mathbf{v} \times \mathbf{v} = \mathbf{0}.$
\end{itemize}
\end{example}

\begin{example}
The determinant map is also alternating:
\[
\phi : \underbrace{\mathbb{F}^n \times \cdots \times \mathbb{F}^n}_{\text{$n$ terms in total}} \to \mathbb{F}, \quad (\mathbf{v}_1, \ldots, \mathbf{v}_n) \mapsto \det([\mathbf{v}_1, \mathbf{v}_2, \cdots, \mathbf{v}_n])
\]
It satisfies:
\begin{itemize}
  \item $\phi$ is $n$-linear (MAT 2040 knowledge),
  \item $\phi$ is alternating (again, from MAT 2040 since the determinant of a matrix with repeated columns is zero).
\end{itemize}
\end{example}

\begin{theorem}[Universal Property for Exterior Power]\label{thm:univ-exterior}
Let 
\[ \mathrm{Obj} := \left\{ \phi : V \times \cdots \times V \to W \,\middle|\, \phi \text{ is $p$-alternating} \right\} .\] 
Then there exists a universal object
\[
(\Lambda : V \times \cdots \times V \longrightarrow E) \in \mathrm{Obj}
\]
such that for every \( (\phi: V \times \dots \times V \to W) \in \mathrm{Obj} \), there exists a unique linear map
\[
\overline{\phi} : E \longrightarrow W
\]
making the following diagram commute:
\[
\begin{tikzcd}[row sep=large, column sep=large]
  V \times \cdots \times V 
    \arrow[r, "\Lambda", dashed, blue] 
    \arrow[dr, "\phi"'] 
    & E 
    \arrow[d, "\overline{\phi}", red] \\
   & W 
\end{tikzcd}
\]
In other words, \(\phi = \overline{\phi} \circ \Lambda.
\)
\end{theorem}

We now construct the universal object $(\Lambda: V \times \dots \times V \to E) \in \mathrm{Obj}$:
\begin{definition}[Wedge Products]\label{def:wedgeproduct}
Let $V$ be a vector space. Consider the subspace $U \leq V^{\otimes p}$ spanned by the vectors
$$U = \mathrm{span}\{ \cdots \otimes {\bf v} \otimes \cdots \otimes {\bf v} \otimes \cdots\ |\ {\bf v} \in V\}$$
with a repeated entry. Then the $p$-\textbf{exterior power} is defined by the quotient space:
\[
\wedge^p V := V^{\otimes p} / U,
\]
Similar to the case of $V^{\otimes p}$, we define the vector
\[
\mathbf{v}_1 \wedge \cdots \wedge \mathbf{v}_p := \mathbf{v}_1 \otimes \cdots \otimes \mathbf{v}_p + U \in \wedge^p V.
\]
\end{definition}

We claim that the mapping
\[
\Lambda : V \times \cdots \times V \to  \wedge^p V,\] 
defined by \[(\mathbf{v}_1, \ldots, \mathbf{v}_p) \mapsto \mathbf{v}_1 \wedge \cdots \wedge \mathbf{v}_p
\]
is in $\mathrm{Obj}$, and it satisfies the universal property of exterior power (c.f. \autoref{thm:univ-exterior}).
\begin{proposition} \label{prop:wedge_alternating}
The $p$-exterior power $\wedge^p V$ is $p$-linear: 
\begin{align*}
\mathbf{v}_1 \wedge \cdots \wedge (a\mathbf{v}_i + b \mathbf{v}_i') \wedge \cdots \wedge \mathbf{v}_p &= a \left(\mathbf{v}_1 \wedge \cdots \wedge \mathbf{v}_i \wedge \cdots \wedge \mathbf{v}_p\right) + b \left(\mathbf{v}_1 \wedge \cdots \wedge \mathbf{v}_i' \wedge \cdots \wedge \mathbf{v}_p\right)
\end{align*}
and alternating:
  \begin{align*}
    \mathbf{v}_1 \wedge \cdots \wedge \mathbf{v} \wedge \cdots \wedge \mathbf{v} \wedge \cdots \wedge \mathbf{v}_p = {\bf 0}_{\wedge^pV}
  \end{align*}
Consequently, $\Lambda: V \times \dots \times V \to \wedge^pV$ is in $\mathrm{Obj}$.
\end{proposition}
\begin{proof}
The proof is trivial - as an example, for the second statement:
$$\mathbf{v}_1 \wedge \cdots \wedge \mathbf{v} \wedge \cdots \wedge \mathbf{v} \wedge \cdots \wedge \mathbf{v}_p := \mathbf{v}_1 \otimes \cdots \otimes \mathbf{v} \otimes \cdots \otimes \mathbf{v} \otimes \cdots \otimes \mathbf{v}_p + U.$$
Since $\mathbf{v}_1 \otimes \cdots \otimes \mathbf{v} \otimes \cdots \otimes \mathbf{v} \in U$ by definition, it is equal to  ${\bf 0} + U = {\bf 0}_{\wedge^p V}$. 
\end{proof}

We leave it to the readers to check that $\Lambda: V \times \dots \times V \to \wedge^p V$ satisfies the universal property for exterior power (\autoref{thm:univ-exterior}). 

\begin{remark} \label{rmk:wedge_alternating}
The wedge product satisfies the sign reversal property:
  \begin{equation} \label{eq:reversal}
    \mathbf{v}_1 \wedge \cdots \wedge \mathbf{v} \wedge \cdots \wedge \mathbf{w} \wedge \cdots \wedge \mathbf{v}_p = -\mathbf{v}_1 \wedge \cdots \wedge \mathbf{w} \wedge \cdots \wedge \mathbf{v} \wedge \cdots \wedge \mathbf{v}_p.
  \end{equation}
For instance, one has:
\[(\mathbf{v} + \mathbf{w}) \wedge (\mathbf{v} + \mathbf{w}) = {\bf 0} \ \Rightarrow \ \mathbf{v} \wedge \mathbf{v} + \mathbf{v} \wedge \mathbf{w} + \mathbf{w} \wedge \mathbf{v} + \mathbf{w} \wedge \mathbf{w} = {\bf 0} \ \Rightarrow \ \mathbf{v} \wedge \mathbf{w} + \mathbf{w} \wedge \mathbf{v} = {\bf 0}.\]
One can easily generalize the above proof to obtain \autoref{eq:reversal}.
\end{remark}


\begin{proposition}
If $\dim(V) = n$ and $0 \leq p \leq n$, then
\[
\dim(\wedge^p V) = \binom{n}{p}.
\]
\end{proposition}
\begin{proof} 
Let $\{\mathbf{v}_1, \dots, \mathbf{v}_n\}$ be a basis of $V$, then $\{ \mathbf{v}_{i_1} \otimes \cdots \otimes \mathbf{v}_{i_p} \mid 1 \leq i_k \leq n\}$ forms a basis of $V^{\otimes p}$. Since $\pi_{\wedge^p V} : V^{\otimes p} \to V^{\otimes p}/U = \wedge^p V$ is surjective, the image 
\[ 
\{ \pi_{\wedge^pV}(\mathbf{v}_{i_1} \otimes \cdots \otimes \mathbf{v}_{i_p}) = \mathbf{v}_{i_1} \wedge \cdots \wedge \mathbf{v}_{i_p} \mid 1 \leq i_k\leq n \}
\]
spans $\wedge^p V$.
    
By \autoref{prop:wedge_alternating} and \autoref{rmk:wedge_alternating}, many of the terms in the above expression are zero or repeated. For instance, using \autoref{rmk:wedge_alternating}, one can reorder the expression $\mathbf{v}_{i_1} \wedge \cdots \wedge \mathbf{v}_{i_p}$, so that (up to $\pm$) the indices
$i_1 < \dots < i_p$ are in ascending order. 
After removing the repeated entries, one conclude that
$$\mathcal{B} := \{ \mathbf{v}_{i_1} \wedge \cdots \wedge \mathbf{v}_{i_p} \mid 1 \leq i_1 < \dots < i_k \leq n\}$$
spans $\wedge^pV.$

We omit the proof that $\mathcal{B}$ is linearly independent - its proof is similar to that of \autoref{thm:tensorbasis}. Consequently, $\mathcal{B}$ is a basis of $\wedge^pV$ which has cardinality equal to $\binom{n}{p}$.
\end{proof}

Let $T: V \to V$ be a linear operator. By applying the universal property for exterior power (\autoref{thm:univ-exterior}), one can obtain as in \autoref{cor:multi_linear_trans}:
\begin{corollary} \label{cor:exterior_transformation}
Let $V$ be a vector space, and $T:V \to V$ be a linear operator. For all $0 \leq p \leq \dim(V)$, there exists a unique linear operator
  \[
  T^{\wedge^p} : \wedge^p V \to \wedge^p V
  \]
defined by:
  \[
  \mathbf{v}_1 \wedge \cdots \wedge \mathbf{v}_p \mapsto T(\mathbf{v}_1) \wedge \cdots \wedge T(\mathbf{v}_p).
  \]
\end{corollary}



\section{The Determinant}
To end this chapter, we define the determinant for all linear operators $T:V \to V$ on finite-dimensional vector space $V$ directly. As a consequence, we will prove that
\[
\det(T) = \det \left( T_{\mathcal{B}, \mathcal{B}} \right) \ \text{for any basis } \mathcal{B} \text{ of } T.
\]
In other words, if $T({\bf x}) = A{\bf x}$ is a matrix transformation, then one has $\det(T) = \det(A)$.

Moreover, we will have a natural way of proving 
$\det(T \circ S) = \det(T) \det(S)$, which in term will imply the well-known formula of $\det(AB) = \det(A) \det(B)$ in MAT2040.

In short, our definition will give generalize the notion of determinant of matrices to linear operators, and give a natural explanation of why determinant is multiplicative.

\begin{definition}[Determinant for Linear Operators] Let $\dim(V) = n$. Then
\( \dim(\wedge^n V) = \binom{n}{n} = 1.\)
By \autoref{cor:exterior_transformation} we have a linear operator 
$$T^{\wedge n} : \wedge^n V \to \wedge^n V.$$
on the $1$-dimensional vector space $\wedge^n V$.

Therefore, for all $\tau \in \wedge^n V$, there exists $\alpha_T \in \mathbb{F}$ such that
\[T^{\wedge n}(\tau) = \alpha_T \tau,\]
and we define the \emph{determinant} of $T$ by:
\(\det(T) := \alpha_T,\) so that
$T^{\wedge n}(\tau) = \det(T) \tau $
Note that this definition does not depend on any choice of basis of $V$.
\end{definition}

To compute the determinant, one can take {\bf any} basis ${\mathbf{v}_1, \dots, \mathbf{v}_n}$ of $V$, and
\[ \wedge^n V = \mathrm{span}(\mathbf{v}_1 \wedge \cdots \wedge \mathbf{v}_n).\] 
Then one computes
$$T^{\wedge n}(\mathbf{v}_1 \wedge \cdots \wedge \mathbf{v}_n) := T(\mathbf{v}_1) \wedge \cdots \wedge T(\mathbf{v}_n) = \det(T) \mathbf{v}_1 \wedge \cdots \wedge \mathbf{v}_n.$$

\begin{example}
\begin{enumerate}
    \item Suppose that $T = I : V \to V$ be the identity operator. Then
\[
I^{\wedge n}(\mathbf{v}_1 \wedge \cdots \wedge \mathbf{v}_n) = I(\mathbf{v}_1) \wedge \cdots \wedge I(\mathbf{v}_n) = \mathbf{v}_1 \wedge \cdots \wedge \mathbf{v}_n = 1 \cdot (\mathbf{v}_1 \wedge \cdots \wedge \mathbf{v}_n).
\]
Therefore, $\det(I) = 1$.

\item Suppose that $T : V \to V$ is diagonalizable with $\{\mathbf{w}_1, \dots, \mathbf{w}_n\}$ forming an eigen-basis of $T$. As a result,
\[
T^{\wedge n}(\mathbf{w}_1 \wedge \cdots \wedge \mathbf{w}_n) = T(\mathbf{w}_1) \wedge T(\mathbf{w}_2) \cdots \wedge T(\mathbf{w}_n) 
=(\lambda_1 \mathbf{w}_1) \wedge \cdots \wedge (\lambda_n \mathbf{w}_n) = (\lambda_1 \cdots \lambda_n)\mathbf{w}_1 \wedge \cdots \wedge \mathbf{w}_n
\]
Therefore, $\det(T) = \lambda_1 \cdots \lambda_n$.
\end{enumerate}
\end{example}

\begin{proposition}
Let $T, S : V \to V$ be linear transformations, then
\[
(T \circ S)^{\wedge^p} : \wedge^p V \to \wedge^p V
\]
with $T^{\wedge^p}, S^{\wedge^p} : \wedge^p V \to \wedge^p V$ satisfies
\[
(T \circ S)^{\wedge^p} = (T^{\wedge^p}) \circ (S^{\wedge^p})
\]
\end{proposition}

\begin{proof}
Pick any basis $\{ \mathbf{v}_{i_1} \wedge \cdots \wedge \mathbf{v}_{i_p} \mid 1 \leq i_1 < \cdots < i_p \leq n \}$ of $\wedge^p V$. Then
\[
(T \circ S)^{\wedge^p}(\mathbf{v}_{i_1} \wedge \cdots \wedge \mathbf{v}_{i_p}) = (T \circ S)(\mathbf{v}_{i_1}) \wedge \cdots \wedge (T \circ S)(\mathbf{v}_{i_p})
\]
On the other hand,
\begin{align*}
    (T^{\wedge^p}) \circ (S^{\wedge^p})(\mathbf{v}_{i_1} \wedge \cdots \wedge \mathbf{v}_{i_p}) = T^{\wedge^p}(S(\mathbf{v}_{i_1}) \wedge \cdots \wedge S(\mathbf{v}_{i_p})) = 
    (T \circ S)(\mathbf{v}_{i_1}) \wedge \cdots \wedge (T \circ S)(\mathbf{v}_{i_p})
\end{align*}
\end{proof}

\[
T : \mathbb{R}^2 \to \mathbb{R}^2
\quad \text{with} \quad
T\begin{pmatrix} x \\ y \end{pmatrix}
= \begin{pmatrix}
a & b \\
c & d
\end{pmatrix}
\begin{pmatrix}
x \\ y
\end{pmatrix}.
\]

\begin{example}
Let $T: \mathbb{F}^2 \to \mathbb{F}^2$ be the matrix transformation given by $T({\bf x}) = A{\bf x}$, where $A = \begin{pmatrix}
    a & b \\ c & d
\end{pmatrix}$.
We take the usual basis $\{{\bf e}_1, {\bf e}_2\}$ to compute the determinant of $T$:
\begin{align*}
\det(T) \cdot ({\bf e}_1 \wedge {\bf e}_2)
&= T({\bf e}_1) \wedge T({\bf e}_2) \\
&= \begin{pmatrix} a \\ c \end{pmatrix} \wedge \begin{pmatrix} b \\ d \end{pmatrix} \\
&= (a {\bf e}_1 + c {\bf e}_2) \wedge (b {\bf e}_1 + d {\bf e}_2) \\
&= (a b) {\bf e}_1 \wedge {\bf e}_1 + (a d) {\bf e}_1 \wedge {\bf e}_2 + (c b) {\bf e}_2 \wedge {\bf e}_1 + (c d) {\bf e}_2 \wedge {\bf e}_2 \\
&= (a d) {\bf e}_1 \wedge {\bf e}_2 + (c b) {\bf e}_2 \wedge {\bf e}_1 \\
&= (a d - b c) {\bf e}_1 \wedge {\bf e}_2.
\end{align*}
where we have used ${\bf e}_1 \wedge {\bf e}_1 = {\bf e}_2 \wedge {\bf e}_2 = {\bf 0}$ and ${\bf e}_1 \wedge {\bf e}_2 = -({\bf e}_2 \wedge {\bf e}_1)$ by \autoref{prop:wedge_alternating} and \autoref{rmk:wedge_alternating}. Therefore, we conclude:
\[
\det(T) = \Delta(A) = a d - b c.
\]
(to avoid confusion, we denote by $\Delta(A)$ the determinant of $A$ as defined in MAT2040.)
\end{example}

For general $n \times n$-matrices $A$, one also has:
\begin{theorem}\label{thm:det_linear_operator_matrix}
Let \( V = \mathbb{F}^n \), and let
\[
T : V \to V, \quad \text{with } T(\mathbf{v}) = A \mathbf{v}, \quad A \in M_{n \times n}(\mathbb{F}).
\]
Then \(\det(T) = \Delta(A).\)
\end{theorem}
\begin{proof}
Let \( \{ {\bf e}_1, \dots, {\bf e}_n \} \) be the usual basis of \( V \cong \mathbb{F}^n \), and  \( \{ {\bf a}_1, \dots, {\bf a}_n \} \) be the columns of $A$. Then:
\begin{equation} \label{eq:determinant}
\begin{aligned}
\det(T) \cdot {\bf e}_1 \wedge \cdots \wedge {\bf e}_n
  &= T^{\wedge n}({\bf e}_1 \wedge \cdots \wedge {\bf e}_n) \\
  &= T({\bf e}_1) \wedge \cdots \wedge T({\bf e}_n) \\
  &= {\bf a}_1 \wedge \cdots \wedge {\bf a}_n.
\end{aligned}
\end{equation}
We wish to show $\Delta(T) = \det(A)$. Indeed, by MAT2040, $\Delta(A)$ is uniquely determined by the following:
\begin{itemize}
    \item $\Delta(I_{n \times n}) = 1$;
    \item $\Delta( \cdots | \alpha{\bf a}_i + \beta {\bf a}_i'| \cdots) = \alpha\Delta( \cdots | {\bf a}_i| \cdots) + \beta\Delta( \cdots | {\bf a}_i'| \cdots)$.
    \item $\Delta( \cdots | {\bf v}| \cdots| {\bf v}| \cdots) = 0$
\end{itemize}
By the multilinearity and alternating property of the wedge product, one can easily see that $\det(T)$ satisfies all the above properties. For instance, for the last point, one can apply \autoref{eq:determinant} and get:
\[
\det(T) {\bf e}_1 \wedge \cdots \wedge {\bf e}_n = \cdots \wedge {\bf v} \wedge \cdots \wedge {\bf v} \wedge \cdots = {\bf 0} = 0({\bf e}_1 \wedge \cdots \wedge {\bf e}_n).
\]
Therefore, \(\det(T) = 0\) as stated in the third point. And the result follows.
\end{proof}

So we have just proved that our definition of $\det$ matches with that of the usual definition of determinant of matrices. Moreover, we have:
\begin{corollary}
\label{cor:det-composition}
\[
\det(T \circ S) = \det(T)\, \det(S)
\]
\end{corollary}

\begin{proof}
Let \( \{\mathbf{v}_1, \dots, \mathbf{v}_n\} \) be a basis of \( V \), so that \( \mathbf{v}_1 \wedge \cdots \wedge \mathbf{v}_n \) is a basis of \( \wedge^n V \). Then:
\begin{align*}
\det(T \circ S) \cdot \mathbf{v}_1 \wedge \cdots \wedge \mathbf{v}_n 
  &= (T \circ S)^{\wedge n} (\mathbf{v}_1 \wedge \cdots \wedge \mathbf{v}_n) \\
  &= T^{\wedge n} \left( S^{\wedge n}(\mathbf{v}_1 \wedge \cdots \wedge \mathbf{v}_n) \right) \\
  &= T^{\wedge n} \left( \det(S) \cdot \mathbf{v}_1 \wedge \cdots \wedge \mathbf{v}_n \right) \\
  &= \det(S) \cdot T^{\wedge n}(\mathbf{v}_1 \wedge \cdots \wedge \mathbf{v}_n) \\
  &= \det(S) \cdot \det(T) \cdot \mathbf{v}_1 \wedge \cdots \wedge \mathbf{v}_n.
\end{align*}
Since \( \mathbf{v}_1 \wedge \cdots \wedge \mathbf{v}_n \neq 0 \), it follows that
\[
\det(T \circ S) = \det(T)\det(S).
\]
\end{proof}




\chapter{Multilinear \& Exterior Products}

\section{Multilinear Tensor Product}
In this section, we will generalize the tensor product of more than 2 vector spaces:
\begin{definition}[Tensor Product among multiple vector spaces]
Let $V_1, \ldots, V_p$ be vector spaces over $\mathbb{F}$. Let 
$$S = \{(\mathbf{v}_1, \ldots, \mathbf{v}_p) \mid \mathbf{v}_i \in V_i\}$$ 
(we assume no relations among distinct elements in $S$), and define $\mathfrak{X} = \text{span}(S)$. Then the tensor product space $V_1 \otimes \cdots \otimes V_p = \mathfrak{X} / \mathfrak{Y}$, where $\mathfrak{Y}$ is the vector subspace of $\mathfrak{X}$ spanned by vectors of the form
$$(\mathbf{v}_1, \ldots, \mathbf{v}_i + \mathbf{v}_i', \ldots, \mathbf{v}_p) - (\mathbf{v}_1, \ldots, \mathbf{v}_i, \ldots, \mathbf{v}_p) - (\mathbf{v}_1, \ldots, \mathbf{v}_i', \ldots, \mathbf{v}_p),$$ 
 and
$$(\mathbf{v}_1, \ldots, \alpha \mathbf{v}_i, \ldots, \mathbf{v}_p) - \alpha (\mathbf{v}_1, \ldots, \mathbf{v}_i, \ldots, \mathbf{v}_p)$$
for $i = 1, 2, \ldots, p$.

Moreover, we define
\[\mathbf{v}_1 \otimes \cdots \otimes \mathbf{v}_p := [(\mathbf{v}_1, \ldots, \mathbf{v}_p) + \mathcal{Y}] \in V_1 \otimes \cdots \otimes V_p.\]
\end{definition}

\begin{remark} \label{rmk:mult_tensor}
Similar to the tensor product of two vector spaces, we have:
  \begin{align*}
\mathbf{v}_1 \otimes \cdots \otimes (\alpha \mathbf{v}_i + \beta \mathbf{v}_i') \otimes \cdots \otimes \mathbf{v}_p = \alpha (\mathbf{v}_1 \otimes \cdots \otimes \mathbf{v}_i \otimes \cdots \otimes \mathbf{v}_p) + \beta (\mathbf{v}_1 \otimes \cdots \otimes \mathbf{v}_i' \otimes \cdots \otimes \mathbf{v}_p).
  \end{align*}
Also, a general vector in $V_1 \otimes \cdots \otimes V_p$ is:
  \[
    \sum_{i=1}^n \left( \mathbf{w}_1^{(i)} \otimes \cdots \otimes \mathbf{w}_p^{(i)} \right), \quad \text{where } \mathbf{w}_j^{(i)} \in V_j, j = 1, \ldots, p.
  \]

Finally, let $\mathcal{B}_i = \{\mathbf{v}_i^{(1)}, \ldots, \mathbf{v}_i^{(d_i)}\}$ be a basis of $V_i$, $i = 1, \ldots, p$, then
  \[
    \mathcal{B} = \left\{ \mathbf{v}_1^{(\alpha_1)} \otimes \cdots \otimes \mathbf{v}_p^{(\alpha_p)} \mid 1 \leq \alpha_i \leq d_i \right\}
  \]
  is a basis of $V_1 \otimes \cdots \otimes V_p$. As a result,
  \[
    \dim(V_1 \otimes \cdots \otimes V_p) = d_1 \times \cdots \times d_p = \dim(V_1) \times \cdots \times \dim(V_p).
  \]
\end{remark}

We now introduce the universal property of multilinear tensor product:
\begin{theorem}[Universal Property of Multilinear Tensor]\label{thm:multi_tensor_universal}
Let 
$$\mathrm{Obj} = \{ \phi : V_1 \times \cdots \times V_p \to W \mid \phi \text{ is a $p$-linear map} \},$$ 
i.e. for any $\phi \in \mathrm{Obj}$:
\[
\phi(\mathbf{v}_1, \ldots, \alpha \mathbf{v}_i + \beta \mathbf{v}_i', \ldots, \mathbf{v}_p) 
= \alpha \phi(\mathbf{v}_1, \ldots, \mathbf{v}_i, \ldots, \mathbf{v}_p)
+ \beta \phi(\mathbf{v}_1, \ldots, \mathbf{v}_i', \ldots, \mathbf{v}_p),
\]
for all $\mathbf{v}_i, \mathbf{v}_i' \in \mathbf{v}_i$, $i = 1, \ldots, p$, $\forall \alpha, \beta \in \mathbb{F}$. Then
\[i : V_1 \times \cdots \times V_p \to V_1 \otimes \cdots \otimes V_p\]
defined by
\[({\bf v}_1, \ldots, {\bf v}_p) \mapsto {\bf v}_1 \otimes \cdots \otimes {\bf v}_p,
\]
is in $\mathrm{Obj}$. Moreover, $(i : V_1 \times \cdots \times V_p \to V_1 \otimes \cdots \otimes V_p) \in \mathrm{Obj}$ satisfies the \emph{universal property} of multilinear tensor product, that is, for all $(\phi: V_1 \times \dots \times V_p \to W) \in \mathrm{Obj}$, there exists a unique linear transformation
\[
\bar{\phi} : V_1 \otimes \cdots \otimes V_p \to W
\]
such that the diagram commutes:
\[
\begin{tikzcd}[column sep=large, row sep=large]
V_1 \times \cdots \times V_p 
\arrow[r, "i", blue] \arrow[dr, "\phi"] &
V_1 \otimes \cdots \otimes V_p 
\arrow[d, "\bar{\phi}", red] \\
& W
\end{tikzcd}
\]
\noindent In other words, \(\phi = \bar{\phi} \circ i\). 
\end{theorem}

As an application of the universal property of multilinear tensor product, we can construct linear transformation of multilinear tensor product spaces:
\begin{corollary} \label{cor:multi_linear_trans}
Let \( T_i : V_i \to V_i' \) be a linear transformation for \( 1 \leq i \leq p \). Then there is a unique linear transformation
\[
(T_1 \otimes \cdots \otimes T_p) : V_1 \otimes \cdots \otimes V_p \to V_1' \otimes \cdots \otimes V_p'
\]
such that
\[
(T_1 \otimes \cdots \otimes T_p)(\mathbf{v}_1 \otimes \cdots \otimes \mathbf{v}_p) = T_1(\mathbf{v}_1) \otimes \cdots \otimes T_p(\mathbf{v}_p).
\]
\end{corollary}

\begin{proof}
Construct the mapping
\[
\phi : V_1 \times \cdots \times V_p \to V_1' \otimes \cdots \otimes V_p'
\]
defined by
\[
(\mathbf{v}_1, \ldots, \mathbf{v}_p) \mapsto T_1(\mathbf{v}_1) \otimes \cdots \otimes T_p(\mathbf{v}_p),
\]
which is indeed \( p \)-linear.  
By the universal property, we obtain the unique linear transformation
\[
\bar{\phi} : V_1 \otimes \cdots \otimes V_p \to V_1' \otimes \cdots \otimes V_p'
\]
satisfying our desired properties.
\end{proof}

\section{Exterior Products}
From now on, we only consider \( V_1 = \cdots = V_p = V \).  
Then for any linear transformation \( T : V \to W \), we define
\[
T^{\otimes p} : V \otimes \cdots \otimes V \to W \otimes \cdots \otimes W.
\]
We use the shorthand notation \( V^{\otimes p} \) to denote
\[
\underbrace{V \otimes \cdots \otimes V}_{\text{$p$ terms in total}}.
\]
\begin{definition}[Alternating Map]
A $p$-linear map $\phi : V \times \cdots \times V \to W$ is called \textbf{alternating} if
\[
\phi(\mathbf{v}_1, \ldots, \mathbf{v}_i, \ldots, \mathbf{v}_j, \ldots, \mathbf{v}_p) = 0_W, \quad \text{whenever } \mathbf{v}_i = \mathbf{v}_j \text{ for } i \neq j.
\]
We also say $\phi$ is \emph{$p$-alternating}.
\end{definition}

\begin{example}
The cross product map is an alternating bilinear map:
\[
\phi : \mathbb{R}^3 \times \mathbb{R}^3 \to \mathbb{R}^3, \quad (\mathbf{v}, \mathbf{w}) \mapsto \mathbf{v} \times \mathbf{w}.
\]
It satisfies:
\begin{itemize}
  \item $\phi$ is bilinear,
  \item $\phi(\mathbf{v}, \mathbf{v}) = \mathbf{v} \times \mathbf{v} = \mathbf{0}.$
\end{itemize}
\end{example}

\begin{example}
The determinant map is also alternating:
\[
\phi : \underbrace{\mathbb{F}^n \times \cdots \times \mathbb{F}^n}_{\text{$n$ terms in total}} \to \mathbb{F}, \quad (\mathbf{v}_1, \ldots, \mathbf{v}_n) \mapsto \det([\mathbf{v}_1, \mathbf{v}_2, \cdots, \mathbf{v}_n])
\]
It satisfies:
\begin{itemize}
  \item $\phi$ is $n$-linear (MAT 2040 knowledge),
  \item $\phi$ is alternating (again, from MAT 2040 since the determinant of a matrix with repeated columns is zero).
\end{itemize}
\end{example}

\begin{theorem}[Universal Property for Exterior Power]\label{thm:univ-exterior}
Let 
\[ \mathrm{Obj} := \left\{ \phi : V \times \cdots \times V \to W \,\middle|\, \phi \text{ is $p$-alternating} \right\} .\] 
Then there exists a universal object
\[
(\Lambda : V \times \cdots \times V \longrightarrow E) \in \mathrm{Obj}
\]
such that for every \( (\phi: V \times \dots \times V \to W) \in \mathrm{Obj} \), there exists a unique linear map
\[
\overline{\phi} : E \longrightarrow W
\]
making the following diagram commute:
\[
\begin{tikzcd}[row sep=large, column sep=large]
  V \times \cdots \times V 
    \arrow[r, "\Lambda", dashed, blue] 
    \arrow[dr, "\phi"'] 
    & E 
    \arrow[d, "\overline{\phi}", red] \\
   & W 
\end{tikzcd}
\]
In other words, \(\phi = \overline{\phi} \circ \Lambda.
\)
\end{theorem}

We now construct the universal object $(\Lambda: V \times \dots \times V \to E) \in \mathrm{Obj}$:
\begin{definition}[Wedge Products]\label{def:wedgeproduct}
Let $V$ be a vector space. Consider the subspace $U \leq V^{\otimes p}$ spanned by the vectors
$$U = \mathrm{span}\{ \cdots \otimes {\bf v} \otimes \cdots \otimes {\bf v} \otimes \cdots\ |\ {\bf v} \in V\}$$
with a repeated entry. Then the $p$-\textbf{exterior power} is defined by the quotient space:
\[
\wedge^p V := V^{\otimes p} / U,
\]
Similar to the case of $V^{\otimes p}$, we define the vector
\[
\mathbf{v}_1 \wedge \cdots \wedge \mathbf{v}_p := \mathbf{v}_1 \otimes \cdots \otimes \mathbf{v}_p + U \in \wedge^p V.
\]
\end{definition}

We claim that the mapping
\[
\Lambda : V \times \cdots \times V \to  \wedge^p V,\] 
defined by \[(\mathbf{v}_1, \ldots, \mathbf{v}_p) \mapsto \mathbf{v}_1 \wedge \cdots \wedge \mathbf{v}_p
\]
is in $\mathrm{Obj}$, and it satisfies the universal property of exterior power (c.f. \autoref{thm:univ-exterior}).
\begin{proposition} \label{prop:wedge_alternating}
The $p$-exterior power $\wedge^p V$ is $p$-linear: 
\begin{align*}
\mathbf{v}_1 \wedge \cdots \wedge (a\mathbf{v}_i + b \mathbf{v}_i') \wedge \cdots \wedge \mathbf{v}_p &= a \left(\mathbf{v}_1 \wedge \cdots \wedge \mathbf{v}_i \wedge \cdots \wedge \mathbf{v}_p\right) + b \left(\mathbf{v}_1 \wedge \cdots \wedge \mathbf{v}_i' \wedge \cdots \wedge \mathbf{v}_p\right)
\end{align*}
and alternating:
  \begin{align*}
    \mathbf{v}_1 \wedge \cdots \wedge \mathbf{v} \wedge \cdots \wedge \mathbf{v} \wedge \cdots \wedge \mathbf{v}_p = {\bf 0}_{\wedge^pV}
  \end{align*}
Consequently, $\Lambda: V \times \dots \times V \to \wedge^pV$ is in $\mathrm{Obj}$.
\end{proposition}
\begin{proof}
The proof is trivial - as an example, for the second statement:
$$\mathbf{v}_1 \wedge \cdots \wedge \mathbf{v} \wedge \cdots \wedge \mathbf{v} \wedge \cdots \wedge \mathbf{v}_p := \mathbf{v}_1 \otimes \cdots \otimes \mathbf{v} \otimes \cdots \otimes \mathbf{v} \otimes \cdots \otimes \mathbf{v}_p + U.$$
Since $\mathbf{v}_1 \otimes \cdots \otimes \mathbf{v} \otimes \cdots \otimes \mathbf{v} \in U$ by definition, it is equal to  ${\bf 0} + U = {\bf 0}_{\wedge^p V}$. 
\end{proof}

We leave it to the readers to check that $\Lambda: V \times \dots \times V \to \wedge^p V$ satisfies the universal property for exterior power (\autoref{thm:univ-exterior}). 

\begin{remark} \label{rmk:wedge_alternating}
The wedge product satisfies the sign reversal property:
  \begin{equation} \label{eq:reversal}
    \mathbf{v}_1 \wedge \cdots \wedge \mathbf{v} \wedge \cdots \wedge \mathbf{w} \wedge \cdots \wedge \mathbf{v}_p = -\mathbf{v}_1 \wedge \cdots \wedge \mathbf{w} \wedge \cdots \wedge \mathbf{v} \wedge \cdots \wedge \mathbf{v}_p.
  \end{equation}
For instance, one has:
\[(\mathbf{v} + \mathbf{w}) \wedge (\mathbf{v} + \mathbf{w}) = {\bf 0} \ \Rightarrow \ \mathbf{v} \wedge \mathbf{v} + \mathbf{v} \wedge \mathbf{w} + \mathbf{w} \wedge \mathbf{v} + \mathbf{w} \wedge \mathbf{w} = {\bf 0} \ \Rightarrow \ \mathbf{v} \wedge \mathbf{w} + \mathbf{w} \wedge \mathbf{v} = {\bf 0}.\]
One can easily generalize the above proof to obtain \autoref{eq:reversal}.
\end{remark}


\begin{proposition}
If $\dim(V) = n$ and $0 \leq p \leq n$, then
\[
\dim(\wedge^p V) = \binom{n}{p}.
\]
\end{proposition}
\begin{proof} 
Let $\{\mathbf{v}_1, \dots, \mathbf{v}_n\}$ be a basis of $V$, then $\{ \mathbf{v}_{i_1} \otimes \cdots \otimes \mathbf{v}_{i_p} \mid 1 \leq i_k \leq n\}$ forms a basis of $V^{\otimes p}$. Since $\pi_{\wedge^p V} : V^{\otimes p} \to V^{\otimes p}/U = \wedge^p V$ is surjective, the image 
\[ 
\{ \pi_{\wedge^pV}(\mathbf{v}_{i_1} \otimes \cdots \otimes \mathbf{v}_{i_p}) = \mathbf{v}_{i_1} \wedge \cdots \wedge \mathbf{v}_{i_p} \mid 1 \leq i_k\leq n \}
\]
spans $\wedge^p V$.
    
By \autoref{prop:wedge_alternating} and \autoref{rmk:wedge_alternating}, many of the terms in the above expression are zero or repeated. For instance, using \autoref{rmk:wedge_alternating}, one can reorder the expression $\mathbf{v}_{i_1} \wedge \cdots \wedge \mathbf{v}_{i_p}$, so that (up to $\pm$) the indices
$i_1 < \dots < i_p$ are in ascending order. 
After removing the repeated entries, one conclude that
$$\mathcal{B} := \{ \mathbf{v}_{i_1} \wedge \cdots \wedge \mathbf{v}_{i_p} \mid 1 \leq i_1 < \dots < i_k \leq n\}$$
spans $\wedge^pV.$

We omit the proof that $\mathcal{B}$ is linearly independent - its proof is similar to that of \autoref{thm:tensorbasis}. Consequently, $\mathcal{B}$ is a basis of $\wedge^pV$ which has cardinality equal to $\binom{n}{p}$.
\end{proof}

Let $T: V \to V$ be a linear operator. By applying the universal property for exterior power (\autoref{thm:univ-exterior}), one can obtain as in \autoref{cor:multi_linear_trans}:
\begin{corollary} \label{cor:exterior_transformation}
Let $V$ be a vector space, and $T:V \to V$ be a linear operator. For all $0 \leq p \leq \dim(V)$, there exists a unique linear operator
  \[
  T^{\wedge^p} : \wedge^p V \to \wedge^p V
  \]
defined by:
  \[
  \mathbf{v}_1 \wedge \cdots \wedge \mathbf{v}_p \mapsto T(\mathbf{v}_1) \wedge \cdots \wedge T(\mathbf{v}_p).
  \]
\end{corollary}



\section{The Determinant}
To end this Chapter, we define the determinant for all linear operators $T:V \to V$ on finite-dimensional vector space $V$ directly. As a consequence, we will prove that
\[
\det(T) = \det \left( (T)_{\mathcal{B}, \mathcal{B}} \right) \ \text{for any basis } \mathcal{B} \text{ of } T.
\]
which implies that if $T({\bf x}) = A{\bf x}$ is a matrix transformation, one has $\det(T) = \det(A)$.

Moreover, we will have a natural way of proving 
$\det(T \circ S) = \det(T) \det(S)$, which in term will imply the well-known formula of $\det(AB) = \det(A) \det(B)$ in MAT2040.

In short, our definition will give generalize the notion of determinant of matrices to linear operators, and give a natural explanation of why determinant is multiplicative.

\begin{definition}[Determinant for Linear Operators] Let $\dim(V) = n$. Then
\( \dim(\wedge^n V) = \binom{n}{n} = 1.\)
By \autoref{cor:exterior_transformation} we have a linear operator 
$$T^{\wedge n} : \wedge^n V \to \wedge^n V.$$
on the $1$-dimensional vector space $\wedge^n V$.

Therefore, for all $\tau \in \wedge^n V$, there exists $\alpha_T \in \mathbb{F}$ such that
\[T^{\wedge n}(\tau) = \alpha_T \tau,\]
and we define the \emph{determinant} of $T$ by:
\(\det(T) := \alpha_T,\) so that
$T^{\wedge n}(\tau) = \det(T) \tau $
Note that this definition does not depend on any choice of basis of $V$.
\end{definition}

To compute the determinant, one can take {\bf any} basis ${\mathbf{v}_1, \dots, \mathbf{v}_n}$ of $V$, and
\[ \wedge^n V = \mathrm{span}(\mathbf{v}_1 \wedge \cdots \wedge \mathbf{v}_n).\] 
Then one computes
$$T^{\wedge n}(\mathbf{v}_1 \wedge \cdots \wedge \mathbf{v}_n) := T(\mathbf{v}_1) \wedge \cdots \wedge T(\mathbf{v}_n) = \det(T) \mathbf{v}_1 \wedge \cdots \wedge \mathbf{v}_n.$$

\begin{example}
\begin{enumerate}
    \item Suppose that $T = I : V \to V$ be the identity operator. Then
\[
I^{\wedge n}(\mathbf{v}_1 \wedge \cdots \wedge \mathbf{v}_n) = I(\mathbf{v}_1) \wedge \cdots \wedge I(\mathbf{v}_n) = \mathbf{v}_1 \wedge \cdots \wedge \mathbf{v}_n = 1 \cdot (\mathbf{v}_1 \wedge \cdots \wedge \mathbf{v}_n).
\]
Therefore, $\det(I) = 1$.

\item Suppose that $T : V \to V$ is diagonalizable with $\{\mathbf{w}_1, \dots, \mathbf{w}_n\}$ forming an eigen-basis of $T$. As a result,
\[
T^{\wedge n}(\mathbf{w}_1 \wedge \cdots \wedge \mathbf{w}_n) = T(\mathbf{w}_1) \wedge T(\mathbf{w}_2) \cdots \wedge T(\mathbf{w}_n) 
=(\lambda_1 \mathbf{w}_1) \wedge \cdots \wedge (\lambda_n \mathbf{w}_n) = (\lambda_1 \cdots \lambda_n)\mathbf{w}_1 \wedge \cdots \wedge \mathbf{w}_n
\]
Therefore, $\det(T) = \lambda_1 \cdots \lambda_n$.
\end{enumerate}
\end{example}

\begin{proposition}
Let $T, S : V \to V$ be linear transformations, then
\[
(T \circ S)^{\wedge^p} : \wedge^p V \to \wedge^p V
\]
with $T^{\wedge^p}, S^{\wedge^p} : \wedge^p V \to \wedge^p V$ satisfies
\[
(T \circ S)^{\wedge^p} = (T^{\wedge^p}) \circ (S^{\wedge^p})
\]
\end{proposition}

\begin{proof}
Pick any basis $\{ \mathbf{v}_{i_1} \wedge \cdots \wedge \mathbf{v}_{i_p} \mid 1 \leq i_1 < \cdots < i_p \leq n \}$ of $\wedge^p V$. Then
\[
(T \circ S)^{\wedge^p}(\mathbf{v}_{i_1} \wedge \cdots \wedge \mathbf{v}_{i_p}) = (T \circ S)(\mathbf{v}_{i_1}) \wedge \cdots \wedge (T \circ S)(\mathbf{v}_{i_p})
\]
On the other hand,
\begin{align*}
    (T^{\wedge^p}) \circ (S^{\wedge^p})(\mathbf{v}_{i_1} \wedge \cdots \wedge \mathbf{v}_{i_p}) = T^{\wedge^p}(S(\mathbf{v}_{i_1}) \wedge \cdots \wedge S(\mathbf{v}_{i_p})) = 
    (T \circ S)(\mathbf{v}_{i_1}) \wedge \cdots \wedge (T \circ S)(\mathbf{v}_{i_p})
\end{align*}
\end{proof}

\[
T : \mathbb{R}^2 \to \mathbb{R}^2
\quad \text{with} \quad
T\begin{pmatrix} x \\ y \end{pmatrix}
= \begin{pmatrix}
a & b \\
c & d
\end{pmatrix}
\begin{pmatrix}
x \\ y
\end{pmatrix}.
\]

\begin{example}
Let $T: \mathbb{F}^2 \to \mathbb{F}^2$ be the matrix transformation given by $T({\bf x}) = A{\bf x}$, where $A = \begin{pmatrix}
    a & b \\ c & d
\end{pmatrix}$.
We take the usual basis $\{{\bf e}_1, {\bf e}_2\}$ to compute the determinant of $T$:
\begin{align*}
\det(T) \cdot ({\bf e}_1 \wedge {\bf e}_2)
&= T({\bf e}_1) \wedge T({\bf e}_2) \\
&= \begin{pmatrix} a \\ c \end{pmatrix} \wedge \begin{pmatrix} b \\ d \end{pmatrix} \\
&= (a {\bf e}_1 + c {\bf e}_2) \wedge (b {\bf e}_1 + d {\bf e}_2) \\
&= (a b) {\bf e}_1 \wedge {\bf e}_1 + (a d) {\bf e}_1 \wedge {\bf e}_2 + (c b) {\bf e}_2 \wedge {\bf e}_1 + (c d) {\bf e}_2 \wedge {\bf e}_2 \\
&= (a d) {\bf e}_1 \wedge {\bf e}_2 + (c b) {\bf e}_2 \wedge {\bf e}_1 \\
&= (a d - b c) {\bf e}_1 \wedge {\bf e}_2.
\end{align*}
where we have used ${\bf e}_1 \wedge {\bf e}_1 = {\bf e}_2 \wedge {\bf e}_2 = {\bf 0}$ and ${\bf e}_1 \wedge {\bf e}_2 = -({\bf e}_2 \wedge {\bf e}_1)$ by \autoref{prop:wedge_alternating} and \autoref{rmk:wedge_alternating}. Therefore, we conclude:
\[
\det(T) = \Delta(A) = a d - b c.
\]
(to avoid confusion, we denote by $\Delta(A)$ the determinant of $A$ as defined in MAT2040.)
\end{example}

For general $n \times n$-matrices $A$, one also has:
\begin{theorem}\label{thm:det_linear_operator_matrix}
Let \( V = \mathbb{F}^n \), and let
\[
T : V \to V, \quad \text{with } T(\mathbf{v}) = A \mathbf{v}, \quad A \in M_{n \times n}(\mathbb{F}).
\]
Then \(\det(T) = \Delta(A).\)
\end{theorem}
\begin{proof}
Let \( \{ {\bf e}_1, \dots, {\bf e}_n \} \) be the usual basis of \( V \cong \mathbb{F}^n \), and  \( \{ {\bf a}_1, \dots, {\bf a}_n \} \) be the columns of $A$. Then:
\begin{equation} \label{eq:determinant}
\begin{aligned}
\det(T) \cdot {\bf e}_1 \wedge \cdots \wedge {\bf e}_n
  &= T^{\wedge n}({\bf e}_1 \wedge \cdots \wedge {\bf e}_n) \\
  &= T({\bf e}_1) \wedge \cdots \wedge T({\bf e}_n) \\
  &= {\bf a}_1 \wedge \cdots \wedge {\bf a}_n.
\end{aligned}
\end{equation}
We wish to show $\Delta(T) = \det(A)$. Indeed, by MAT2040, $\Delta(A)$ is uniquely determined by the following:
\begin{itemize}
    \item $\Delta(I_{n \times n}) = 1$;
    \item $\Delta( \cdots | \alpha{\bf a}_i + \beta {\bf a}_i'| \cdots) = \alpha\Delta( \cdots | {\bf a}_i| \cdots) + \beta\Delta( \cdots | {\bf a}_i'| \cdots)$.
    \item $\Delta( \cdots | {\bf v}| \cdots| {\bf v}| \cdots) = 0$
\end{itemize}
By the multilinearity and alternating property of the wedge product, one can easily see that $\det(T)$ satisfies all the above properties. For instance, for the last point, one can apply \autoref{eq:determinant} and get:
\[
\det(T) {\bf e}_1 \wedge \cdots \wedge {\bf e}_n = \cdots \wedge {\bf v} \wedge \cdots \wedge {\bf v} \wedge \cdots = {\bf 0} = 0({\bf e}_1 \wedge \cdots \wedge {\bf e}_n).
\]
Therefore, \(\det(T) = 0\) as stated in the third point. And the result follows.
\end{proof}

So we have just proved that our definition of $\det$ matches with that of the usual definition of determinant of matrices. Moreover, we have:
\begin{corollary}
\label{cor:det-composition}
\[
\det(T \circ S) = \det(T)\, \det(S)
\]
\end{corollary}

\begin{proof}
Let \( \{\mathbf{v}_1, \dots, \mathbf{v}_n\} \) be a basis of \( V \), so that \( \mathbf{v}_1 \wedge \cdots \wedge \mathbf{v}_n \) is a basis of \( \wedge^n V \). Then:
\begin{align*}
\det(T \circ S) \cdot \mathbf{v}_1 \wedge \cdots \wedge \mathbf{v}_n 
  &= (T \circ S)^{\wedge n} (\mathbf{v}_1 \wedge \cdots \wedge \mathbf{v}_n) \\
  &= T^{\wedge n} \left( S^{\wedge n}(\mathbf{v}_1 \wedge \cdots \wedge \mathbf{v}_n) \right) \\
  &= T^{\wedge n} \left( \det(S) \cdot \mathbf{v}_1 \wedge \cdots \wedge \mathbf{v}_n \right) \\
  &= \det(S) \cdot T^{\wedge n}(\mathbf{v}_1 \wedge \cdots \wedge \mathbf{v}_n) \\
  &= \det(S) \cdot \det(T) \cdot \mathbf{v}_1 \wedge \cdots \wedge \mathbf{v}_n.
\end{align*}
Since \( \mathbf{v}_1 \wedge \cdots \wedge \mathbf{v}_n \neq 0 \), it follows that
\[
\det(T \circ S) = \det(T)\det(S).
\]
\end{proof}






\end{document}
