\chapter{Introduction to Tensor Products}

\section{Motivation}
Let \( U, V, W \) be vector spaces. We want to study bilinear maps \( f : V \times W \to U \), i.e., for all \( {\bf v}, {\bf v}_1, {\bf v}_2 \in V \), \( {\bf w}, {\bf w}_1, {\bf w}_2 \in W \), \( a, b, c, d \in \mathbb{F} \), one has
\[
f(a{\bf v}_1 + b{\bf v}_2, {\bf w}) = a f({\bf v}_1, {\bf w}) + b f({\bf v}_2, {\bf w})
\]
\[
f({\bf v}, c{\bf w}_1 + d{\bf w}_2) = c f(v, {\bf w_1}) + d f({\bf v}, {\bf w}_2)
\]

\begin{example}
\begin{itemize}
\item \( f : \mathbb{R}^n \times \mathbb{R}^n \to \mathbb{R} \) defined by the usual inner product \( ({\bf u}, {\bf v}) \mapsto \langle {\bf u}, {\bf v} \rangle \).


    \item \( f : M_{n \times n}(\mathbb{F}) \times M_{n \times n}(\mathbb{F}) \to M_{n \times n}(\mathbb{F}) \) defined by matrix multiplication \( f(A, B) = AB \).

    \item \( f : \mathbb{F}[x] \times \mathbb{F}[x] \to \mathbb{F}[x] \) defined by multiplication \( f(p(x), q(x)) = p(x) q(x) \).
    
    \item \( f : \mathbb{F}[x] \times \mathbb{F}[x] \to \mathbb{F} \) defined by \( f(p(x), q(x)) = p(1) q(2) \).
    
    \item \( f : \mathbb{R}^3 \times \mathbb{R}^3 \) defined by the \emph{cross product} \( f({\bf u}, {\bf v}) = {\bf u} \times {\bf v} \) .
\end{itemize}
\end{example}

\noindent Unfortunately, bilinear maps are almost always \textbf{not a linear transformation}. For instance, in the last example, one has:
\[
f(3 \mathbf{v}, 3 \mathbf{w}) = (3 \mathbf{v}) \times (3 \mathbf{w}) = 9 \mathbf{v} \times \mathbf{w} \neq 3 f(\mathbf{v}, \mathbf{w}).
\]

Since \( f \) is not a linear transformation, one cannot apply any of the tools (e.g., matrix representations, rank-nullity theorem, etc.) we developed in this course so far to study \( f \).

Indeed, the fundamental issue is that the vector space structure of \( V \times W \) is not suited to studying bilinear maps.

\medskip

As a consequence, we begin by giving an abstract, category-theoretic definition of tensor product \( V \otimes W \).

\subsection{Universal Property of Tensor Products}
\begin{definition}[Universal Property of Tensor Product]\label{def:univtensor}
Let \(V,W\) be vector spaces. Consider the set
\[
\text{Obj} :=  \{ \phi  : V \times  W \rightarrow  U \mid  \phi \text{ is a bilinear map }\}.
\]
We say the {\bf tensor product space} \(\mathcal{T}\), or the bilinear map \(\left( {i : V \times  W \rightarrow  \mathcal{T}}\right)  \in Obj\) satisfies the {\bf universal property of tensor product} if for any \((\phi  : V \times  W \rightarrow U) \in  \mathrm{{Obj}}\), there exists an unique linear transformation \(\color{red} T_{\phi } : \mathcal{T} \rightarrow  U\) such that the diagram below commutes:


\begin{center}
\begin{tikzcd}[row sep=large, column sep=large]
V \times W \arrow[r, "i", blue] \arrow[rd, "\phi"']  & \mathcal{T}  \arrow[d, "T_{\phi}"', dashed, red]\\
& U 
\end{tikzcd} \quad \quad \quad i.e. \(\phi  = {\color{red} T_{\phi}} \circ  i\).
\end{center} 

\end{definition}

\noindent
In other words, rather than studying the \textbf{bilinear map} \( \phi \), it is better to study the \textbf{linear transformation} \( T_\phi \). Since \( \phi = T_\phi \circ i \), \( T_\phi \) contains all the information about \( \phi \), and one can apply all the theorems we know about linear algebra to study \( T_\phi \)!

\bigskip

\noindent
\textbf{\Large The question is: Does the tensor product space \( \mathcal{T} \) exist?}

\bigskip

\noindent
In the next section, we will construct \( \mathcal{T} \) explicitly, and show that it satisfies the properties we mentioned above.


\section{Tensor Product Space}
We begin with defining the tensor product $\mathcal{T} = V \otimes W$ of two vector spaces $V$ and $W$. This can be generalized into any (countable) number of vector spaces.


\begin{definition} Let \(V,W\) be vector spaces. Let \(S = \{ \left( {\mathbf{v},\mathbf{w}}\right)  \mid  \mathbf{v} \in  V,\mathbf{w} \in  W\}\), we define

\[
\mathfrak{X} = \operatorname{span}\left( S\right).
\]
\end{definition}

\begin{remark}
    Note that we assume no relations on the elements \(\left( {\mathbf{v},\mathbf{w}}\right)  \in  \mathcal{S}\). In other words, $1\cdot ({\bf v}, {\bf w})$ and $1 \cdot ({\bf v}', {\bf w}') \in \mathfrak{X}$ are linearly independent unless ${\bf v} = {\bf v}'$ and ${\bf w} = {\bf w}'$ 
    
    For example, if ${\bf w} \neq {\bf 0}$, then ${\bf w} \neq 2{\bf w}$. Therefore, $(0,{\bf w})$ and $(0, 2{\bf w})$ are linearly independent and hence:
    \begin{align*} 
    2\cdot \left( {0,\mathbf{w}}\right)  &\neq  \left( {0,2\mathbf{w}}\right)\\
3\cdot \left( {0,\mathbf{w}}\right) &\neq 1\cdot \left( {0,\mathbf{w}}\right)  + 1\cdot\left( {0,{2\bf w}}\right)  \end{align*}
Similarly, if ${\bf v}, {\bf w} \neq {\bf 0}$:
\begin{align*}
\left(\mathbf{v},\mathbf{w}\right) &\neq \left( \mathbf{v},0\right)  +  \left( 0,\mathbf{w}\right). \end{align*}
The only legitimate relationship in $\mathfrak{X}$ is
\[
2\cdot \left( \mathbf{v},{\bf w}\right)  + 3\cdot \left(\mathbf{v},\mathbf{w}\right)  = 5\left( {\mathbf{v},\mathbf{w}}\right) ,
\]
yet it is not equal to $(5{\bf v},5{\bf w})$.

\noindent In other words, \(\mathcal{S}\) is a basis of \(\mathfrak{X}\), and consequently \(\mathfrak{X}\) is of uncountable dimension.
\end{remark}

\begin{definition}[Tensor Product of $V$ and $W$] Let \(\mathfrak{Y} \leq  \mathfrak{X}\) be a vector subspace spanned by vectors of the form
\[
\left\{  {1\left( {{\bf v}_1+{\bf v}_2,\mathbf{w}}\right)  - 1\left( {{\bf v}_1,\mathbf{w}}\right)  - 1\left( {{\bf v}_2,\mathbf{w}}\right) }\right\}  ,\quad \left\{  {1\left( {\mathbf{v},{\bf w}_1 + {\bf w}_2}\right)  - 1\left( {\mathbf{v},{\bf w}_1}\right)  - 1\left( {\mathbf{v},{\bf w}_2}\right) }\right\}
\]
and
\[
\{ 1\left( {k\mathbf{v},\mathbf{w}}\right)  - k\left( {\mathbf{v},\mathbf{w}}\right)  \mid  k \in  \mathbb{F}\},
\quad
\{ 1\left( {\mathbf{v},k\mathbf{w}}\right)  - k\left( {\mathbf{v},\mathbf{w}}\right)  \mid  k \in  \mathbb{F}\}
\]
for all ${\bf v}, {\bf v}_1, {\bf v}_2 \in V$, ${\bf w}, {\bf w}_1, {\bf w}_2 \in W$. Then the {\bf tensor product} \(V \otimes  W\) is defined by

\[
V \otimes  W := \mathfrak{X}/\mathfrak{Y}
\]

\noindent Also, for ${\bf v} \in V$ and ${\bf w} \in W$, we define 
\[\mathbf{v} \otimes  \mathbf{w} := \left( {\mathbf{v},\mathbf{w}}\right)  + \mathfrak{Y} \quad \in  \mathfrak{X}/\mathfrak{Y} = V \otimes W.\]
\end{definition}

\noindent Using our definition of $V \otimes W$, the expression ${\bf v} \otimes {\bf w} \in V \otimes W$ is `bilinear', for instance:
\begin{equation} \label{eq:tensorrule1}
\begin{aligned}
\left( {{\bf v}_1 + {\bf v}_2}\right)  \otimes  \mathbf{w} &= \left( {{\bf v}_1 + {\bf v}_2,\mathbf{w}}\right)  + \mathfrak{Y} \\
&= \left( {{\bf v}_1 + {\bf v}_2,\mathbf{w}}\right)  - \left\lbrack  {\left( {{\bf v}_1 + {\bf v}_2,\mathbf{w}}\right)  - \left( {{\bf v}_1,\mathbf{w}}\right)  - \left( {{\bf v}_2,\mathbf{w}}\right) }\right\rbrack   + \mathfrak{Y}
\\
&= 0\left( {{\bf v}_1 + {\bf v}_2,\mathbf{w}}\right)  + \left( {{\bf v}_1,\mathbf{w}}\right)  + \left( {{\bf v}_2,\mathbf{w}}\right)  + \mathfrak{Y}
\\
&= \left\lbrack  {\left( {{\bf v}_1,\mathbf{w}}\right)  + \mathfrak{Y}}\right\rbrack   + \left\lbrack  {\left( {{\bf v}_2,\mathbf{w}}\right)  + \mathfrak{Y}}\right\rbrack
\\
&= {\bf v}_1 \otimes  {\bf w} + {\bf v}_2 \otimes  {\bf w}
\end{aligned}
\end{equation}
Similarly, one can check that
\begin{equation} \label{eq:tensorrule2}
\begin{aligned}
\mathbf{v} \otimes  \left( {{\bf w}_1 + {\bf w}_2}\right)  &= \left( {\mathbf{v} \otimes  {\bf w}_1}\right)  + \left( {\mathbf{v} \otimes  {\bf w}_2}\right)
\\
\left( {k\mathbf{v}}\right)  \otimes  \mathbf{w} &= k\left( {\mathbf{v} \otimes  \mathbf{w}}\right)
\\
\mathbf{v} \otimes  \left( {k\mathbf{w}}\right)  &= k\left( {\mathbf{v} \otimes  \mathbf{w}}\right)
\end{aligned}
\end{equation}
Making use of the rules above, we present an example of arithmetic on tensor product spaces:
\begin{example} 
Let \(V = W = {\mathbb{R}}^2\), with
\({\mathbf{e}}_1 = \left( \begin{array}{l} 1 \\  0 \end{array}\right) ,\;{\mathbf{e}}_2 = \left( \begin{array}{l} 0 \\  1 \end{array}\right) .
\)
Then
\begin{align*}
\left( \begin{array}{l} 3 \\  1 \end{array}\right)  \otimes  \left( \begin{matrix}  - 4 \\  2 \end{matrix}\right)  &= \left( {3{\mathbf{e}}_1 + 2{\mathbf{e}}_2}\right)  \otimes  \left( {-4{\mathbf{e}}_1 + 2{\mathbf{e}}_2}\right)
\\
&= \left( {3{\mathbf{e}}_1}\right)  \otimes  \left( {-4{\mathbf{e}}_1 + 2{\mathbf{e}}_2}\right)  + \left( {\mathbf{e}}_2\right)  \otimes  \left( {-4{\mathbf{e}}_1 + 2{\mathbf{e}}_2}\right)
\\
&= \left( {3{\mathbf{e}}_1}\right)  \otimes  \left( {-4{\mathbf{e}}_1}\right)  + \left( {3{\mathbf{e}}_1}\right)  \otimes  \left( {2{\mathbf{e}}_2}\right)  + \left( {\mathbf{e}}_2\right)  \otimes  \left( {-4{\mathbf{e}}_1}\right)  + {\mathbf{e}}_2 \otimes  \left( {2{\mathbf{e}}_2}\right)
\\
&=  - {12}\left( {{\mathbf{e}}_1 \otimes  {\mathbf{e}}_1}\right)  + 6\left( {{\mathbf{e}}_1 \otimes  {\mathbf{e}}_2}\right)  - 4\left( {{\mathbf{e}}_2 \otimes  {\mathbf{e}}_1}\right)  + 2\left( {{\mathbf{e}}_2 \otimes  {\mathbf{e}}_2}\right)
\end{align*}
Exercise: Check that \({\mathbf{e}}_1 \otimes  {\mathbf{e}}_2 + {\mathbf{e}}_2 \otimes  {\mathbf{e}}_1\) cannot be re-written as
\[
\left( {a{\mathbf{e}}_1 + b{\mathbf{e}}_2}\right)  \otimes  \left( {c{\mathbf{e}}_1 + d{\mathbf{e}}_2}\right).
\]
for any $a,b,c,d \in  \mathbb{R}$.
\end{example}

\begin{remark}
The product space \(V \times  W\) is different from the tensor product space \(V \otimes  W\) in the following sense:

(a) \(\left( {\mathbf{v},\mathbf{0}}\right)  \neq  {\mathbf{0}}_{V \times  W}\) in \(V \times  W\) ; but \(\mathbf{v} \otimes  {\bf 0} \in  {\bf 0}_{V \otimes  W}\), since

\[
{\bf v} \otimes  0 = {\bf v} \otimes  \left( {0\mathbf{w}}\right)
= 0\left( {\bf v} \otimes  {\bf w}\right) = {\bf 0}_{V \otimes  W}
\]


(b) \(\left( {{\bf v}_1,{\bf w}_1}\right)  + \left( {{\bf v}_2,{\bf w}_2}\right)  = \left( {{\bf v}_1 + {\bf v}_2,{\bf w}_1 + {\bf w}_2}\right)\) ; but \({\bf v}_1 \otimes  {\bf w}_1 + {\bf v}_2 \otimes  {\bf w}_2\) cannot be

simplified further in general, unless (for instance) \({\bf v}_1 = {\bf v}_2\), so that:

\[
\mathbf{v} \otimes  {\bf w}_1 + \mathbf{v} \otimes  {\bf w}_2 = \mathbf{v} \otimes  \left( {{\bf w}_1 + {\bf w}_2}\right)
\]

As we saw in the exercise above, a general element in $V \otimes W$ {\bf cannot} is not necessarily of the form $v \otimes w$.
How does a general element in $V \otimes W$ look like?

We begin with a general element in \(\mathfrak{X}\) :
\[
{a}_1\left( {{\bf v}_1,{\bf w}_1}\right)  + \cdots  + {a}_k\left( {{\bf v}_k,{\bf w}_k}\right) ,
\]

where \(\left( {{\bf v}_{i},{\bf w}_{i}}\right)\) are distinct. Then a general element in \(\mathfrak{X}/\mathfrak{Y} \mathrel{\text{ := }} V \otimes  W\) looks like:
\begin{align*}
{a}_1\left( {{\bf v}_1,{\bf w}_1}\right)  + \cdots  + {a}_k\left( {{\bf v}_k,{\bf w}_k}\right)  + \mathfrak{Y} &= {a}_1\left( {\left( {{\bf v}_1,{\bf w}_1}\right)  + \mathfrak{Y}}\right)  + \cdots  + {a}_k\left( {\left( {{\bf v}_k,{\bf w}_k}\right)  + \mathfrak{Y}}\right)
\\
&= {a}_1\left( {{\bf v}_1 \otimes  {\bf w}_1}\right)  + \cdots  + {a}_k\left( {{\bf v}_k \otimes  {\bf w}_k}\right)
\\
&= \left( {{a}_1{\bf v}_1}\right)  \otimes  {\bf w}_1 + \cdots  + \left( {{a}_k{\bf v}_k}\right)  \otimes  {\bf w}_k
\end{align*}
Therefore, a general element in \(V \otimes  W\) is of the form
\begin{equation} \label{eq:generaltensor}
{\bf v}^{(1)} \otimes  {\bf w}^{(1)} + \cdots  + {\bf v}^{(k)} \otimes  {\bf w}^{(k)} \quad \quad ({\bf v}^{(i)} \in  V, \quad {\bf w}^{(i)} \in  W).
\end{equation}
\end{remark}

\begin{theorem}\label{thm: univ-binlinear-map} 
The bilinear map
$$i : V \times W \rightarrow V \otimes W$$ 
defined by 
$$i({\bf v}, {\bf w}) := {\bf v} \otimes {\bf w}$$ 
is in \textnormal{Obj} in \autoref{def:univtensor} (i.e. $i$ is a bilinear map). Moreover, by taking $\mathcal{T} := V \otimes W$ in \autoref{def:univtensor}, it satisfies the universal property of tensor products.
\end{theorem}
\begin{proof}
By Equations \eqref{eq:tensorrule1} and \eqref{eq:tensorrule2}, \( i \) satisfies:
\begin{align*}
    i(\mathbf{v}_1 + \mathbf{v}_2, \mathbf{w}) &= i(\mathbf{v}_1, \mathbf{w}) + i(\mathbf{v}_2, \mathbf{w}), \\
    i(\mathbf{v}, \mathbf{w}_1 + \mathbf{w}_2) &= i(\mathbf{v}, \mathbf{w}_1) + i(\mathbf{v}, \mathbf{w}_2), \\
    i(a \mathbf{v}, \mathbf{w}) &= a \cdot i(\mathbf{v}, \mathbf{w}), \quad i(\mathbf{v}, b \mathbf{w}) = b \cdot i(\mathbf{v}, \mathbf{w}),
\end{align*}
so \( i \in \textnormal{Obj} \) as in \autoref{def:univtensor}.

Now, let \( (\phi : V \times W \to U) \in \textnormal{Obj} \) be any bilinear map into some vector space \( U \). Since \( \phi \) is bilinear, it respects the relations that generate \( \mathfrak{Y} \). Therefore, we can define a linear map
\[
T_{\phi} : V \otimes W \to U, \quad T_{\phi}(\mathbf{v} \otimes \mathbf{w}) := \phi(\mathbf{v}, \mathbf{w}),
\]
and extend linearly to all of \( V \otimes W \) (recall all elements of $V \otimes W$ are of the form ${\bf v}^{(1)} \otimes {\bf w}^{(1)} + \dots + {\bf v}^{(k)} \otimes {\bf w}^{(k)}$). This map is well-defined because if two elements \( (\mathbf{v}, \mathbf{w}) \) and \( (\mathbf{v}', \mathbf{w}') \) are equivalent in \( \mathfrak{X}/\mathfrak{Y} \), i.e. $({\bf v},{\bf w}) - ({\bf v}',{\bf w}') \in \mathfrak{Y}$, then their images under \( \phi \) are equal due to bilinearity.

It follows directly from the definition that the diagram
\[
\begin{tikzcd}[row sep=large, column sep=large]
V \times W \arrow[r, "i", blue] \arrow[rd, "\phi"'] & V \otimes W \arrow[d, "T_{\phi}"', dashed, red] \\
& U
\end{tikzcd}
\]
commutes, i.e., \( \phi = T_\phi \circ i \).

For uniqueness: suppose there exists another linear map \( T' : V \otimes W \to U \) such that \( \phi = T' \circ i \). Then both \( T_\phi \) and \( T' \) agree on all pure tensors \( \mathbf{v} \otimes \mathbf{w} \), and hence, by linearity, on all of \( V \otimes W \). So \( T' = T_\phi \).

Therefore, the pair \( (V \otimes W, i) \) satisfies the universal property of the tensor product.
\end{proof}

\section{Basis of \(V \otimes W\)}

Given that \(\left\{  {{\bf v}_1,\ldots ,{\bf v}_n}\right\}\) is a basis of \(V\), and \(\left\{  {{\bf w}_1,\ldots ,{\bf w}_{m}}\right\}\) a basis of \(W\), we aim to find a basis of \(V \otimes  W\) using \({\bf v}_{i}\) ’s and \({\bf w}_{i}\) ’s.

\begin{proposition}\label{prop: tensor-span}
    The set \(\left\{  {{\bf v}_{i} \otimes  {\bf w}_{j} \mid  1 \leq  i \leq  n,1 \leq  j \leq  m}\right\}\) spans the tensor product space \(V \otimes  W\).
\end{proposition}

\begin{proof} Consider any \(\mathbf{v} \in  V\) and \(\mathbf{w} \in  W\), and we want to express \(\mathbf{v} \otimes  \mathbf{w}\) in terms of \({\bf v}_{i},{\bf w}_{j}\). Suppose that \(\mathbf{v} = \alpha_1{\bf v}_1 + \cdots  + \alpha_n{\bf v}_n\) and \(\mathbf{w} = \beta_1{\bf w}_1 + \cdots  + \beta_{m}{\bf w}_{m}\).

Substituting \(\mathbf{v} = \alpha_1{\bf v}_1 + \cdots  + \alpha_n{\bf v}_n\) into the expression \(\mathbf{v} \otimes  \mathbf{w}\), we imply

\begin{align*}
\mathbf{v} \otimes  \mathbf{w} &= \left( {\alpha_1{\bf v}_1 + \cdots  + \alpha_n{\bf v}_n}\right)  \otimes  \mathbf{w} \\
&= \left( {\alpha_1{\bf v}_1}\right)  \otimes  {\bf w}_1 + \cdots  + \left( {\alpha_n{\bf v}_n}\right)  \otimes  {\bf w}_n \\
&= \alpha_1\left( {{\bf v}_1 \otimes  \mathbf{w}}\right)  + \cdots  + \alpha_n\left( {{\bf v}_n \otimes  \mathbf{w}}\right)
\end{align*}

Similarly, for each \({\bf v}_{i} \otimes  \mathbf{w},i = 1,\ldots ,n\),

\[
{\bf v}_{i} \otimes  \mathbf{w} = \beta_1\left( {{\bf v}_{i} \otimes  {\bf w}_1}\right)  + \cdots  + \beta_{m}\left( {{\bf v}_{i} \otimes  {\bf w}_{m}}\right) .
\]

Therefore,

\begin{equation}\label{eq: tensor-expansion}
    \mathbf{v} \otimes  \mathbf{w} = \mathop{\sum }\limits_{{i = 1}}^n\mathop{\sum }\limits_{{j = 1}}^{m}\alpha_{i}\beta_{j}\left( {{\bf v}_{i} \otimes  {\bf w}_{j}}\right)
\end{equation}



By \autoref{eq:generaltensor}, any vector in \(V \otimes  W\) is of the form
\(
{\bf v}^{\left( 1\right) } \otimes  {\bf w}^{\left( 1\right) } + \cdots  + {\bf v}^{\left( k \right) } \otimes  {\bf w}^{\left( k \right) }
\), and by \autoref{eq: tensor-expansion}, each \({\bf v}^{\left( \ell \right) } \otimes  {\bf w}^{\left( \ell \right) },\ell = 1,\ldots ,k\), can be expressed as

\[
{\bf v}^{\left( \ell\right) } \otimes  {\bf w}^{\left( \ell\right) } = \mathop{\sum }\limits_{{i = 1}}^n\mathop{\sum }\limits_{{j = 1}}^{m}\alpha_{i}^{\left( \ell\right) }\beta_{j}^{\left( \ell\right) }\left( {{\bf v}_{i} \otimes  {\bf w}_{j}}\right)
\]
Therefore,
\[
{\bf v}^{\left( 1\right) } \otimes  {\bf w}^{\left( 1\right) } + \cdots  + {\bf v}^{\left( k \right) } \otimes  {\bf w}^{\left( k \right) } = \mathop{\sum }\limits_{{\ell = 1}}^{k}\mathop{\sum }\limits_{{i = 1}}^n\mathop{\sum }\limits_{{j = 1}}^{m}\alpha_{i}^{\left( k\right) }\beta_{j}^{\left( k\right) }\left( {{\bf v}_{i} \otimes  {\bf w}_{j}}\right)
\]

In other words, \(\left\{  {{\bf v}_{i} \otimes  {\bf w}_{j} \mid  1 \leq  i \leq  n,1 \leq  j \leq  m}\right\}\) spans \(V \otimes  W\).
\end{proof}


\begin{theorem} \label{thm:tensorbasis}
    A basis of \(V \otimes  W\) is \(\left\{  {{\bf v}_{i} \otimes  {\bf w}_{j} \mid  1 \leq  i \leq  n,1 \leq  j \leq  m}\right\}\)
\end{theorem} 

\begin{proof}
By \autoref{prop: tensor-span}, it suffices to show that the set \(\left\{  {{\bf v}_{i} \otimes  {\bf w}_{j} \mid  1 \leq  i \leq  n,1 \leq  j \leq  m}\right\}\) is linear independent. Suppose that

\begin{equation}\label{eq: tensor-linear-independece}
    \mathop{\sum }\limits_{{i = 1}}^n\mathop{\sum }\limits_{{j = 1}}^n\alpha_{ij}\left( {{\bf v}_{i} \otimes  {\bf w}_{j}}\right)  = \mathbf{0} \tag{13.3}
\end{equation}


Suppose that \(\left\{  {{\phi }_1,\ldots ,{\phi }_n}\right\}\) is a dual basis of \({V}^{ * }\), and \(\left\{  {{\psi }_1,\ldots ,{\psi }_{m}}\right\}\) is a dual basis of \({W}^{ * }\). Define the mapping
\[
{\pi }_{p,q} : \;V \times  W \rightarrow  \mathbb{F}
\]
by 
$$\pi_{p,q}(\mathbf{v},\mathbf{w}) := {\phi }_{p}\left( \mathbf{v}\right) {\psi }_{q}\left( \mathbf{w}\right).$$
Then one can check that \({\pi }_{p,q}\) is bilinear: for instance,
\begin{align*}
{\pi }_{p,q}\left( {a{\bf v}_1 + b{\bf v}_2,\mathbf{w}}\right)  = {\phi }_{p}\left( {a{\bf v}_1 + b{\bf v}_2}\right) {\psi }_{q}\left( \mathbf{w}\right)
&= \left( {a{\phi }_{p}\left( {\bf v}_1\right)  + b{\phi }_{p}\left( {\bf v}_2\right) }\right) {\psi }_{q}\left( \mathbf{w}\right)\\
&= a{\phi }_{p}\left( {\bf v}_1\right) {\psi }_{q}\left( \mathbf{w}\right)  + b{\phi }_{p}\left( {\bf v}_2\right) {\psi }_{q}\left( \mathbf{w}\right)\\
&= a{\pi }_{p,q}\left( {{\bf v}_1,\mathbf{w}}\right)  + b{\pi }_{p,q}\left( {{\bf v}_2,\mathbf{w}}\right) .
\end{align*}
similarly, we can check that \({\pi }_{p,q}\left( {\mathbf{v},a{\bf w}_1 + b{\bf w}_2}\right)  = a{\pi }_{p,q}\left( {\mathbf{v},{\bf w}_1}\right)  +\)  \(b{\pi }_{p,q}\left( {v,{\bf w}_2}\right)\). Therefore, \({\pi }_{p,q} \in\) Obj. By the universal property of the tensor product (Theorem \ref{thm: univ-binlinear-map}), \({\pi }_{p,q}\) induces the unique {\it linear transformation}
\[
\Pi_{p,q} : \;V \otimes  W \rightarrow  \mathbb{F}
\quad \text{ with }\;\Pi_{p,q}\left( {\mathbf{v} \otimes  \mathbf{w}}\right)  = {\pi }_{p,q}\left( {\mathbf{v},\mathbf{w}}\right)
\]

In other words, \(\Pi_{p,q}\left( {\mathbf{v} \otimes  \mathbf{w}}\right)  = {\phi }_{p}\left( \mathbf{v}\right) {\psi }_{q}\left( \mathbf{w}\right)\).

\begin{itemize}
\item Applying the mapping \({\Pi }_{p,q}\) on both sides of \autoref{eq: tensor-linear-independece}, we imply
\end{itemize}

\[
{\Pi }_{p,q}\left( {\mathop{\sum }\limits_{{i = 1}}^n\mathop{\sum }\limits_{{j = 1}}^n\alpha_{ij}\left( {{\bf v}_{i} \otimes  {\bf w}_{j}}\right) }\right)  = {\Pi }_{p,q}\left( \mathbf{0}\right)
\]

Or equivalently,

\[
\mathop{\sum }\limits_{{i = 1}}^n\mathop{\sum }\limits_{{j = 1}}^n\alpha_{ij}{\Pi }_{p,q}\left( {{\bf v}_{i} \otimes  {\bf w}_{j}}\right)  = 0,
\]

i.e.,

\[
\mathop{\sum }\limits_{{i = 1}}^n\mathop{\sum }\limits_{{j = 1}}^n\alpha_{ij}{\phi }_{p}\left( {\bf v}_{i}\right) {\psi }_{q}\left( {\bf w}_{j}\right)  = \alpha_{p,q} = 0
\]

Following this procedure, we can argue that \(\alpha_{ij} = 0\) for all $i$ and \(j\).
\end{proof}

\begin{corollary} If \(\dim \left( V\right) ,\dim \left( W\right)  < \infty\), then \(\dim \left( {V \otimes  W}\right)  = \dim \left( V\right) \dim \left( W\right)\)
\end{corollary}

\begin{proof} Check dimension of the basis of \(V \otimes  W\).
\end{proof}

\section{An Application of Universal Property}
As we saw in the proof of Theorem \ref{thm:tensorbasis}, the universal property (\autoref{thm: univ-binlinear-map}) is very helpful. In particular, whenever \(\phi  : V \times  W \rightarrow  U\) is a {\it bilinear map}, \autoref{thm: univ-binlinear-map} implies that we can induce an unique {\it linear transformation} \(\psi  : V \otimes  W \rightarrow  U\). Let's try another example for applying universal property: 

\begin{theorem}
For any vector spaces \(U\) and \(V\),
\[
V \otimes  U \cong  U \otimes  V.
\]
\end{theorem}

\begin{proof}
Construct the mapping
\[
\phi  : \;V \times  U \rightarrow  U \otimes  V
\]
by \(\phi \left( {\mathbf{v},\mathbf{u}}\right)  = \mathbf{u} \otimes  \mathbf{v}
\). Then \(\phi\) is bilinear: for instance,

\[
\phi \left( {a{\bf v}_1 + b{\bf v}_2,\mathbf{u}}\right)  = u \otimes  \left( {a{\bf v}_1 + b{\bf v}_2}\right)
= a\left( {\mathbf{u} \otimes  {\bf v}_1}\right)  + b\left( {u \otimes  {\bf v}_2}\right)
= {a\phi }\left( {{\bf v}_1,\mathbf{u}}\right)  + {b\phi }\left( {{\bf v}_2,\mathbf{u}}\right).
\]
Therefore, \(\phi  \in  \mathrm{{Obj}}\). By Theorem \autoref{thm: univ-binlinear-map}, we induce a unique linear transformation
\[
\Phi  : \;V \otimes  U \rightarrow  U \otimes  V
\]
satisfying \(\Phi \left( {\mathbf{v} \otimes  \mathbf{u}}\right)  = \mathbf{u} \otimes  \mathbf{v}
\).

Similarly, we may induce the linear transformation
\[
\Psi  : \;U \otimes  V \rightarrow  V \otimes  U
\]
satisfying \(\Psi \left( {\mathbf{u} \otimes  \mathbf{v}}\right)  = \mathbf{v} \otimes  \mathbf{u}\).

Given any \(\mathop{\sum }\limits_{i}{\mathbf{u}}_{i} \otimes  {\bf v}_{i} \in  U \otimes  V\), observe that
\[
\left( {\Phi  \circ  \Psi }\right) \left( {\mathop{\sum }\limits_{i}{\mathbf{u}}_{i} \otimes  {\bf v}_{i}}\right)  = \Phi \left( {\mathop{\sum }\limits_{i}\Psi \left( {{\mathbf{u}}_{i} \otimes  {\bf v}_{i}}\right) }\right)
= \Phi \left( {\mathop{\sum }\limits_{i}{\bf v}_{i} \otimes  {\mathbf{u}}_{i}}\right)
= \mathop{\sum }\limits_{i}\Phi \left( {{\bf v}_{i} \otimes  {\mathbf{u}}_{i}}\right)
= \mathop{\sum }\limits_{i}{\mathbf{u}}_{i} \otimes  {\bf v}_{i}
\]
Therefore, \(\Phi  \circ  \Psi  = {\operatorname{id}}_{U \otimes  V}\). Similarly, \(\Psi  \circ  \Phi  = {\operatorname{id}}_{V \otimes  U}\). Therefore,
$U \otimes  V \cong  V \otimes  U$.
\end{proof} 

\section{Tensor Product of Linear Transformation}
 Given two linear transformations \(T : V \rightarrow  {V}^{\prime }\) and \(S : W \rightarrow  {W}^{\prime }\), we want to use the universal property of tensor product to construct the linear transformation between tensor product spaces:
\[
T \otimes  S : V \otimes  W \rightarrow  {V}^{\prime } \otimes  {W}^{\prime }
\]

\begin{proposition} \label{prop:tensor_linear_trans}
Suppose that \(T : V \rightarrow  {V}^{\prime }\) and \(S : W \rightarrow  {W}^{\prime }\) are linear transformations, then there exists a unique linear transformation
\[
T \otimes  S : V \otimes  W \rightarrow  {V}^{\prime } \otimes  {W}^{\prime }
\]
satisfying \(\left( {T \otimes  S}\right) \left( {\bf v} \otimes  {\bf w}\right)  = T\left({\bf v}\right)  \otimes  S\left( {\bf w}\right)
\) for all ${\bf v} \in V$ and ${\bf w} \in W$.
\end{proposition} 

\begin{proof}
Define the map
\[
T \times  S : \;V \times  W \rightarrow  {V}^{\prime } \otimes  {W}^{\prime }
\]
by \(\left( {T \times  S}\right) \left( {\bf v},{\bf w}\right)  := T\left( {\bf v}\right)  \otimes  S\left({\bf w}\right)
\).

Then $T \times S$ is bilinear: for instance, we can show that
\[
\left( {T \times  S}\right) \left( {a{\bf v}_1 + b{\bf v}_2,{\bf w}}\right)  = a\left( {T \times  S}\right) \left( {{\bf v}_1,{\bf w}}\right)  + b\left( {T \times  S}\right) \left( {{\bf v}_2,{\bf w}}\right)
\]
Therefore, \(T \times  S \in\) Obj. By \autoref{thm: univ-binlinear-map}, there exists a unique {\it linear transformation}
\[
T \otimes  S\ :V \otimes  W \rightarrow  {V}^{\prime } \otimes  {W}^{\prime }
\]
with the properties as described in the proposition.
\end{proof} 

\noindent \textbf{Notation Warning.} We use the notation \(T \otimes  S\) for the linear transformation defined above. Does it possess the properties of tensor product? For instance, do we have the following equality:
\[
\left( {a{T}_1 + b{T}_2}\right)  \otimes  S = a\left( {{T}_1 \otimes  S}\right)  + b\left( {{T}_2 \otimes  S}\right) ?
\]
This will be proved in \autoref{prop:tensorlineartrans} below, which justifies our notation of $T \otimes S$.

\begin{example}\label{ex: notation-verification}
\item
Let \(V = {V}^{\prime } = {\mathbb{F}}^2\) and \(W = {W}^{\prime } = {\mathbb{F}}^{3}\). Define the linear transformations 
$$T :  V \rightarrow  V, \quad \quad S: W \rightarrow  W$$
by:
\[
T({\bf v}) := A{\bf v} = \left( \begin{array}{ll} a & b \\  c & d \end{array}\right){\bf v}, \quad \quad S({\bf w}) := B{\bf w} =\left( \begin{array}{lll} p & q & r \\  s & t & u \\  v & w & x \end{array}\right){\bf w}. 
\]
How does \(T \otimes  S : V \otimes  W \rightarrow  V \otimes  W\) look like? Suppose \(\left\{  {{e}_1,{\bf e}_2}\right\}  ,\left\{  {{\bf f}_1,{\bf f}_2,{\bf f}_{3}}\right\}\) are the usual bases of \(V\) and \(W\), respectively. By \autoref{thm:tensorbasis}, a basis of
\(V \otimes  W\) is given by:
\[
\mathcal{C} = \left\{  {{\bf e}_1 \otimes  {\bf f}_1,{\bf e}_1 \otimes  {\bf f}_2,{\bf e}_1 \otimes  {\bf f}_{3},{\bf e}_2 \otimes  {\bf f}_1,{\bf e}_2 \otimes  {\bf f}_2,{\bf e}_2 \otimes  {\bf f}_{3}}\right\}  .
\]
Now compute \(\left( {T \otimes  S}\right) \left( {{\bf e}_{i} \otimes  {\bf f}_{j}}\right)\) for \(i = 1,2\) and \(j = 1,2,3\). For instance,
\begin{align*}
&\left( {T \otimes  S}\right) \left( {{\bf e}_1 \otimes  {\bf f}_1}\right)\\ 
&= T\left( {\bf e}_1\right)  \otimes  S\left( {\bf f}_1\right)\\
&= \left( {a{\bf e}_1 + c{\bf e}_2}\right)  \otimes  \left( {p{\bf f}_1 + s{\bf f}_2 + v{\bf f}_{3}}\right)
\\
&= \left( {ap}\right) {\bf e}_1 \otimes  {\bf f}_1 + \left( {as}\right) {\bf e}_1 \otimes  {\bf f}_2 + \left( {av}\right) {\bf e}_1 \otimes  {\bf f}_{3} + \left( {cp}\right) {\bf e}_2 \otimes  {\bf f}_1 + \left( {cs}\right) {\bf e}_2 \otimes  {\bf f}_2 + \left( {cv}\right) {\bf e}_2 \otimes  {\bf f}_{3}.
\end{align*}
Therefore, the linear transformation \(\left( {T \otimes  S}\right)\) has a $6 \times 6$ matrix representation:
\[
{\left( T \otimes  S\right) }_{\mathcal{C},\mathcal{C}} = \left( \begin{array}{ll} {aB} & {bB} \\  {cB} & {dB} \end{array}\right) ,
\]
Th above matrix is called the {\bf Kronecker Tensor Product} of $A$ and $B$ (see the command kron in MATLAB for details).
\end{example}

\begin{proposition}
More generally, given the linear operator \(T : V \rightarrow  V\) and \(S : W \rightarrow  W\), let \(\mathcal{A} = \left\{  {{\bf v}_1,\ldots ,{\bf v}_n}\right\}  ,\mathcal{B} = \left\{  {{\bf w}_1,\ldots ,{\bf w}_{m}}\right\}\) be a basis of \(V,W\) respectively, with

\[
 {\left(T \right)}_{\mathcal{A},\mathcal{A}} = \left( {a}_{ij} \right)\mathrel{\text{ := }} A, \;{\left(S\right)}_{\mathcal{B},\mathcal{B}}  = \left( {b}_{ij}\right)  \mathrel{\text{ := }} B
\]

As a result, \({\left( T \otimes  S\right) }_{C,C} = A \otimes  B\), where \(C = \left\{  {{\bf v}_1 \otimes  {\bf w}_1,\ldots ,{\bf v}_n \otimes  {\bf w}_{m}}\right\}\), and \(A \otimes  B\) denotes the Kronecker tensor product, defined as the matrix

\[
\left( \begin{matrix} {a}_{1,1}B & \cdots & {a}_{1,n}B \\  \vdots &  \ddots  & \vdots \\  {a}_{n,1}B & \cdots & {a}_{n,n}B \end{matrix}\right) .
\]
\end{proposition} 
\begin{proof}
Following the similar procedure as in \autoref{ex: notation-verification} and applying the relation

\[
\left( {T \otimes  S}\right) \left( {{\bf v}_{i} \otimes  {\bf w}_{j}}\right)  = T\left( {\bf v}_{i}\right)  \otimes  S\left( {\bf w}_{j}\right)
= \left( {\mathop{\sum }\limits_{{k = 1}}^n{a}_{ki}{\bf v}_{k}}\right)  \otimes  \left( {\mathop{\sum }\limits_{{\ell  = 1}}^{m}{b}_{\ell j}{\bf w}_{\ell }}\right)
= \mathop{\sum }\limits_{{k = 1}}^n\mathop{\sum }\limits_{{\ell  = 1}}^{m}\left( {{a}_{ki}{b}_{\ell j}}\right) {\bf v}_{k} \otimes  {\bf w}_{\ell }
\]
\end{proof} 

\begin{proposition} \label{prop:tensorlineartrans}
The operation \(T \otimes  S\) satisfies all the properties of tensor product. For example,

\[
\left( {a{T}_1 + b{T}_2}\right)  \otimes  S = a\left( {{T}_1 \otimes  S}\right)  + b\left( {{T}_2 \otimes  S}\right)
\]

\[
T \otimes  \left( {c{S}_1 + d{S}_2}\right)  = c\left( {T \otimes  {S}_1}\right)  + d\left( {T \otimes  {S}_2}\right)
\]

Therefore, the usage of the notion " \(\otimes\) " is justified for the definition of \(T \otimes  S\).
\end{proposition}

\begin{proof}
using matrix multiplication. For instance, consider the operation \(\left( {T + {T}^{\prime }}\right)  \otimes  S\), with

\({\left( T\right) }_{\mathcal{A},\mathcal{A}} = \left( {a}_{ij}\right) ,{\left( {T}^{\prime }\right) }_{\mathcal{A},\mathcal{A}} = \left( {c}_{ij}\right) ,{\left( S\right) }_{\mathcal{B},\mathcal{B}} = \left( {b}_{ij}\right).\)

We compute its matrix representation directly:

\begin{align*}
{\left( \left( T + {T}^{\prime }\right)  \otimes  S\right) }_{\mathcal{C},\mathcal{C}} &= {\left( T + {T}^{\prime }\right) }_{\mathcal{A},\mathcal{A}} \otimes  {\left( S\right) }_{\mathcal{B},\mathcal{B}}\\
&= \left\lbrack  {{\left( T\right) }_{\mathcal{A},\mathcal{A}} + {\left( {T}^{\prime }\right) }_{\mathcal{A},\mathcal{A}}}\right\rbrack   \otimes  {\left( S\right) }_{\mathcal{B},\mathcal{B}}\\
&= {\left( T\right) }_{\mathcal{A},\mathcal{A}} \otimes  {\left( S\right) }_{\mathcal{B},\mathcal{B}} + {\left( {T}^{\prime }\right) }_{\mathcal{A},\mathcal{A}} \otimes  {\left( S\right) }_{\mathcal{B},\mathcal{B}}
\end{align*}

where the last equality is by the additive rule for Kronecker product for matrices. Therefore,

\[
{\left( \left( T + {T}^{\prime }\right)  \otimes  S\right) }_{\mathcal{C},\mathcal{C}} = {\left( T \otimes  S\right) }_{\mathcal{C},\mathcal{C}} + {\left( {T}^{\prime } \otimes  S\right) }_{\mathcal{C},\mathcal{C}} \Rightarrow  \left( {T + {T}^{\prime }}\right)  \otimes  S = T \otimes  S + {T}^{\prime } \otimes  S
\]
\end{proof} 

\noindent Alternatively, we study the linear transformations by applying them on a basis of $V \otimes W$. Another way of the proof is by computing
\(\left( {\left( {T + {T}^{\prime }}\right)  \otimes  S}\right) \left( {{\bf v}_{i} \otimes  {\bf w}_{j}}\right),
\)
where \(\left\{  {{\bf v}_{i} \otimes  {\bf w}_{j} \mid  1 \leq  i \leq  n,1 \leq  j \leq  m}\right\}\) forms a basis of \(\left( {T + {T}^{\prime }}\right)  \otimes  S\) :
\begin{align*}
\left( {\left( {T + {T}^{\prime }}\right)  \otimes  S}\right) \left( {{\bf v}_{i} \otimes  {\bf w}_{j}}\right)  &= \left( {T + {T}^{\prime }}\right) \left( {\bf v}_{i}\right)  \otimes  S\left( {\bf w}_{j}\right)
\\
&= \left( {T\left( {\bf v}_{i}\right)  + {T}^{\prime }\left( {\bf v}_{i}\right) }\right)  \otimes  S\left( {\bf w}_{j}\right)
\\
&= T\left( {\bf v}_{i}\right)  \otimes  S\left( {\bf w}_{j}\right)  + {T}^{\prime }\left( {\bf v}_{i}\right)  \otimes  S\left( {\bf w}_{j}\right)
\\
&= \left( {T \otimes  S}\right) \left( {{\bf v}_{i} \otimes  {\bf w}_{j}}\right)  + \left( {{T}^{\prime } \otimes  S}\right) \left( {{\bf v}_{i} \otimes  {\bf w}_{j}}\right)
\end{align*}
Since \(\left( {\left( {T + {T}^{\prime }}\right)  \otimes  S}\right) \left( {{\bf v}_{i} \otimes  {\bf w}_{j}}\right)\) coincides with \(\left( {T \otimes  S + {T}^{\prime } \otimes  S}\right) \left( {{\bf v}_{i} \otimes  {\bf w}_{j}}\right)\) for all basis vectors \({\bf v}_{i} \otimes  {\bf w}_{j} \in  C\), we imply
\(\left( {T + {T}^{\prime }}\right)  \otimes  S = T \otimes  S + {T}^{\prime } \otimes  S 
\).
\qed

\begin{proposition}
Let \(A,C: V \to V\) be linear operators of \(V\), and \(B,D: W \to W\) be linear operators on \(W\), then
\[
\left( {A \otimes  B}\right)  \circ  \left( {C \otimes  D}\right)  = \left( {AC}\right)  \otimes  \left( {BD}\right).
\] 
\end{proposition}

\begin{proposition}
Define linear operators \(A : V \rightarrow  V\) and \(B : W \rightarrow  W\) with \(\dim \left( V\right) ,\dim \left( W\right)  <\)  \(\infty\). Then
\[
\det \left( {A \otimes  B}\right)  = {\left( \det \left( A\right) \right) }^{\dim \left( W\right) }{\left( \det \left( B\right) \right) }^{\dim \left( V\right) }
\]
\end{proposition} 
\begin{proof}
    Homework question.
\end{proof}

\begin{corollary}
There exists a linear transformation
\[
\Phi : \operatorname{Hom}\left( {V,V}\right)  \otimes  \operatorname{Hom}\left( {W,W}\right)  \rightarrow  \operatorname{Hom}\left( {V \otimes  W,V \otimes  W}\right)
\]
with \(\Phi(A \otimes  B) :=  A \otimes  B\), where the input of \(\Phi\) is a linear transformation $A:V \to V$ tensored with another linear transformation $B: W \to W$, and the output is the tensored linear transformation on $V \otimes W$ (c.f. \autoref{prop:tensor_linear_trans}).

Moreover, if $\dim(V), \dim(W) < \infty$, then $\Phi$ is an isomorphism.
\end{corollary} 

\begin{proof}
Construct the mapping

\[
\Phi  : \operatorname{Hom}\left( {V,V}\right)  \times  \operatorname{Hom}\left( {W,W}\right)  \rightarrow  \operatorname{Hom}\left( {V \otimes  W,V \otimes  W}\right)
\]
with \(\Phi \left( {A,B}\right)  = A \otimes  B
\). Then \(\Phi\) is indeed bilinear: for instance,
\[
\Phi \left( {{pA} + {qC},B}\right)  = \left( {{pA} + {qC}}\right)  \otimes  B
= p\left( {A \otimes  B}\right)  + q\left( {C \otimes  B}\right)
= {p\Phi }\left( {A,B}\right)  + {q\Phi }\left( {C,B}\right)
\]
So the first statement follows from the universal property of tensor product.

If \(\dim \left( V\right) ,\dim \left( W\right)  < \infty\), we imply
\begin{align*}
\dim(\operatorname{Hom}\left( {V,V}\right)  \otimes  \operatorname{Hom}\left( {W,W}\right)) &= \dim \left( {\operatorname{Hom}\left( {V,V}\right) }\right) \cdot \dim \left( {\operatorname{Hom}\left( {W,W}\right) }\right)
\\
&= \left\lbrack  {\dim \left( V\right) \dim \left( V\right) }\right\rbrack   \cdot  \left\lbrack  {\dim \left( W\right) \dim \left( W\right) }\right\rbrack   \\
&= {\left\lbrack  \dim \left( V\right) \dim \left( W\right) \right\rbrack  }^2
\\
&= {\left\lbrack  \dim \left( V \otimes  W\right) \right\rbrack  }^2
\\
&= \dim \left( {\operatorname{Hom}\left( {V \otimes  W,V \otimes  W}\right) }\right)
\end{align*}
Therefore, is \(\Phi\) is an isomorphism.    
\end{proof}

As a result, every linear operator \(\alpha  : V \otimes  W \rightarrow V \otimes  W\) can be expressed as
\[
\alpha  = {A}_1 \otimes  {B}_1 + \cdots  + {A}_{k} \otimes  {B}_{k}
\]
where \({A}_{i} : V \rightarrow  V\) and \({B}_{j} : W \rightarrow  W\).