\chapter{Multilinear \& Exterior Products}

\section{Multilinear Tensor Product}
In this section, we will generalize the tensor product of more than 2 vector spaces:
\begin{definition}[Tensor Product among multiple vector spaces]
Let $V_1, \ldots, V_p$ be vector spaces over $\mathbb{F}$. Let 
$$S = \{(\mathbf{v}_1, \ldots, \mathbf{v}_p) \mid \mathbf{v}_i \in V_i\}$$ 
(we assume no relations among distinct elements in $S$), and define $\mathfrak{X} = \text{span}(S)$. Then the tensor product space $V_1 \otimes \cdots \otimes V_p = \mathfrak{X} / \mathfrak{Y}$, where $\mathfrak{Y}$ is the vector subspace of $\mathfrak{X}$ spanned by vectors of the form
$$(\mathbf{v}_1, \ldots, \mathbf{v}_i + \mathbf{v}_i', \ldots, \mathbf{v}_p) - (\mathbf{v}_1, \ldots, \mathbf{v}_i, \ldots, \mathbf{v}_p) - (\mathbf{v}_1, \ldots, \mathbf{v}_i', \ldots, \mathbf{v}_p),$$ 
 and
$$(\mathbf{v}_1, \ldots, \alpha \mathbf{v}_i, \ldots, \mathbf{v}_p) - \alpha (\mathbf{v}_1, \ldots, \mathbf{v}_i, \ldots, \mathbf{v}_p)$$
for $i = 1, 2, \ldots, p$.

Moreover, we define
\[\mathbf{v}_1 \otimes \cdots \otimes \mathbf{v}_p := [(\mathbf{v}_1, \ldots, \mathbf{v}_p) + \mathcal{Y}] \in V_1 \otimes \cdots \otimes V_p.\]
\end{definition}

\begin{remark} \label{rmk:mult_tensor}
Similar to the tensor product of two vector spaces, we have:
  \begin{align*}
\mathbf{v}_1 \otimes \cdots \otimes (\alpha \mathbf{v}_i + \beta \mathbf{v}_i') \otimes \cdots \otimes \mathbf{v}_p = \alpha (\mathbf{v}_1 \otimes \cdots \otimes \mathbf{v}_i \otimes \cdots \otimes \mathbf{v}_p) + \beta (\mathbf{v}_1 \otimes \cdots \otimes \mathbf{v}_i' \otimes \cdots \otimes \mathbf{v}_p).
  \end{align*}
Also, a general vector in $V_1 \otimes \cdots \otimes V_p$ is:
  \[
    \sum_{i=1}^n \left( \mathbf{w}_1^{(i)} \otimes \cdots \otimes \mathbf{w}_p^{(i)} \right), \quad \text{where } \mathbf{w}_j^{(i)} \in V_j, j = 1, \ldots, p.
  \]

Finally, let $\mathcal{B}_i = \{\mathbf{v}_i^{(1)}, \ldots, \mathbf{v}_i^{(d_i)}\}$ be a basis of $V_i$, $i = 1, \ldots, p$, then
  \[
    \mathcal{B} = \left\{ \mathbf{v}_1^{(\alpha_1)} \otimes \cdots \otimes \mathbf{v}_p^{(\alpha_p)} \mid 1 \leq \alpha_i \leq d_i \right\}
  \]
  is a basis of $V_1 \otimes \cdots \otimes V_p$. As a result,
  \[
    \dim(V_1 \otimes \cdots \otimes V_p) = d_1 \times \cdots \times d_p = \dim(V_1) \times \cdots \times \dim(V_p).
  \]
\end{remark}

We now introduce the universal property of multilinear tensor product:
\begin{theorem}[Universal Property of Multilinear Tensor]\label{thm:multi_tensor_universal}
Let 
$$\mathrm{Obj} = \{ \phi : V_1 \times \cdots \times V_p \to W \mid \phi \text{ is a $p$-linear map} \},$$ 
i.e. for any $\phi \in \mathrm{Obj}$:
\[
\phi(\mathbf{v}_1, \ldots, \alpha \mathbf{v}_i + \beta \mathbf{v}_i', \ldots, \mathbf{v}_p) 
= \alpha \phi(\mathbf{v}_1, \ldots, \mathbf{v}_i, \ldots, \mathbf{v}_p)
+ \beta \phi(\mathbf{v}_1, \ldots, \mathbf{v}_i', \ldots, \mathbf{v}_p),
\]
for all $\mathbf{v}_i, \mathbf{v}_i' \in \mathbf{v}_i$, $i = 1, \ldots, p$, $\forall \alpha, \beta \in \mathbb{F}$. Then
\[i : V_1 \times \cdots \times V_p \to V_1 \otimes \cdots \otimes V_p\]
defined by
\[({\bf v}_1, \ldots, {\bf v}_p) \mapsto {\bf v}_1 \otimes \cdots \otimes {\bf v}_p,
\]
is in $\mathrm{Obj}$. Moreover, $(i : V_1 \times \cdots \times V_p \to V_1 \otimes \cdots \otimes V_p) \in \mathrm{Obj}$ satisfies the \emph{universal property} of multilinear tensor product, that is, for all $(\phi: V_1 \times \dots \times V_p \to W) \in \mathrm{Obj}$, there exists a unique linear transformation
\[
\bar{\phi} : V_1 \otimes \cdots \otimes V_p \to W
\]
such that the diagram commutes:
\[
\begin{tikzcd}[column sep=large, row sep=large]
V_1 \times \cdots \times V_p 
\arrow[r, "i", blue] \arrow[dr, "\phi"] &
V_1 \otimes \cdots \otimes V_p 
\arrow[d, "\bar{\phi}", red] \\
& W
\end{tikzcd}
\]
\noindent In other words, \(\phi = \bar{\phi} \circ i\). 
\end{theorem}

As an application of the universal property of multilinear tensor product, we can construct linear transformation of multilinear tensor product spaces:
\begin{corollary} \label{cor:multi_linear_trans}
Let \( T_i : V_i \to V_i' \) be a linear transformation for \( 1 \leq i \leq p \). Then there is a unique linear transformation
\[
(T_1 \otimes \cdots \otimes T_p) : V_1 \otimes \cdots \otimes V_p \to V_1' \otimes \cdots \otimes V_p'
\]
such that
\[
(T_1 \otimes \cdots \otimes T_p)(\mathbf{v}_1 \otimes \cdots \otimes \mathbf{v}_p) = T_1(\mathbf{v}_1) \otimes \cdots \otimes T_p(\mathbf{v}_p).
\]
\end{corollary}

\begin{proof}
Construct the mapping
\[
\phi : V_1 \times \cdots \times V_p \to V_1' \otimes \cdots \otimes V_p'
\]
defined by
\[
(\mathbf{v}_1, \ldots, \mathbf{v}_p) \mapsto T_1(\mathbf{v}_1) \otimes \cdots \otimes T_p(\mathbf{v}_p),
\]
which is indeed \( p \)-linear.  
By the universal property, we obtain the unique linear transformation
\[
\bar{\phi} : V_1 \otimes \cdots \otimes V_p \to V_1' \otimes \cdots \otimes V_p'
\]
satisfying our desired properties.
\end{proof}

\section{Exterior Products}
From now on, we only consider \( V_1 = \cdots = V_p = V \).  
Then for any linear transformation \( T : V \to W \), we define
\[
T^{\otimes p} : V \otimes \cdots \otimes V \to W \otimes \cdots \otimes W.
\]
We use the shorthand notation \( V^{\otimes p} \) to denote
\[
\underbrace{V \otimes \cdots \otimes V}_{\text{$p$ terms in total}}.
\]
\begin{definition}[Alternating Map]
A $p$-linear map $\phi : V \times \cdots \times V \to W$ is called \textbf{alternating} if
\[
\phi(\mathbf{v}_1, \ldots, \mathbf{v}_i, \ldots, \mathbf{v}_j, \ldots, \mathbf{v}_p) = 0_W, \quad \text{whenever } \mathbf{v}_i = \mathbf{v}_j \text{ for } i \neq j.
\]
We also say $\phi$ is \emph{$p$-alternating}.
\end{definition}

\begin{example}
The cross product map is an alternating bilinear map:
\[
\phi : \mathbb{R}^3 \times \mathbb{R}^3 \to \mathbb{R}^3, \quad (\mathbf{v}, \mathbf{w}) \mapsto \mathbf{v} \times \mathbf{w}.
\]
It satisfies:
\begin{itemize}
  \item $\phi$ is bilinear,
  \item $\phi(\mathbf{v}, \mathbf{v}) = \mathbf{v} \times \mathbf{v} = \mathbf{0}.$
\end{itemize}
\end{example}

\begin{example}
The determinant map is also alternating:
\[
\phi : \underbrace{\mathbb{F}^n \times \cdots \times \mathbb{F}^n}_{\text{$n$ terms in total}} \to \mathbb{F}, \quad (\mathbf{v}_1, \ldots, \mathbf{v}_n) \mapsto \det([\mathbf{v}_1, \mathbf{v}_2, \cdots, \mathbf{v}_n])
\]
It satisfies:
\begin{itemize}
  \item $\phi$ is $n$-linear (MAT 2040 knowledge),
  \item $\phi$ is alternating (again, from MAT 2040 since the determinant of a matrix with repeated columns is zero).
\end{itemize}
\end{example}

\begin{theorem}[Universal Property for Exterior Power]\label{thm:univ-exterior}
Let 
\[ \mathrm{Obj} := \left\{ \phi : V \times \cdots \times V \to W \,\middle|\, \phi \text{ is $p$-alternating} \right\} .\] 
Then there exists a universal object
\[
(\Lambda : V \times \cdots \times V \longrightarrow E) \in \mathrm{Obj}
\]
such that for every \( (\phi: V \times \dots \times V \to W) \in \mathrm{Obj} \), there exists a unique linear map
\[
\overline{\phi} : E \longrightarrow W
\]
making the following diagram commute:
\[
\begin{tikzcd}[row sep=large, column sep=large]
  V \times \cdots \times V 
    \arrow[r, "\Lambda", dashed, blue] 
    \arrow[dr, "\phi"'] 
    & E 
    \arrow[d, "\overline{\phi}", red] \\
   & W 
\end{tikzcd}
\]
In other words, \(\phi = \overline{\phi} \circ \Lambda.
\)
\end{theorem}

We now construct the universal object $(\Lambda: V \times \dots \times V \to E) \in \mathrm{Obj}$:
\begin{definition}[Wedge Products]\label{def:wedgeproduct}
Let $V$ be a vector space. Consider the subspace $U \leq V^{\otimes p}$ spanned by the vectors
$$U = \mathrm{span}\{ \cdots \otimes {\bf v} \otimes \cdots \otimes {\bf v} \otimes \cdots\ |\ {\bf v} \in V\}$$
with a repeated entry. Then the $p$-\textbf{exterior power} is defined by the quotient space:
\[
\wedge^p V := V^{\otimes p} / U,
\]
Similar to the case of $V^{\otimes p}$, we define the vector
\[
\mathbf{v}_1 \wedge \cdots \wedge \mathbf{v}_p := \mathbf{v}_1 \otimes \cdots \otimes \mathbf{v}_p + U \in \wedge^p V.
\]
\end{definition}

We claim that the mapping
\[
\Lambda : V \times \cdots \times V \to  \wedge^p V,\] 
defined by \[(\mathbf{v}_1, \ldots, \mathbf{v}_p) \mapsto \mathbf{v}_1 \wedge \cdots \wedge \mathbf{v}_p
\]
is in $\mathrm{Obj}$, and it satisfies the universal property of exterior power (c.f. \autoref{thm:univ-exterior}).
\begin{proposition} \label{prop:wedge_alternating}
The $p$-exterior power $\wedge^p V$ is $p$-linear: 
\begin{align*}
\mathbf{v}_1 \wedge \cdots \wedge (a\mathbf{v}_i + b \mathbf{v}_i') \wedge \cdots \wedge \mathbf{v}_p &= a \left(\mathbf{v}_1 \wedge \cdots \wedge \mathbf{v}_i \wedge \cdots \wedge \mathbf{v}_p\right) + b \left(\mathbf{v}_1 \wedge \cdots \wedge \mathbf{v}_i' \wedge \cdots \wedge \mathbf{v}_p\right)
\end{align*}
and alternating:
  \begin{align*}
    \mathbf{v}_1 \wedge \cdots \wedge \mathbf{v} \wedge \cdots \wedge \mathbf{v} \wedge \cdots \wedge \mathbf{v}_p = {\bf 0}_{\wedge^pV}
  \end{align*}
Consequently, $\Lambda: V \times \dots \times V \to \wedge^pV$ is in $\mathrm{Obj}$.
\end{proposition}
\begin{proof}
The proof is trivial - as an example, for the second statement:
$$\mathbf{v}_1 \wedge \cdots \wedge \mathbf{v} \wedge \cdots \wedge \mathbf{v} \wedge \cdots \wedge \mathbf{v}_p := \mathbf{v}_1 \otimes \cdots \otimes \mathbf{v} \otimes \cdots \otimes \mathbf{v} \otimes \cdots \otimes \mathbf{v}_p + U.$$
Since $\mathbf{v}_1 \otimes \cdots \otimes \mathbf{v} \otimes \cdots \otimes \mathbf{v} \in U$ by definition, it is equal to  ${\bf 0} + U = {\bf 0}_{\wedge^p V}$. 
\end{proof}

We leave it to the readers to check that $\Lambda: V \times \dots \times V \to \wedge^p V$ satisfies the universal property for exterior power (\autoref{thm:univ-exterior}). 

\begin{remark} \label{rmk:wedge_alternating}
The wedge product satisfies the sign reversal property:
  \begin{equation} \label{eq:reversal}
    \mathbf{v}_1 \wedge \cdots \wedge \mathbf{v} \wedge \cdots \wedge \mathbf{w} \wedge \cdots \wedge \mathbf{v}_p = -\mathbf{v}_1 \wedge \cdots \wedge \mathbf{w} \wedge \cdots \wedge \mathbf{v} \wedge \cdots \wedge \mathbf{v}_p.
  \end{equation}
For instance, one has:
\[(\mathbf{v} + \mathbf{w}) \wedge (\mathbf{v} + \mathbf{w}) = {\bf 0} \ \Rightarrow \ \mathbf{v} \wedge \mathbf{v} + \mathbf{v} \wedge \mathbf{w} + \mathbf{w} \wedge \mathbf{v} + \mathbf{w} \wedge \mathbf{w} = {\bf 0} \ \Rightarrow \ \mathbf{v} \wedge \mathbf{w} + \mathbf{w} \wedge \mathbf{v} = {\bf 0}.\]
One can easily generalize the above proof to obtain \autoref{eq:reversal}.
\end{remark}


\begin{proposition}
If $\dim(V) = n$ and $0 \leq p \leq n$, then
\[
\dim(\wedge^p V) = \binom{n}{p}.
\]
\end{proposition}
\begin{proof} 
Let $\{\mathbf{v}_1, \dots, \mathbf{v}_n\}$ be a basis of $V$, then $\{ \mathbf{v}_{i_1} \otimes \cdots \otimes \mathbf{v}_{i_p} \mid 1 \leq i_k \leq n\}$ forms a basis of $V^{\otimes p}$. Since $\pi_{\wedge^p V} : V^{\otimes p} \to V^{\otimes p}/U = \wedge^p V$ is surjective, the image 
\[ 
\{ \pi_{\wedge^pV}(\mathbf{v}_{i_1} \otimes \cdots \otimes \mathbf{v}_{i_p}) = \mathbf{v}_{i_1} \wedge \cdots \wedge \mathbf{v}_{i_p} \mid 1 \leq i_k\leq n \}
\]
spans $\wedge^p V$.
    
By \autoref{prop:wedge_alternating} and \autoref{rmk:wedge_alternating}, many of the terms in the above expression are zero or repeated. For instance, using \autoref{rmk:wedge_alternating}, one can reorder the expression $\mathbf{v}_{i_1} \wedge \cdots \wedge \mathbf{v}_{i_p}$, so that (up to $\pm$) the indices
$i_1 < \dots < i_p$ are in ascending order. 
After removing the repeated entries, one conclude that
$$\mathcal{B} := \{ \mathbf{v}_{i_1} \wedge \cdots \wedge \mathbf{v}_{i_p} \mid 1 \leq i_1 < \dots < i_k \leq n\}$$
spans $\wedge^pV.$

We omit the proof that $\mathcal{B}$ is linearly independent - its proof is similar to that of \autoref{thm:tensorbasis}. Consequently, $\mathcal{B}$ is a basis of $\wedge^pV$ which has cardinality equal to $\binom{n}{p}$.
\end{proof}

Let $T: V \to V$ be a linear operator. By applying the universal property for exterior power (\autoref{thm:univ-exterior}), one can obtain as in \autoref{cor:multi_linear_trans}:
\begin{corollary} \label{cor:exterior_transformation}
Let $V$ be a vector space, and $T:V \to V$ be a linear operator. For all $0 \leq p \leq \dim(V)$, there exists a unique linear operator
  \[
  T^{\wedge^p} : \wedge^p V \to \wedge^p V
  \]
defined by:
  \[
  \mathbf{v}_1 \wedge \cdots \wedge \mathbf{v}_p \mapsto T(\mathbf{v}_1) \wedge \cdots \wedge T(\mathbf{v}_p).
  \]
\end{corollary}



\section{The Determinant}
To end this Chapter, we define the determinant for all linear operators $T:V \to V$ on finite-dimensional vector space $V$ directly. As a consequence, we will prove that
\[
\det(T) = \det \left( (T)_{\mathcal{B}, \mathcal{B}} \right) \ \text{for any basis } \mathcal{B} \text{ of } T.
\]
which implies that if $T({\bf x}) = A{\bf x}$ is a matrix transformation, one has $\det(T) = \det(A)$.

Moreover, we will have a natural way of proving 
$\det(T \circ S) = \det(T) \det(S)$, which in term will imply the well-known formula of $\det(AB) = \det(A) \det(B)$ in MAT2040.

In short, our definition will give generalize the notion of determinant of matrices to linear operators, and give a natural explanation of why determinant is multiplicative.

\begin{definition}[Determinant for Linear Operators] Let $\dim(V) = n$. Then
\( \dim(\wedge^n V) = \binom{n}{n} = 1.\)
By \autoref{cor:exterior_transformation} we have a linear operator 
$$T^{\wedge n} : \wedge^n V \to \wedge^n V.$$
on the $1$-dimensional vector space $\wedge^n V$.

Therefore, for all $\tau \in \wedge^n V$, there exists $\alpha_T \in \mathbb{F}$ such that
\[T^{\wedge n}(\tau) = \alpha_T \tau,\]
and we define the \emph{determinant} of $T$ by:
\(\det(T) := \alpha_T,\) so that
$T^{\wedge n}(\tau) = \det(T) \tau $
Note that this definition does not depend on any choice of basis of $V$.
\end{definition}

To compute the determinant, one can take {\bf any} basis ${\mathbf{v}_1, \dots, \mathbf{v}_n}$ of $V$, and
\[ \wedge^n V = \mathrm{span}(\mathbf{v}_1 \wedge \cdots \wedge \mathbf{v}_n).\] 
Then one computes
$$T^{\wedge n}(\mathbf{v}_1 \wedge \cdots \wedge \mathbf{v}_n) := T(\mathbf{v}_1) \wedge \cdots \wedge T(\mathbf{v}_n) = \det(T) \mathbf{v}_1 \wedge \cdots \wedge \mathbf{v}_n.$$

\begin{example}
\begin{enumerate}
    \item Suppose that $T = I : V \to V$ be the identity operator. Then
\[
I^{\wedge n}(\mathbf{v}_1 \wedge \cdots \wedge \mathbf{v}_n) = I(\mathbf{v}_1) \wedge \cdots \wedge I(\mathbf{v}_n) = \mathbf{v}_1 \wedge \cdots \wedge \mathbf{v}_n = 1 \cdot (\mathbf{v}_1 \wedge \cdots \wedge \mathbf{v}_n).
\]
Therefore, $\det(I) = 1$.

\item Suppose that $T : V \to V$ is diagonalizable with $\{\mathbf{w}_1, \dots, \mathbf{w}_n\}$ forming an eigen-basis of $T$. As a result,
\[
T^{\wedge n}(\mathbf{w}_1 \wedge \cdots \wedge \mathbf{w}_n) = T(\mathbf{w}_1) \wedge T(\mathbf{w}_2) \cdots \wedge T(\mathbf{w}_n) 
=(\lambda_1 \mathbf{w}_1) \wedge \cdots \wedge (\lambda_n \mathbf{w}_n) = (\lambda_1 \cdots \lambda_n)\mathbf{w}_1 \wedge \cdots \wedge \mathbf{w}_n
\]
Therefore, $\det(T) = \lambda_1 \cdots \lambda_n$.
\end{enumerate}
\end{example}

\begin{proposition}
Let $T, S : V \to V$ be linear transformations, then
\[
(T \circ S)^{\wedge^p} : \wedge^p V \to \wedge^p V
\]
with $T^{\wedge^p}, S^{\wedge^p} : \wedge^p V \to \wedge^p V$ satisfies
\[
(T \circ S)^{\wedge^p} = (T^{\wedge^p}) \circ (S^{\wedge^p})
\]
\end{proposition}

\begin{proof}
Pick any basis $\{ \mathbf{v}_{i_1} \wedge \cdots \wedge \mathbf{v}_{i_p} \mid 1 \leq i_1 < \cdots < i_p \leq n \}$ of $\wedge^p V$. Then
\[
(T \circ S)^{\wedge^p}(\mathbf{v}_{i_1} \wedge \cdots \wedge \mathbf{v}_{i_p}) = (T \circ S)(\mathbf{v}_{i_1}) \wedge \cdots \wedge (T \circ S)(\mathbf{v}_{i_p})
\]
On the other hand,
\begin{align*}
    (T^{\wedge^p}) \circ (S^{\wedge^p})(\mathbf{v}_{i_1} \wedge \cdots \wedge \mathbf{v}_{i_p}) = T^{\wedge^p}(S(\mathbf{v}_{i_1}) \wedge \cdots \wedge S(\mathbf{v}_{i_p})) = 
    (T \circ S)(\mathbf{v}_{i_1}) \wedge \cdots \wedge (T \circ S)(\mathbf{v}_{i_p})
\end{align*}
\end{proof}

\[
T : \mathbb{R}^2 \to \mathbb{R}^2
\quad \text{with} \quad
T\begin{pmatrix} x \\ y \end{pmatrix}
= \begin{pmatrix}
a & b \\
c & d
\end{pmatrix}
\begin{pmatrix}
x \\ y
\end{pmatrix}.
\]

\begin{example}
Let $T: \mathbb{F}^2 \to \mathbb{F}^2$ be the matrix transformation given by $T({\bf x}) = A{\bf x}$, where $A = \begin{pmatrix}
    a & b \\ c & d
\end{pmatrix}$.
We take the usual basis $\{{\bf e}_1, {\bf e}_2\}$ to compute the determinant of $T$:
\begin{align*}
\det(T) \cdot ({\bf e}_1 \wedge {\bf e}_2)
&= T({\bf e}_1) \wedge T({\bf e}_2) \\
&= \begin{pmatrix} a \\ c \end{pmatrix} \wedge \begin{pmatrix} b \\ d \end{pmatrix} \\
&= (a {\bf e}_1 + c {\bf e}_2) \wedge (b {\bf e}_1 + d {\bf e}_2) \\
&= (a b) {\bf e}_1 \wedge {\bf e}_1 + (a d) {\bf e}_1 \wedge {\bf e}_2 + (c b) {\bf e}_2 \wedge {\bf e}_1 + (c d) {\bf e}_2 \wedge {\bf e}_2 \\
&= (a d) {\bf e}_1 \wedge {\bf e}_2 + (c b) {\bf e}_2 \wedge {\bf e}_1 \\
&= (a d - b c) {\bf e}_1 \wedge {\bf e}_2.
\end{align*}
where we have used ${\bf e}_1 \wedge {\bf e}_1 = {\bf e}_2 \wedge {\bf e}_2 = {\bf 0}$ and ${\bf e}_1 \wedge {\bf e}_2 = -({\bf e}_2 \wedge {\bf e}_1)$ by \autoref{prop:wedge_alternating} and \autoref{rmk:wedge_alternating}. Therefore, we conclude:
\[
\det(T) = \Delta(A) = a d - b c.
\]
(to avoid confusion, we denote by $\Delta(A)$ the determinant of $A$ as defined in MAT2040.)
\end{example}

For general $n \times n$-matrices $A$, one also has:
\begin{theorem}\label{thm:det_linear_operator_matrix}
Let \( V = \mathbb{F}^n \), and let
\[
T : V \to V, \quad \text{with } T(\mathbf{v}) = A \mathbf{v}, \quad A \in M_{n \times n}(\mathbb{F}).
\]
Then \(\det(T) = \Delta(A).\)
\end{theorem}
\begin{proof}
Let \( \{ {\bf e}_1, \dots, {\bf e}_n \} \) be the usual basis of \( V \cong \mathbb{F}^n \), and  \( \{ {\bf a}_1, \dots, {\bf a}_n \} \) be the columns of $A$. Then:
\begin{equation} \label{eq:determinant}
\begin{aligned}
\det(T) \cdot {\bf e}_1 \wedge \cdots \wedge {\bf e}_n
  &= T^{\wedge n}({\bf e}_1 \wedge \cdots \wedge {\bf e}_n) \\
  &= T({\bf e}_1) \wedge \cdots \wedge T({\bf e}_n) \\
  &= {\bf a}_1 \wedge \cdots \wedge {\bf a}_n.
\end{aligned}
\end{equation}
We wish to show $\Delta(T) = \det(A)$. Indeed, by MAT2040, $\Delta(A)$ is uniquely determined by the following:
\begin{itemize}
    \item $\Delta(I_{n \times n}) = 1$;
    \item $\Delta( \cdots | \alpha{\bf a}_i + \beta {\bf a}_i'| \cdots) = \alpha\Delta( \cdots | {\bf a}_i| \cdots) + \beta\Delta( \cdots | {\bf a}_i'| \cdots)$.
    \item $\Delta( \cdots | {\bf v}| \cdots| {\bf v}| \cdots) = 0$
\end{itemize}
By the multilinearity and alternating property of the wedge product, one can easily see that $\det(T)$ satisfies all the above properties. For instance, for the last point, one can apply \autoref{eq:determinant} and get:
\[
\det(T) {\bf e}_1 \wedge \cdots \wedge {\bf e}_n = \cdots \wedge {\bf v} \wedge \cdots \wedge {\bf v} \wedge \cdots = {\bf 0} = 0({\bf e}_1 \wedge \cdots \wedge {\bf e}_n).
\]
Therefore, \(\det(T) = 0\) as stated in the third point. And the result follows.
\end{proof}

So we have just proved that our definition of $\det$ matches with that of the usual definition of determinant of matrices. Moreover, we have:
\begin{corollary}
\label{cor:det-composition}
\[
\det(T \circ S) = \det(T)\, \det(S)
\]
\end{corollary}

\begin{proof}
Let \( \{\mathbf{v}_1, \dots, \mathbf{v}_n\} \) be a basis of \( V \), so that \( \mathbf{v}_1 \wedge \cdots \wedge \mathbf{v}_n \) is a basis of \( \wedge^n V \). Then:
\begin{align*}
\det(T \circ S) \cdot \mathbf{v}_1 \wedge \cdots \wedge \mathbf{v}_n 
  &= (T \circ S)^{\wedge n} (\mathbf{v}_1 \wedge \cdots \wedge \mathbf{v}_n) \\
  &= T^{\wedge n} \left( S^{\wedge n}(\mathbf{v}_1 \wedge \cdots \wedge \mathbf{v}_n) \right) \\
  &= T^{\wedge n} \left( \det(S) \cdot \mathbf{v}_1 \wedge \cdots \wedge \mathbf{v}_n \right) \\
  &= \det(S) \cdot T^{\wedge n}(\mathbf{v}_1 \wedge \cdots \wedge \mathbf{v}_n) \\
  &= \det(S) \cdot \det(T) \cdot \mathbf{v}_1 \wedge \cdots \wedge \mathbf{v}_n.
\end{align*}
Since \( \mathbf{v}_1 \wedge \cdots \wedge \mathbf{v}_n \neq 0 \), it follows that
\[
\det(T \circ S) = \det(T)\det(S).
\]
\end{proof}



