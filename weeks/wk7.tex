\chapter{Minimal Polynomial \& Cayley-Hamilton Theorem}
\section{ Minimal Polynomial}

\begin{definition}[Linear Operator Induced From Polynomial]
Let \( f(x) := a_m x^m + \cdots + a_0 \in \mathbb{F}[x] \), and let \( T : V \to V \) be a linear operator. Then the operator
\[
f(T) := a_m T^m + \cdots + a_1 T + a_0 I : V \to V
\]
is called the linear operator induced by the polynomial \( f(x) \).
\end{definition}

The composition of linear operators is not commutative, i.e. \( S \circ T \neq T \circ S \) in general. This follows similarly to the fact that matrix multiplication is not commutative. However, we always have
  \[
  f(T) T = T f(T),
  \]
for any polynomial \( f \in \mathbb{F}[x] \) - Indeed, we can show that \( T^n T = T T^n \) for all \( n \) by induction. Now for \( f(x) = \sum_i a_i x^i \), then
  \[
  f(T) T = \sum_i a_i T^i T = \sum_i a_i T T^i = T \sum_i a_i T^i = T f(T). 
  \]
More generally, one can apply induction argument again to show that the composition of polynomial-induced operators is commutative:
  \[
  f(T) g(T) = g(T) f(T),
  \]
for all \( f(x), g(x) \in \mathbb{F}[x] \).

\begin{definition}[Minimal Polynomial]
Let \( T : V \to V \) be a linear operator. The \emph{minimal polynomial} \( m_T(x) \) is the unique monic polynomial of least degree such that
\[
m_T(T) = \mathbf{0}_{V \to V},
\]
where \( \mathbf{0}_{V \to V} \) denotes the zero operator in \( \operatorname{Hom}(V, V) \).
\end{definition}

\begin{example}[Minimal Polynomials of Matrices]
\leavevmode
\begin{enumerate}
  \item Let \( A = \begin{pmatrix} 1 & 0 \\ 0 & 1 \end{pmatrix} \), then
  \[
  \mathcal{X}_A(x) = (x - 1)^2, \quad m_A(x) = x - 1,
  \]
  since \( A - I = 0 \).

  \item Let \( B = \begin{pmatrix} 1 & 1 \\ 0 & 1 \end{pmatrix} \), then
  \[
  \mathcal{X}_B(x) = (x - 1)^2.
  \]
  Could the minimal polynomial be \( x - 1 \)? No, because
  \[
  B - I = \begin{pmatrix} 0 & 1 \\ 0 & 0 \end{pmatrix} \neq 0.
  \]
  However,
  \[
  (B - I)^2 = \begin{pmatrix} 0 & 0 \\ 0 & 0 \end{pmatrix} \Rightarrow m_B(x) = (x - 1)^2.
  \]
\end{enumerate}
\end{example}

Two natural questions arise:
\begin{enumerate}
  \item Does \( m_T(x) \) always exist? Is it unique?
  \item What is the relationship between \( m_T(x) \) and \( \mathcal{X}_T(x) \)?
\end{enumerate}

Regarding the existence of minimal polynomial, note that $m_T(x)$ may not exist if \( V \) is infinite-dimensional:
\begin{example}
Let \( V = \mathbb{R}[x] \), and define
\[
T : V \to V, \quad p(x) \mapsto \int_0^x p(t)\, dt.
\]
Then, for monomials: \( T(x^n) = \frac{1}{n+1} x^{n+1} \). Suppose \( m_T(x) = x^n + \cdots + a_0 \), then
\[
m_T(T) = T^n + \cdots + a_0 I = 0.
\]
Applying this to \( x \in V \), we would have:
\[
m_T(T)(x) = \frac{1}{n!} x^n + a_{n-1} \frac{1}{(n-1)!} x^{n-1} + \cdots + a_0 x = 0.
\]
But this is a contradiction since the LHS is a nonzero polynomial while the RHS is zero. Hence, no such minimal polynomial exists.
\end{example}

\subsection{ Properties of the Minimal Polynomial}

\begin{proposition}[Existence]\label{prop:minpoly-existence}
The minimal polynomial \( m_T(x) \) always exists if \( \dim(V) = n < \infty \).
\end{proposition}

\begin{proof}
We consider the set
\[
\{ I, T, T^2, \dots, T^{n^2} \} \subseteq \operatorname{Hom}(V, V),
\]
which lies in a vector space $\mathrm{Hom}(V,V)$ of dimension \( n^2 \). Therefore, this set is linearly dependent, and there exist scalars \( a_0, \dots, a_{n^2} \), not all zero, such that
\[
a_0 I + a_1 T + \cdots + a_{n^2} T^{n^2} = 0.
\]
That is, there exists a nonzero polynomial \( g(x) \in \mathbb{F}[x] \) with \( \deg(g) < n^2 \) such that \( g(T) = 0 \). Thus a minimal-degree such polynomial exists.
\end{proof}

\begin{proposition}[Uniqueness]\label{prop:minpoly-uniqueness}
The minimal polynomial \( m_T(x) \), if it exists, is unique.
\end{proposition}

\begin{proof}
Suppose \( f_1(x) \neq f_2(x) \) are two minimal polynomials with the same minimal degree. Then their difference
\[
g(x) = f_1(x) - f_2(x) \neq 0,
\]
satisfies \( \deg(g) < \deg(f_1) \) and
\[
g(T) = f_1(T) - f_2(T) = 0.
\]
After scaling \( g(x) \) to be monic, we obtain a polynomial of strictly lower degree annihilating \( T \), contradicting the minimality of \( f_1 \). Hence \( m_T(x) \) is unique.
\end{proof}

\begin{proposition}[Divisibility]\label{prop:minpoly-divides}
Let \( f(x) \in \mathbb{F}[x] \) satisfy \( f(T) = 0 \). Then
\[
m_T(x) \mid f(x).
\]
\end{proposition}

\begin{proof}
By the division algorithm, write
\[
f(x) = q(x) m_T(x) + r(x), \quad \deg(r) < \deg(m_T).
\]
Then
\[
f(T) = q(T) m_T(T) + r(T) = 0 + r(T) = r(T).
\]
Since \( f(T) = 0 \), we get \( r(T) = 0 \). But then \( r(x) \) is a polynomial of smaller degree than \( m_T \) that annihilates \( T \), contradicting the minimality of \( m_T \). Thus \( r(x) \equiv 0 \), and \( m_T(x) \mid f(x) \).
\end{proof}

\begin{proposition}[Similarity Invariance]\label{prop:minpoly-similarity}
Let \( A, B \in \mathbb{F}^{n \times n} \) be similar, i.e., \( B = P^{-1} A P \) for some invertible matrix \( P \). Then
\[
m_A(x) = m_B(x).
\]
\end{proposition}

\begin{proof}
Let \( m_A(x) = x^k + a_{k-1}x^{k-1} + \cdots + a_0 \). Then
\[
m_A(B) = m_A(P^{-1} A P) = P^{-1} m_A(A) P = P^{-1}(0)P = 0.
\]
Hence \( m_A(x) \) annihilates \( B \), so by \autoref{prop:minpoly-divides}, \( m_B(x) \mid m_A(x) \). A symmetric argument shows \( m_A(x) \mid m_B(x) \). Since both are monic, we conclude
\[
m_A(x) = m_B(x).
\]
\end{proof}

\begin{remark}
\autoref{prop:minpoly-similarity} states that the minimal polynomial is invariant under similarity. In fact, the characteristic polynomial is similarity-invariant as well.
\end{remark}

% From now on we assume finite-dimensionality.
% Define standard notation for minimal polynomial.

We now assume \( \dim(V) < \infty \). The minimal polynomial \( m_T(x) \) denotes the unique monic polynomial of least degree such that
\[
m_T(T) = 0 \in \operatorname{Hom}(V, V).
\]



\section{Cayley–Hamilton Theorem}
\subsection{Statement of the Theorem}
We begin with an example: consider the matrix \( A = \begin{pmatrix} 1 & 0 \\ 0 & 2 \end{pmatrix} \), and its associated linear operator on \( \mathbb{F}^2 \). The characteristic polynomial of \( A \) is
\[
\chi_A(x) = (x - 1)(x - 2).
\]
Note that the minimal polynomial \( m_A(x) \) cannot have degree one. Otherwise, \( m_A(x) = x - k \) for some scalar \( k \), and we would have
\[
m_A(A) = A - k I = \begin{pmatrix} 1 - k & 0 \\ 0 & 2 - k \end{pmatrix} \neq 0,
\]
for any \( k \in \mathbb{F} \), contradicting the definition of minimal polynomial. Therefore, \( m_A(x) \) must be
\[
m_A(x) = (x - 1)(x - 2),
\]
which coincides with the characteristic polynomial. In fact, one has:

\begin{theorem}[Cayley–Hamilton]\label{thm:cayley-hamilton}
Let \( T : V \to V \) be a linear operator on a finite-dimensional vector space over a field \( \mathbb{F} \). Then the characteristic polynomial \( \mathcal{X}_T(x) \) annihilates \( T \), i.e.,
\[
\mathcal{X}_T(T) = 0.
\]
Equivalently, the minimal polynomial \( m_T(x) \) divides the characteristic polynomial:
\[
m_T(x) \mid \mathcal{X}_T(x).
\]
\end{theorem}

In the course of the proof of \autoref{thm:cayley-hamilton}, we would assume 
$\mathcal{X}_T(x) = (x - \lambda_1)^{e_1} \cdots (x-\lambda_k)^{e_k}$
can be factorized into linear terms. However, this is not always possible over arbitrary \( \mathbb{F} \). For instance, consider
\[
A = \begin{pmatrix} 0 & -1 \\ 1 & 0 \end{pmatrix} \in M_{2 \times 2}(\mathbb{R}).
\]
Its characteristic polynomial is \( \mathcal{X}_A(x) = x^2 + 1 \), which cannot be factorized into linear terms in \( \mathbb{R}[x] \). However, we may always extend the field to its {\bf algebraic closure} 
\[ \overline{\mathbb{F}} \supseteq \mathbb{F} \] (e.g. $\mathbb{C} \supseteq \mathbb{R}$) so that \( \mathcal{X}_A(x) \in \mathbb{F}[x] \subseteq \overline{\mathbb{F}}[x]\) always admits the factorization (e.g. $\mathcal{X}_A(x) = (x - i)(x +1)$ in $\mathbb{C}[x]$).

Our strategy to prove \autoref{thm:cayley-hamilton} is:
\begin{itemize}
    \item Consider the case where \( m_T(x), \mathcal{X}_T(x)\) are in \(\overline{\mathbb{F}}[x]\) 
    \item Show that \(m_T(x) \mid \mathcal{X}_T(x)\) under \(\overline{\mathbb{F}}[x]\).
    \item Since both \(m_T(x)\) and \(\mathcal{X}_T(x)\) are in $\mathbb{F}[x]$, one must also have \(m_T(x) \mid \mathcal{X}_T(x)\) under \(\mathbb{F}[x]\).
\end{itemize}

\subsection{Invariant Subspace}
Before we give the full proof, we develop tools that allow us to decompose the vector space using invariant subspaces.
\begin{definition}[Invariant Subspace]\label{def:invariant-subspace}
Let \( T : V \to V \) be a linear operator. A subspace \( W \leq V \) is called \( T \)-invariant if
\[
T(W) \leq W.
\]
\end{definition}

\begin{remark}
If \( W \leq V \) is \( T \)-invariant, then the restriction \( T|_W : W \to W \) defines a well-defined linear operator.
\end{remark}

\begin{example}\label{ex:invariant-subspace}
The following are examples of \( T \)-invariant subspaces:
\begin{enumerate}
    \item The whole space \( V \) is trivially \( T \)-invariant.
    \item For each eigenvalue \( \lambda \), the eigenspace \( \ker(T - \lambda I) \) is \( T \)-invariant.
    \item More generally, for any polynomial \( g \in \mathbb{F}[x] \), the space
    \[
    U = \ker(g(T)) \leq V
    \]
    is \( T \)-invariant.

    \emph{Proof:} If \( {\bf v} \in \ker(g(T)) \), then \( g(T)({\bf v}) = 0 \). We want to show \( T({\bf v}) \in \ker(g(T)) \), i.e., \( g(T)(T({\bf v})) = 0 \). Observe:
    \[
    g(T)(T({\bf v})) = (a_m T^m + \cdots + a_1 T + a_0 I)(T({\bf v})) = T(g(T)({\bf v})) = T({\bf 0}) = {\bf 0}.
    \]
    \item If \( {\bf v} \in \ker(T - \lambda I) \), then \( \operatorname{span}\{{\bf v}\} \) is a \( T \)-invariant subspace.
\end{enumerate}
\end{example}

\begin{proposition}\label{prop:block-diagonal}
Let \( T : V \to V \) be a linear operator, and suppose \( W \leq V \) is a \( T \)-invariant subspace. Then we have the induced linear operators:
\begin{align}
\left. T \right|_W & : W \to W, \quad \mathbf{w} \mapsto T(\mathbf{w}) \label{eq:restriction-map} \\
\widetilde{T} & : V/W \to V/W, \quad \mathbf{v} + W \mapsto T(\mathbf{v}) + W \label{eq:quotient-map}
\end{align}
(The well-definedness of \( \widetilde{T} \) is proved in Homework 2, Exercise 4.)

With respect to a suitable basis, the matrix representation of \( T \) has the block form:
\[
[T]_{\mathcal{B},\mathcal{B}} =
\begin{pmatrix}
[\left. T \right|_W]_{\mathcal{C}, \mathcal{C}} & * \\
0 & [\widetilde{T}]_{\overline{\mathcal{B}}, \overline{\mathcal{B}}}
\end{pmatrix}
\]
where \( \mathcal{C} = \{ \mathbf{v}_1, \dots, \mathbf{v}_k \} \) is a basis of \( W \), and \( \overline{\mathcal{B}} = \{ \mathbf{v}_{k+1} + W, \dots, \mathbf{v}_n + W \} \) is a basis of \( V/W \).

It follows that the characteristic polynomial of \( T \) satisfies:
\[
\mathcal{X}_T(x) = \mathcal{X}_{\left. T \right|_W}(x) \cdot \mathcal{X}_{\widetilde{T}}(x).
\]
\end{proposition}

\subsection{Triangularizable Transformation}
As mentioned before, we will now focus on the case when 
\[
\mathcal{X}_T(x) = (x - \lambda_1) \cdots (x - \lambda_n) \in \mathbb{F}[x]
\]
can be factorized into linear terms (for instance, when $\mathbb{F} = \overline{\mathbb{F}}$ is algebraically closed.
\begin{proposition}\label{prop:upper-triangular-form}
Suppose
\[
\mathcal{X}_T(x) = (x - \lambda_1) \cdots (x - \lambda_n)
\]
where the \( \lambda_i \in \mathbb{F} \) are not necessarily distinct. Then there exists a basis \( \mathcal{A} \) of \( V \) such that
\[
T_{\mathcal{A}, \mathcal{A}} =
\begin{pmatrix}
\lambda_1 & * & \cdots & * \\
0 & \lambda_2 & \cdots & * \\
\vdots & \vdots & \ddots & \vdots \\
0 & 0 & \cdots & \lambda_n
\end{pmatrix}.
\]
In such a case, we call $T$ a {\bf triangularizable} linear transformation.
\end{proposition}

Obviously, the converse of the above proposition also holds, i.e. if $T:V \to V$ is triangularizable, then $\mathcal{X}_T(x) = (x - \lambda_1) \cdots (x - \lambda_n)$ can be factorized into linear terms. We will not make use of this fact in the proof of \autoref{thm:cayley-hamilton}, though.
\begin{proof}
We proceed by induction on \( n = \dim V \).

\textbf{Base Case:} For \( n = 1 \), the matrix of \( T \) is scalar \( (\lambda_1) \), and the statement holds.

\textbf{Induction Hypothesis:} Assume the result holds for all dimensions \( < n \). We show it holds for \( \dim V = n \).

\textbf{Step 1: Existence of eigenvector.}
Let \( \lambda_1 \) be an eigenvalue of \( T \). By the fundamental theorem of algebra (applied over \( \mathbb{C} \)), there exists an eigenvector \( \mathbf{v} \in V \setminus \{{\bf 0}\} \) such that
\[
T(\mathbf{v}) = \lambda_1 \mathbf{v}.
\]

\textbf{Step 2: Invariant subspace and quotient.}
Define \( W = \operatorname{span}\{\mathbf{v}\} \leq V \), which is \( T \)-invariant. By \autoref{prop:block-diagonal}, \( T \) induces a well-defined operator \( \widetilde{T} : V/W \to V/W \) with characteristic polynomial
\[
\mathcal{X}_{\widetilde{T}}(x) = (x - \lambda_2) \cdots (x - \lambda_n).
\]

\textbf{Step 3: Apply induction.}
By the induction hypothesis, there exists a basis \( \overline{\mathcal{C}} = \{ \mathbf{w}_2 + W, \dots, \mathbf{w}_n + W \} \) of \( V/W \) such that
\[
\widetilde{T}_{\overline{\mathcal{C}}, \overline{\mathcal{C}}} =
\begin{pmatrix}
\lambda_2 & * & \cdots & * \\
0 & \lambda_3 & \cdots & * \\
\vdots & \vdots & \ddots & \vdots \\
0 & 0 & \cdots & \lambda_n
\end{pmatrix}.
\]

\textbf{Step 4: Lift basis to \( V \).}
Define \( \mathcal{A} := \{ \mathbf{v}, \mathbf{w}_2, \dots, \mathbf{w}_n \} \). Then \( \mathcal{A} \) is a basis of \( V \) (see Homework 2, Exercise 2), and in this basis we have:
\[
T_{\mathcal{A}, \mathcal{A}} =
\begin{pmatrix}
\lambda_1 & * \\
0 & \widetilde{T}_{\overline{\mathcal{C}}, \overline{\mathcal{C}}}
\end{pmatrix}
=
\begin{pmatrix}
\lambda_1 & * & * & * \\
0 & \lambda_2 & \cdots & * \\
0 & \cdots & \ddots & * \\
0 & 0 & \cdots & \lambda_n
\end{pmatrix}
\]
as given in \autoref{prop:block-diagonal}. So the result follows.
\end{proof}
\begin{proposition}\label{prop:cayley-hamilton-diagonalizable}
Suppose that the characteristic polynomial of a linear operator \( T \) is
\[
\mathcal{X}_{T}(x) = (x - \lambda_1)\cdots(x - \lambda_n) \in \mathbb{F}[x].
\]
Then
\[
\mathcal{X}_{T}(T) = \mathbf{0}_{V \to V}.
\]
In other words, \autoref{thm:cayley-hamilton} holds for $T$.
\end{proposition}


\begin{proof}
Since ${\mathcal{X}}_{T}(x) = (x - \lambda_1) \cdots (x - \lambda_n)$, we imply $T$ is triangularizable under some basis $\mathcal{A}$, i.e.
$$T_{\mathcal{A},\mathcal{A}} = \begin{pmatrix}
\lambda_1 & \times & \cdots & \times \\
0 & \lambda_2 & \ddots & \vdots \\
\vdots & \ddots & \ddots & \times \\
0 & \cdots & 0 & \lambda_n
\end{pmatrix}.$$
Note that
\begin{itemize}
\item $\ast \mapsto {\left( \ast \right)}_{\mathcal{A},\mathcal{A}}$ is an isomorphism between $\operatorname{Hom}(V,V)$ and ${M}_{n \times n}(\mathbb{F})$ (c.f. \autoref{thm: coordinate-isomorphism} with $S = T$ and $\mathcal{A} = \mathcal{B} = \mathcal{C}$).
\item ${\left( T^m \right)}_{\mathcal{A},\mathcal{A}} = \left[ T_{\mathcal{A},\mathcal{A}} \right]^m$ for any $m$ (c.f. \autoref{prop:functoriality}).
\end{itemize}
Therefore, if
${\mathcal{X}}_{T}(x) = x^n + \alpha_{n-1}x^{n-1} + \dots + \alpha_1 x + \alpha_0,$ then
\begin{align*}
[{\mathcal{X}}_{T}(T)]_{\mathcal{A},\mathcal{A}} &= 
[T^n + \alpha_{n-1}T^{n-1} + \dots + \alpha_1 T + \alpha_0I]_{\mathcal{A},\mathcal{A}} \\
&= (T_{\mathcal{A},\mathcal{A}})^n + \alpha_{n-1}(T_{\mathcal{A},\mathcal{A}})^{n-1} + \dots + \alpha_1 T_{\mathcal{A},\mathcal{A}} + \alpha_0I_{\mathcal{A},\mathcal{A}} \\
&= {\mathcal{X}}_{T}\left( T_{\mathcal{A},\mathcal{A}}\right)
\end{align*}
Therefore,
$$\mathcal{X}_{T}(T) = \mathbf{0}_{V \to V} \quad \Leftrightarrow \quad [\mathcal{X}_{T}(T)]_{\mathcal{A},\mathcal{A}} = {\bf 0}_{n \times n} \quad \Leftrightarrow \quad {\mathcal{X}}_{T}\left( T_{\mathcal{A},\mathcal{A}}\right) = {\bf 0}_{n \times n},$$
and it suffices to show ${\mathcal{X}}_{T}\left( T_{\mathcal{A},\mathcal{A}}\right)$ is the zero matrix. To see so, note that
\[
{\mathcal{X}}_{T}\left( T_{\mathcal{A},\mathcal{A}} \right) = \left( T_{\mathcal{A},\mathcal{A}} - \lambda_1 \mathbf{I} \right) \cdots \left( T_{\mathcal{A},\mathcal{A}} - \lambda_n \mathbf{I} \right),
\]
where $T_{\mathcal{A},\mathcal{A}}$ is as given above. Write

$$M_i := \left( T_{\mathcal{A},\mathcal{A}} - \lambda_i \mathbf{I} \right) = \begin{pmatrix}
\lambda_1 - \lambda_i & \times & \cdots & \cdots &\times \\
0 & \ddots & \ddots & & \vdots \\
\vdots & \ddots & {\bf 0} & \ddots& \vdots \\
\vdots &  & \ddots &\ddots  &\times \\
0 & \cdots & \cdots & 0 &\lambda_n - \lambda_i
\end{pmatrix}$$
so that ${\mathcal{X}}_{T}\left( T_{\mathcal{A},\mathcal{A}}\right) = M_1\cdots M_i \cdots M_n$. Then
\begin{align*}
{\mathcal{X}}_{T}\left( T_{\mathcal{A},\mathcal{A}}\right)
\begin{pmatrix}
x_1 \\
\vdots \\
x_{n-1} \\
x_n
\end{pmatrix}
&=
M_1\cdots M_{n-1} M_n \begin{pmatrix}
x_1 \\
\vdots \\
x_{n-1} \\
x_n
\end{pmatrix} \\
&= M_1\cdots M_{n-1}\begin{pmatrix}
\lambda_1 - \lambda_n & \times &  \cdots &\times \\
0 & \ddots & \ddots & \vdots \\
\vdots &  \ddots &\lambda_{n-1} - \lambda_n  &\times \\
0  & \cdots & 0 & 0
\end{pmatrix}\begin{pmatrix}
x_1 \\
\vdots \\
x_{n-1} \\
x_n
\end{pmatrix} \\
&= M_1\cdots M_{n-1}\begin{pmatrix}
x_1' \\
\vdots \\
x_{n-1}' \\
0
\end{pmatrix}
\end{align*}
So we `kill off' the last coordinate of $\mathbf{v} =  \begin{pmatrix}
x_1 \\
\vdots \\
x_n 
\end{pmatrix}$. Continuing the process, one will kill of the $n-1$, $n-2$, $\dots$ until the top entry of ${\bf v}$, i.e.
\[
{\mathcal{X}}_{T}\left( T_{\mathcal{A},\mathcal{A}}\right){\bf v} = M_n \cdots M_1 \mathbf{v} = {\bf 0}.
\]
Consequently, ${\mathcal{X}}_{T}\left( T_{\mathcal{A},\mathcal{A}} \right)$ is a zero matrix.
\end{proof}

\section{Proof of Cayley–Hamilton Theorem}
Now we are ready to give a proof for the Cayley-Hamilton Theorem:
\begin{theorem}[Cayley–Hamilton]\label{thm:cayley-hamilton-weaker}
Let \( T : V \to V \) be a linear operator on a finite-dimensional vector space over a field \( \mathbb{F} \). Then the characteristic polynomial \( \mathcal{X}_T(x) \) annihilates \( T \), i.e.,
\[
\mathcal{X}_T(T) = {\bf 0}.
\]
\end{theorem}

\begin{proof}
Suppose that \( \mathcal{X}_T(x) = x^n + a_{n - 1}x^{n - 1} + \cdots + a_0 \in \mathbb{F}[x] \). By considering an algebraically closed field \( \overline{\mathbb{F}} \supseteq \mathbb{F} \), we write
\begin{align}
\mathcal{X}_T(x) &= x^n + a_{n - 1}x^{n - 1} + \cdots + a_0, \quad a_i \in \mathbb{F} \label{eq:charpoly-explicit} \\
                 &= (x - \lambda_1)\cdots(x - \lambda_n), \quad \lambda_i \in \overline{\mathbb{F}}. \label{eq:charpoly-factor}
\end{align}
By applying \autoref{prop:cayley-hamilton-diagonalizable}, we conclude \( \mathcal{X}_T(T) = {\bf 0} \), where the coefficients in the formula \( \mathcal{X}_T(T) = {\bf 0} \) with respect to \( T \) lie in \( \overline{\mathbb{F}} \).

Then we argue that these coefficients are actually in \( \mathbb{F} \). Expand the full expression for \( \mathcal{X}_T(T) \):
\begin{align}
\mathcal{X}_T(T) &= (T - \lambda_1 I)\cdots(T - \lambda_n I) \label{eq:poly-subst-product} \\
                 &= T^n - (\lambda_1 + \cdots + \lambda_n) T^{n-1} + \cdots + (-1)^n \lambda_1 \cdots \lambda_n I \label{eq:poly-subst-expanded} \\
                 &= T^n + a_{n - 1} T^{n - 1} + \cdots + a_0 I. \label{eq:poly-subst-original}
\end{align}
The equality \eqref{eq:poly-subst-original} holds because the coefficients of the polynomials in \eqref{eq:charpoly-explicit} and \eqref{eq:charpoly-factor} are equal.

Therefore, we conclude that \( \mathcal{X}_T(T) = 0 \) over the field \( \mathbb{F} \).
\end{proof}

\begin{corollary}\label{cor:cayley-hamiton-cor}
$m_T(x) \mid \mathcal{X}_T(x)$. More precisely, if
\[
\mathcal{X}_T(x) = [p_1(x)]^{e_1} \cdots [p_k(x)]^{e_k}, \quad e_i > 0 \text{ for all } i,
\]
where $p_i$ are distinct, monic, and irreducible polynomials, then
\[
m_T(x) = [p_1(x)]^{f_1} \cdots [p_k(x)]^{f_k}, \quad \text{for some } 0 < f_i \leq e_i \text{ for all } i.
\]
\end{corollary}

\begin{proof}
The statement $m_T(x) \mid \mathcal{X}_T(x)$ follows from \autoref{thm:cayley-hamilton-weaker} and \autoref{prop:minpoly-divides}. Therefore, $0 < f_i \leq e_i$ for all $i$.

Suppose on the contrary that $f_i = 0$ for some $i$. Without loss of generality, let $i = 1$.

It’s clear that $\gcd(p_1, p_j) = 1$ for all $j \neq 1$, which implies
\[
a(x)p_1(x) + b(x)p_j(x) = 1, \quad \text{for some } a(x), b(x) \in \mathbb{F}[x].
\]
Considering the field extension $\overline{\mathbb{F}} \supseteq \mathbb{F}$, we have
\[
p_1(x) = (x - \mu_1) \cdots (x - \mu_\ell).
\]
For any root $\mu_m$ of $p_1$, $m = 1, \ldots, \ell$, we have
\[
a(\mu_m)p_1(\mu_m) + b(\mu_m)p_j(\mu_m) = 1 \Rightarrow b(\mu_m)p_j(\mu_m) = 1 \Rightarrow p_j(\mu_m) \neq 0,
\]
i.e., $\mu_m$ is not a root of $p_j$ for all $j \neq 1$.

Therefore, $\mu_m$ is a root of $\mathcal{X}_T(x)$ but not a root of $m_T(x)$. Then $\mu_m$ is an eigenvalue of $T$, i.e., $T \mathbf{v} = \mu_m \mathbf{v}$ for some $\mathbf{v} \neq 0$.

Recall that $m_{T,\mathbf{v}} = x - \mu_m$, so $m_{T,\mathbf{v}} = x - \mu_m \mid m_T(x)$, which contradicts the assumption $f_1 = 0$.
\end{proof}

\begin{example}\label{ex:minpoly-examples}
We can use \autoref{cor:cayley-hamiton-cor}, a stronger version of the Cayley–Hamilton Theorem, to determine minimal polynomials:

\begin{enumerate}
  \item For the matrix \( A = \begin{pmatrix} 0 & -1 \\ 1 & 1 \end{pmatrix} \), we compute
  \[
  \mathcal{X}_A(x) = \left(x^2 + x + 1\right)^1.
  \]
  Since \( x^2 + x + 1 \) is irreducible over \( \mathbb{R} \), we conclude
  \[
  m_A(x) = x^2 + x + 1.
  \]

  \item For the matrix
  \[
  A = \begin{pmatrix}
  1 & 1 & 0 & 0 \\
  0 & 1 & 0 & 0 \\
  0 & 0 & 2 & 0 \\
  0 & 0 & 0 & 2
  \end{pmatrix},
  \]
  we compute
  \[
  \mathcal{X}_A(x) = (x - 1)^2(x - 2)^2.
  \]
  By \autoref{cor:cayley-hamiton-cor}, both \( (x - 1) \) and \( (x - 2) \) must divide \( m_A(x) \), so the minimal polynomial must be one of:
  \[
  (x - 1)^2(x - 2)^2, \quad (x - 1)(x - 2)^2, \quad (x - 1)^2(x - 2), \quad (x - 1)(x - 2).
  \]
  By trial and error, we find that
  \[
  m_A(x) = (x - 1)^2(x - 2).
  \]
\end{enumerate}
\end{example}